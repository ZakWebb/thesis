%======================================================================
%   Thesis Front Information
%======================================================================

%%%%%%%%%%%%%%%%%%%%%%%%%%%%%%
% Title Page

% The title page is counted as page `i' but we need to suppress the
% page number.  We also don't want any headers or footers.
\pagestyle{empty}
\pagenumbering{roman}

% The contents of the title page are specified in the "titlepage"
% environment.
\begin{titlepage}
        \begin{center}
        \vspace*{1.0cm}

        \Huge
        {\bf The computational power of many-body systems}

        \vspace*{1.0cm}

        \normalsize
        by \\

        \vspace*{1.0cm}

        \Large
        Zak Webb \\

        \vspace*{3.0cm}

        \normalsize
        A thesis \\
        presented to the University of Waterloo \\ 
        in fulfillment of the \\
        thesis requirement for the degree of \\
        Doctor of Philosophy \\
        in \\
        Physics and Astronomy (Quantum Information) \\

        \vspace*{2.0cm}

        Waterloo, Ontario, Canada, 2016 \\

        \vspace*{1.0cm}

        \copyright\ Zak Webb, 2016 \\
        \end{center}
\end{titlepage}

% The rest of the front pages should contain no headers and be numbered using Roman numerals starting with `ii'
\pagestyle{plain}
\setcounter{page}{2}

\cleardoublepage % Ends the current page and causes all figures and tables that have so far appeared in the input to be printed.

 


%%%%%%%%%%%%%%%%%%%%%%%%%%%%%%
% Declaration Page

\begin{center}\textbf{Author's Declaration}\end{center}

  \noindent
I hereby declare that I am the sole author of this thesis. This is a true copy of the thesis, including any required final revisions, as accepted by my examiners.

  \bigskip
  
  \noindent
I understand that my thesis may be made electronically available to the public.

\cleardoublepage


%%%%%%%%%%%%%%%%%%%%%%%%%%%%%%
% Abstract

\begin{center}\textbf{Abstract}\end{center}

Many-body systems are well known throughout physics to be hard problems to exactly solve, but much of this is folklore resulting from the lack of an analytic solution to these systems.  This thesis attempts to classify the complexity inherent in many of these systems, and give quantitative results for why the problems are hard.  In particular, we analyze the many-particle system corresponding to a multi-particle quantum walk, showing that the time evolution of such systems on a polynomial sized graph is universal for quantum computation, and thus determining how a particular state evolves is as hard as an arbitrary quantum computation.  We then analyze the ground energy properties of related systems, showing that for bosons, bounding the ground energy of the same Hamiltonian with a fixed number of particles is \QMA-complete.  Similar techniques provide a novel proof that quantum walk is universal for quantum computing, and constructs a \QMA-complete problem that does not reference quantum mechanics.  


\cleardoublepage

%%%%%%%%%%%%%%%%%%%%%%%%%%%%%%
% Acknowledgements

\begin{center}\textbf{Acknowledgements}\end{center}

I would very much like to thank my advisor, Andrew Childs, for his great help and support during my graduate studies at the University of Waterloo.  His research guidance and support during hard times was very helpful, and this thesis would not be written without him.

I would also like to thank David Gosset, who acted as a sounding board, and guided much of my research.  He was a fountain of useful ideas, with an ability to conceive of complex solutions to difficult problems.  His work ethic made many of the results in this thesis possible.

I'd like to thank Chris Pugh, who got me started back on track, as well as Norbert L\"{u}tkenhaus.  To all of my friends at the Institute for Quantum Computing, from the optics groups of my early years to the eclectic bunch now eating cookies at 4, thank you.
\cleardoublepage

%%%%%%%%%%%%%%%%%%%%%%%%%%%%%%
% Dedication

\begin{center}\textbf{Dedication}\end{center}

This is dedicated to everyone who has helped me get through troubled times.
\cleardoublepage

%%%%%%%%%%%%%%%%%%%%%%%%%%%%%%
% Table of Contents

\renewcommand\contentsname{Table of Contents}
\tableofcontents
\cleardoublepage
\phantomsection

%%%%%%%%%%%%%%%%%%%%%%%%%%%%%%
% List of Tables

%\addcontentsline{toc}{chapter}{List of Tables}
%\listoftables
%\cleardoublepage
%\phantomsection		% allows hyperref to link to the correct page

%%%%%%%%%%%%%%%%%%%%%%%%%%%%%%
% L I S T   O F   F I G U R E S

\addcontentsline{toc}{chapter}{List of Figures}
\listoffigures
\cleardoublepage
\phantomsection		% allows hyperref to link to the correct page


%%%%%%%%%%%%%%%%%%%%%%%%%%%%%%
%  L I S T   O F   S Y M B O L S

% To include a Nomenclature section
% \addcontentsline{toc}{chapter}{\textbf{Nomenclature}}
% \renewcommand{\nomname}{Nomenclature}
% \printglossary
% \cleardoublepage
% \phantomsection % allows hyperref to link to the correct page
% \newpage

% Change page numbering back to Arabic numerals
\pagenumbering{arabic}

