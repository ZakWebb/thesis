%======================================================================
%   Zak Webb
%   Ph. D. Thesis
%   Department of Physics and Astronomy
%   University of Waterloo
%======================================================================
 
\documentclass[11pt,oneside]{book}

%%%%%%%%%%%%%%%%%%%%%%%%%%%%%%
% Included Packages

\usepackage{amsmath,amssymb,amstext,amsthm}
\usepackage[pdftex]{graphicx}
\usepackage{etoolbox}
\usepackage{subfiles}
\usepackage[margin=1in]{geometry}
\usepackage{thm-restate}
\usepackage{mdframed}
\usepackage[lofdepth,lotdepth,labelformat=simple]{subfig}
\usepackage{enumitem}
\usepackage{nicefrac}
\usepackage{thmtools}
\usepackage{pgfplots}


%%%%%%%%%%%%%%%%%%%%%%%%%%%%%%
% Print/Online Version

\newtoggle{print}              % Is this electronic or print?
\togglefalse{print}

%%%%%%%%%%%%%%%%%%%%%%%%%%%%%%
% Page Setup

\iftoggle{print}{
  \addtolength{\oddsidemargin}{0.125in}
  \addtolength{\textwidth}{-0.125in}
}{
}

%%%%%%%%%%%%%%%%%%%%%%%%%%%%%%
%ensure "pages left intentionally blank" don't have numbers
\let\origdoublepage\cleardoublepage
\newcommand{\clearemptydoublepage}{%
  \clearpage{\pagestyle{empty}\origdoublepage}}
\let\cleardoublepage\clearemptydoublepage

%%%%%%%%%%%%%%%%%%%%%%%%%%%%%%
% Hyperref Macros

\usepackage[pdftex,letterpaper=true,pagebackref=false]{hyperref}

\hypersetup{
    plainpages=false,       % needed if Roman numbers in frontpages
    pdfpagelabels=true,     % adds page number as label in Acrobat's page count
    bookmarks=true,         % show bookmarks bar
    unicode=false,          % non-Latin characters in Acrobat��s bookmarks
    pdftoolbar=true,        % show Acrobat��s toolbar
    pdfmenubar=true,        % show Acrobats menu
    pdffitwindow=false,     % window fit to page when opened
    pdfstartview={FitH},    % fits the width of the page to the window
    pdftitle={The computational power of many-body systems},    
    pdfauthor={Zak Webb},    % author
    pdfsubject={Physics and Astronomy Thesis},  % subject
    colorlinks=true,        % false: boxed links; true: colored links
    linkcolor=blue,         % color of internal links
    citecolor=green,        % color of links to bibliography
    filecolor=magenta,      % color of file links
    urlcolor=cyan,           % color of external links
    breaklinks=true,
}

  % if we are using paper, don't have color links 
\iftoggle{print}{
  \hypersetup{	
    citecolor=black,
    filecolor=black,
    linkcolor=black,
    urlcolor=black
  }
}{}

\newcommand{\eq}[1]{\hyperref[eq:#1]{(\ref*{eq:#1})}}
\renewcommand{\sec}[1]{\hyperref[sec:#1]{\mbox{Section~\ref*{sec:#1}}}}
\newcommand{\chap}[1]{\hyperref[chap:#1]{\mbox{Chapter~\ref*{chap:#1}}}}
\newcommand{\app}[1]{\hyperref[app:#1]{\mbox{Appendix~\ref*{app:#1}}}}
\newcommand{\thm}[1]{\hyperref[thm:#1]{\mbox{Theorem~\ref*{thm:#1}}}}
\newcommand{\lem}[1]{\hyperref[lem:#1]{\mbox{Lemma~\ref*{lem:#1}}}}
\newcommand{\defn}[1]{\hyperref[defn:#1]{\mbox{Definition~\ref*{defn:#1}}}}
\newcommand{\ex}[1]{\hyperref[ex:#1]{\mbox{Example~\ref*{ex:#1}}}}
\newcommand{\fct}[1]{\hyperref[fct:#1]{\mbox{Fact~\ref*{fct:#1}}}}
\newcommand{\fig}[1]{\hyperref[fig:#1]{\mbox{Figure~\ref*{fig:#1}}}}
\newcommand{\cor}[1]{\hyperref[cor:#1]{\mbox{Corollary~\ref*{cor:#1}}}}
\newcommand{\figs}[1]{\hyperref[fig:#1]{\mbox{Figures~\ref*{fig:#1}}}}
\newcommand{\propo}[1]{\hyperref[prop:#1]{\mbox{Proposition~\ref*{fig:#1}}}}
\newcommand{\subfig}[1]{\mbox{\protect\subref{fig:#1}}}




%%%%%%%%%%%%%%%%%%%%%%%%%%%%%%
% TikZ stuff

\usepackage{tikz}
\usetikzlibrary{calc}
\usetikzlibrary{external}
\usetikzlibrary{decorations.markings,decorations.pathreplacing}
\usetikzlibrary{fadings, shadings, patterns}
\tikzexternalize[prefix=plots/]

\newcounter{mycount}

\pgfmathdeclarefunction{gauss}{2}{%
  \pgfmathparse{1/(#2*sqrt(2*pi))*exp(-((x-#1)^2)/(2*#2^2))}%
}


%%%%%%%%%%%%%%%%%%%%%%%%%%%%%%
% Bra/Ket Macros
\newcommand{\tinyspace}{\mspace{1mu}}
\newcommand{\microspace}{\mspace{2mu}}

\newcommand{\ket}[1]{|\microspace #1 \microspace \rangle}
\newcommand{\bra}[1]{\langle\microspace #1 \microspace |}
\newcommand{\braket}[2]{\langle\microspace #1 \microspace | \microspace #2\microspace \rangle}
\newcommand{\braMidket}[3]{\langle\microspace #1\microspace |\microspace #2
	\microspace |\microspace #3\microspace\rangle}
\newcommand{\xval}[1]{\langle \microspace #1 \microspace \rangle}
\newcommand{\ketbra}[2]{\ket{#1}\mspace{-2mu}\bra{#2}}
\newcommand{\den}[1]{\ketbra{#1}{#1}}

\newcommand{\bigket}[1]{\big|\microspace #1 \microspace \big\rangle}
\newcommand{\bigbra}[1]{\big\langle\microspace #1 \microspace \big|}
\newcommand{\bigbraket}[2]{\big\langle\microspace #1 \microspace \big| \microspace #2\microspace \big\rangle}


\newcommand{\Norm}[1]{\left\|{#1}\right\|}
\newcommand{\norm}[1]{\|{#1}\|}

%%%%%%%%%%%%%%%%%%%%%%%%%%%%%%
% Math Operator Macros

\DeclareMathOperator{\poly}{poly}
\DeclareMathOperator{\spn}{span}
\DeclareMathOperator{\Sym}{Sym}
\DeclareMathOperator{\ASym}{Asym}
\DeclareMathOperator{\adj}{adj}
\DeclareMathOperator{\sign}{sign}

%%%%%%%%%%%%%%%%%%%%%%%%%%%%%%
% Scattering states macro
\newcommand{\scat}{\text{sc}}

%%%%%%%%%%%%%%%%%%%%%%%%%%%%%%
% Set Name Macros

\newcommand{\NN}{\mathbb{N}}
\newcommand{\ZZ}{\mathbb{Z}}
\newcommand{\QQ}{\mathbb{Q}}
\newcommand{\RR}{\mathbb{R}}
\newcommand{\CC}{\mathbb{C}}
\newcommand{\II}{\mathbb{I}}
\newcommand{\FF}{\mathbb{F}}
\newcommand{\posint}{\ZZ^+}

\newcommand{\OO}{\mathcal{O}}

\newcommand{\HHH}{\mathcal{H}}
\newcommand{\AAA}{\mathcal{A}}
\newcommand{\BBB}{\mathcal{B}}


%%%%%%%%%%%%%%%%%%%%%%%%%%%%%%
% Adjusting Counter Depth  

\setcounter{secnumdepth}{3}
\setcounter{tocdepth}{3}

%%%%%%%%%%%%%%%%%%%%%%%%%%%%%%
% Complexity Class Macros
  
\newcommand{\class}[1]{\textbf{\textsc{#1}}}
\newcommand{\PP}{\class{P}}
\newcommand{\NP}{\class{NP}}
\newcommand{\BQP}{\class{BQP}}
\newcommand{\QMA}{\class{QMA}}

\newcommand{\QCircVer}{\textsc{Quantum Circuit Verification}}
\newcommand{\QWE}{\textsc{Quantum Walk Evolution}}

%%%%%%%%%%%%%%%%%%%%%%%%%%%%%%
% Complexity Class Macros


%%%%%%%%%%%%%%%%%%%%%%%%%%%%%%
% Todo Command

\newcommand{\todo}[1]{\textcolor{red}{[\textbf{TO~DO:}~\textit{#1}]} }
\newcommand{\missingcite}[1]{\textcolor{red}{[\textbf{CITE:}~\textit{#1}]} }

%%%%%%%%%%%%%%%%%%%%%%%%%%%%%%
% Theorem Setup

\theoremstyle{definition}
\newtheorem{definition}{Definition}
\newtheorem{problem}{Problem}
\newtheorem{example}{Example}

\theoremstyle{plain}
\newtheorem{theorem}{Theorem}
\newtheorem{lemma}{Lemma}
\newtheorem{fact}{Fact}
\newtheorem{proposition}{Proposition}
\newtheorem{corollary}{Corollary}
\newtheorem{type}{Type}

%%%%%%%%%%%%%%%%%%%%%%%%%%%%%%%
%  Standardized names

\newcommand{\med}{\text{med}}
\newcommand{\Tgate}{\mathrm{T}}
\newcommand{\CP}{\mathrm{CP}}
\newcommand{\CD}{\mathrm{C}\theta}
\newcommand{\XX}{\mathrm{X}}
\newcommand{\CZ}{\mathrm{CZ}}
\newcommand{\CNOT}{\mathrm{CNOT}}
\newcommand{\SWAP}{\mathrm{SWAP}}

\DeclareMathOperator{\AP}{AP}
\DeclareMathOperator{\In}{In}
\DeclareMathOperator{\Cube}{Cube}
\DeclareMathOperator{\diam}{diam}
\DeclareMathOperator{\diag}{diag}
\DeclareMathOperator{\new}{new}
\DeclareMathOperator{\tr}{Tr}
\DeclareMathOperator{\pr}{Pr}
\DeclareMathOperator{\erf}{erf}
\DeclareMathOperator{\erfi}{erfi}

\newcommand{\Had}{\mathrm{H}}

\newcommand{\dmin}{d_{\text{min}}}
\newcommand{\dmax}{d_{\text{max}}}


%%%%%%%%%%%%%%%%%%%%%%%%%%%%%%%
%  Referencing across subfiles

\usepackage{zref-xr}

\zexternaldocument{../chapters/introduction.tex}
\zexternaldocument{../chapters/mathematical_preliminaries.tex}
\zexternaldocument{../chapters/scattering_on_graphs.tex}
\zexternaldocument{../chapters/SP_universality.tex}
\zexternaldocument{../chapters/MP_universality.tex}
\zexternaldocument{../chapters/SP_ground.tex}
\zexternaldocument{../chapters/MP_ground.tex}
\zexternaldocument{../chapters/Spin_ground.tex}
\zexternaldocument{../chapters/conclusions.tex}

