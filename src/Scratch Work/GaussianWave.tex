\documentclass[../thesis-main/thesis-main]{subfiles}
\begin{document}

\begin{theorem}
  Let $\widehat{G}$ be an $(N+m)$-vertex graph, let $G$ be the graph obtained from $\widehat{G}$ by attaching $N$ semi-infinite paths to the first $N$ of its vertices, and let $S$ be the corresponding $S$-matrix.  Let $\ket{\psi_j(0)}$ be a cut-off Gaussian distribution with momentum $k$ and standard deviation $\sigma$ centered at $\mu$, with the cut-off at a distance $L$ from the center.  Namely, let 
  \begin{equation}
    \ket{\psi^j(0)} = \gamma \sum_{x = \mu - L}^{\mu + L} e^{- i k x} e^{- (x - \mu)^2/2\sigma^2} \ket{x,j}  \label{eq:guass_start_defn},
  \end{equation}
  where $\gamma$ is the normalization of $\ket{\psi^j(0)}$.  Then let us define the state 
  \begin{align}
    \ket{\alpha^j(t)} &= \gamma e^{-2 i t \cos k} \sum_{x = \mu(t) -L}^{\mu(t)+L} e^{-i k x} e^{ -(x - \mu(t))^2/2\sigma^2} 
    			 \ket{x,j} & t < \frac{\mu-L}{2|\sin k|}\\
			 \gamma e^{-2 i t \cos k} \sum_{x = \mu(t)-L}^{\mu(t) + L} \sum_{q=1}^N  S_{qj}(k) e^{i k x} e^{-(x + \mu(t))^2/2\sigma^2} \ket{x,q} & t > \frac{\mu+L}{2|\sin k|},\end{cases}\label{eq:guass_approximation_defn}
  \end{equation}
  where
  \begin{equation}
    \mu(t) = \mu - \lceil 2 t \sin(k)\rceil.
  \end{equation}
  If $\sigma = c_1 \frac{ L}{\sqrt{\log L}}$ for some constant $c_1$, we then have that for $0 < t < \frac{\mu - L}{2|\sin k|}$ and for $\frac{\mu + L}{2|\sin k|} < t < c_2 L$ (assuming the bounds make sense),  
  \begin{equation}
    \norm{e^{-i H_g t} \ket{\psi^j(0)} - \ket{\alpha^j (t)}} \leq \chi \sqrt{\frac{\log L}{L}},
  \end{equation}  
  for some constant $\chi$.
\end{theorem}

To simply many of these equations, it will be useful to define a function
\begin{equation}
  h_L^\sigma(\phi) = \sum_{n = -L}^L e^{i \phi n} e^{-\frac{ x^2}{2\sigma^2}}. 
  \label{eq:h_L_defn}
\end{equation}
This function is closely related to the $\Theta(z,q)$ is the Jacobi theta function, for which we refer the reader to Chapter 10 of \cite{SSCA} for a broad overview.  $\Theta$ is defined for all complex $z$ and all $q$ with positive imaginary part, and is related to our $h$ as 
\begin{equation}
  h_{\infty}^\sigma(\phi) = \sum_{n = -\infty}^\infty e^{ i \phi n} e^{ - \frac{ n^2}{2\sigma^2}} = \Theta\Bigg(\frac{\phi}{2\pi}, \frac{i}{2\pi \sigma^2} \Bigg).
\end{equation}

Using this equality, we can then use Theorem 1.6 from Chapter 10 of \cite{SSCA} to see that
\begin{equation}
  h_{\infty}^\sigma (\phi) = \Theta\Bigg(\frac{\phi}{2\pi}, \frac{i}{2\pi \sigma^2} \Bigg) = \sqrt{2\pi} \sigma e^{ - \frac{\sigma^2\phi^2}{2}} \Theta \big(i \phi\sigma^2, 2\pi i \sigma^2 \big) =  \sqrt{2\pi} \sigma e^{ - \frac{\sigma^2\phi^2}{2}}  h_{\infty}^{1/(2\pi \sigma)}\big(2\pi i \phi \sigma^2 \big).
  \label{eq:discrete_fourier_transform}
\end{equation}
This can be viewed as a discrete version of a Fourier transform, as the summand goes from a Gaussian distribution with standard deviation $\sigma$ to one that has standard deviation $\sigma^{-1}$.  Additionally, note that the argument to the $h$ function is now complex.

Let us now give some bounds comparing the various $h_L^\sigma$.  Note that this simple bound for the normal distribution came from \cite{Cook09}.  Assuming that $L > 0$, and that $\phi$ is real, we have that
\begin{align}
  \big|h_{\infty}^{\sigma}(\phi) - h_{L}^{\sigma}(\phi) \big| &= \Big| \sum_{n=L+1}^\infty 2 \cos (n\phi) e^{- \frac{n^2}{2\sigma^2}}\Big|\\
   &\leq 2 \sum_{n = L+1}^\infty e^{ -\frac{n^2}{2\sigma^2}}\\ 
   &\leq 2 \int_{L}^\infty e^{- \frac{x^2}{2\sigma^2}} dx\\
   &= 2 \sigma \int_{L/\sigma}^\infty e^{- \frac{u^2}{2}}{du}\\
   &< 2 \sigma \int_{L/\sigma}^\infty \frac{\sigma u}{L} e^{-\frac{u^2}{2}} du\\
   &= \frac{2 \sigma^2}{L} e^{- \frac{L^2}{2\sigma^2}}\label{eq:hL_bound},
\end{align}
while if $L = 0$ we instead have
\begin{align}
  \big|h_{\infty}^{\sigma}(\phi) - 1 \big| &= \Big| \sum_{n=1}^\infty 2 \cos (n\phi) e^{- \frac{n^2}{2\sigma^2}}\Big|\\
   &\leq 2e^{- \frac{1}{2\sigma^2}} +  \sum_{n = 2}^\infty e^{ -\frac{n^2}{2\sigma^2}}\\ 
   &\leq 2e^{- \frac{1}{2\sigma^2}} + 2 \int_{1}^\infty e^{- \frac{x^2}{2\sigma^2}} dx\\
   &< 2 e^{- \frac{1}{2\sigma^2}} +  2\sigma^2 e^{-\frac{1}{2\sigma^2}}\\
   &= 2(1+ \sigma^2) e^{-\frac{1}{2\sigma^2}}.
\end{align}
These two bounds will allow us to approximate many of the relevant numbers.  

We will now try to bound the size of $h_\infty^\sigma(\phi)$, for small (but real) $\sigma$ and imaginary $\phi$ (so as to use the discrete Fourier transform).  In particular, if we assume that $1 > \sigma^2 |\phi|$, we will have
\begin{align}
  h_\infty^\sigma(\phi) &= \sum_{n=-\infty}^\infty e^{i n \phi} e^{- \frac{n^2}{2\sigma^2}}\\
    &= 1 + \sum_{n=1}^\infty 2 \cos(n \phi) e^{-\frac{n^2}{2\sigma^2}}\\
    &\leq 1 + 2 \sum_{n=1}^\infty e^{ -\frac{n^2}{2\sigma^2} + n |\phi|}\\
%    &= 1 + 2 e^{ \frac{\sigma^2|\phi|^2 }{2}}\sum_{n=1}^\infty  \exp\Big[ -\frac{n^2}{2\sigma^2} + n |\phi| - \frac{ \sigma^2 |\phi|^2}{2}  \Big]\\
    &< 1 + 2 e^{- \frac{1}{2\sigma^2} + |\phi|} + 2 e^{ \frac{\sigma^2|\phi|^2 }{2}}\sum_{n=2}^\infty \exp\Big[ -\frac{1}{2}\Big(\frac{n}{\sigma} - \sigma |\phi|\Big)^2 \Big]\\
    &< 1 + 2 e^{ - \frac{1}{2\sigma^2} + |\phi|} + 2 e^{ \frac{\sigma^2|\phi|^2 }{2}}\int_{1}^\infty  \exp\Big[ -\frac{1}{2}\Big(\frac{x}{\sigma} - \sigma |\phi|\Big)^2 \Big]dx\\
    &= 1 + 2 e^{-\frac{1}{2\sigma^2} + |\phi|} + 2 \sigma e^{ \frac{\sigma^2|\phi|^2 }{2}} \int_{\frac{1}{\sigma} - \sigma|\phi|}^\infty e^{-\frac{u^2}{2}} du\\
    &< 1 + 2 e^{-\frac{1}{2\sigma^2} + |\phi|} + \frac{2 \sigma e^{\frac{\sigma^2|\phi|^2}{2}}}{\frac{1}{\sigma} - \sigma |\phi|} \int_{\frac{1}{\sigma} - \sigma|\phi|}^\infty u e^{-\frac{u^2}{2}} du\\
    &= 1 + 2 e^{-\frac{1}{2\sigma^2} + |\phi|} + \frac{2 \sigma}{\frac{1}{\sigma} - \sigma |\phi|} e^{-\frac{1}{2\sigma^2} + \phi}\\
    &= 1 + 2 \Bigg[ \frac{ 1 + (1-|\phi|) \sigma^2}{1 - \sigma^2|\phi|}\Bigg]e^{-\frac{1}{2\sigma^2} + \phi} \label{eq:h_bound}
\end{align}


We will want to show that $\ket{\alpha^j(t)}$ is always normalized for $t < \frac{\mu - L}{2|\sin k|}$ and for $t > \frac{\mu + L}{2|\sin k|}$.  In the first case, we have that 
\begin{align}
   \braket{\alpha_j(t)}{\alpha_j(t)} &= \gamma^2 \sum_{x = \mu(t) - L}{\mu(t) + L} e^{ -\frac{(x-\mu(t))^2}{\sigma^2}} = \gamma^2 h_{L}^{\sigma/\sqrt{2}} (0) = 1,
\end{align}
where we used the fact that the approximated wavepacket had not yet hit the finite graph.  In the second case, we instead have
\begin{align}
  \braket{\alpha_j(t)}{\alpha_j(t)} &= \gamma^2 \sum_{x = \mu(t) - L}{\mu(t) + L} \sum_{q = 1}^N S_{qj(k)}^*S_{qj}(k) e^{ -\frac{(x-\mu(t))^2}{\sigma^2}} \\
   &= \gamma^2 \sum_{x = \mu(t) - L}{\mu(t) + L}  e^{ -\frac{(x-\mu(t))^2}{\sigma^2}} = 1
\end{align}
where we used the fact that $S$ is a unitary matrix, and our previous results.  

Additionally, it will be helpful to actually have bounds on $\gamma^{-2}$.  In particular, we have that
\begin{align}
  \gamma^{-2} &= h_{L}^{\sigma/\sqrt{2}}(0)   < h_{\infty}^{\sigma/\sqrt{2}}(0)
     = \sqrt{\pi} \sigma h_{\infty}^{1/(\pi \sigma \sqrt{2})} (0)
     \leq \sqrt{\pi} \sigma \Big[ 1 + 2 \Big( 1 + \frac{1}{2\pi^2\sigma^2}\Big) e^{ - \pi^2\sigma^2}\Big]
\end{align}
as an upper bound, while
\begin{align}
  \gamma^{-2} &= h_{L}^{\sigma/\sqrt{2}}(0) = h_{\infty}^{\sigma/\sqrt{2}}(0) - \big( h_{L}^{\sigma/\sqrt{2}}(0) - h_\infty^{\sigma/\sqrt{2}}(0)\big) \geq \sqrt{\pi} \sigma - \frac{\sigma^2}{L} e^{-\frac{L^2}{\sigma^2}} \\
   &= \sqrt{\pi} \sigma \Big(1 - \frac{\sigma}{L\sqrt{\pi}} e^{ - \frac{L^2}{\sigma^2}}\Big) 
\end{align}
can be used as a lower bound.  However, these bounds are not particular nice to use and thus we will use the slightly weaker bounds of
\begin{align}
  \gamma^{-2} & \leq \sqrt{\pi}\sigma \big[ 1 + 3 e^{-\pi^2 \sigma^2}\big]\label{eq:gaussian_gamma_upper_bound}
\end{align}
and
\begin{align}
  \gamma^{-2} & \geq \sqrt{\pi} \sigma \big[ 1 - e^{-\frac{L^2}{\sigma^2}}\big],\label{eq:gaussian_gamma_lower_bound}
\end{align}
where we assume that $L > \sigma > 1$.

\begin{proof}



The main idea behind this proof will be to show that $\ket{\psi_j(0)}$ and $\ket{\alpha_j(t)}$ are both well approximated by a Gaussian distribution over momentum states near $k$, and then show that the time evolved Gaussian approximation of $\ket{\psi_j(0)}$ is well approximated by the Gaussian approximation for $\ket{\alpha_j(t)}$.  

In particular, let us examine the inner product between a scattering state $\ket{\scat_j(k+\phi)}$ and $\ket{\alpha_j(t)}$.  We can see that
\begin{align}
  &\braket{\scat_j(k+\phi)}{\alpha_j(t)}\nonumber\\
  &\qquad= \begin{cases} 
      \gamma e^{-2 i t \cos k} \sum_{x=\mu(t)-L}^{\mu(t)+L} \Big(e^{i \phi x} + S_{qj}^*(k+\phi) e^{-i(2k + \phi)x}\Big) e^{ -\frac{(x-\mu(t))^2}{2\sigma^2}}
        & t < \frac{\mu - L}{2|\sin(k)|}\\
      \gamma e^{-2 i t \cos k} \sum_{x=\mu(t)-L}^{\mu(t)+L} \sum_{q=1}^N \Big(\delta_{qj} e^{i (2k+\phi) x}\\
      \qquad \qquad \qquad \qquad\qquad  + S_{qj}^*(k+\phi) S_{qj}(k) e^{-i\phi x}\Big) e^{ -\frac{(x+\mu(t))^2}{2\sigma^2}}& t > \frac{\mu + L}{2|\sin(k)|}
      \end{cases} 
 \end{align}
While this looks like somewhat complicated expression, most of the amplitude results from the terms multiplied by $e^{\pm i \phi x}$.  If we approximate $S_{qj}^*(k+\phi)$ by $S_{qj}^*(k)$, for either of the time ranges the above expression can be seen as $e^{-2it \cos k} e^{ i \phi \mu(t)} h_L^\sigma(\phi)$ plus some small error terms.  Additionally, for $\phi$ small compared to $\sigma$, $h_\infty^\sigma(\phi) \propto e^{-\frac{\sigma^2\phi^2}{2}}$,  so let us define
\begin{align}
  \ket{w_j(t)} = \eta e^{- 2 i t \cos k}\int_{-\delta}^{\delta} \frac{d\phi}{2\pi} e^{ i \phi \mu(t)} e^{-\frac{\sigma^2\phi^2}{2}}\ket{\scat_{j}(k+\phi)} \label{eq:gaussian_w_defn}
\end{align}
where $\delta$ is a constant that we will define later, and where $\eta$ is an approximate normalization factor defined as
\begin{align}
  \eta^{-2} = \int_{-\infty}^{\infty} \frac{d\phi}{2\pi} e^{-\sigma^2\phi^2} = \frac{1}{2 \sqrt{\pi} \sigma}
\end{align}
The state $\ket{w_j(t)}$ will be our Gaussian approximation to the state $\ket{\alpha_j(t)}$.  

Note that the states $\ket{w_j(t)}$ are not exactly normalized, but that 
\begin{align}
  \braket{w_j(t)}{w_j(t)} &= \eta^2  \int_{-\delta}^\delta \frac{d\phi}{2\pi}  e^{-\sigma^2\phi^2} =  \eta^2  \int_{-\infty}^\infty \frac{d\phi}{2\pi}  e^{-\sigma^2\phi^2} - 2 \eta^2 \int_{\delta}^\infty \frac{d\phi}{2\pi} e^{-\sigma^2 \phi^2} \\
  &= 1 - \frac{2 \sigma}{\sqrt{\pi}} \int_{\delta}^\infty d\phi e^{-\sigma^2\phi^2}.
\end{align}
Hence, we have that
\begin{align}
  \braket{w_j(t)}{w_j(t)} & \geq 1 - \frac{1}{\delta \sigma \sqrt{\pi}} e^{-\sigma^2\delta^2}
\end{align}
and
\begin{align}
  \braket{w_j(t)}{w_j(t)} & \leq 1 - \frac{ 2\sigma \delta}{\sqrt{\pi}(2 \sigma^2 \delta^2 + 1)}e^{- \sigma^2 \delta^2}
\end{align}


\begin{comment}

In particular, we can use the bound of \eq{h_bound} to see
\begin{align}
  \eta^{-2} 
  &\leq 2  \sigma^2\int_{0}^\delta d\phi \Big[ h_{\infty}^{1/(2\pi\sigma)} \big(2 \pi i \phi \sigma^2\big)\Big]^2\\
  &\leq 2\sigma^2 \int_{0}^\delta d\phi \Bigg(1 + 2 \Bigg[ \frac{ 1 + (1 - 2\pi \phi \sigma^2) \frac{1}{4\pi^2\sigma^2}}{1 - \frac{1}{4\pi^2 \sigma^2} (2\pi\phi\sigma^2)}\Bigg]e^{-\frac{(2\pi\sigma)^2}{2} + 2\pi \phi\sigma^2}\Bigg)^2\\
  & = 2 \sigma^2 \int_{0}^\delta d\phi \Bigg( 1 + 2 \Bigg[ 1 + \frac{1}{2\pi \sigma^2}\frac{1}{2\pi- \phi}\Bigg] e^{2\pi \sigma^2 (\phi - \pi)}\Bigg)^2\\
  & < 2 \sigma^2 \delta\Bigg( 1 + 2 \Bigg[ 1 + \frac{1}{2\pi \sigma^2}\frac{1}{2\pi - \delta}\Bigg] e^{2\pi \sigma^2 (\delta - \pi)}\Bigg)^2
\end{align}
and if $\delta < \pi/2$, we find that
\begin{equation}
  \eta^{-2}  < 2 \sigma^2 \delta \big( 1 + 4 e^{-\pi^2 \sigma^2}\big).
\end{equation}
If we then note that for purely imaginary $\theta$, $h_L^\sigma(\theta) < h_{L+1}^\sigma(\theta)$ as the additional terms are always positive, and that $h_0^\sigma(\theta) = 1$, we can also bound this value by
\begin{equation}
  \eta^{-2} > 2 \sigma^2 \int_{0}^\delta d \phi e^{ - \sigma^2 \phi^2}
   > 2\sigma^2 \int_{0}^\delta d\phi e^{-\sigma^2\delta^2}= 2 \sigma^2 \delta e^{-\sigma^2\delta^2}
\end{equation}

\end{comment}

We will also want to know the overlap of these approximations with the states that they will approximate, namely $\ket{\alpha_j(t)}$.  If $t < \frac{\mu - L}{2|\sin k|}$, then we have that
\begin{align}
  \braket{w_j(t)}{\alpha_j(t)} &= \eta e^{2 i t \cos k} \int_{-\delta}^\delta \frac{d\phi}{2\pi} \braket{\scat_j(k+\phi)}{\alpha_j(t)}  e^{-\frac{\sigma^2\phi^2}{2}} e^{ - i \phi \mu(t)}\\
  &= \gamma \eta \int_{-\delta}^\delta \frac{d\phi}{ 2\pi} e^{-\frac{\sigma^2\phi^2}{2}} \big( h_L^\sigma(\phi) + S_{jj}^*(k+\phi) e^{-2 i (\phi + k) \mu(t)}h_L^\sigma(2k+\phi) \big)\\
  &= \gamma \eta \int_{-\delta}^\delta \frac{d\phi}{2\pi} e^{-\frac{\sigma^2\phi^2}{2}} h_\infty^\sigma(\phi) +\gamma \eta\int_{-\delta}^\delta \frac{d\phi}{2\pi}e^{-\frac{\sigma^2\phi^2}{2}}  \big( h_L^\sigma(\phi) - h_\infty^\sigma(\phi)\big) \nonumber \\
  & \qquad + \gamma \eta \int_{-\delta}^\delta \frac{d\phi}{2\pi} S_{jj}^*(k+\phi) e^{-2 i (\phi + k)\mu(t)}e^{-\frac{\sigma^2\phi^2}{2}}  h_\infty^\sigma(2k + \phi)  \nonumber\\
  &\qquad + \gamma \eta \int_{-\delta}^\delta \frac{d\phi}{2\pi} S_{jj}^*(k+\phi) e^{-2 i (\phi + k)\mu(t)}e^{-\frac{\sigma^2\phi^2}{2}}  \big(h_L^\sigma(2k+\phi) - h_\infty^\sigma(2k + \phi) \big).
\end{align}
If $t > \frac{\mu + L}{2|\sin k|}$, we instead have 
\begin{align}
  \braket{w_j(t)}{\alpha_j(t)} &= \eta e^{2 i t \cos k} \int_{-\delta}^\delta \frac{d\phi}{2\pi} \braket{\scat_j(k+\phi)}{\alpha_j(t)} e^{-\frac{\sigma^2\phi^2}{2}} e^{ - i \phi \mu(t)}\\
    &= \gamma\eta \int_{-\delta}^\delta \frac{d\phi}{2\pi} e^{2 i (k+\phi)\mu(t)} S_{jj}(k) e^{-\frac{\sigma^2\phi^2}{2}} h_{L}^\sigma(2k+\phi)\nonumber\\
    &\qquad +\gamma\eta \sum_{q=1}^N \int_{-\delta}^\delta \frac{d\phi}{2\pi}  S_{qj}^*(k+\phi) S_{qj}(k)e^{-\frac{\sigma^2\phi^2}{2}}  h_L^\sigma(\phi)\\
    &= \gamma \eta \int_{-\delta}^\delta \frac{d\phi}{2\pi} e^{-\frac{\sigma^2\phi^2}{2}} h_\infty^\sigma(\phi)
      + \gamma \eta\sum_{q=1}^N \int_{-\delta}^\delta \frac{d\phi}{2\pi} \big(S_{qj}^*(k+\phi )- S_{qj}^*(k)\big) S_{qj}(k)e^{-\frac{\sigma^2\phi^2}{2}} h_{\infty}^\sigma(\phi)\nonumber\\
    &\qquad  + \gamma \eta\sum_{q=1}^N \int_{-\delta}^{\delta} \frac{d\phi}{2\pi} S_{qj}^*(k+\phi)S_{qj}(k) e^{-\frac{\sigma^2\phi^2}{2}} \big(h_L^\sigma(\phi) - h_\infty^\sigma(\phi)\big) \nonumber\\
    &\qquad + \gamma \eta \int_{-\delta}^\delta \frac{d\phi}{2\pi} e^{2 i (k+\phi)\mu(t)} S_{jj}(k) e^{-\frac{\sigma^2\phi^2}{2}}  h_\infty^\sigma(2k+\phi)\nonumber \\
    &\qquad + \gamma \eta \int_{-\delta}^\delta \frac{d\phi}{2\pi} e^{2 i (k+\phi)\mu(t)} S_{jj}(k)e^{-\frac{\sigma^2\phi^2}{2}}  \big(h_{L}^\sigma(2k+\phi) - h_\infty^\sigma(2k+\phi )\big).
\end{align}
In both cases, most of the norm comes from the first term, and the rest will be small error terms.  In particular, we see that
\begin{align}
   \int_{-\delta}^\delta \frac{d\phi}{2\pi} e^{-\frac{\sigma^2\phi^2}{2}} h_\infty^\sigma(\phi) &= \frac{\sigma}{\sqrt{2\pi}} \int_{-\delta}^\delta d\phi e^{-\sigma^2 \phi^2} h_{\infty}^{1/(2\pi\sigma)}(2\pi i \sigma^2 \phi)\\
   &\leq \frac{2\sigma}{\sqrt{2\pi}} \int_0^\delta d\phi e^{ - \sigma^2 \phi^2} \Big[ 1 + 2 \Big( 1 + \frac{1}{2\pi \sigma^2} \frac{1}{ 2\pi - \phi}\Big) e^{ - 2\pi \sigma^2 (\pi - \phi)}\Big]\\
   & \leq \frac{2\sigma}{\sqrt{2\pi}} \Big[ 1 + 2 \Big( 1 + \frac{1}{2\pi \sigma^2} \frac{1}{ 2\pi - \delta }\Big) e^{ - 2\pi \sigma^2 (\pi - \delta)}\Big] \int_{0}^{\delta} d\phi e^{ - \sigma^2 \phi^2} \\
   &\leq \frac{1}{\sqrt{2}} \Big[ 1 + 3 e^{ - \pi^2 \sigma^2}\Big]\label{eq:gaussian_approx_gh}
\end{align}
where we assumed that $\delta < \frac{\pi}{2}$. If we also note that $h_L^\sigma(i\phi) \geq 1$ for all $L$ and all real $\phi$, we have 
\begin{align}
  \int_{-\delta}^\delta \frac{d\phi}{2\pi} e^{-\frac{\sigma^2\phi^2}{2}} h_\infty^\sigma(\phi) &= \frac{\sigma}{\sqrt{2\pi}} \int_{-\delta}^\delta d\phi e^{-\sigma^2 \phi^2} h_{\infty}^{1/(2\pi\sigma)}(2\pi i \sigma^2 \phi)\\
  &\geq \frac{\sigma}{\sqrt{2\pi}} \int_{-\delta}^\delta d\phi e^{-\sigma^2\phi^2} \\
  &= \sigma \sqrt{2\pi} \eta^{-2} \braket{w_j(t)}{w_j(t)}\\
  & \geq \frac{1}{\sqrt{2}}\Big( 1 - \frac{1}{\delta \sigma \sqrt{\pi}} e^{-\sigma^2\delta^2}\Big).
\end{align}

Now let us assume that $t < \frac{\mu - L}{2|\sin k |}$ and examine the first case.  We can use the bound of \eq{hL_bound} to see that
\begin{align}
  \Big| \int_{-\delta}^\delta \frac{d\phi}{2\pi} e^{-\frac{\sigma^2\phi^2}{2}} \big(h_{L}^\sigma(\phi) - h_{\infty}^\sigma(\phi)\big) \Big| &\leq \int_{-\delta}^\delta \frac{d\phi}{2\pi} e^{-\frac{\sigma^2\phi^2}{2}} \big| h_L^\sigma(\phi) - h_\infty^\sigma(\phi)\big|\\
   &\leq \int_{-\delta}^\delta \frac{d\phi}{2\pi}  \frac{2\sigma^2}{L} e^{- \frac {L^2}{2\sigma^2}}e^{-\frac{\sigma^2\phi^2}{2}}\\
   & \leq \frac{\sigma^2}{\pi L} e^{ - \frac{L^2}{2\sigma^2}} \int_{-\infty}^\infty d\phi e^{ - \frac{\sigma^2 \phi^2}{2}}\\
   &= \sqrt{\frac{2}{\pi}} \frac{\sigma}{L} e^{-\frac{L^2}{2\sigma^2}}\label{eq:gaussian_approx_L_to_infinity},
\end{align}
For the third term, we can continue to use the relations for the $h$ functions and note that
\begin{align}
 &\Big| \int_{-\delta}^\delta \frac{d\phi}{2\pi} S_{jj}^*(k+\phi) e^{- 2 i (\phi + k)\mu(t)} h_\infty^\sigma(2 k+\phi) e^{-\frac{\sigma^2\phi^2}{2}} \Big| \nonumber\\
  & \qquad \leq \int_{-\delta}^\delta \frac{d\phi}{2\pi} h_{\infty}^\sigma(2k+\phi) e^{-\frac{\sigma^2\phi^2}{2}} \\
  & \qquad = \int_{-\delta}^\delta  \frac{d\phi}{\sqrt{2\pi}} \sigma e^{ -\frac{\sigma^2}{2} \big( (2k+ \phi)^2 + \phi^2\big)} h_{\infty}^{1/(2\pi\sigma)}(2\pi i (2k+\phi) \sigma^2) \\
  &\qquad \leq \frac{\sigma}{\sqrt{2\pi}} \int_{-\delta}^{\delta} d\phi e^{ -\frac{\sigma^2}{2} \big( (2k+ \phi)^2 + \phi^2\big)}\Big( 1 + 2 \Big[ 1 + \frac{1}{2\pi \sigma^2} \frac{1}{ 2\pi - |2k + \phi|}\Big] e^{- 2\pi \sigma^2 (\pi - |2k + \phi|)}\Big)\label{eq:gaussian_k_offset_midstep}
\end{align}
where we used equation \eq{h_bound} in the third step.  If $2|k|+ \delta < \pi$, then $h_{\infty}^{1/(2\pi\sigma)} (2k+\phi)$ can be bounded by $2$ for $\sigma > (\pi - 2|k| - \delta)^{-1}$.  We then have
\begin{align}
 \Big| \int_{-\delta}^\delta \frac{d\phi}{2\pi} S_{jj}^*(k+\phi) e^{- 2 i (\phi + k)\mu(t)} h_\infty^\sigma(2 k+\phi) e^{-\frac{\sigma^2\phi^2}{2}}\Big|&\leq \sqrt{\frac{2}{\pi}} \sigma \int_{-\delta}^\delta d\phi e^{- \frac{\sigma^2}{2} \big( (2k + \phi)^2 + \phi^2\big)}\\
  & = \sqrt{\frac{2}{\pi}} \sigma \int_{-\delta}^{\delta} d\phi e^{ -\sigma^2 k^2 - \sigma^2 (k + \phi)^2}\\
  & \leq \sqrt{\frac{2}{\pi}} \sigma e^{-\sigma^2 k^2} \int_{-\infty}^\infty d\phi e^{ -\sigma^2 \phi^2}\\
  & = \sqrt{2} e^{-\sigma^2 k^2}.
\end{align}
However, if we instead have that  $2|k| + \delta > \pi$, then we can instead bound equation \eq{gaussian_k_offset_midstep}  as
\begin{align}
&\Big| \int_{-\delta}^\delta \frac{d\phi}{2\pi} S_{jj}^*(k+\phi) e^{- 2 i (\phi + k)\mu(t)} h_\infty^\sigma(2 k+\phi)e^{-\frac{\sigma^2\phi^2}{2}}\Big| \nonumber\\
  & \qquad \leq \sqrt{\frac{2}{\pi}} \sigma \int_{-\delta}^{\delta}d\phi e^{ -\frac{\sigma^2}{2} \big( (2k+ \phi)^2 + \phi^2\big)}\Bigg( 1 + 2 \Bigg[ 1 + \frac{1}{2\pi \sigma^2} \frac{1}{ 2\pi - |2k + \phi|}\Bigg] e^{- 2\pi \sigma^2 (\pi - |2k + \phi|)}\Bigg)\\
  &\qquad \leq 4 \sqrt{\frac{2}{\pi}}\sigma \int_{-\delta}^{\delta} d\phi e^{ -\frac{\sigma^2}{2} \big( (2k+ \phi)^2 + \phi^2\big)} e^{2 \pi \sigma^2 (2 |k| + \delta - \pi)}\\
  &\qquad =4 \sqrt{\frac{2}{\pi}}\sigma \int_{-\delta}^{\delta} d\phi  e^{  2\sigma^2 \big(-k^2 +2\pi |k| -\pi^2 + \pi \delta - k\phi - \frac{\phi^2}{2} \big)}\\
  &\qquad =4 \sqrt{\frac{2}{\pi}}\sigma e^{2\sigma^2[-(\pi - |k|)^2 + \pi \delta]}\int_{-\delta}^{\delta} d\phi e^{  -\sigma^2 (\phi^2 + 2k\phi )}\\
  &\qquad \leq 4 \sqrt{\frac{2}{\pi}}\sigma e^{2\sigma^2[-(\pi - |k|)^2 + (\pi + |k|) \delta]} \int_{-\delta}^{\delta} d\phi e^{- \sigma^2 \phi^2}\\
  & \qquad \leq 4\sqrt{2} e^{ -\sigma^2 (\pi - |k|)^2}
  \end{align}
where we assumed that $\delta < \frac{(\pi-|k|)^2}{\pi + |k|}$.  In either case, we have that 
\begin{align}
\Big| \int_{-\delta}^\delta \frac{d\phi}{2\pi} S_{jj}^*(k+\phi) e^{- 2 i (\phi + k)\mu(t)} h_\infty^\sigma(2 k+\phi) e^{-\frac{\sigma^2\phi^2}{2}}\Big|
  & \leq 4 \sqrt{2} e^{ - \sigma^2 \delta}\label{eq:gaussian_cross_term_bound}.
\end{align}
Finally, we can again use the same argument as in equation  \eq{gaussian_approx_L_to_infinity}  to see that in to bound the final term as
\begin{align}
  & \Big| \int_{-\delta}^\delta \frac{d\phi}{2\pi} S_{jj}^*(k+\phi) e^{2 i (\phi + k)\mu(t)} \big( h_L^\sigma (2k + \phi) - h_{\infty}^\sigma(2k+\phi)\big) e^{-\frac{\sigma^2\phi^2}{2}}\Big| \nonumber\\
  & \qquad \leq  \int_{-\delta}^\delta \frac{d\phi}{2\pi} \big|h_L^\sigma(2k + \phi) - h_\infty^\sigma(2k+\phi)\big| e^{-\frac{\sigma^2\phi^2}{2}}\\
  & \qquad \leq \sqrt{\frac{2}{\pi}} \frac{\sigma}{L} e^{-\frac{L^2}{2\sigma^2}}
\end{align}
where we used \eq{gaussian_approx_L_to_infinity} in the last line.  This is sufficient for us to bound the norm of $\braket{w_j(t)}{\alpha_j(t)}$ for small $t$.

With all of these bounds, we then have that if $t < \frac{\mu - L}{2|\sin k|}$, since most of the amplitude of $\braket{\alpha_j(t)}{w_j(t)}$ is on the term equal to $\braket{w_j(t)}{w_j(t)}$, 
\begin{align}
  \norm{\ket{\alpha_j(t)} - \ket{w_j(t)}}^2 &= \braket{\alpha_j(t)} {\alpha_j(t)} + \braket{w_j(t)}{w_j(t)} - 2 \Re\big[ \braket{\alpha_j(t)}{w_j(t)} \big]\\
  & \leq 1 + \braket{w_j(t)}{w_j(t)}\nonumber\\
  &\qquad  - 2\gamma\eta  \Bigg[ \frac{1}{\sqrt{2}} \braket{w_j(t)}{w_j(t)} - \sqrt{\frac{2}{\pi}} \frac{\sigma}{L} e^{-\frac{L^2}{2\sigma^2}} - 4 \sqrt{2} e^{-\sigma^2 \delta} - \sqrt{\frac{2}{\pi}} \frac{\sigma}{L} e^{-\frac{L^2}{2\sigma^2}}  \Bigg] \\
  & = 1 + \big( 1 - \sqrt{2} \gamma \eta\big) \braket{w_j(t)}{w_j(t)} + 4 \sqrt{2} \gamma \eta \Big[\frac{ \sigma}{\sqrt{\pi} L} e^{ - \frac{L^2}{2\sigma^2}} + 2 e^{ - \sigma^2 \delta} \Big] 
\end{align}
We can then use our bounds on $\gamma$ from equations \eq{gaussian_gamma_upper_bound} and \eq{gaussian_gamma_lower_bound}, along with the bounds on $\braket{w_j(t)}{w_j(t)}$ to see that
\begin{align}
  &\norm{\ket{\alpha_j(t)} - \ket{w_j(t)}}^2 \nonumber\\
  &\qquad \leq 1 + \Big[ 1 - \sqrt{2} \sqrt{2\sqrt{\pi} \sigma} \frac{1}{\sqrt{ \sqrt{\pi} \sigma}} \Big( 1 + 3 e^{ - \pi^2\sigma^2} \Big)^{-1/2} \Big] \Big( 1 - \frac{1}{\delta \sigma \sqrt{\pi}} e^{-\sigma^2\delta^2}\Big)\nonumber\\
  &\qquad \qquad + 4 \sqrt{2} \sqrt{2\sqrt{\pi} \sigma} \frac{1}{\sqrt{ \sqrt{\pi} \sigma}} \Big[ 1 - e^{-\frac{L^2}{2\sigma^2}}\Big]^{-1/2}  \Big[\frac{ \sigma}{\sqrt{\pi} L} e^{ - \frac{L^2}{2\sigma^2}} + 2 e^{ - \sigma^2 \delta} \Big] \\
  & \qquad \leq 1 + \Big[ 1 - 2 + 3 e^{-\pi^2 \sigma^2}\Big] \Big( 1 - \frac{1}{\delta \sigma \sqrt{\pi}} e^{-\sigma^2\delta^2}\Big)
    + 8 \Big[1 + e^{ - \frac{L^2}{\sigma^2}}\Big]  \Big[\frac{ \sigma}{\sqrt{\pi} L} e^{ - \frac{L^2}{2\sigma^2}} + 2 e^{ - \sigma^2 \delta} \Big] \\
  &\qquad\leq  3 e^{ - \pi^2 \sigma^2}  + \frac{1}{\delta \sigma \sqrt{\pi}} e^{-\sigma^2\delta^2} + \frac{ 16\sigma}{\sqrt{\pi} L} e^{ - \frac{L^2}{2\sigma^2}} + 32 e^{ - \sigma^2 \delta}\\
  &\qquad \leq 36 e^{ - \sigma^2 \delta}  + \frac{16 \sigma}{\sqrt{\pi}L} e^{ - \frac{L^2}{2\sigma^2}}
\end{align}
where we assumed that $\sigma > (\delta \sqrt{\pi})^{-1}$.


If we now examine the cases for which $t > \frac{\mu + L}{2|\sin k|}$, we have some slightly different errors to bound, but most will utilize the same tricks.  The one that needs different ideas, however, will utilize the fact that $S$ is a matrix of bounded rational functions.  In particular, the Lipschitz constant
\begin{align}
  \Gamma &= \max_{q,j\in [N]} \max_{p\in [-\pi,\pi]} \Bigg| \frac{d}{dk'} S_{qj}(k') \Big|_{k' = p} \Bigg|
\end{align}
is well defined.  Using this, we then have that
\begin{align}
  &\Big|\int_{-\delta}^\delta \frac{d\phi}{2\pi} \big(S_{qj}^*(k+\phi )- S_{qj}^*(k)\big) S_{qj}(k) e^{-\frac{\sigma^2\phi^2}{2}}h_{\infty}^\sigma(\phi)\Big| \nonumber\\
   & \qquad \leq \int_{-\delta}^\delta \frac{d\phi}{2\pi} \big|S_{qj}^*(k+\phi )- S_{qj}^*(k)\big|e^{-\frac{\sigma^2\phi^2}{2}}h_{\infty}^\sigma(\phi)\\
   & \qquad\leq \frac{\kappa}{\pi} \int_{0}^\delta d\phi \phi e^{-\frac{\sigma^2\phi^2}{2}}h_\infty^\sigma(\phi)\\
   & \qquad = \kappa\sigma \sqrt{\frac{2}{\pi}}\int_0^\delta d\phi \phi e^{-\sigma^2 \phi^2} h_{\infty}^{1/(2\pi\sigma)} (2\pi i \sigma^2 \phi)\\
   & \qquad \leq \kappa\sigma \sqrt{\frac{2}{\pi}} \big( 1 + 3 e^{-\pi^2\sigma^2}\big) \int_{0}^\delta d\phi \phi e^{ -\sigma^2\phi^2}\\
   & \qquad = \frac{\kappa}{\sigma \sqrt{2\pi}} \big(1 + 3 e^{-\pi^2 \sigma^2}\big) \big( 1 - e^{-\sigma^2\delta^2}\big)
\end{align}
where we used our previous bounds on $h_\infty^\sigma(\phi)$.  We also have
\begin{align}
  \Big|\int_{-\delta}^\delta \frac{d\phi}{2\pi} S_{qj}^*(k+\phi) S_{qj}(k) \big(h_{L}^\sigma (\phi) - h_\infty^\sigma(\phi)\big) e^{-\frac{\sigma^2\phi^2}{2}}\Big|
    & \leq \int_{-\delta}^\delta \frac{d\phi}{2\pi}  \big|h_{L}^\sigma (\phi) - h_\infty^\sigma(\phi)\big| e^{-\frac{\sigma^2\phi^2}{2}}\\
    & \leq \sqrt{\frac{2}{\pi}} \frac{\sigma}{L} e^{-\frac{L^2}{2\sigma^2}}
\end{align}
where we use equation \eq{gaussian_approx_L_to_infinity} in the last line.  Continuing to extend our previous results, we can utilize equation \eq{gaussian_cross_term_bound} to see that
\begin{align}
  \Bigg| \int_{-\delta}^\delta \frac{d\phi}{2\pi} e^{2 i (k+\phi) \mu(t)} S_{jj}(k) e^{-\frac{\sigma^2\phi^2}{2}} h_{\infty}^\sigma(2k+\phi)\Bigg| 
    & \leq \int_{-\delta}^\delta \frac{d\phi}{2\pi}  e^{-\frac{\sigma^2\phi^2}{2}} h_{\infty}^\sigma(2k+\phi) \leq 4 \sqrt{2} e^{ - \sigma^2 \delta}.
\end{align}
Finally, we can again use equation \eq{gaussian_approx_L_to_infinity} to show
\begin{align}
  &\Bigg| \int_{-\delta}^\delta \frac{d\phi}{2\pi} e^{2 i (k+\phi) \mu(t)} S_{jj}(k) \big( h_{L}^\sigma(2k + \phi) - h_{\infty}^\sigma(2k + \phi) \big)e^{-\frac{\sigma^2\phi^2}{2}}\Bigg|  \nonumber\\
  &\qquad\leq \int_{-\delta}^\delta \frac{d\phi}{2\pi} \big| h_{L}^\sigma(2k + \phi) - h_{\infty}^\sigma(2k + \phi) \big|e^{-\frac{\sigma^2\phi^2}{2}}\\
  & \qquad \leq \sqrt{\frac{2}{\pi}} \frac{\sigma}{L} e^{-\frac{L^2}{2\sigma^2}}.
\end{align}

We now have the ability to bound the error arising from approximating $\ket{\alpha_j(t)}$ for $t > \frac{\mu + L}{2|\sin k|}$.  In particular, we have
\begin{align}
    \norm{\ket{\alpha_j(t)} - \ket{w_j(t)}}^2 &= \braket{\alpha_j(t)} {\alpha_j(t)} + \braket{w_j(t)}{w_j(t)} - 2 \Re\big[ \braket{\alpha_j(t)}{w_j(t)} \big]\\
  & \leq 1 + \braket{w_j(t)}{w_j(t)} - 2\gamma \eta \Bigg[  \frac{1}{\sqrt{2}} \braket{w_j(t)}{w_j(t)} \nonumber\\
  & \qquad  - \sum_{q=1}^N  \frac{\kappa}{\sigma \sqrt{2\pi}}\big(1 + 3 e^{-\pi^2 \sigma^2}\big)\big(1 - e^{-\sigma^2 \delta^2}\big) - \sum_{q=1}^N \sqrt{\frac{2}{\pi}} \frac{\sigma}{L} e^{-\frac{L^2}{2\sigma^2}}\nonumber\\
  &\qquad  - 4 \sqrt{2} e^{-\sigma^2 \delta} - \sqrt{\frac{2}{\pi}} \frac{\sigma}{L} e^{-\frac{L^2}{2\sigma^2}}\Bigg]\\
  &\leq 1 + \big(1 - \sqrt{2}\gamma \eta \big) \braket{w_j(t)}{w_j(t)} + 2\gamma\eta \Bigg[ \frac{N \kappa}{\sigma \sqrt{2\pi}} \big(1 + 3 e^{-\pi^2 \sigma^2}\big) \nonumber\\
  &\qquad + \sqrt{\frac{2}{\pi}} \frac{(N+1)\sigma}{L} e^{ - \frac{L^2}{2\sigma^2}} + 4 \sqrt{2} e^{-\sigma^2\delta} \Bigg].
\end{align}
If we use the same bounds as for $t < \frac{\mu -L}{2|\sin k|}$ we find 
\begin{align}
  & \norm{\ket{\alpha_j(t)} - \ket{w_j(t)}}^2 \\
  & \qquad \leq 1 + \big(1 - 2 + 3 e^{-\pi^2 \sigma^2}\big) \Big(1 - \frac{1}{\delta \sigma \sqrt{\pi}} e^{-\sigma^2 \delta^2} \Big) \nonumber\\
  &\qquad \qquad \qquad + 2\sqrt{2} \Big( 1 + e^{ - \frac{L^2}{\sigma^2}} \Big)\Bigg( \frac{N \kappa}{\sigma}  + \sqrt{\frac{2}{\pi}} \frac{(N+1)\sigma}{L} e^{-\frac{L^2}{2\sigma^2} }+ 4 \sqrt{2} e^{-\sigma^2 \delta}  \Bigg)\\
  & \qquad \leq 3 e^{-\pi^2 \sigma^2} + \frac{1}{\delta \sigma \sqrt{\pi}} e^{ - \sigma^2 \delta^2} + 4 \sqrt{2} \Bigg( \frac{N \kappa}{\sigma}  + \sqrt{\frac{2}{\pi}} \frac{(N+1)\sigma}{L} e^{-\frac{L^2}{2\sigma^2} }+ 4 \sqrt{2} e^{-\sigma^2 \delta}\Bigg) \\
  & \qquad \leq 36  e^{-\sigma^2 \delta} + \frac{32N\kappa}{\sigma} + \frac{8 (N+1)}{\sqrt{\pi}} \frac{\sigma}{L} e^{-\frac{L^2}{2\sigma^2}}
\end{align}


At this point, we can now bound the error that arises when approximating $\ket{\alpha_j(t)}$ by a Gaussian for any $t < \frac{\mu -L}{2 |\sin(k)|}$ and for any $t > \frac{\mu +L}{2|\sin(k)|}$.  However, we don't yet know how well $\ket{\alpha_j(t)}$ approximates $\ket{\psi_j(t)}$.  Note, however, that $\ket{\alpha_j(0)} = \ket{\psi_j(0)}$, and thus we already have an approximation to the initial state.  We can then look easily time-evolve this approximation, and compare it to the Gaussian approximation for $\ket{\alpha_j(t)}$. 

Let us define
\begin{align}
  \ket{v_j(0)} = \ket{w_j(0)} = \eta \int_{-\delta}^\delta \frac{d\phi}{2\pi} e^{ i \phi \mu} e^{ - \frac{\sigma^2 \phi^2}{2}}\ket{\scat_j(k+\phi)}
\end{align}
and then define 
\begin{align}
  \ket{v_j(t)} = e^{-i H t} \ket{v_j(0)} = \eta \int_{-\delta}^\delta \frac{d\phi}{2\pi} e^{ i \phi \mu - 2 i t \cos(k+\phi)} e^{ - \frac{\sigma^2 \phi^2}{2}} \ket{\scat_j(k+\phi)}.
\end{align}
We then want to compare this time evolved state with our approximation to $\ket{\alpha_j(t)}$.  We can see 
\begin{align}
  \braket{v_j(t)}{w_j(t)} &= \eta^2 \int_{-\delta}^\delta \frac{d\phi}{2\pi} e^{ 2i t \cos(k+\phi) - 2 i t \cos(k)} e^{ i \phi \mu - i \phi \mu - i \phi \lceil 2t \sin k\rceil} e^{-\sigma^2 \phi^2}\\
  &= \eta^2 \int_{-\delta}^\delta \frac{d\phi}{2\pi} e^{-\sigma^2\phi^2} - \eta^2 \int_{-\delta}^\delta \frac{d\phi}{2\pi}\big( 1 - e^{ 2i t \cos(k+\phi) - 2 i t \cos(k) +i \phi \lceil 2t \sin k\rceil} \big)e^{-\sigma^2\phi^2}.
\end{align}
The first term is simply the norm of both $\ket{v_j(t)}$ and $\ket{w_j(t)}$, while the third term can be bounded as
\begin{align}
 & \Bigg| \int_{-\delta}^\delta \frac{d\phi}{2\pi}\big( 1 - e^{ 2i t \cos(k+\phi) - 2 i t \cos(k) + i \phi \lceil 2t \sin k\rceil} \big)e^{-\sigma^2\phi^2}\Bigg| \\
 &\qquad\leq \int_{-\delta}^\delta \frac{d\phi}{2\pi}\big| 1 - e^{ 2i t \cos(k+\phi) - 2 i t \cos(k) + i \phi \lceil 2t \sin k\rceil} \big|e^{-\sigma^2\phi^2}\\
  &\qquad\leq \int_{-\delta}^\delta \frac{d\phi}{2\pi}\big|2 t \cos(k+\phi) - 2 t \cos(k) + \phi \lceil 2t \sin k\rceil \big|e^{-\sigma^2\phi^2}\\
  &\qquad\leq \int_{-\delta}^\delta \frac{d\phi}{2\pi} \Big(2t \big|\cos(k) \cos(\phi) - \sin(k)\sin(\phi) - \cos(k) + \phi\sin(k)\big|+|\phi|\Big)e^{-\sigma^2\phi^2}\\
  &\qquad\leq \int_{-\delta}^\delta \frac{d\phi}{2\pi} \Big( t \big| \cos(k) \phi^2 + \sin(k) |\phi|^3\big| + |\phi|\Big) e^{-\sigma^2\phi^2}\\
  &\qquad \leq 2 \int_{0}^\delta \frac{d \phi}{2\pi} \big( 2 t \phi^2 + \phi\big) e^{-\sigma^2\phi^2}\\
  & \qquad \leq \frac{1}{2\pi\sigma^2} + \frac{t}{2\sqrt{\pi} \sigma^3}.
\end{align}
Noting that the norm of $\braket{v_j(t)}{v_j(t)}$ doesn't change with time, we have that
\begin{align}
  \norm{\ket{v_j(t)} - \ket{w_j(t)}}^2 &= \braket{v_j(t)}{v_j(t)} + \braket{w_j(t)}{w_j(t)} - \braket{v_j(t)}{w_j(t)} - \braket{w_j(t)}{v_j(t)}\\
  &\leq 2 \eta^2 \int_{\delta}^\delta \frac{d\phi}{2\pi} e^{-\sigma^2 \phi^2} - 2\eta^2 \int_{\delta}^\delta \frac{d\phi}{2\pi }e^{ - \sigma^2 \phi^2} + \frac{2\eta^2}{2\pi \sigma^2} + \frac{2 \eta^2 t}{2\sqrt{\pi} \sigma^3}\\
  &= 2 \frac{\sqrt{\pi}}{ \sigma} + \frac{2 t }{\sigma^2}.
\end{align}

Finally, we can combine these three bounds, noting that $\ket{\psi_j(0)} = \ket{\alpha_j(0)}$, and that $\norm{\ket{w_j(t)} - \ket{\alpha_j(t)}} $ is larger for $ t > \frac{\mu+L}{2|\sin k|}$ than for $t < \frac{\mu- L}{2|\sin k|}$.  In particular, we have for all $t \leq \frac{\mu -L}{2 |\sin k|}$ and for all $t > \frac{\mu + L}{2|\sin k|}$ that
\begin{align}
  &\norm{ \ket{\psi_j(t)} - \ket{\alpha_j(t)}}\nonumber\\
   &\qquad \leq \norm{ \ket{\psi_j}{(t)} - \ket{v_j(t)}} + \norm{\ket{ v_j(t)} - \ket{w_j(t)}} + \norm{\ket{w_j(t)} - \ket{\alpha_j(t)}}\\
  & \qquad \leq \Bigg[ 36 e^{-\sigma^2 \delta} + \frac{16 \sigma}{\sqrt{\pi} L} e^{-\frac{L^2}{\sigma^2}} \Bigg]^{1/2} + \Bigg[ \frac{2 \sqrt{\pi}}{\sigma} + \frac{2t}{\sigma^2}\Bigg]^{1/2} \nonumber\\
  &\qquad \qquad \qquad + \Bigg[36 e^{ -\sigma^2 \delta} + \frac{32 N \kappa}{\sigma} + \frac{8 (N+1)}{\sqrt{\pi}} \frac{\sigma}{L} e^{-\frac{L^2}{2\sigma^2}} \Bigg] ^{1/2}.
\end{align}
If we then assume that $\sigma = \frac{ c_1 L}{\sqrt{\log L}}$ for some constant $c_1$, and that $t < c_2 L$ for some constant $c_2$, we have that for $L$ large enough
\begin{align}
  &\norm{ \ket{\psi_j(t)} - \ket{\alpha_j(t)}}\nonumber\\
   & \qquad \leq \Big( \frac{ c_3}{\sqrt{\log{L}}} \frac{1}{L} \Big)^{1/2} + \Big(\frac{c_4\sqrt{ \log L}}{L} + \frac{ c_2 \log L}{L}   \Big)^{1/2} + \Big( \frac{ c_5\sqrt{\log L}}{L} + \frac{c_6}{L\sqrt{\log L}}\Big)^{1/2}\\
   &\qquad \leq \chi  \sqrt{\frac{\log L}{L}}
\end{align}
for some constant $\chi$.

\end{proof}

Now that we have a bound for the single particle truncated scattering, it will be useful to also have a truncated two-particle scattering on an infinite path.

In particular, let $k\in(-\pi,\pi)$, and let $\mu \in \NN$.  We will define a Gaussian wavepacket centered at $\mu$, with momentum $k$, standard deviation $\sigma$, and cutoff $L$ as the state
\begin{align}
  \ket{\chi_{\mu,k}} = \gamma \sum_{x=\mu-L}^{\mu+L} e^{i k x} e^{ -\frac{(x-\mu)^2}{2\sigma^2}} \ket{x},
\end{align}
where $\gamma^2$ is a normalization factor given by $\gamma^{-2} = h_L^{\sigma/\sqrt{2}}(0)$.  Note that this is nearly the same state as for single-particle scattering, but now we don't have to deal with the multiple semi-infinite paths and the graph $\widehat{G}$.
%%%%%%%%%%%%%%%%%%%%%%%%%%%%%%%%%%%
\begin{theorem}
Let $H^{(2)}$ be a two-particle Hamiltonian of the form \eq{something} with interaction range at most $C$.  Let $\theta_{\pm}(p_1,p_2)$ be given by equation \eq{something_else}.  Let $k_1\in (-\pi,0)$ and let $k_2 \in (0,\pi)$, let $L,\mu, \nu \in \NN$ with $L>0$ and $\mu < \nu - 2 L$, and let $\sigma > 0$.  Let us then define the states
\begin{align}
  \ket{\psi(0)}_{\pm} = \frac{1}{\sqrt{2}} \big( \ket{\chi_{\mu, k_1}} \ket{\chi_{\nu, k_2}} \pm  \ket{\chi_{\nu,k_2}}\ket{\chi_{\mu,k_1}}\big),
\end{align}
and 
\begin{align}
  &\ket{\alpha (t)}_{\pm} \nonumber\\
  &\qquad= \frac{e^{-2 i t (\cos(k_1) + \cos(k_2))}}{\sqrt{2}} 
    e^{i \theta_{\pm}(t)} \big( \ket{\chi_{\mu(t),k_1}}\ket{\chi_{\nu(t),k_2}}  \pm \ket{\chi_{\nu(t),k_2}}\ket{\chi_{\mu(t),k_1}}\big) ,
\end{align}
where
\begin{align}
  \mu(t) = \mu - \lceil 2 t \sin k_1\rceil, \quad
  \nu(t) = \nu - \lceil 2 t \sin k_2\rceil, \quad \text{and} \quad 
  \theta_{\pm}(t)=\begin{cases}0 & t < \frac{ \nu - \mu - 2 L}{2 \sin k_2 - 2\sin k_1}\\
  \theta_{\pm}(k_1,k_2)& t > \frac{\nu - \mu + 2 L}{2 \sin k_2 - 2 \sin k_1}.\end{cases}
\end{align}
If $\sigma = \frac{ L}{2\sqrt{\log L}}$, and if $0 \leq t < \frac{ \nu - \mu - 2 L}{2 \sin k_2 - 2\sin k_1}$ or if $\frac{\nu - \mu + 2L}{2 \sin k_2 - 2\sin k_1}< t < c L$ for some constant $c$, then 
\begin{align}
  \norm{e^{ - i H^2 t} \ket{\psi(0)}_{\pm} - \ket{\alpha(t)}}_{\pm} \leq \chi_2 \sqrt{\frac{\log L}{L}}
\end{align}
for some constant $\chi_2$.
\end{theorem}
The proof of this theorem is going to follow the exact same method as for the single-particle finite wavepacket proof.

\begin{proof}
The main idea behind this proof will be to show that $\ket{\psi(0)}$ and $\ket{\alpha (t)}$ are both well approximated by a Gaussian distribution over eigenstates of $H^{(2)}$ with momenta near $k_1$ and $k_2$, and then show that time evolving the Gaussian approximation for $\ket{\psi(0)}$ is well approximated by the Gaussian approximation for $\ket{\alpha(t)}$.  
  
In particular, let us examine the inner product between an eigenstate $\ket{\scat (p_1;p_2)}_{\pm}$ and $\ket{\alpha(t)}_{\pm}$.  Note that $\ket{\scat(p_1;p_2)}_{\pm}$ has no overlap with $\ket{\alpha(t)}_{\mp}$, and 
\begin{align}
  &{}_{\pm}\braket{\scat(p_1;p_2)}{\alpha(t)}_\pm \nonumber\\
  & \qquad= \frac{e^{-2 i  t (\cos(k_1) + \cos(k_2)) + i\theta_{\pm}(t)}}{\sqrt{2}} \big({}_{\pm}\braket{\scat (p_1;p_2)}{\chi_{\mu(t),k_1},\chi_{\nu(t),k_2}}  \pm {}_{\pm}\braket{\scat(p_1;p_2)}{\chi_{\nu(t),k_2},\chi_{\mu(t),k_1}}\big).
\end{align}
We will then want to investigate the overlap between $\ket{\scat(p_1;p_2)}$ and the cut-off Gaussian approximations.  In particular, if we assume that $s - r > 2 L$, then 
\begin{align}
   &_{\pm}\braket{\scat(p_1;p_2)}{\chi_{r,k_1},\chi_{s,k_2}}\nonumber\\
    &\qquad= \gamma^2\sum_{x=r-L}^{r+L} \sum_{y=s-L}^{L} e^{ik_1x}e^{-\frac{(x-r)^2}{2\sigma^2}} e^{i k_2 y}e^{-\frac{(y-s)^2}{2\sigma^2}} \braket{\scat(p_1;p_2)}{x,y}\\
   &\qquad= \gamma^2\sum_{x=r-L}^{r+L} \sum_{y=s-L}^{s+L} e^{i (k_1 x + k_2 y)} e^{-\frac{(x-r)^2 + (y-s)^2}{2\sigma^2}} \frac{e^{i p_1 \frac{x+y}{2}}}{\sqrt{2}} \Big( e^{  i p_2 | x - y|} \pm e^{-i\theta_{\pm}(p_1,p_2) - i p_2 |x-y|}\Big)\\
   &\qquad = \gamma^2\frac{e^{i (k_1 r + k_2 s + p_1 \frac{r+s}{2} + p_2 (s - r))}}{\sqrt{2}} \sum_{x=-L}^{L} \sum_{y=-L}^L e^{ -\frac{ x^2 + y^2}{2\sigma^2}}e^{i ( k_1 + \frac{p_1}{2} -p_2) x} e^{i (k_2 + \frac{p_1}{2} + p_2)y} \nonumber\\
   &\qquad \qquad + \gamma^2 \frac{e^{i (k_1 r + k_2 s + p_1 \frac{r+s}{2} + p_2 (r-s) - i \theta_\pm(p_1,p_2))}}{\sqrt{2}} \sum_{x=-L}^{L} \sum_{y=-L}^L e^{ -\frac{ x^2 + y^2}{2\sigma^2}}e^{i ( k_1 + \frac{p_1}{2} +p_2) x} e^{i (k_2 + \frac{p_1}{2} - p_2)y}\\
   &\qquad = \gamma^2\frac{e^{i (k_1 + \frac{p_1}{2} - p_2) r + i (k_2 + \frac{p_1}{2} + p_2)s}}{\sqrt{2}} h_L^\sigma\bigg( k_1 + \frac{p_1}{2} - p_2\bigg) h_{L}^\sigma \bigg(k_2 + \frac{p_1}{2} + p_2\bigg)\nonumber\\
   &\qquad \qquad +\gamma^2 \frac{e^{i (k_1 + \frac{p_1}{2} + p_2) r + i (k_2 + \frac{p_1}{2} - p_2)s - i\theta_{\pm}(p_1,p_2)}}{\sqrt{2}} h_L^\sigma\bigg( k_1 + \frac{p_1}{2} + p_2\bigg) h_{L}^\sigma \bigg(k_2 + \frac{p_1}{2} - p_2\bigg).
\end{align}
If $s - r < -2L$, then nearly the same argument holds, and we find
\begin{align}
   &_{\pm}\braket{\scat(p_1;p_2)}{\chi_{r,k_1},\chi_{s,k_2}}\nonumber\\
   &\qquad = \gamma^2 \frac{e^{i (k_1 + \frac{p_1}{2} + p_2) r + i (k_2 + \frac{p_1}{2} - p_2)s}}{\sqrt{2}} h_L^\sigma\bigg( k_1 + \frac{p_1}{2} + p_2\bigg) h_{L}^\sigma \bigg(k_2 + \frac{p_1}{2} - p_2\bigg)\nonumber\\
   &\qquad \qquad +\gamma^2\frac{e^{i (k_1 + \frac{p_1}{2} - p_2) r + i (k_2 + \frac{p_1}{2} + p_2)s - i\theta_{\pm}(p_1,p_2)}}{\sqrt{2}} h_L^\sigma\bigg( k_1 + \frac{p_1}{2} - p_2\bigg) h_{L}^\sigma \bigg(k_2 + \frac{p_1}{2} + p_2\bigg).
\end{align}
Putting these bounds together, if we define $p_1 = -(k_1 + k_2)$ and $2 p_2 = k_1 - k_2$, we find that for $t < \frac{\nu - \mu - 2L}{2 \sin k_2 - 2 \sin k_1}$
\begin{align}
  &{}_{\pm}\braket{\scat(p_1 + \phi_1 ; p_2 + \phi_2)}{\alpha(t)}_{\pm} \nonumber\\
  &\qquad    = \gamma^2e^{-2 i  t (\cos(k_1) + \cos(k_2))} e^{i \phi_1\frac{\mu(t) + \nu(t)}{2}} \Bigg[e^{i \phi_2 (\nu(t) - \mu(t)) } h_{L}^\sigma \bigg(\frac{\phi_1}{2} - \phi_2\bigg) h_L^\sigma\bigg( \frac{\phi_1}{2} + \phi_2\bigg)\nonumber\\
  &\qquad\qquad \pm e^{ i (2 p_2+\phi_2)(\mu(t) - \nu(t)) - i \theta_{\pm}(p_1 + \phi_1, p_2+\phi_2)}h_L^\sigma\bigg(\frac{\phi_1}{2} - 2p_2 - \phi_2 \bigg) h_L^\sigma\bigg(\frac{\phi_1}{2} + 2 p_2 + \phi_2\bigg)\Bigg].
\end{align}
In nearly the same manner, if $ t > \frac{\nu - \mu + 2 L}{2 \sin k_2 - 2 \sin k_1}$, we have
\begin{align}
  &{}_{\pm}\braket{\scat(p_1 + \phi_1 ; p_2 + \phi_2)}{\alpha(t)}_{\pm} \nonumber\\
  &\qquad    = \gamma^2e^{-2 i  t (\cos(k_1) + \cos(k_2)) + i \theta_{\pm}(p_1,p_2)} e^{i\phi_1\frac{\mu(t) + \nu(t)}{2}} \Bigg[e^{i \phi_2 (\nu(t) - \mu(t)) - i \theta_{\pm}(p_1 + \phi_1, p_2+\phi_2) } h_{L}^\sigma \bigg(\frac{\phi_1}{2} - \phi_2\bigg) h_L^\sigma\bigg( \frac{\phi_1}{2} + \phi_2\bigg)\nonumber\\
  &\qquad\qquad \pm e^{ i (2 p_2+\phi_2)(\mu(t) - \nu(t)) }h_L^\sigma\bigg(\frac{\phi_1}{2} - 2p_2 - \phi_2 \bigg) h_L^\sigma\bigg(\frac{\phi_1}{2} + 2 p_2 + \phi_2\bigg)\Bigg].
\end{align}

With these useful inner products, we can now define our Gaussian approximations.  In particular, we will define the states
\begin{align}
  \ket{w(t)} &= \eta e^{ - 2 i t (\cos k_1 + \cos k_2)} \int_{-\delta}^\delta \int_{-\delta}^\delta \frac{d\phi_1 d\phi_2}{4\pi^2} e^{i \phi_1 \big(\frac{\mu(t) + \nu(t)}{2}\big)} e^{i  \phi_2 (\nu(t) - \mu(t))} e^{ -\frac{ \sigma^2 \phi_1^2}{4}} e^{ -\sigma^2 \phi_2^2} \ket{\scat(p_1 + \phi_1; p_2 + \phi_2)}_{\pm}
\end{align} 
where
\begin{align}
  \eta^{-2} &= \int_{-\infty}^\infty \int_{-\infty}^\infty \frac{d\phi_1 d\phi_2}{4\pi^2} e^{ -\sigma^2 \big(\frac{\phi_1^2}{2} + 2 \phi_2^2 \big)} = \frac{1}{4\pi\sigma^2}.
\end{align}
While the states $\ket{w(t)}$ are not exactly normalized, we have that
\begin{align}
   \braket{w(t)}{w(t)} &= \eta^2 \int_{-\delta}^\delta \int_{-\delta}^\delta \frac{d\phi_1d\phi_2}{4\pi^2}  e^{ -\sigma^2 \big(\frac{\phi_1^2}{2} + 2 \phi_2^2 \big)} &= 1  - \frac{\eta^2}{\pi^2} \int_\delta^{\infty} \int_{\delta}^\infty d\phi_1d\phi_2 e^{ - \sigma^2 \big(\frac{\phi_1^2}{2} + 2 \phi_2^2\big)}\\
   & \geq  1 - \frac{1}{\pi \delta^2 \sigma^2}e^{- \frac{5 \sigma^2 \delta^2}{2}},
\end{align}
and as the second term on the right hand side is non-negative, we have that $\braket{w(t)}{w(t)} \leq 1$.

As in the single-particle case, we will want to show that the overlap between $\ket{w}$ and $\ket{\alpha(t)}_\pm$ is nearly 1.  In particular, for $0 \leq t < \frac{\nu(t) - \mu(t) - 2 L}{2 \sin k_2 - 2 \sin k_1}$, we then have that
\begin{align}
  &\braket{w(t)}{\alpha(t)}_\pm \nonumber\\
  &\qquad= \eta \gamma^2  \int_{-\delta}^\delta \int_{-\delta}^\delta \frac{d\phi_1 d \phi_2}{4\pi^2}  e^{-\frac{\sigma^2 \phi_1^2}{4}}  e^{- \sigma^2 \phi_2^2}\Big[h_L^\sigma \big(\frac{\phi_1}{2} - \phi_2 \big) h_L^\sigma\Big( \frac{\phi_1}{2} + \phi_2 \Big) \nonumber\\
  & \qquad \quad\pm e^{ i \theta_{\pm}(p_1+\phi_1, p_2+\phi_2)} e^{ 2 i( p_2 + \phi_2)(\mu(t) - \nu(t))} h_L^\sigma\Big(\frac{\phi_1}{2} - 2 p_2 - \phi_2\Big) h_L^\sigma\Big(\frac{\phi_1}{2}+ 2 p_2 + \phi_2\Big)\Big]\\
  & \qquad = \eta\gamma^2 \int_{-\delta}^\delta \int_{-\delta}^\delta \frac{d\phi_1 d\phi_2}{4\pi^2} e^{- \frac{\sigma^2 \phi_1^2}{4}}e^{-\sigma^2 \phi_2^2} \Bigg[ h_\infty^\sigma(\frac{\phi_1}{2} - \phi_{2} ) h_{\infty}^\sigma ( \frac{\phi_1}{2} + \phi_2) \nonumber\\
  &\qquad\quad 
   + \Big[h_L^{\sigma} \Big(\frac{\phi_1}{2} + \phi_2 \Big)  + h_\infty^\sigma \Big(\frac{\phi_1}{2} + \phi_2 \Big)\Big] \Big[ h_{L}^\sigma\Big(\frac{\phi_1}{2} - \phi_2 \Big) - h_\infty^\sigma\Big(\frac{\phi_1}{2} - \phi_2\Big)\Big]\nonumber\\
   &\qquad \quad\pm e^{ i \theta_{\pm}(p_1+\phi_1, p_2+\phi_2)} e^{ 2 i( p_2 + \phi_2)(\mu(t) - \nu(t))}  \Big[h_L^{\sigma} \Big(\frac{\phi_1}{2} + 2p_2 +\phi_2 \Big)  + h_\infty^\sigma \Big(\frac{\phi_1}{2} + 2p_2+ \phi_2 \Big)\Big] \nonumber\\
   &\qquad \quad \qquad \times\Big[ h_{L}^\sigma\Big(\frac{\phi_1}{2} -2p_2 - \phi_2 \Big) - h_\infty^\sigma\Big(\frac{\phi_1}{2} -2p_2- \phi_2\Big)\Big]\nonumber\\
   &\qquad \quad\pm e^{ i \theta_{\pm}(p_1+\phi_1, p_2+\phi_2)} e^{ 2 i( p_2 + \phi_2)(\mu(t) - \nu(t))}h_\infty^{\sigma} \Big(\frac{\phi_1}{2} + 2p_2 +\phi_2 \Big) h_\infty^{\sigma} \Big(\frac{\phi_1}{2} - 2p_2 -\phi_2 \Big) \Bigg].
\end{align}
While this expression looks rather complicated, most of the amplitude is a result of the first term in the integrand, and the rest of the terms are simply small error terms resulting from approximating $h_L^\sigma$ by $h_\infty^\sigma$ or cross terms between the pre- and post-scattering amplitudes.  As such, we will bound each term individually, giving upper and lower bounds on the first term, and only upper bounds on the norm of the latter terms.

For the first term, we have
\begin{align}
  &\int_{-\delta}^\delta \int_{-\delta}^{\delta} \frac{ d\phi_1 d\phi_2}{4\pi^2} e^{ - \frac{\sigma^2\phi_1^2}{4} - \sigma^2 \phi_2^2} h_\infty^\sigma \Big(\frac{\phi_1}{2} - \phi_2\Big) h_\infty^\sigma \Big( \frac{\phi_1}{2} + \phi_2\Big)\nonumber\\
  & \qquad =  \frac{\sigma^2}{2\pi} \int_{-\delta}^\delta \int_{-\delta}^\delta d\phi_1 d\phi_2 e^{ -\frac{\sigma^2 \phi_1^2}{2} - 2\sigma^2 \phi_2^2} h_\infty^{1/(2\pi \sigma)} \Big[2 \pi i \sigma^2 \Big(\frac{\phi_1}{2} - \phi_2 \Big)\Big]h_\infty^{1/(2\pi \sigma)} \Big[2 \pi i \sigma^2 \Big(\frac{\phi_1}{2} +\phi_2 \Big)\Big]\\
 & \qquad \geq \frac{\sigma^2}{2\pi} \int_{-\delta}^\delta \int_{-\delta}^\delta d\phi_1 d\phi_2 e^{ -\frac{\sigma^2 \phi_1^2}{2} - 2\sigma^2 \phi_2^2}\\
 & \qquad = \frac{1}{2} \braket{w(t)}{w(t)}
\end{align}
where we used the fact that $h_\infty^\sigma(i \phi) \geq 1$.  Using the upper bounds on $h$, we also have that
\begin{align}
&\int_{-\delta}^\delta \int_{-\delta}^{\delta} \frac{ d\phi_1 d\phi_2}{4\pi^2} e^{ - \frac{\sigma^2\phi_1^2}{4} - \sigma^2 \phi_2^2} h_\infty^\sigma \Big(\frac{\phi_1}{2} - \phi_2\Big) h_\infty^\sigma \Big( \frac{\phi_1}{2} + \phi_2\Big)\nonumber\\
& \qquad \leq \frac{\sigma^2}{2\pi} \int_{-\delta}^\delta \int_{-\delta}^\delta d\phi_1 d\phi_2 e^{ -\frac{\sigma^2 \phi_1^2}{2} - 2\sigma^2 \phi_2^2} \Bigg[1 + 2 \Big(1 + \frac{ 1}{\pi \sigma^2} \frac{1}{4 \pi -  | \phi_1 - 2\phi_2|} \Big)e^{-2\pi^2 \sigma^2 + \pi \sigma^2 (\phi_1 - 2 \phi_2)} \Bigg]\nonumber\\
& \qquad \qquad \times   \Bigg[1 + 2 \Big(1 + \frac{ 1}{\pi \sigma^2} \frac{1}{4 \pi -  | \phi_1+ 2\phi_2|} \ \Big)e^{-2\pi^2 \sigma^2 +   \pi \sigma^2 (\phi_1 + 2 \phi_2)} \Bigg]\\
& \qquad \leq \frac{1}{2} \Big[1 + 3 e^{ -  \pi^2 \sigma^2 \big(2 \pi - 3\delta\big)} \Big]^2  \braket{w(t)}{w(t)}
\end{align}
where we assumed that $2\pi > 3 \delta$, and that $\sigma \geq 1$. 

For the second term, note that for any $\varphi_1$, $\varphi_2$, and for any $L > \sigma \geq 1$, we have that
\begin{align}
  &\big| \big(h_L^\sigma(\varphi_1) + h_{\infty}^\sigma(\varphi_1)\big)\big(h_L^\sigma(\varphi_2) - h_\infty^\sigma(\varphi_2) \big) \big| \nonumber\\
  &\qquad\leq \big| h_L^\sigma(\varphi_1) + h_{\infty}^\sigma(\varphi_1)\big| \frac{2\sigma^2}{L} e^{-\frac{L^2}{2\sigma^2}}\\
   & \qquad\leq \frac{2 \sigma^2}{L} e^{-\frac{L^2}{2\sigma^2} }\Big [ 2 + \big|h_L^\sigma (\varphi_2) - h_\infty^\sigma(\varphi_2) \big| + 2 \big| 1 - h_\infty^\sigma(\varphi_2) \big| \Big]\\
   & \qquad \leq  \frac{2 \sigma^2}{L} e^{-\frac{L^2}{2\sigma^2} }\Big [ 2 +   \frac{2 \sigma^2}{L} e^{-\frac{L^2}{2\sigma^2} }  + 4 (1 + \sigma^2) e^{ -\frac{1}{2\sigma^2}}\Big]\\
   & \qquad \leq \frac{24 \sigma^4}{L}  e^{ - \frac{L^2}{2\sigma^2}}.
\end{align}
If we then use this bound for the integrand, we have 
\begin{align}
  &\Bigg | \int_{-\delta}^\delta \int_{-\delta}^\delta \frac{d\phi_1 d\phi_2}{4\pi^2} e^{- \frac{\sigma^2 \phi_1^2}{4}}e^{-\sigma^2 \phi_2^2}  \Big[h_L^{\sigma} \Big(\frac{\phi_1}{2} + \phi_2 \Big)  + h_\infty^\sigma \Big(\frac{\phi_1}{2} + \phi_2 \Big)\Big] \Big[ h_{L}^\sigma\Big(\frac{\phi_1}{2} - \phi_2 \Big) - h_\infty^\sigma\Big(\frac{\phi_1}{2} - \phi_2\Big)\Big]\Bigg|\nonumber\\
  & \qquad \leq  \int_{-\delta}^\delta \int_{-\delta}^\delta \frac{d\phi_1 d\phi_2}{4\pi^2} e^{- \frac{\sigma^2 \phi_1^2}{4}}e^{-\sigma^2 \phi_2^2}  \frac{24 \sigma^4}{L}  e^{-\frac{L^2}{2\sigma^2}}\\
  & \qquad \leq \frac{6\sigma^4}{\pi^2 L}  e^{ - \frac{ L^2}{2\sigma^2}}\int_{-\infty}^\infty \int_{-\infty}^\infty  d\phi_1 d\phi_2 e^{- \frac{\sigma^2 \phi_1^2}{4}}e^{-\sigma^2 \phi_2^2}\\
  & \qquad = \frac{ 12 \sigma^2}{\pi L}  e^{ - \frac{L^2}{2\sigma^2}}.
\end{align}

For the third term, we can use the same argument, and we have
\begin{align}
    &\Bigg | \int_{-\delta}^\delta \int_{-\delta}^\delta \frac{d\phi_1 d\phi_2}{4\pi^2} e^{- \frac{\sigma^2 \phi_1^2}{4}}e^{-\sigma^2 \phi_2^2} e^{ i \theta_{\pm}(p_1+\phi_1, p_2+\phi_2)} e^{ 2 i( p_2 + \phi_2)(\mu(t) - \nu(t))}   \nonumber\\
   &\quad \times\Big[h_L^{\sigma} \Big(\frac{\phi_1}{2} + 2p_2 +\phi_2 \Big)  + h_\infty^\sigma \Big(\frac{\phi_1}{2} + 2p_2+ \phi_2 \Big)\Big]\nonumber\\
   &\quad \times\Big[ h_{L}^\sigma\Big(\frac{\phi_1}{2} -2p_2 - \phi_2 \Big) - h_\infty^\sigma\Big(\frac{\phi_1}{2} -2p_2- \phi_2\Big)\Big]\Bigg|\nonumber\\
   &\qquad \qquad \leq \int_{-\delta}^\delta \frac{d\phi_1 d\phi_2}{4\pi^2} e^{- \frac{\sigma^2 \phi_1^2}{4}}e^{-\sigma^2 \phi_2^2} \frac{24 \sigma^4}{L} e^{ - \frac{L^2}{2\sigma^2}} \leq \frac{ 12 \sigma^2}{\pi L} e^{ - \frac{L^2}{2\sigma^2}}.
\end{align}

For the fourth term, we instead need to use the fact that $h(\phi)$ rapidly decreases for large $\phi$.  In particular, if we assume that $\sigma^{-1} < 2\pi - |k_1| - |k_2|$, we have
\begin{align}
    &\Bigg | \int_{-\delta}^\delta \int_{-\delta}^\delta \frac{d\phi_1 d\phi_2}{4\pi^2} e^{- \frac{\sigma^2 \phi_1^2}{4}}e^{-\sigma^2 \phi_2^2} e^{ i \theta_{\pm}(p_1+\phi_1, p_2+\phi_2)} e^{ 2 i( p_2 + \phi_2)(\mu(t) - \nu(t))}   h_\infty^{\sigma} \Big(\frac{\phi_1}{2} + 2p_2 +\phi_2 \Big) h_\infty^{\sigma} \Big(\frac{\phi_1}{2} - 2p_2 -\phi_2 \Big) \Bigg| \nonumber\\
    &  \qquad \leq  \int_{-\delta}^\delta \int_{-\delta}^\delta \frac{d\phi_1 d\phi_2}{4\pi^2} e^{- \frac{\sigma^2 \phi_1^2}{4}}e^{-\sigma^2 \phi_2^2} h_\infty^{\sigma} \Big(\frac{\phi_1}{2} + 2p_2 +\phi_2 \Big) h_\infty^{\sigma} \Big(\frac{\phi_1}{2} - 2p_2 -\phi_2 \Big)\\
    &\qquad = \frac{\sigma^2}{2\pi}\int_{-\delta}^\delta \int_{-\delta}^\delta {d\phi_1 d\phi_2}  e^{- \sigma^2 \phi_1^2}e^{-\sigma^2 (\phi_2^2 + (2p_2 + \phi_2)^2)} \nonumber\\
    &\qquad \qquad \qquad \times h_{\infty}^{1/(2\pi\sigma)} \Big[2\pi i \sigma^2 \Big(\frac{\phi_1}{2} + 2p_2 +\phi_2 \Big)\Big] h_\infty^{1/(2\pi\sigma)}\Big[ 2\pi i \sigma^2  \Big(\frac{\phi_1}{2} - 2p_2 -\phi_2 \Big) \Big]\\
    & \leq  \frac{\sigma^2}{2\pi}\int_{-\delta}^\delta \int_{-\delta}^\delta {d\phi_1 d\phi_2}  e^{- \sigma^2 \phi_1^2}e^{-\sigma^2 (\phi_2^2 + (2p_2 + \phi_2)^2)}\big( 1 + 3 e^{ - 2 \pi \sigma^2 ( \pi - | 2 p_2 + \phi_2 - \frac{\phi_1}{2} | )}\big) \big( 1 + 3 e^{ - 2 \pi \sigma^2 ( \pi - | 2 p_2 + \phi_2 + \frac{\phi_1}{2} | )}\big) 
\end{align}
At this point, if $\pi > |k_1| + |k_2|$, we can choose $\delta$ so that $\pi > |k_1| + |k_2| + 2\delta$ and thus 
\begin{align}
    &\Bigg | \int_{-\delta}^\delta \int_{-\delta}^\delta \frac{d\phi_1 d\phi_2}{4\pi^2} e^{- \frac{\sigma^2 \phi_1^2}{4}}e^{-\sigma^2 \phi_2^2} e^{ i \theta_{\pm}(p_1+\phi_1, p_2+\phi_2)} e^{ 2 i( p_2 + \phi_2)(\mu(t) - \nu(t))}   h_\infty^{\sigma} \Big(\frac{\phi_1}{2} + 2p_2 +\phi_2 \Big) h_\infty^{\sigma} \Big(\frac{\phi_1}{2} - 2p_2 -\phi_2 \Big) \Bigg| \nonumber\\
    &  \qquad \leq \frac{8\sigma^2}{\pi}\int_{-\delta}^\delta \int_{-\delta}^\delta {d\phi_1 d\phi_2}  e^{- \sigma^2 \phi_1^2}e^{-\sigma^2 (\phi_2^2 + (2p_2 + \phi_2)^2)}\\
    & \qquad \leq \frac{8 \sigma}{\sqrt{\pi}} \int_{-\delta}^\delta d\phi_2 e^{ - 2\sigma^2 ( p_2^2 + (p_2 + \phi_2)^2)}\\
    &\qquad \leq 4\sqrt{2} e^{ -2 \sigma^2 p_2^2} = 4\sqrt{2} e^{-\frac{\sigma^2}{2} (k_2-k_1)^2}.
\end{align}
However, if $\pi \leq |k_1| + |k_2|$, we can instead bound the functions approximating $h$ by their largest value, which is attained when $2\phi_2 \pm \phi_1 = 3 \delta$.  In particular, we have
\begin{align}
    &\Bigg | \int_{-\delta}^\delta \int_{-\delta}^\delta \frac{d\phi_1 d\phi_2}{4\pi^2} e^{- \frac{\sigma^2 \phi_1^2}{4}}e^{-\sigma^2 \phi_2^2} e^{ i \theta_{\pm}(p_1+\phi_1, p_2+\phi_2)} e^{ 2 i( p_2 + \phi_2)(\mu(t) - \nu(t))}   h_\infty^{\sigma} \Big(\frac{\phi_1}{2} + 2p_2 +\phi_2 \Big) h_\infty^{\sigma} \Big(\frac{\phi_1}{2} - 2p_2 -\phi_2 \Big) \Bigg| \nonumber\\
    &  \qquad \leq \frac{8\sigma^2}{\pi}\int_{-\delta}^\delta \int_{-\delta}^\delta {d\phi_1 d\phi_2}  e^{- \sigma^2 \phi_1^2}e^{-\sigma^2 (\phi_2^2 + (2p_2 + \phi_2)^2)} e^{ 4 \pi \sigma^2 ( 2|p_2| + \frac{3\delta}{2} - \pi)} \\
    & \qquad \leq \frac{8 \sigma}{\sqrt{\pi}} e^{4\pi \sigma^2 (2|p_2| + \frac{3}{2}\delta -\pi) - 2 \sigma^2 p_2^2 } \int_{-\delta}^\delta d\phi_2 e^{ -2\sigma^2( p_2 + 2 p_2 \phi_2 + \phi_2^2)}\\
    &\qquad \leq \frac{16\delta}{\sqrt{\pi}}e^{-\sigma^2 ( 4\pi^2  - 8 \pi |p_2|  + 4 \sigma^2 p_2^2 - 6 \pi \delta + 2\delta^2 - 4 \delta |p_2| )}\\
    & \qquad \leq \frac{16 \delta}{\sqrt{\pi}} e^{ -\sigma^2 (4 [\pi - |p_2|]^2 - (4|p_2| + 6\pi)\delta)}\\
    &\qquad \leq \frac{ 16 \delta}{\sqrt{\pi}} e^{ - 3 \sigma^2 \delta}
\end{align}
where we assume that $\delta < \frac{ (\pi - |p_2|)^2}{4|p_2| + 6\pi}$.  We can then put these two bounds together, if we assume that $\delta < p_2^2$ and that $\delta <  \frac{ (\pi - |p_2|)^2}{4|p_2| + 6\pi}$, so that 
\begin{align}
    &\Bigg | \int_{-\delta}^\delta \int_{-\delta}^\delta \frac{d\phi_1 d\phi_2}{4\pi^2} e^{- \frac{\sigma^2 \phi_1^2}{4}}e^{-\sigma^2 \phi_2^2} e^{ i \theta_{\pm}(p_1+\phi_1, p_2+\phi_2)} e^{ 2 i( p_2 + \phi_2)(\mu(t) - \nu(t))}   h_\infty^{\sigma} \Big(\frac{\phi_1}{2} + 2p_2 +\phi_2 \Big) h_\infty^{\sigma} \Big(\frac{\phi_1}{2} - 2p_2 -\phi_2 \Big) \Bigg| \nonumber\\
    & \qquad \leq \frac{16 }{\sqrt{\pi}} e^{ - 2 \sigma^2 \delta}.
\end{align}

We can now combine these results, and thus show that $\ket{\alpha(t)}_{\pm}$ is well approximated by $\ket{w(t)}$ for $0\leq t < \frac{\nu(t) - \mu(t) - 2 L}{2 \sin k_2 - 2 \sin k_1}$.  In particular, we have
\begin{align}
   &\norm{\ket{\alpha(t)}_{\pm} - \ket{w(t)}}^2\nonumber\\
    &\qquad\leq {}_{\pm}\braket{\alpha(t)}{\alpha(t)}_{\pm} + \braket{w(t)}{w(t)} - \braket{w(t)}{\alpha(t)}_{\pm} - {}_\pm \braket{\alpha(t)}{w(t)}\\
    &\qquad \leq 1 + \braket{w(t)}{w(t)} - 2 \eta\gamma^2 \Big(\frac{1}{2} \braket{w(t)}{w(t)} - \frac{12 \sigma^2}{\pi L} e^{ -\frac{L^2}{2\sigma^2}} - \frac{12 \sigma^2}{\pi L} e^{ - \frac{L^2}{2\sigma^2}} -  \frac{16}{\sqrt{\pi}} e^{-2\sigma^2 \delta}\Big).
\end{align}
For large enough $\sigma$, $L$, and small enough $\delta$, we have that the term multiplying $\eta\gamma$ is negative, and thus we need to give a lower bound on $\eta\gamma^2$.  We know the value of $\eta$, and as a lower bound on $\gamma^2$ we have
\begin{align}
  \gamma^2 = \frac{1}{h_L^{\sigma/\sqrt{2}}(0)} \geq \frac{1}{\sqrt{\pi} \sigma (1 + 3 e^{-\pi^2\sigma^2})} \geq \frac{1}{\sqrt{\pi} \sigma} \big(1 - 3 e^{-\pi^2 \sigma^2}\big).
\end{align}
From this, and using our upper bounds on the norm of $\braket{w(t)}{w(t)}$, we have
\begin{align}
 &\norm{\ket{\alpha(t)}_{\pm} - \ket{w(t)}}^2 \nonumber\\
 &\qquad \leq 1 + \braket{w(t)}{w(t)} - 4 \big(1 - 3 e^{-\pi^2 \sigma^2} \big) \Big(\frac{1}{2} \braket{w(t)}{w(t)} - \frac{24 \sigma^2}{\pi L} e^{ -\frac{L^2}{2\sigma^2}}  -  \frac{16}{\sqrt{\pi}} e^{-2\sigma^2 \delta}\Big)\\
 & \qquad \leq 1 - \braket{w(t)}{w(t)} (1 -6 e^{-\pi^2 \sigma^2}) + \frac{96 \sigma^2}{\pi L} e^{-\frac{L^2}{2\sigma^2}} + \frac{64}{\sqrt{\pi}} e^{-2\sigma^2 \delta}\\
 & \qquad \leq 1 - \Big( 1 - \frac{1}{\pi \delta^2 \sigma^2} e^{-\frac{5 \sigma^2 \delta^2}{2}} \Big)(1 -6 e^{-\pi^2 \sigma^2}) + \frac{96 \sigma^2}{\pi L} e^{-\frac{L^2}{2\sigma^2}} + \frac{64}{\sqrt{\pi}} e^{-2\sigma^2 \delta}\\
 & \qquad \leq \frac{1}{\pi \delta^2 \sigma^2} e^{-\frac{5 \sigma^2 \delta^2}{2}} + 6 e^{-\pi^2 \sigma^2} + \frac{96 \sigma^2}{\pi L} e^{-\frac{L^2}{2\sigma^2}} + \frac{64}{\sqrt{\pi}} e^{-2\sigma^2 \delta}\\
 &\qquad \leq 44 e^{-2 \sigma^2 \delta} + \frac{32 \sigma^2}{L} e^{-\frac{L^2}{2\sigma^2}}.
\end{align}

Let us now examine how close $\ket{w(t)}$ and $\ket{\alpha(t)}_\pm$ are for $t > \frac{\nu(t) - \mu(t) + 2L}{2\sin k_2 - 2 \sin k_1}$.  In particular, we have for these times that
\begin{align}
  &\braket{w(t)}{\alpha(t)}_\pm \nonumber\\
  &\qquad= \eta \gamma^2 e^{i \theta_{\pm}(p_1,p_2)} \int_{-\delta}^\delta \int_{-\delta}^\delta \frac{d\phi_1 d \phi_2}{4\pi^2}  e^{-\frac{\sigma^2 \phi_1^2}{4}}  e^{- \sigma^2 \phi_2^2}\Big[e^{-i \theta_{\pm}(p_1 + \phi_1, p_2 + \phi_2)} h_L^\sigma \big(\frac{\phi_1}{2} - \phi_2 \big) h_L^\sigma\Big( \frac{\phi_1}{2} + \phi_2 \Big) \nonumber\\
  & \qquad \quad\pm e^{ 2 i( p_2 + \phi_2)(\mu(t) - \nu(t))} h_L^\sigma\Big(\frac{\phi_1}{2} - 2 p_2 - \phi_2\Big) h_L^\sigma\Big(\frac{\phi_1}{2}+ 2 p_2 + \phi_2\Big)\Big]\\
  & \qquad = \eta\gamma^2 e^{i \theta_{\pm}(p_1,p_2)}  \int_{-\delta}^\delta \int_{-\delta}^\delta \frac{d\phi_1 d\phi_2}{4\pi^2} e^{- \frac{\sigma^2 \phi_1^2}{4}}e^{-\sigma^2 \phi_2^2} \Bigg[ e^{-i \theta_{\pm}(p_1,p_2) }h_\infty^\sigma(\frac{\phi_1}{2} - \phi_{2} ) h_{\infty}^\sigma ( \frac{\phi_1}{2} + \phi_2) \nonumber\\
  &\qquad\quad \Big(e^{- i\theta_{\pm}(p_1 + \phi_1, p_2 + \phi_2)} - e^{- i \theta_{\pm}(p_1, p_2)}  \Big) h_\infty^\sigma(\frac{\phi_1}{2} - \phi_{2} ) h_{\infty}^\sigma ( \frac{\phi_1}{2} + \phi_2) \nonumber\\
 &\qquad \quad  + e^{- i\theta_{\pm}(p_1 + \phi_1, p_2 + \phi_2)} \Big[h_L^{\sigma} \Big(\frac{\phi_1}{2} + \phi_2 \Big)  + h_\infty^\sigma \Big(\frac{\phi_1}{2} + \phi_2 \Big)\Big] \Big[ h_{L}^\sigma\Big(\frac{\phi_1}{2} - \phi_2 \Big) - h_\infty^\sigma\Big(\frac{\phi_1}{2} - \phi_2\Big)\Big]\nonumber\\
   &\qquad \quad\pm  e^{ 2 i( p_2 + \phi_2)(\mu(t) - \nu(t))}  \Big[h_L^{\sigma} \Big(\frac{\phi_1}{2} + 2p_2 +\phi_2 \Big)  + h_\infty^\sigma \Big(\frac{\phi_1}{2} + 2p_2+ \phi_2 \Big)\Big] \nonumber\\
   &\qquad \quad \qquad \times\Big[ h_{L}^\sigma\Big(\frac{\phi_1}{2} -2p_2 - \phi_2 \Big) - h_\infty^\sigma\Big(\frac{\phi_1}{2} -2p_2- \phi_2\Big)\Big]\nonumber\\
   &\qquad \quad\pm  e^{ 2 i( p_2 + \phi_2)(\mu(t) - \nu(t))}h_\infty^{\sigma} \Big(\frac{\phi_1}{2} + 2p_2 +\phi_2 \Big) h_\infty^{\sigma} \Big(\frac{\phi_1}{2} - 2p_2 -\phi_2 \Big) \Bigg].
\end{align}
This is nearly an identical overlap as for the small $t$ case, except for the changing angle $\theta_\pm$.  As such, we can bound most of the terms as before.

The first term in the bound is simply
\begin{align}
  &e^{i\theta_\pm(p_1,p_2)} \int_{-\delta}^\delta \int_{-\delta}^\delta \frac{d\phi_1 d\phi_2}{4\pi^2} e^{- \frac{\sigma^2 \phi_1^2}{4}}e^{-\sigma^2 \phi_2^2} e^{-i \theta_{\pm}(p_1,p_2) }h_\infty^\sigma(\frac{\phi_1}{2} - \phi_{2} ) h_{\infty}^\sigma ( \frac{\phi_1}{2} + \phi_2) \\
  &  \qquad = \int_{-\delta}^\delta \int_{-\delta}^\delta \frac{d\phi_1 d\phi_2}{4\pi^2} e^{- \frac{\sigma^2 \phi_1^2}{4}}e^{-\sigma^2 \phi_2^2}h_\infty^\sigma(\frac{\phi_1}{2} - \phi_{2} ) h_{\infty}^\sigma ( \frac{\phi_1}{2} + \phi_2) \\
  & \qquad \geq \frac{1}{2} \braket{w(t)}{w(t)}
\end{align}
as in the small $t$ case.

The second term is slightly more complicated.  However, remember from equation ?????? \todo{find the correct equation} that $\theta_\pm(p_1+\phi_1,p_2+\phi_2)$ are bounded rational functions of $e^{i\phi_1}$ and $e^{i\phi_2}$.  As such, they are differentiable as functions of both $\phi_1$ and $\phi_2$ on some neighborhood $U$ of $(0,0)$.  Let us assume that $\delta$ is chosen so that $[-\delta,\delta]\times [-\delta,\delta] \subset U$, and now let
\begin{align}
  \Gamma = \max_{[-\delta,\delta]\times [-\delta,\delta]} \big|\nabla e^{i \theta_{\pm}(p_1 +\phi_1, p_2 +\phi_2)}\big|
\end{align}
From this, we then have that
\begin{align}
  &\Bigg|\int_{-\delta}^\delta \int_{-\delta}^\delta \frac{d\phi_1 d\phi_2}{4\pi^2} e^{- \frac{\sigma^2 \phi_1^2}{4}}e^{-\sigma^2 \phi_2^2}  \big(e^{- i\theta_{\pm}(p_1 + \phi_1, p_2 + \phi_2)} - e^{- i \theta_{\pm}(p_1, p_2)}  \big) h_\infty^\sigma(\frac{\phi_1}{2} - \phi_{2} ) h_{\infty}^\sigma ( \frac{\phi_1}{2} + \phi_2)\Bigg|\nonumber\\
  & \qquad \leq \int_{-\delta}^\delta \int_{-\delta}^\delta \frac{d\phi_1 d\phi_2}{4\pi^2} e^{- \frac{\sigma^2 \phi_1^2}{4}}e^{-\sigma^2 \phi_2^2}  \big(\phi_1^2 + \phi_2^2 \big) \Gamma h_\infty^\sigma(\frac{\phi_1}{2} - \phi_{2} ) h_{\infty}^\sigma ( \frac{\phi_1}{2} + \phi_2)\\
  & \qquad = \frac{\Gamma\sigma^2}{2\pi} \int_{-\delta}^\delta \int_{-\delta}^\delta {d\phi_1 d\phi_2} e^{- \frac{\sigma^2 \phi_1^2}{2}}e^{-2\sigma^2 \phi_2^2}( \phi_1^2 + \phi_2^2) h_\infty^{1/(2\pi \sigma)} \Big(2 \pi i \sigma^2 \Big[\frac{\phi_1}{2} - \phi_2 \Big]\Big) h_\infty^{1/(2\pi \sigma)}  \Big(2 \pi i \sigma^2 \Big[\frac{\phi_1}{2} + \phi_2 \Big]\Big) \\
  & \qquad \leq \frac{\Gamma\sigma^2}{2\pi} (1 + 3 e^{- \pi^2 \sigma^2}) \int_{-\infty}^\infty \int_{-\infty}^\infty {d\phi_1 d\phi_2} e^{- \frac{\sigma^2 \phi_1^2}{2}}e^{-2\sigma^2 \phi_2^2}( \phi_1^2 + \phi_2^2)\\
  & \qquad \leq \frac{5\Gamma }{2\sigma^2}.
\end{align}

The third term can be bounded exactly as the second term for small $t$.  In particular, we have bounds on the differences and sums of $h$, and thus 
\begin{align}
  &\Bigg| \int_{-\delta}^\delta \int_{-\delta}^\delta \frac{d\phi_1 d\phi_2}{4\pi^2} e^{- \frac{\sigma^2 \phi_1^2}{4}}e^{-\sigma^2 \phi_2^2}e^{- i\theta_{\pm}(p_1 + \phi_1, p_2 + \phi_2)}\nonumber\\
  &\qquad \times \Big[h_L^{\sigma} \Big(\frac{\phi_1}{2} + \phi_2 \Big)  + h_\infty^\sigma \Big(\frac{\phi_1}{2} + \phi_2 \Big)\Big] \Big[ h_{L}^\sigma\Big(\frac{\phi_1}{2} - \phi_2 \Big) - h_\infty^\sigma\Big(\frac{\phi_1}{2} - \phi_2\Big)\Big]\Bigg|\nonumber\\
  &\qquad\qquad \leq \int_{-\delta}^\delta \int_{-\delta}^\delta \frac{d\phi_1 d\phi_2}{4\pi^2} e^{- \frac{\sigma^2 \phi_1^2}{4}}e^{-\sigma^2 \phi_2^2} \frac{24 \sigma^4}{L} e^{-\frac{L^2}{2\sigma^2}}\\
  &\qquad \qquad \leq \frac{12 \sigma^2}{\pi L} e^{-\frac{L^2}{2\sigma^2}}.
\end{align}
In a similar manner, we have that the fourth term can be bounded as the third for small $t$.
\begin{align}
  &\Bigg| \int_{-\delta}^\delta \int_{-\delta}^\delta \frac{d\phi_1 d\phi_2}{4\pi^2} e^{- \frac{\sigma^2 \phi_1^2}{4}}e^{-\sigma^2 \phi_2^2}e^{- 2i (p_2 + \phi_2)(\mu(t) - \nu(t))} \Big[h_L^{\sigma} \Big(\frac{\phi_1}{2} +2p_2 + \phi_2 \Big)  + h_\infty^\sigma \Big(\frac{\phi_1}{2} +2p_2 + \phi_2 \Big)\Big] \nonumber\\
  &\qquad \times\Big[ h_{L}^\sigma\Big(\frac{\phi_1}{2} -2 p_2 - \phi_2 \Big) - h_\infty^\sigma\Big(\frac{\phi_1}{2} -2 p_2 - \phi_2\Big)\Big]\Bigg|\nonumber\\
  & \qquad \qquad \leq \int_{-\delta}^\delta \int_{-\delta}^\delta \frac{d\phi_1 d\phi_2}{4\pi^2} e^{- \frac{\sigma^2 \phi_1^2}{4}}e^{-\sigma^2 \phi_2^2} \frac{24 \sigma^4}{L} e^{-\frac{L^2}{2\sigma^2}}\\
  &\qquad \qquad \leq \frac{12 \sigma^2}{\pi L} e^{-\frac{L^2}{2\sigma^2}}.
\end{align}

Finally, we have that the final term can be bounded as
\begin{align}
  &\Bigg| \int_{-\delta}^\delta \int_{-\delta}^\delta \frac{d\phi_1 d\phi_2}{4\pi^2} e^{- \frac{\sigma^2 \phi_1^2}{4}}e^{-\sigma^2 \phi_2^2} e^{2i (p_2 + \phi_2)(\mu(t) - \nu(t))}h_\infty^\sigma \Big(\frac{\phi_1}{2} -2p_2 - \phi_2 \Big)h_\infty^\sigma \Big(\frac{\phi_1}{2} +2p_2 + \phi_2 \Big)\Bigg| \nonumber\\
  & \qquad \leq \int_{-\delta}^\delta \int_{-\delta}^\delta \frac{d\phi_1 d\phi_2}{4\pi^2} e^{- \frac{\sigma^2 \phi_1^2}{4}}e^{-\sigma^2 \phi_2^2}h_\infty^\sigma \Big(\frac{\phi_1}{2} -2p_2 - \phi_2 \Big)h_\infty^\sigma \Big(\frac{\phi_1}{2} +2p_2 + \phi_2 \Big)\\
  & \qquad \leq \frac{16}{\sqrt{\pi}} e^{-2\sigma^2 \delta}
\end{align} 
where we used the bounds for small $t$.

We can then use the bounds on all of the individual terms to show that $\ket{w(t)}$ and $\ket{\alpha(t)}$ are close.  In particular, we have 
\begin{align}   
  &\norm{\ket{\alpha(t)}_{\pm} - \ket{w(t)}}^2\nonumber\\
    &\qquad\leq {}_{\pm}\braket{\alpha(t)}{\alpha(t)}_{\pm} + \braket{w(t)}{w(t)} - \braket{w(t)}{\alpha(t)}_{\pm} - {}_\pm \braket{\alpha(t)}{w(t)}\\
    &\qquad \leq 1 + \braket{w(t)}{w(t)} - 2 \eta\gamma^2 \Big(\frac{1}{2} \braket{w(t)}{w(t)} - \frac{5 \Gamma}{2\sigma^2}- \frac{12 \sigma^2}{\pi L} e^{ -\frac{L^2}{2\sigma^2}} - \frac{12 \sigma^2}{\pi L} e^{ - \frac{L^2}{2\sigma^2}} -  \frac{16}{\sqrt{\pi}} e^{-2\sigma^2 \delta}\Big).
\end{align}
If we again use our bounds on $\gamma^2$ and on $\braket{w(t)}{w(t)}$, we find
\begin{align}
    &\norm{\ket{\alpha(t)}_{\pm} - \ket{w(t)}}^2\nonumber\\
    & \qquad \leq 1 + \braket{w(t)}{w(t)} - 4 (1 - 3 e^{-\pi^2 \sigma^2})\Big(\frac{1}{2} \braket{w(t)}{w(t)} - \frac{5 \Gamma}{2\sigma^2}- \frac{24 \sigma^2}{\pi L} e^{ - \frac{L^2}{2\sigma^2}} -  \frac{16}{\sqrt{\pi}} e^{-2\sigma^2 \delta}\Big)\\
    & \qquad \leq 1 - \braket{w(t)}{w(t)} ( 1 - 6 e^{-\pi^2 \sigma^2}) + \frac{5 \Gamma}{2\sigma^2}+ \frac{24 \sigma^2}{\pi L} e^{ - \frac{L^2}{2\sigma^2}} +  \frac{16}{\sqrt{\pi}} e^{-2\sigma^2 \delta}\\
    & \qquad \leq 1 - \Big(1 - \frac{1}{\pi \delta^2 \sigma^2} e^{- \frac{5 \sigma^2 \delta^2}{2}} \Big)( 1 - 6 e^{-\pi^2 \sigma^2}) + \frac{5 \Gamma}{2\sigma^2}+ \frac{24 \sigma^2}{\pi L} e^{ - \frac{L^2}{2\sigma^2}} +  \frac{16}{\sqrt{\pi}} e^{-2\sigma^2 \delta}\\
    & \qquad \leq \frac{1}{\pi \delta^2 \sigma^2}e^{- \frac{5 \sigma^2 \delta^2}{2}}  + 6 e^{-\pi^2 \sigma^2} +\frac{5 \Gamma}{2\sigma^2}+ \frac{24 \sigma^2}{\pi L} e^{ - \frac{L^2}{2\sigma^2}} +  \frac{16}{\sqrt{\pi}} e^{-2\sigma^2 \delta}\\
    & \qquad \leq 44e^{-2\sigma^2 \delta} + \frac{32\sigma^2}{L} e^{- \frac{L^2}{2\sigma^2}} + \frac{5 \Gamma}{2\sigma^2}.
\end{align}

With these good bounds on our approximations to the expected time-evolved states, it will now be useful to determine the overlap of the actual time-evolved state with our approximations.  Note that $\ket{\alpha(0)}_{\pm} = \ket{\psi(0)}_{\pm}$, and thus we have that $\ket{w(0)}$ is a good approximation to the initial state.  If we define the state
\begin{align}
  \ket{v(t)} &= e^{-i H^{(2)}t} \ket{w(0)} \\
  &= \eta \int_{-\delta}^\delta \int_{-\delta}^\delta \frac{d\phi_1 d\phi_2}{4\pi^2} e^{i \phi_1 \big(\frac{\mu + \nu}{2}\big)} e^{i  \phi_2 (\nu - \mu)} e^{ -\frac{ \sigma^2 \phi_1^2}{4}} e^{ -\sigma^2 \phi_2^2} e^{- 4 i t \cos\big(\frac{p_1+\phi_1}{2}\big)\cos(p_2+\phi_2)} \ket{\scat(p_1 + \phi_1; p_2 + \phi_2)}_{\pm}
\end{align}
we will have that this state does not change its overlap with $\ket{\psi(t)}$ over time (i.e., $\norm{\ket{v(t)} - \ket{\psi(t)}} = \norm{\ket{v(0)} - \ket{\psi(0)}}$).  Additionally, we will have that $\ket{v(t)}$ and $\ket{w(t)}$ will be close in norm.  As expected, we have
\begin{align}
   \braket{v(t)}{w(t)} &= \eta^2 \int_{-\delta}^\delta \int_{-\delta}^\delta \frac{d\phi_1 d\phi_2}{4\pi^2} e^{i \phi_1 \big(\frac{\mu(t) + \nu(t)}{2} - \frac{\mu +\nu}{2} \big)} e^{i  \phi_2 (\nu(t) - \mu(t) -\nu + \mu)} \nonumber\\
   & \qquad \times e^{ -\frac{ \sigma^2 \phi_1^2}{2}} e^{ -2\sigma^2 \phi_2^2} e^{ 2 i t\big( 2 \cos\big(\frac{p_1+\phi_1}{2}\big)\cos(p_2+\phi_2) - \cos(k_1) - \cos(k_2)\big)} \\
   & = \braket{w(t)}{w(t)} - \eta^2 \int_{-\delta}^\delta \int_{-\delta}^\delta \frac{d\phi_1 d\phi_2}{4\pi^2}  e^{ -\frac{ \sigma^2 \phi_1^2}{2}} e^{ -2\sigma^2 \phi_2^2}\nonumber\\
   & \qquad \times \bigg[ 1 - e^{i \phi_1 \big(\frac{\mu(t) + \nu(t)}{2} - \frac{\mu +\nu}{2} \big)} e^{i  \phi_2 (\nu(t) - \mu(t) -\nu + \mu)}e^{ 2 i t\big( 2 \cos\big(\frac{p_1+\phi_1}{2}\big)\cos(p_2+\phi_2) - \cos(k_1) - \cos(k_2)\big)}\bigg] .
\end{align}
We can then bound the value of the integrand, using the fact that $|1 - e^{i\theta}| \leq \theta$.  We then have
\begin{align}
  &\bigg| 1 - e^{i \phi_1 \big(\frac{\mu(t) + \nu(t)}{2} - \frac{\mu +\nu}{2} \big)} e^{i  \phi_2 (\nu(t) - \mu(t) -\nu + \mu)}e^{ 2 i t\big( 2 \cos\big(\frac{p_1+\phi_1}{2}\big)\cos(p_2+\phi_2) - \cos(k_1) - \cos(k_2)\big)}\bigg|\nonumber\\
  & \qquad \leq \bigg|-\frac{\phi_1}{2} \big[ \lceil 2 t \sin k_1\rceil + \lceil 2t \sin k_2\rceil \big] - \phi_2 \big[ \lceil 2 t \sin k_2\rceil - \lceil 2t \sin k_1\rceil\big]  \nonumber\\
  &\qquad \qquad + 2 t \Big[ 2 \cos\Big( \frac{p_1 +\phi_1}{2}\Big) \cos(p_2 + \phi_2) - \cos(k_1) - \cos(k_2) \Big]\bigg|\\
  & \qquad \leq |\phi_1| + 2 |\phi_2| + 2 t \bigg| 2\cos\Big( \frac{p_1 +\phi_1}{2}\Big) \cos(p_2 + \phi_2) - \cos(k_1) - \cos(k_2)  \nonumber\\
  & \qquad \qquad - \frac{\phi_1}{2} \big[ \sin k_1 +   \sin k_2\big] - \phi_2 \big[  \sin k_2 -  \sin k_1\big]\bigg|\\
  &  \qquad \leq |\phi_1| + 2 |\phi_2| + 2 t \bigg| \cos\Big(- k_2 + \frac{\phi_1}{2} + \phi_2 \Big)  + \cos\Big(- k_1 + \frac{\phi_1}{2} - \phi_2 \Big)- \cos(k_1) - \cos(k_2)  \nonumber\\
  & \qquad \qquad - \Big(\frac{\phi_1}{2} + \phi_2 \Big)\sin k_2  -\Big(\frac{\phi_1}{2} -  \phi_2\Big) \sin k_1 \bigg|\\
  & \qquad \leq |\phi_1| + 2 |\phi_2| + 2 t \bigg| \cos(k_2) \big[\cos\big( \frac{\phi_1}{2} + \phi_2 \big) - 1\big] - \sin(k_2) \big[\big( \frac{\phi_1}{2} + \phi_2 \big) - \sin\big( \frac{\phi_1}{2} + \phi_2 \big) \big]  \bigg|\nonumber\\
  & \qquad \qquad + 2t \bigg| \cos(k_1) \Big[\cos\Big(\frac{\phi_1}{2} - \phi_2 \Big) - 1 \Big] - \sin(k_1) \Big[ \frac{\phi_1}{2} - \phi_2 - \sin\Big(  \frac{\phi_1}{2} - \phi_2\Big) \Big] \bigg|\\
  & \qquad \leq |\phi_1| + 2 |\phi_2| + 2 t  \Big (\frac{\phi_1}{2} + \phi_2\Big)^2 + 2t \Big( \frac{\phi_1}{2} - \phi_2\Big)^2\\
  & \qquad \leq |\phi_1| +\frac{t}{2} \phi_1^2 + 2 |\phi_2| + 4 t\phi_2^2.
\end{align}
We can then use this in the bound, so that
\begin{align}
  & \Bigg| \int_{-\delta}^\delta \int_{-\delta}^\delta \frac{d\phi_1 d\phi_2}{4\pi^2}  e^{ -\frac{ \sigma^2 \phi_1^2}{2}} e^{ -2\sigma^2 \phi_2^2} \bigg[ 1 - e^{i \phi_1 \big(\frac{\mu(t) + \nu(t)}{2} - \frac{\mu +\nu}{2} \big)} e^{i  \phi_2 (\nu(t) - \mu(t) -\nu + \mu)}\nonumber\\
  & \qquad \qquad \qquad \times e^{ 2 i t\big( 2 \cos\big(\frac{p_1+\phi_1}{2}\big)\cos(p_2+\phi_2) - \cos(k_1) - \cos(k_2)\big)}\bigg] \Bigg|\nonumber\\
  & \qquad \leq  \int_{-\delta}^\delta \int_{-\delta}^\delta \frac{d\phi_1 d\phi_2}{4\pi^2}  e^{ -\frac{ \sigma^2 \phi_1^2}{2}} e^{ -2\sigma^2 \phi_2^2} \Big[|\phi_1| + \frac{t}{2} \phi_1^2 + 2 |\phi_2| + 4 t \phi_2^2 \Big]\\
  & \qquad \leq  \int_{0}^\infty \int_{0}^\infty \frac{d\phi_1 d\phi_2}{\pi^2}  e^{ -\frac{ \sigma^2 \phi_1^2}{2}} e^{ -2\sigma^2 \phi_2^2} \Big[\phi_1 + \frac{t}{2} \phi_1^2 + 2 \phi_2+ 4 t \phi_2^2 \Big]\\
  & = \frac{3 t}{8\pi \sigma^4} + \frac{\sqrt{2\pi}}{2\pi^2 \sigma^3}.
\end{align}

With this, we can then bound the difference between $\ket{v(t)}$ and $\ket{w(t)}$.  In particular, we have
\begin{align}
  \norm{\ket{v(t)} - \ket{w(t)}}^2 &= \braket{v(t)}{v(t)} + \braket{w(t)}{w(t)} - \braket{v(t)}{w(t)} - \braket{w(t)}{v(t)}\\
  & \leq  2 \braket{w(t)}{w(t)} - 2 \Big[ \braket{w(t)}{w(t)}  -\eta^2 \frac{3t}{8\pi \sigma^4} - \eta^2 \frac{\sqrt{2\pi}}{2 \pi^2 \sigma^3}\Big]\\
  & = \frac{3t}{\sigma^2} + \frac{4 \sqrt{2}}{\sqrt{\pi} \sigma}.
\end{align}

Finally, we can now bound the norm of the difference between the time evolved $\ket{\psi(t)}$ and the approximation $\ket{\alpha(t)}_{\pm}$.  In particular, if we once again note that $\ket{\alpha(0)}_\pm = \ket{\psi(0)}_{\pm}$, and remember that $\ket{v(t)} = e^{- i H^{(2)} t} \ket{v(0)}$, we have
\begin{align}
  &\norm{ \ket{\psi(t)}_{\pm} - \ket{\alpha(t)}_{\pm}} \nonumber\\
  &\qquad\leq \norm{\ket{\psi(t)_\pm} - \ket{v(t)}} + \norm{ \ket{\alpha(t)}_{\pm} - \ket{w(t)}} + \norm{\ket{v(t)} - \ket{w(t)}}\\
   & \qquad\leq \Big( 44 e^{-2 \sigma^2 \delta} + \frac{32 \sigma^2}{L} e^{-\frac{L^2}{2\sigma^2}}\Big)^{1/2} +   \Big(44e^{-2\sigma^2 \delta} + \frac{32\sigma^2}{L} e^{- \frac{L^2}{2\sigma^2}} + \frac{5 \Gamma}{2\sigma^2} \Big)^{1/2}  + \Big(\frac{3t}{\sigma^2} + \frac{4\sqrt{2}}{\sqrt{\pi} \sigma}\Big)^{1/2}.
\end{align}
If we have chosen $\sigma = \frac{ L}{2 \sqrt{\log(L)}}$, we then have that for $L$ large enough, if $0 < t < \frac{\nu - \mu - 2L}{2\sin k_2 - 2 \sin k_1}$ or $\frac{\nu - \mu + 2L}{2\sin k_2 - 2 \sin k_1} < t < c L$
\begin{align}
  &\norm{ \ket{\psi(t)}_{\pm} - \ket{\alpha(t)}_{\pm}} \nonumber\\
  &\qquad\leq \Big( \frac{9}{L \log( L)} \Big)^{1/2} + \Big( \frac{10}{L \log( L)} \Big)^{1/2} + \Big( \frac{12 t \log L}{L^2} + \frac{16 \sqrt{2 \log L}}{\sqrt{\pi} L}\Big)^{1/2}\\
  & \qquad \leq \chi_2 \sqrt{\frac{ \log L}{L}}
\end{align}
where $\chi_2$ is a constant.
\end{proof}
%%%%%%%%%%%%%%%%%%%%%%%%%%%%%%%%%%%%%%




\biblio
\end{document}