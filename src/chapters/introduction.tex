%======================================================================
%   Zak Webb
%   Ph. D. Thesis
%   Department of Physics and Astronomy
%   University of Waterloo
% 
%   Introduction Chapter
%======================================================================

\documentclass[../thesis-main/thesis-main]{subfiles}
\begin{document}


\chapter{Introduction}
\label{chap:introduction}

The first thing I would like to mention is that this thesis is not complete.  I ran out of time and was unable to complete the thesis.  I had to submit something to the graduate office, and this is it.  I do apologize, and I expect to have major revisions required.  At this point, I do not believe that any one chapter is actually completed, and in fact many of the state results have not been completely written up.  Basically, I recognize that this thesis is defensible as is, and am planning major revisions.  


When examining the physics literature, the three-body problem is extremely well known, as is its usual impossibility to solve.  In these cases, people generally mean that there does not exist a closed form solution in general, but can we quantify exactly how hard the problem is?

In particular, if one is working with some particular many body system, how hard is it to compute various attributes about the system.  I'd really like to know the answer.

While such questions have not generally been asked in physics, classifying the computation power of a problem in terms of the necessary resources in order to solve it is a foundational idea in computer science.  The entire field of computational complexity arose in the attempt to classify these problems.  This thesis will attempt to use tools founded in this field and apply them to the various physical systems.




%======================================================================
%   Section: Quantum Walk
%======================================================================

\section{Quantum walk}
Over the years, randomness proved itself as a useful tool, allowing access to physical systems that are too large to accurately simulate.  By assuming the dynamics of such systems can be modelled as independent events, Markov chains provide insight to the structure of the dynamics.  These ideas can then be cast into the framework of random walk, where the generating Markov matrix describes the weighted, directed graph on which the walk takes place.  

Using the principle of "put quantum in front," one can then analyze what happens when the random dynamics are replaced by unitary dynamics.  This actually poses a little difficulty, as there is no obvious way to make a random walk have unitary dynamics.  In particular, there are many ways that the particle can arrive at one particular vertex in the underlying graph, and thus after arriving at the vertex there is no way to reverse the dynamics.

There are (at least) two ways to get around this.  One continues with the discrete-time structure of a random walk, and keeps track of a "direction" in addition to the position of the particle.  Each step of the walk is then a movement in the chosen direction followed by a unitary update to the direction register.   These Szegedy walks are extremely common in the literature, and go by the name of "discrete-time quantum walk."

Another way to get around the reversibility problem is to generalize the continuous-time model of random walks.  In particular, assuming that the underlying graph is symmetric, we look at the unitary generated by taking the adjacency matrix of the graph as a Hamiltonian.  This is a one-parameter family of unitaries, and thus easily reversible.  The ``continuous-time quantum walk" model is the one we'll be focussing on in this thesis.

%======================================================================
%   Section: Many-body Systems
%======================================================================

\section{Many-body systems}

Everything in nature has many particles, and the reason that physicists are so interested in smaller dynamics is the relatively understandable fact that many-body systems are extremely complicated.  The entire branch of statistical physics was created in an attempt to make a coherent understanding of these large systems, since writing down the dynamics of every particle is impossible in general.  

Along these lines, many models of simple interactions between particles exist in the literature.  As an example, on can consider a lattice of occupation sites, where bosons can sit at any point in the lattice.  Without interactions between the particles, the dynamics are easily understood as decoupled plane waves.  However, by including even a simple energy penalty when multiple particles occupy the same location (i.e. particles don't like to bunch), we no longer have a closed form solution and are required to look at things such as the Bethe ansatz.  

%======================================================================
%   Section: Computational Complexity
%======================================================================

\section{Computational complexity}

While this is a physics thesis, much of my work is focused on understanding the computational power of these physical systems, and as such an understanding of the classification framework is in order.  I should explain some of the motivation behind this area of research, giving a bit of background for the physicist.

These classifications are generally described by languages, or subsets of all possible 0-1 strings.  In particular, given some string $x$, the requisite power in order to determine whether the string belongs to a language or not describes the complexity of the language.  

%%======================================================================
%%   Section: Hamiltonian Complexity
%%======================================================================
%
%\section{Hamiltonian Complexity}
%
%I want at least give an idea of where these results are coming from.  Basically, I should mention the idea of complexity measures related to Hamiltonians, such as area laws, quantum expanders, matrix product states, etc.
%
%Should I explain qPCP? I'm not sure.  I should really just explain that this is the nice intersection between physics and CS, with a large focus on understand physical systems, and how information and computation relate to our physical world.




%======================================================================
%   Section: Layout
%======================================================================

\section{Layout of thesis}

With all of this background in mind, we'd like to give a basic understanding of what this thesis is going to entail.  The underlying theme of this thesis is to understand the computational power of quantum walk, when restricted to various questions.  As such, we will look both at the single and multi-particle cases.  Note that most of this thesis is based on the papers \cite{MPQW}, \cite{momentum_switches}, \cite{BHQMA}, and \cite{xy_model}.  While I am an author on all four papers, Andrew Childs and David Gosset are co-authors on all four papers, while Daniel Nagaj and Mouktik Raha are co-authors on \cite{momentum_switches}.

However, in \chap{mathematical_preliminaries}, we will at first define various terms related to computational complexity that will be of use to us, as well as some lemmas that might be of independent interest.

In \chap{graph_scattering} we will describe single-particle scattering on graphs.  In particular, we will give some simple motivations and understanding of what is going on in a simple case.  This chapter will also describe some basic algorithmic uses for the single particle case.  This paper will include some review of previous papers 
\todo{Cite quantum walk papers}
that I have not written, as well as a broad overview of techniques used in \cite{MPQW} and \cite{momentum_switches}.

At this point, we will transition into understanding the computational power of time evolving according to single- and multi-particle quantum walks.  In particular, \chap{sp_universality} will include a novel proof that quantum walk of a single particle on an exponentially sized graph for polynomial time is universal quantum computing, using techniques slightly different than that of \cite{Childs11}\todo{Find correct citation}.  While this proof has not been submitted as a paper in any journal, it makes use of many of the techniques of \cite{MPQW}.  In \chap{mp_universality}, we extend this result to show that a multi-particle quantum walk with almost any finite-range interaction is universal for quantum computing, the main result of \cite{MPQW}.

With the computational power of time evolved quantum walk, we will want to understand the ground energy problem of the quantum walk.  In particular, \chap{sp_ground}  shows that determining whether the ground energy of a sparse, row-computable graph is above or below some threshold is \QMA-complete, which is work is found in an appendix of the \cite{BHQMA} paper.  As this corresponds exactly to the ground energy problem of a single particle quantum walk on an exponentially large, but specifiable, graph, this shows that the ground energy problem for single-particle quantum walk is \QMA-complete. \chap{mp_ground} then expands on this result, and shows that the ground energy problem for multi-particle quantum walk with bosons on simple graphs is \QMA complete.  While this result follows the proof techniques of \cite{MPQW} and \cite{xy_model}, the extension to arbitrary finite-range interactions for bosons is novel.

With the quantum walk interactions out of the way, \chap{spin_ground} makes use of these results on the multi-particle ground energy problem to study the ground energy problem of various spin systems.  

Finally, \chap{conclusions} concludes with some discussion of these results, along with some avenues for future research.




\end{document}