%======================================================================
%   Zak Webb
%   Ph. D. Thesis
%   Department of Physics and Astronomy
%   University of Waterloo
% 
%   Introduction Chapter
%======================================================================

\documentclass[../thesis-main/thesis-main]{subfiles}
\begin{document}


\chapter{Introduction}
\label{chap:introduction}

``The three-body problem is impossible.''

This statement is well known to any physicist, as it is perhaps the most simple classical dynamics problem for which there is no closed form solution.  There are some caveats, and special cases for which a solution is known, but this lack of a general solution makes this a ``hard'' problem.  However, for short time-periods, we can use numerical methods to approximate the evolution of three-bodies, and thus the question is how hard is this problem.  More generally, if one is working with some particular physical model, how hard is it to compute various attributes about that model.

While such questions have not generally been asked in physics, classifying the computation power of a problem in terms of the necessary resources in order to solve it is a foundational idea in computer science.  The entire field of computational complexity arose in the attempt to classify these problems.  This thesis will attempt to use tools from complexity theory, and apply them to some particular physical models.  Along the way, we will prove that some physical systems are unfeasible to solve in any reasonable time frame, and find some novel systems that can be used as a universal quantum computer.

\section{Hamiltonian Complexity}

This idea for using complexity theoretic ideas in order to study physical models in not novel; in 1982, Barahona \cite{Bar82} showed that computing the ground state of an Ising spin glass in a nonuniform magnetic field is a \NP-hard problem.  If we restrict ourselves to quantum problems, however, things become more recent. In 1999, Alexei Kitaev \cite{KSV02} showed that the $5$-local Hamiltonian problem was \QMA-complete, and thus gave a reason for the difficulty in computing the ground energy of these problems.  Various improvements have shown the \QMA-completeness of similar problems, with a reduced locality to $3$-local\cite{KR03}, $2$-local \cite{KKR04}, and finally $2$-local Hamiltonians with interactions between qubits restricted to a two-dimensional lattice \cite{OT05}.  Other Hamiltonians with various restrictions have also been shown to be \QMA-complete, including Hamiltonians in one dimension \cite{AGIK09,GI09}, frustration-free Hamiltonians \cite{Bra06,GN13}, and stoquastic Hamiltonians (Hamiltonians with no ``sign problem'') \cite{BDOT08,BT09}.

In most of these problems, the proof that a given system is \QMA-complete is done by either reducing the Local-Hamiltonian problem to the problem in question, or else encoding a ``history'' state in the ground state of the model.  In either case, the models need to have some way to encode each problem instance.  This is often done by having a sum of many different Hamiltonians, each varying independently, but there are other ways to encode problem instances.  Some have only a finite-set of models with varying strengths \cite{CM13}, some have a consistent background Hamiltonian with location dependent terms \cite{SV09}. 

In addition to these questions about the computational complexity of various Hamiltonian problems, Hamiltonian complexity also investigates area laws, and entanglement structures.  These other areas of Hamiltonian complexity are not directly related to this thesis, and I will thus avoid their discussion.


%======================================================================
%   Section: Quantum Walk
%======================================================================

\section{Quantum walk}

With the discussion of various physical models, and how they encode complexity, we should briefly describe the main model we will study in this thesis.  In particular, most of this thesis is devoted to studying multi-particle quantum walk, a generalization of quantum walk, which is itself a generalization of random walk.  These models are all defined in terms of some underlying graph on which particles move with a time-independent model.  

In particular, we study the continuous-time model of quantum walk, in which time-evolution is generated by a Hamiltonian equal to the adjacency matrix.  The single-particle model of computation has been very successful, as quantum walk is an intuitive framework for developing quantum algorithms. This framework has lead to examples of exponential speedups over classical computation \cite{CCDFGS03}, as well as optimal algorithms for element distinctness \cite{Amb07} and formula evaluation \cite{FGG08}.  Additionally, it is known that this model is universal for quantum computing \cite{Chi09}.

Unfortunately, multi-particle quantum walk has not been as successful a theory.   It has previously been considered as an algorithmic tool for solving graph isomorphism \cite{gamble2010}, but not has some known limitations \cite{Smi11}. Other previous work on multi-particle quantum walk has focused on two-particle quantum walk \cite{sheridan2006,PA07,BLMS09,PLM10,OBB11,lahini2012,SSVMCRO12} and multi-particle quantum walk without interactions \cite{sheridan2006,PA07,BLMS09,PLM10,OBB11,Rohde11}. 

If we focus on the case of multi-particle quantum walk without interactions, we can see that while non-interacting bosonic quantum walks may be difficult to simulate classically \cite{AA11}, such systems are probably not capable of performing universal quantum computation.  For fermionic systems the situation is even clearer: non-interacting fermions can be efficiently simulated on a classical computer~\cite{TD02}.

However, this lack of theoretical backing has not stopped experimentalists from studying multi-particle walks.  In particular, we have that the two-particle bosonic quantum walk experiments \cite{BLMS09,PLM10,OBB11,SSVMCRO12} have already occurred, although interactions with larger number of particles are likely to be difficult to implement.


%======================================================================
%   Section: Layout
%======================================================================

\section{Layout of thesis}

With all of this background in mind, we'd like to give a basic understanding of what this thesis is going to entail.  The underlying theme of this thesis is to understand the computational power of quantum walk, and its multi-particle generalizations.  Note that most of this thesis is based on the papers \cite{MPQW}, \cite{MomSwitches}, \cite{BHQMA}, and \cite{XYQMA}.  While I am an author on all four papers, Andrew Childs and David Gosset are co-authors on all four papers, while Daniel Nagaj and Mouktik Raha are co-authors on \cite{MomSwitches}.  Additionally, while the proof structures and overall themes are the same between the papers and this thesis, I have extended several of the results from these papers, from improved error bounds over \cite{MPQW} to an extension of the results of \cite{BHQMA} to more general interactions. 

The first portion of this thesis, in \chap{mathematical_preliminaries}, we will cover some basic terms related to computational complexity that a physicist reading this thesis might not be familiar with.  Additionally, some technical lemmas that might be of independent interest and do not fit elsewhere in this thesis are placed here.

The first chapter with content related to quantum walk is \chap{scattering_on_graphs}, we will describe single-particle scattering on graphs.  In particular, we will give some simple motivation, and a broad overview of the the model of graph scattering.  This paper will include some review of previous papers \cite{Chi09,CG12} that I have not written, as well as some technical results from \cite{MPQW}.  We then describe some gadgets that can be utilized via graph scattering in \chap{scattering_gadgets}, which will be used elsewhere in the thesis.  

At this point, we will transition into understanding the computational power of time evolving according to single- and multi-particle quantum walks.  In particular, \chap{SP_universality} will include a novel proof that quantum walk of a single particle on an exponentially sized graph for polynomial time is universal quantum computing, using techniques slightly different than that of \cite{Chi09}.  While this proof has not been submitted as a paper in any journal, it makes use of many of the techniques of \cite{MPQW}.  In \chap{MP_universality}, we extend this result to show that a multi-particle quantum walk with almost any finite-range interaction is universal for quantum computing, the main result of \cite{MPQW}.

With the computational power of time evolved quantum walk, we will want to understand the ground energy problem of the quantum walk.  In particular, \chap{SP_ground} shows that determining whether the ground energy of a sparse, row-computable graph is above or below some threshold is \QMA-complete, which is work is found in an appendix of the \cite{BHQMA} paper.  As this corresponds exactly to the ground energy problem of a single particle quantum walk on an exponentially large, but specifiable, graph, this shows that the ground energy problem for single-particle quantum walk is \QMA-complete. 

\chap{MP_ground} then expands on this result, and shows that the ground energy problem for multi-particle quantum walk with bosons on simple graphs is \QMA-complete.  While this result follows the proof techniques of \cite{MPQW} and \cite{XYQMA}, the extension to arbitrary finite-range interactions for bosons is novel.

Finally, \chap{Conclusions} concludes with some discussion of these results, along with some avenues for future research.


\biblio{}

\end{document}