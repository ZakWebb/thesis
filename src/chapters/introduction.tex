%======================================================================
%   Zak Webb
%   Ph. D. Thesis
%   Department of Physics and Astronomy
%   University of Waterloo
% 
%   Introduction Chapter
%======================================================================

\documentclass[../thesis-main/thesis-main]{subfiles}
\begin{document}


\chapter{Introduction}

Well, I'd like to see where this goes.




%======================================================================
%   Section: Quantum Walk
%======================================================================

\section{Quantum walk}
Over the years, randomness proved itself as a useful tool, allowing access to physical systems that are too large to accurately simulate.  By assuming the dynamics of such systems can be modelled as independent events, Markov chains provide insight to the structure of the dynamics.  These ideas can then be cast into the framework of random walk, where the generating Markov matrix describes the weighted, directed graph on which the walk takes place.  

Using the principle of "put quantum in front," one can then analyze what happens when the random dynamics are replaced by unitary dynamics.  This actually poses a little difficulty, as there is no obvious way to make a random walk have unitary dynamics.  In particular, there are many ways that the particle can arrive at one particular vertex in the underlying graph, and thus after arriving at the vertex there is no way to reverse the dynamics.

There are (at least) two ways to get around this.  One continues with the discrete-time structure of a random walk, and keeps track of a "direction" in addition to the position of the particle.  Each step of the walk is then a movement in the chosen direction followed by a unitary update to the direction register.   These Szegedy walks are extremely common in the literature, and go by the name of "discrete-time quantum walk."

Another way to get around the reversibility problem is to generalize the continuous-time model of random walks.  In particular, assuming that the underlying graph is symmetric, we look at the unitary generated by taking the adjacency matrix of the graph as a Hamiltonian.  This is a one-parameter family of unitaries, and thus easily reversible.  The "continuous-time quantum walk" model is the one we'll be focussing on in this thesis.

%======================================================================
%   Section: Many-body Systems
%======================================================================

\section{Many-body systems}

Everything in nature has many particles, and the reason that physicists are so interested in smaller dynamics is the relatively understandable fact that many-body systems are extremely complicated.  The entire branch of statistical physics was created in an attempt to make a coherent understanding of these large systems, since writing down the dynamics of every particle is impossible in general.  

Along these lines, many models of simple interactions between particles exist in the literature.  As an example, on can consider a lattice of occupation sites, where bosons can sit at any point in the lattice.  Without interactions between the particles, the dynamics are easily understood as decoupled plane waves.  However, by including even a simple energy penalty when multiple particles occupy the same location (i.e. particles don't like to bunch), we no longer have a closed form solution and are required to look at things such as the Bethe ansatz.  

%======================================================================
%   Section: Computational Complexity
%======================================================================

\section{Computational complexity}

While this is a physics thesis, much of my work is focused on understanding the computational power of these physical systems, and as such an understanding of the classification framework is in order.  

These classifications are generally described by languages, or subsets of all possible 0-1 strings.  In particular, given some string $x$, the requisite power in order to determine whether the string belongs to a language or not describes the complexity of the language.  

Basically, I should define what a language is, compare with promise problems, and compare \PP and \NP  with \BQP  and \QMA .

%======================================================================
%   Section: Hamiltonian Complexity
%======================================================================

\section{Hamiltonian Complexity}

I want at least give an idea of where these results are coming from.  Basically, I should mention the idea of complexity measures related to Hamiltonians, such as area laws, quantum expanders, matrix product states, etc.

%======================================================================
%   Section: Notation and requisite mathematics
%======================================================================

\section{Notation and requisite mathematics}

I'm realizing that I'll need to include some information about graphs, positive-semidefinite matrices, etc.  This will probably be the section on necessary terminology and mathematics, but I'm not sure what will go in here.

\begin{lemma}[Nullspace Projection Lemma]
Let $H_A$ and $H_B$ be positive semi-definite matrices.  Suppose that the nullspace, $S$, of $H_A$ is nonempty, and that 
\begin{equation}
  \gamma\big(H_B|_S\big) \geq c > 0 \qquad \text{and} \qquad \gamma(H_A) \geq d > 0.
\end{equation}
Then,
\begin{equation}
  \gamma(H_A + H_B) \geq \frac{c d}{d + \norm{H_B}} .
\end{equation}
\end{lemma}
\begin{proof}
Let $|\psi\rangle$ be a normalized state satisfying 
\begin{equation}
\langle\psi|H_{A}+H_{B}|\psi\rangle=\gamma(H_{A}+H_{B}).
\end{equation}
Let $\Pi_{S}$ be the projector onto the nullspace of $H_{A}$. First suppose that $\Pi_{S}|\psi\rangle=0$, in which case 
\begin{equation}
\langle\psi|H_{A}+H_{B}|\psi\rangle\geq\langle\psi|H_{A}|\psi\rangle\geq\gamma(H_{A})
\end{equation}
and the result follows. On the other hand, if $\Pi_{S}|\psi\rangle\neq0$ then we can write 
\begin{equation}
|\psi\rangle=\alpha|a\rangle+\beta|a^{\perp}\rangle
\end{equation}
with $|\alpha|^{2}+|\beta|^{2}=1$, $\alpha\neq0$, and two normalized states $|a\rangle$ and $|a^{\perp}\rangle$ such that $|a\rangle\in S$ and $|a^{\perp}\rangle\in S^{\perp}$. (If $\beta=0$ then we may choose $|a^{\perp}\rangle$ to be an arbitrary state in $S^{\perp}$ but in the following we fix one specific choice for concreteness.) Note that any state $|\phi\rangle$ in the nullspace of $H_{A}+H_{B}$ satisfies $H_{A}|\phi\rangle=0$ and hence $\langle\phi|a^{\perp}\rangle=0$. Since $\langle\phi|\psi\rangle=0$ and $\alpha\neq0$ we also see that $\langle\phi|a\rangle=0$. Hence any state
\begin{equation}
|f(q,r)\rangle=q|a\rangle+r|a^{\perp}\rangle
\end{equation}
is orthogonal to the nullspace of $H_{A}+H_{B}$, and
\begin{equation}
\gamma(H_{A}+H_{B})=\min_{|q|^{2}+|r|^{2}=1}\langle f(q,r)|H_{A}+H_{B}|f(q,r)\rangle.
\end{equation}

Within the subspace $Q$ spanned by $\ket{a}$ and $\ket{a^\perp}$, note that
\begin{equation}
  H_A|_Q = \begin{pmatrix} w & v^*\\
    v & z\end{pmatrix} \qquad H_B|_Q = \begin{pmatrix} 0 & 0\\
    0 & y \end{pmatrix}
\end{equation}
where $w=\langle a|H_{B}|a\rangle$, $v=\langle a^{\perp}|H_{B}|a\rangle$, $y=\langle a^{\perp}|H_{A}|a^{\perp}\rangle$, and $z=\langle a^{\perp}|H_{B}|a^{\perp}\rangle$, and that we are interested in the smaller eigenvalue of 
\begin{equation}
M = H_A|_Q + H_B|_Q =  \begin{pmatrix}
w & v^{*}\\
v & y+z
\end{pmatrix}.
\end{equation}
Letting $\epsilon_+$ and $\epsilon_-$ be the two eigenvalues of $M$ with $\epsilon_+\geq \epsilon_-$, note that 
\begin{equation}
  \epsilon_+ = \norm{M} \leq \norm{H_A|_Q} + \norm{H_B|_Q} \leq y+ \norm{H_B|_Q} \leq y + \norm{H_B},
\end{equation}
where we have used the Cauchy interlacing theorem to note that $\norm{H_B|_Q} \leq \norm{H_B}$.
Additionally, we have that
\begin{equation}
  \epsilon_+ \epsilon_- = \det (M) = w (y+z) - |v|^2 \geq wy
\end{equation}
where we used the fact that $H_B|_Q$ is positive-semidefinite.  Putting this together, we have that
\begin{equation}
 \gamma(H_A + H_B) = \min_{|q|^{2}+|r|^{2}=1}\langle f(q,r)|H_{A}+H_{B}|f(q,r)\rangle = \epsilon_- \geq \frac{w y}{y + \norm{H_B}}.
\end{equation}
As the right hand side increased monotonically with both $w$ and $y$, and as $w \geq \gamma(H_B|_S) \geq c$ and $y \geq \gamma(H_A) \geq d$, we have
\begin{equation}
  \gamma(H_A + H_B) \geq \frac{c d}{d + \norm{H_B}}
\end{equation}
as required.
\end{proof}

%======================================================================
%   Section: Layout
%======================================================================

\section{Layout of thesis}

Most of these results have been previously published.

Chapter 2 will be taken from momentum switch/universatily paper

chapter 3 will be taken from universality paper

chapter 4 will be taken from universality paper

chapter 5 will be taken from BH-qma paper and the new one

chapter 6 will be taken from BH-qma paper and the new one

chapter 7 will have open questions from several papers.


\end{document}