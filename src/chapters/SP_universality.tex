%======================================================================
%   Zak Webb
%   Ph. D. Thesis
%   Department of Physics and Astronomy
%   University of Waterloo
% 
%   Universality of single-particle scattering
%======================================================================


\documentclass[../thesis-main/thesis-main]{subfiles}
\begin{document}

\chapter{Universality of single-particle scattering}

\section{Finite truncation}

I think I should include theorem 1 here (maybe)

\begin{theorem} Let $\widehat{G}$ be an $(N+m)$-vertex graph.  Let $G$ be the graph obtained from $\widetilde{G}$ by attaching semi-infinite paths to the first $N$ of its vertices, and let $S$ be the corresponding $S$-matrix.  Let $H_G$ be the quantum walk Hamiltonian of equation \missingcite{correct equation}. Let $k\in (-\pi,0)$, $M,L\in \NN$, $j\in [N]$, and 
\begin{equation}
  \ket{\psi^j(0)} = \frac{1}{\sqrt{L}} \sum_{x=M+1}^{M4L} e^{-i k x} \ket{x,j}.
\end{equation}
Let $c_0$ be a constant independent of $L$.  Then, for all $0 \leq t \leq c_0 L$,
\begin{equation}
   \Big\|e^{-i H_G t}\ket{\psi^j(0)} - \ket{\alpha^j(t)}\Big\| = \OO(L^{-1/4})
\end{equation}
where
\begin{equation}
  \ket{\alpha^j(t)} = \frac{1}{\sqrt{L}} e^{-2 i t \cos k} \sum_{x=1}^\infty \sum_{q=1}^N (\delta_{qj} e^{-ikx} R(x- \lfloor 2t \sin k\rfloor ) + S_{qj}(k) e^{ikx} R(-x - \lfloor 2t \sin k\rfloor ) ) \ket{x,q}
\end{equation}
with
\begin{equation}
  R(l) = \begin{cases} 1 & \text{ if } l- M \in [L]\\
    0 & \text{ otherwise.}
    \end{cases}
\end{equation}
\end{theorem}
n this section we prove \thm{singlepart}. The proof is based on (and follows closely) the calculation from the appendix of reference \cite{FGG08}.

Recall from \eq{single_particle_states} that the scattering eigenstates of $H_{G}^{(1)}$ have the
form
\[
\langle x,q|\text{sc}_{j}(k)\rangle=e^{-ikx}\delta_{qj}+e^{ikx}S_{qj}(k)\]
 for each $k\in(-\pi,0)$. 

Before delving into the proof, we first establish that the state $|\alpha^{j}(t)\rangle$ is approximately normalized. This state is not normalized at all times $t$. However, $\langle\alpha^{j}(t)|\alpha^{j}(t)\rangle=1+\O(L^{-1})$, as we now show:
\begin{align*}
\langle\alpha^{j}(t)|\alpha^{j}(t)\rangle & =\frac{1}{L} \sum_{x=1}^{\infty}\left|e^{-ikx}R(x-\left\lfloor 2t\sin k\right\rfloor)+S_{jj}(k)e^{ikx}R(-x-\left\lfloor 2t\sin k\right\rfloor)\right|^{2} \\
 &\quad   +\frac{1}{L}\sum_{q\neq j}\sum_{x=1}^{\infty}|S_{qj}(k)|^{2} R(-x-\left\lfloor 2t\sin k\right\rfloor) \\
 & = \frac{1}{L}\sum_{x=1}^{\infty}\left[R(x-\left\lfloor 2t\sin k\right\rfloor )+R(-x-\left\lfloor 2t\sin k\right\rfloor)\right] \\
 &  \quad +\frac{1}{L}\sum_{x=1}^{\infty}\left(e^{-2ikx}S_{jj}^{\ast}(k)+e^{2ikx}S_{jj}(k)\right)R(x-\left\lfloor 2t\sin k\right\rfloor)R(-x-\left\lfloor 2t\sin k\right\rfloor) \\
 & = 1\! +\! \frac{1}{L}\sum_{x=1}^{\infty}
 	\left(e^{-2ikx}S_{jj}^{\ast}(k)\! +\! e^{2ikx}S_{jj}(k)\right)
   R(x\! -\! \left\lfloor 2t\sin k\right\rfloor)
   R(-x\! -\!\left\lfloor 2t\sin k\right\rfloor)
   \! +\!\O(L^{-1})
\end{align*}
where we have used unitarity of $S$ in the second step.
When it is nonzero, the second term can be written as
\[
\frac{1}{L}\sum_{x=1}^{b}\left(e^{-2ikx}S_{jj}^{\ast}(k)+e^{2ikx}S_{jj}(k)\right)
\]
where $b$ is the maximum positive integer such that $\{-b,b\}\subset\{M+1+\left\lfloor 2t\sin k\right\rfloor,\ldots,M+L+\left\lfloor 2t\sin k\right\rfloor \}$. Performing
the sums, we get
\begin{align*}
\left|\frac{1}{L}\sum_{x=1}^{b}\left(e^{-2ikx}S_{jj}^{\ast}(k)+e^{2ikx}S_{jj}(k)\right)\right| 
&= \frac{1}{L} \left|S_{jj}^{\ast}(k)e^{-2ik} \frac{e^{-2ikb}-1}{e^{-2ik}-1} 
+
S_{jj}(k)e^{2ik} \frac{e^{2ikb}-1}{e^{2ik}-1} \right|\\
 & \leq \frac{2}{L|{\sin k}|} .
\end{align*}

Thus we have $\langle\alpha^{j}(t)|\alpha^{j}(t)\rangle=1+\O(L^{-1})$.

\begin{proof}[Proof of \thm{singlepart}]
Define
\[
|\psi^{j}(t)\rangle=e^{-iH_{G}^{(1)}t}|\psi^{j}(0)\rangle
\]
and write
\[
|\psi^{j}(t)\rangle=|w^{j}(t)\rangle+|v^{j}(t)\rangle
\]
where
\[
|w^{j}(t)\rangle=\int_{-\epsilon}^{\epsilon}\frac{d\phi}{2\pi}e^{-2it\cos\left(k+\phi\right)}\sum_{q=1}^{N}|\text{sc}_{q}(k+\phi)\rangle\langle\text{sc}_{q}(k+\phi)|\psi^{j}(0)\rangle
\]
and $\langle w^{j}(t)|v^{j}(t)\rangle=0$. We take $\epsilon=\frac{\left|\sin k\right|}{2\sqrt{L}}$.
Now \[
\langle\text{sc}_{q}(k+\phi)|\psi^{j}(0)\rangle=\frac{1}{\sqrt{L}}\sum_{x=M+1}^{M+L}\left(e^{i\phi x}\delta_{qj}+e^{-i\left(2k+\phi\right)x}S_{qj}^{\ast}(k+\phi)\right),\]
 so \[
|w^{j}(t)\rangle=|w_{A}^{j}(t)\rangle+\sum_{q=1}^{N}|w_{B}^{q,j}(t)\rangle\]
 where \begin{align*}
|w_{A}^{j}(t)\rangle & = \int_{-\epsilon}^{\epsilon}\frac{d\phi}{2\pi}e^{-2it\cos\left(k+\phi\right)}f(\phi)|\text{sc}_{j}(k+\phi)\rangle\\
|w_{B}^{q,j}(t)\rangle & = \int_{-\epsilon}^{\epsilon}\frac{d\phi}{2\pi}e^{-2it\cos\left(k+\phi\right)}g_{qj}(\phi)|\text{sc}_{q}(k+\phi)\rangle
\end{align*}
with
\begin{align*}
f(\phi) & = \frac{1}{\sqrt{L}}\sum_{x=M+1}^{M+L}e^{i\phi x}\\
g_{qj}(\phi) & = \frac{1}{\sqrt{L}}\sum_{x=M+1}^{M+L}e^{-i\left(2k+\phi\right)x}S_{qj}^{\ast}(k+\phi).\end{align*}
We will see that $\ket{\psi^j(t)} \approx \ket{w^j(t)} \approx \ket{w^j_A(t)} \approx \ket{\alpha^j(t)}$.

Now
\begin{align*}
\langle w_{A}^{j}(t)|w_{A}^{j}(t)\rangle & = \int_{-\epsilon}^{\epsilon}\frac{d\phi}{2\pi}\left|f(\phi)\right|^{2}
 = \frac{1}{L} \int_{-\epsilon}^{\epsilon}\frac{d\phi}{2\pi}\frac{\sin^{2}(\frac{1}{2}L\phi)}{\sin^{2}(\frac{1}{2}\phi)}
 \end{align*}
but
\[
\frac{1}{L} \int_{-\pi}^{\pi}\frac{d\phi}{2\pi}\frac{\sin^{2}(\frac{1}{2}L\phi)}{\sin^{2}(\frac{1}{2}\phi)}=1
\]
and
\begin{align}
\frac{1}{L}\left(\int_{\epsilon}^{\pi}+\int_{-\pi}^{-\epsilon}\right)\frac{d\phi}{2\pi}\frac{\sin^{2}(\frac{1}{2}L\phi)}{\sin^{2}(\frac{1}{2}\phi)} & = \frac{2}{L} \int_{\epsilon}^{\pi}\frac{d\phi}{2\pi}\frac{\sin^{2}(\frac{1}{2}L\phi)}{\sin^{2}(\frac{1}{2}\phi)}\nonumber \\
 & \leq \frac{2}{L}\int_{\epsilon}^{\pi}\frac{d\phi}{2\pi}\frac{\pi^{2}}{\phi^{2}}\nonumber \\
 & \leq \frac{\pi}{L\epsilon}.\label{eq:fbound}
\end{align}
Therefore
\[
1\geq\langle w_{A}^{j}(t)|w_{A}^{j}(t)\rangle\geq1-\frac{\pi}{L\epsilon}.
\]
Similarly,
\[
\langle w_{B}^{qj}(t)|w_{B}^{qj}(t)\rangle=\int_{-\epsilon}^{\epsilon}\frac{d\phi}{2\pi}\frac{\left|S_{qj}(k+\phi)\right|^{2}}{L}\frac{\sin^{2}(\frac{1}{2}L(2k+\phi))}{\sin^{2}(\frac{1}{2}(2k+\phi))},\]
and, using the unitarity of $S$,
\begin{align*}
\sum_{q=1}^{N}\langle w_{B}^{qj}(t)|w_{B}^{qj}(t)\rangle & = \frac{1}{L}\int_{-\epsilon}^{\epsilon}\frac{d\phi}{2\pi}\frac{\sin^{2}(\frac{1}{2}L(2k+\phi))}{\sin^{2}(\frac{1}{2}(2k+\phi))}\\
 & \leq \frac{1}{L} \int_{-\epsilon}^{\epsilon}\frac{d\phi}{2\pi}\frac{1}{\sin^{2}(\frac{1}{2}(2k+\phi))}.
\end{align*}
Now $|{\sin(k+{\phi}/{2}) - \sin k}| \leq {|\phi|}/{2}$ (by the mean value theorem). So
\begin{align*}
\sin^{2}\left(k+\frac{\phi}{2}\right) & \geq \left(\left|\sin k\right|-\left|\frac{\phi}{2}\right|\right)^{2}.
\end{align*}
Since $\epsilon=\frac{\left|\sin k\right|}{2\sqrt{L}}<\left|\sin k\right|$
we then have \begin{align*}
\sum_{q=1}^{N}\langle w_{B}^{qj}(t)|w_{B}^{qj}(t)\rangle & \leq \frac{1}{L} \int_{-\epsilon}^{\epsilon}\frac{d\phi}{2\pi}\frac{4}{\sin^{2}k}\\
 & = \frac{4\epsilon}{\pi L\sin^{2}k}.\end{align*}
 Hence \begin{align*}
\langle w^{j}(t)|w^{j}(t)\rangle & \geq \langle w_{A}^{j}(t)|w_{A}^{j}(t)\rangle-2\left|\sum_{q=1}^{N}\langle w_{A}^{j}(t)|w_{B}^{qj}(t)\rangle\right|\\
 & \geq 1-\frac{\pi}{L\epsilon}-2\left\Vert \sum_{q=1}^{n}|w_{B}^{qj}(t)\rangle\right\Vert \\
 & \geq 1-\frac{\pi}{L\epsilon}-2\sum_{q=1}^{n}\left\Vert |w_{B}^{qj}(t)\rangle\right\Vert \\
 & \geq 1-\frac{\pi}{L\epsilon}-4\sqrt{\frac{\epsilon N}{\pi L\sin^{2}k}},\end{align*}
so
\[
\langle v^{j}(t)|v^{j}(t)\rangle\leq\frac{\pi}{L\epsilon}+4\sqrt{\frac{\epsilon N}{\pi L\sin^{2}k}}
\]
since $\langle v^{j}(t)|v^{j}(t)\rangle+\langle w^{j}(t)|w^{j}(t)\rangle=1$.
Thus
\begin{align*}
\left\Vert |\psi^{j}(t)\rangle-|w_{A}^{j}(t)\rangle\right\Vert  & = \left\Vert |v^{j}(t)\rangle+\sum_{q=1}^{N}|w_{B}^{qj}(t)\rangle\right\Vert \\
 & \leq \left(\frac{\pi}{L\epsilon}+4\sqrt{\frac{\epsilon N}{\pi L\sin^{2}k}}\right)^{\frac{1}{2}}+2\sqrt{\frac{\epsilon N}{\pi L\sin^{2}k}}.
\end{align*}
With our choice $\epsilon=\frac{\left|\sin k\right|}{2\sqrt{L}}$,
we have $\norm{|\psi^{j}(t)\rangle-|w_{A}^{j}(t)\rangle} =\O(L^{-{1}/
{4}})$.

We now show that
\begin{equation}
\left\Vert |w_{A}^{j}(t)\rangle-|\alpha^{j}(t)\rangle\right\Vert =\O(L^{-{1}/{8}}).\label{eq:alpha_w_bound}\end{equation}
Letting \[
P=\sum_{q=1}^{N}\sum_{x=1}^{\infty}|x,q\rangle\langle x,q|\]
be the projector onto the semi-infinite paths, to show equation \eq{alpha_w_bound}
it is sufficient to show that
\begin{equation}
\left\Vert P|w_{A}^{j}(t)\rangle-|\alpha^{j}(t)\rangle\right\Vert = \O(L^{-{1}/{4}})
\label{eq:bound_inside_lines}
\end{equation}
since this implies that
\begin{align*}
\left\Vert P|w_{A}^{j}(t)\rangle\right\Vert  & = \left\Vert |\alpha^{j}(t)\rangle\right\Vert + \O(L^{-{1}/{4}})\\
 & = 1+\O(L^{-{1}/{4}})
\end{align*}
and hence
\begin{align}
\left\Vert \left(1-P\right)|w_{A}^{j}(t)\rangle\right\Vert ^{2} & = \left\Vert |w_{A}^{j}(t)\rangle\right\Vert ^{2}-\left\Vert P|w_{A}^{j}(t)\rangle\right\Vert ^{2}\nonumber \\
 & \leq 1-\left(1+\O(L^{-{1}/{4}})\right)\nonumber \\
 & = \O(L^{-{1}/{4}}).\label{eq:bound_outside_lines}
\end{align}
From the above formula we now see that inequality \eq{bound_inside_lines}
implies \eq{alpha_w_bound}.

Noting that
\[
\frac{1}{\sqrt{L}}R(l)=\int_{-\pi}^{\pi}\frac{d\phi}{2\pi}e^{-i\phi l}f(\phi),
\]
we write
\begin{align}
\langle x,q|\alpha^{j}(t)\rangle & = e^{-2it\cos k}\left(\delta_{qj}e^{-ikx}\int_{-\pi}^{\pi}\frac{d\phi}{2\pi}e^{-i\phi\left(x-\left\lfloor 2t\sin k\right\rfloor \right)}f(\phi)\right.\nonumber\\
&\qquad\qquad\qquad  \left. +S_{qj}(k)e^{ikx}\int_{-\pi}^{\pi}\frac{d\phi}{2\pi}e^{-i\phi\left(-x-\left\lfloor 2t\sin k\right\rfloor \right)}f(\phi)\right).\label{eq:alpha_mat_elements}
\end{align}
On the other hand,
\begin{equation}
\langle x,q|w_{A}^{j}(t)\rangle=\int_{-\epsilon}^{\epsilon}\frac{d\phi}{2\pi}e^{-2it\cos\left(k+\phi\right)}f(\phi)\left(e^{-i\left(k+\phi\right)x}\delta_{qj}+e^{i\left(k+\phi\right)x}S_{qj}(k+\phi)\right).\label{eq:w_a_mat_elements}
\end{equation}
Using equations \eq{alpha_mat_elements} and \eq{w_a_mat_elements}
we can write\[
P|w_{A}^{j}(t)\rangle=|\alpha^{j}(t)\rangle+\sum_{i=1}^{7}|c_{i}^{j}(t)\rangle\]
 where $P|c_{i}^{j}(t)\rangle=|c_{i}^{j}(t)\rangle$ and \begin{align*}
\langle x,q|c_{1}^{j}(t)\rangle & = \delta_{qj}e^{-2it\cos k}e^{-ikx}\int_{-\pi}^{\pi}\frac{d\phi}{2\pi}e^{-i\phi x}f(\phi)\left(e^{2it\phi\sin k}-e^{i\phi\left\lfloor 2t\sin k\right\rfloor }\right)\\
\langle x,q|c_{2}^{j}(t)\rangle & = S_{qj}(k)e^{-2it\cos k}e^{ikx}\int_{-\pi}^{\pi}\frac{d\phi}{2\pi}e^{i\phi x}f(\phi)\left(e^{2it\phi\sin k}-e^{i\phi\left\lfloor 2t\sin k\right\rfloor }\right)\\
\langle x,q|c_{3}^{j}(t)\rangle & = -\delta_{qj}e^{-2it\cos k}e^{-ikx}\left(\int_{\epsilon}^{\pi}+\int_{-\pi}^{-\epsilon}\right)\frac{d\phi}{2\pi}e^{-i\phi x}f(\phi)e^{2it\phi\sin k}\\
\langle x,q|c_{4}^{j}(t)\rangle & = -S_{qj}(k)e^{-2it\cos k}e^{ikx}\left(\int_{\epsilon}^{\pi}+\int_{-\pi}^{-\epsilon}\right)\frac{d\phi}{2\pi}e^{i\phi x}f(\phi)e^{2it\phi\sin k}\\
\langle x,q|c_{5}^{j}(t)\rangle & = \delta_{qj}e^{-ikx}\int_{-\epsilon}^{\epsilon}\frac{d\phi}{2\pi}e^{-i\phi x}f(\phi)\left(e^{-2it\cos\left(k+\phi\right)}-e^{-2it\cos k+2it\phi\sin k}\right)\\
\langle x,q|c_{6}^{j}(t)\rangle & = S_{qj}(k)e^{ikx}\int_{-\epsilon}^{\epsilon}\frac{d\phi}{2\pi}e^{i\phi x}f(\phi)\left(e^{-2it\cos\left(k+\phi\right)}-e^{-2it\cos k+2it\phi\sin k}\right)\\
\langle x,q|c_{7}^{j}(t)\rangle & = e^{ikx}\int_{-\epsilon}^{\epsilon}\frac{d\phi}{2\pi}e^{i\phi x}e^{-2it\cos\left(k+\phi\right)}f(\phi)\left(S_{qj}(k+\phi)-S_{qj}(k)\right).
\end{align*}
We now bound the norm of each of these states:
\begin{align*}
\langle c_{1}^{j}(t)|c_{1}^{j}(t)\rangle & = \sum_{q=1}^{N}\sum_{x=1}^{\infty}\left|\delta_{qj}e^{-2it\cos k}e^{-ikx}\int_{-\pi}^{\pi}\frac{d\phi}{2\pi}e^{-i\phi x}f(\phi)\left(e^{2it\phi\sin k}-e^{i\phi\left\lfloor 2t\sin k\right\rfloor }\right)\right|^{2}\\
 & \leq \sum_{q=1}^{N}\sum_{x=-\infty}^{\infty}\left|\delta_{qj}e^{-2it\cos k}e^{-ikx}\int_{-\pi}^{\pi}\frac{d\phi}{2\pi}e^{-i\phi x}f(\phi)\left(e^{2it\phi\sin k}-e^{i\phi\left\lfloor 2t\sin k\right\rfloor }\right)\right|^{2}\\
 & = \int_{-\pi}^{\pi}\frac{d\phi}{2\pi}\left|f(\phi)\right|^{2}\left|e^{2it\phi\sin k}-e^{i\phi\left\lfloor 2t\sin k\right\rfloor }\right|^{2}\\
 & \leq \int_{-\pi}^{\pi}\frac{d\phi}{2\pi}\left|f(\phi)\right|^{2}\left(2t\phi\sin k-\left\lfloor 2t\sin k\right\rfloor \phi\right)^{2}\\
 & \leq \int_{-\pi}^{\pi}\frac{d\phi}{2\pi}\left|f(\phi)\right|^{2}\phi^{2}
\end{align*}
where we have used the facts that $|{e^{is}-1}|^{2} \leq s^{2}$
for $s\in\mathbb{R}$ and $\left|2t\sin k -\left\lfloor 2t\sin k\right\rfloor \right|<1$. In the above we performed the sum over $x$ using the identity
\[
\sum_{x=-\infty}^{\infty} e^{i(\phi-\tilde{\phi})x}=2\pi\delta(\phi-\tilde{\phi}) \text{ for } \phi,\tilde{\phi}\in (-\pi,\pi).
\]
We use this fact repeatedly in the following calculations.
Continuing, we get
\begin{align*}
\langle c_{1}^{j}(t)|c_{1}^{j}(t)\rangle & \leq \frac{1}{L} \int_{-\pi}^{\pi}\frac{d\phi}{2\pi}\frac{\sin^{2}(\frac{1}{2}L\phi)}{\sin^{2}(\frac{1}{2}\phi)}\phi^{2}\\
 & \le \frac{1}{L} \int_{-\pi}^{\pi}\frac{d\phi}{2\pi}\frac{1}{\sin^{2}(\frac{1}{2}\phi)}\phi^{2}\\
 & \leq \frac{\pi^{2}}{L}
\end{align*}
using the fact that $\sin^{2}({\phi}/{2})\geq{\phi^{2}}/{\pi^{2}}$
for $\phi\in[-\pi,\pi]$. Similarly we bound $\langle c_{2}^{j}(t)|c_{2}^{j}(t)\rangle\leq{\pi^{2}}/{L}$. 

Using equation \eq{fbound} we get 
\begin{align*}
\langle c_{3}^{j}(t)|c_{3}^{j}(t)\rangle & \leq \left(\int_{\epsilon}^{\pi}+\int_{-\pi}^{-\epsilon}\right)\frac{d\phi}{2\pi}\left|f(\phi)\right|^{2}\\
 & \leq \frac{\pi}{L\epsilon}\end{align*}
and similarly for $\langle c_{4}^{j}(t)|c_{4}^{j}(t)\rangle$.
Next, we have
\begin{align*}
\langle c_{5}^{j}(t)|c_{5}^{j}(t)\rangle & \leq \int_{-\epsilon}^{\epsilon}\frac{d\phi}{2\pi}\left|f(\phi)\right|^{2}\left|e^{-2it\cos\left(k+\phi\right)}-e^{-2it\cos k+2it\phi\sin k}\right|^{2}\\
 & \leq \int_{-\epsilon}^{\epsilon}\frac{d\phi}{2\pi}\left|f(\phi)\right|^{2}\left(2t\cos\left(k+\phi\right)-2t\cos k+2t\phi\sin k\right)^{2}\\
 & = \int_{-\epsilon}^{\epsilon}\frac{d\phi}{2\pi}\left|f(\phi)\right|^{2}\left(2t\cos k\left(\cos\phi-1\right)+2t\sin k\left(\phi-\sin\phi\right)\right)^{2}\\
 & \leq \int_{-\epsilon}^{\epsilon}\frac{d\phi}{2\pi}\left|f(\phi)\right|^{2}4t^{2}\phi^{4}\\
 & = \frac{4t^{2}}{L}\int_{-\epsilon}^{\epsilon}\frac{d\phi}{2\pi}\frac{\sin^{2}(\frac{1}{2}L\phi)}{\sin^{2}(\frac{1}{2}\phi)}\phi^{4}\\
 & \leq \frac{4t^{2}}{L}\int_{-\epsilon}^{\epsilon}\frac{d\phi}{2\pi}\pi^{2}\phi^{2}\\
 & = \frac{4\pi}{3L}t^{2}\epsilon^{3}\end{align*}
 and we have the same bound for $|c_{6}^{j}(t)\rangle$. Finally, 
\begin{align*}
\langle c_{7}^{j}(t)|c_{7}^{j}(t)\rangle & \leq \int_{-\epsilon}^{\epsilon}\frac{d\phi}{2\pi}\left|f(\phi)\right|^{2}\sum_{q=1}^{N}\left|S_{qj}(k+\phi)-S_{qj}(k)\right|^{2}.\end{align*}
Now, for each $q\in\{1,\ldots,N\}$,
\[
\left|S_{qj}(k+\phi)-S_{qj}(k)\right| \leq \Gamma |\phi|
\]
where the Lipschitz constant
\[
\Gamma = \max_{q,j\in\{1,\ldots,N\}} \max_{k' \in [-\pi,\pi]}\left|\frac{d}{dk'}S_{qj}(k')\right|
\]
is well defined since each matrix element $S_{qj}(k')$ is a
bounded rational function of $e^{ik'}$, as can be seen from equation \eq{smatrixgamma}. Hence
\begin{align*}
\langle c_{7}^{j}(t)|c_{7}^{j}(t)\rangle & \leq \int_{-\epsilon}^{\epsilon}\frac{d\phi}{2\pi}\left|f(\phi)\right|^{2}N\Gamma^{2}\phi^{2}\\
 & = \frac{N\Gamma^{2}}{L}\int_{-\epsilon}^{\epsilon}\frac{d\phi}{2\pi}\frac{\sin^{2}(\frac{1}{2}L\phi)}{\sin^{2}(\frac{1}{2}\phi)}\phi^{2}\\
 & \leq \frac{N\Gamma^{2}}{L}\int_{-\epsilon}^{\epsilon}\frac{d\phi}{2\pi}\pi^{2}\\
 & = N\Gamma^{2}\frac{\pi\epsilon}{L}.
\end{align*}
Now using the bounds on the norms of each of these states we get
\begin{align*}
\left\Vert P|w_{A}^{j}(t)\rangle-|\alpha^{j}(t)\rangle\right\Vert  & \leq 2\frac{\pi}{\sqrt{L}}+2\sqrt{\frac{\pi}{L\epsilon}}+2\sqrt{\frac{4\pi}{3L}t^{2}\epsilon^{3}}+\sqrt{N\Gamma^{2}\frac{\pi\epsilon}{L}}\\
 & = \O(L^{-{1}/{4}})
\end{align*}
using the choice $\epsilon=\frac{|{\sin p}|}{2\sqrt{L}}$ and the fact that $t = \O(L)$. 
\end{proof}

Note that this analysis assumes that $N = \O(1)$, which is the case in our applications of \thm{singlepart}.

\section{Using scattering for simple computation}


\section{Encoded two-qubit gates}
\section{Single-qubit blocks}
\section{Combining blocks}

It might be worthwhile to include a new proof of universal computation of single-particle scattering in this model.

\end{document}