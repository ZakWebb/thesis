%======================================================================
%   Zak Webb
%   Ph. D. Thesis
%   Department of Physics and Astronomy
%   University of Waterloo
% 
%   Mathematical Preliminaries
%======================================================================


\documentclass[../thesis-main/thesis-main]{subfiles}
\begin{document}

\chapter{Mathematical Preliminaries}

\section{Complexity Theory}

Definition of QMA, languages, etcetera.

\section{Various Mathematical Lemmas}

\subsection{Truncation Lemma}

\begin{lemma}[Truncation Lemma]
\label{lem:trunc}
Let $H$ be a Hamiltonian acting on a Hilbertspace $\mathcal{H}$ and let $\ket{\Phi}\in\mathcal{H}$ be a normalized state. Let
$\mathcal{K}$ be a subspace of $\mathcal{H}$, let $P$ be the projector onto $\mathcal{K}$,
and let $\tilde{H}=PHP$ be the Hamiltonian within this subspace. Suppose
that, for some $T>0$, $W\in\{H,\tilde{H}\}$, $N_0\in\natural$,
and $\delta>0$, we have, for all $0\leq t\leq T$, 
\begin{align*}
e^{-iWt}|\Phi\rangle & = |\gamma(t)\rangle+|\epsilon(t)\rangle \text{ with }
\left\Vert |\epsilon(t)\rangle\right\Vert \leq \delta
\end{align*}
and
\begin{align*}
  (1-P) H^{r}|\gamma(t)\rangle & = 0 \text{ for all } r\in\{0,1,\ldots, N_0-1\}.
\end{align*}
Then, for all $0\leq t \leq T$, 
\[
  \Norm{\left(e^{-iHt}-e^{-i\tilde{H}t}\right)|\Phi\rangle}
  \leq \left(\frac{4e\norm{H}t}{N_0} + 2 \right) 
        \left(\delta + 2^{-N_0}(1+\delta)\right).
\]
\end{lemma}

\begin{proposition}
\label{pro:trunc_prop}Let $H$ be a Hamiltonian acting on a Hilbert
space $\mathcal{H}$, and let $\ket{\Phi}\in\mathcal{H}$ be a normalized state. Let
$\mathcal{K}$ be a subspace of $\mathcal{H}$ such that there exists an $N_0\in\natural$
so that for all $\ket{\alpha}\in \mathcal{K}^{\perp}$ and for all $n\in\{0,1,2,\ldots, N_0-1\}$, $\bra{\alpha}H^{n}\ket{\Phi}=0$.
Let $P$ be the projector onto $\mathcal{K}$ and let $\tilde{H}=PHP$ be
the Hamiltonian within this subspace.  Then
\[
\norm{e^{-it\tilde{H}}\ket{\Phi}-e^{-itH}\ket{\Phi}} \leq 2\left(\frac{e\norm{H}t}{N_0}\right)^{N_0}.
\]
\end{proposition}

\begin{proof}
Define $\ket{\Phi(t)}$ and $\ket{\tilde\Phi(t)}$
as
\[
\ket{\Phi(t)}=e^{-itH}\ket{\Phi}=\sum_{k=0}^{\infty}\frac{(-it)^{k}}{k!}H^{k}\ket{\Phi}\qquad\ket{\tilde\Phi(t)}=e^{-it\tilde{H}}\ket{\Phi}=\sum_{k=0}^{\infty}\frac{(-it)^{k}}{k!}\tilde{H}^{k}\ket{\Phi}.
\]
Note that by assumption, $\tilde{H}^{k}\ket{\Phi}=H^{k}\ket{\Phi}$ for all $k< N_0$, and thus the first $N_0$ terms in the two above sums are equal. Looking at the difference between these two states, we have
\begin{align*}
\norm{\ket{\Phi(t)}-\ket{\tilde\Phi(t)}} & =\Norm{\sum_{k=0}^{\infty}\frac{(-it)^{k}}{k!}\left(H^{k}-\tilde{H}^{k}\right)\ket{\Phi}}\\
 & =\Norm{\sum_{k=0}^{N_0-1}\frac{(-it)^{k}}{k!}\left(H^{k}-\tilde{H}^{k}\right)\ket{\Phi}-\sum_{k=N_0}^{\infty}\frac{(-it)^{k}}{k!}\left(H^{k}-\tilde{H}^{k}\right)\ket{\Phi}}\\
 & \leq\sum_{k=N_0}^{\infty}\frac{t^{k}}{k!}\left(\norm{H}^{k}+\norm{\tilde{H}}^{k}\right) \\
 & \leq 2\sum_{k=N_0}^{\infty}\frac{t^{k}}{k!} \norm{H}^{k}
\end{align*}
where the last step uses the fact that $\norm{\tilde{H}}\leq\norm{P}\norm{H}\norm{P} = \norm{H}$.  Thus for any $c \ge 1$, we have
\begin{align*}
\norm{\ket{\Phi(t)}-\ket{\tilde\Phi(t)}}
 & \leq\frac{2}{c^{N_0}}\sum_{k=N_0}^{\infty}\frac{(ct)^{k}}{k!}\norm{H}^{k}\\
 & \leq\frac{2}{c^{N_0}}\exp(ct\norm{H}).
\end{align*}
We obtain the best bound by choosing $c=N_0/\norm{Ht}$, which gives
\[
  \norm{\ket{\Phi(t)}-\ket{\tilde\Phi(t)}}
  \le 2\left(\frac{e\norm{H}t}{N_0}\right)^{N_0}
\]
as claimed.  (If $c < 1$ then the bound is trivial.)
\end{proof}

\begin{proposition}
\label{pro:hybrid}Let $U_1,\ldots, U_n$ and $V_1,\ldots, V_n$  be unitary operators.  Then for any $\ket{\psi}$,
\begin{equation}
  \Norm{\left(\prod_{i=n}^1 U_i - \prod_{i=n}^1 V_i \right) \ket{\psi}}   \le \sum_{j=1}^n \Bigg\|{(U_j - V_j)\prod_{i=j-1}^1 U_i \ket{\psi}}\Bigg\|.
\end{equation}
\end{proposition}

\begin{proof}
The proof is by induction on $n$.  The case $n=1$ is obvious.  For the induction step, we have \begin{align}   \Norm{\left(\prod_{i=n}^1 U_i - \prod_{i=n}^1 V_i \right) \ket{\psi}}   &= \Norm{\left(\prod_{i=n}^1 U_i - V_n \prod_{i=n-1}^1 U_i             + V_n \prod_{i=n-1}^1 U_i - \prod_{i=n}^1 V_i \right) \ket{\psi}} \\   &\le \Norm{(U_n - V_n) \prod_{i=n-1}^1 U_i \ket{\psi}}       +\Norm{\left(\prod_{i=n-1}^1 U_i - \prod_{i=n-1}^1 V_i \right) \ket{\psi}} \\   &\le \sum_{j=1}^n \Norm{(U_j - V_j)\prod_{i=j-1}^1 U_i \ket{\psi}} \end{align} where the last step uses the induction hypothesis. 
\end{proof}

\begin{proof}[Proof of {\lem{trunc}}]
For $M \in \natural$ write
\begin{align*}
  \norm{(e^{-iHt}-e^{-i\tilde{H}t}) \ket{\Phi}}
  &= \Norm{\left(\left(e^{-iH\frac{t}{M}}\right)^{M} -
     \left(e^{-i\tilde{H}\frac{t}{M}}\right)^{M}\right) \ket{\Phi}} \\
  & \leq \sum_{j=1}^{M} \Norm{\left(e^{-iH\frac{t}{M}} - 
     e^{-i\tilde{H}\frac{t}{M}}\right) e^{-iW\left(j-1\right)\frac{t}{M}}
     \ket{\Phi}} \\
  & \leq \sum_{j=1}^{M} 
    \Norm{\left(e^{-iH\frac{t}{M}} - e^{-i\tilde{H}\frac{t}{M}}\right)
    \left(\ket{\gamma(\tfrac{(j-1)t}{M})} +
          \ket{\epsilon(\tfrac{(j-1)t}{M})}\right)} \\
  & \leq 2M\delta+\sum_{j=1}^{M} 
    \Norm{ \left(e^{-iH\frac{t}{M}} - e^{-i\tilde{H}\frac{t}{M}}\right)
    \frac{\ket{\gamma(\tfrac{(j-1)t}{M})}} 
         {\Norm{\ket{\gamma(\tfrac{(j-1)t}{M})}}}}
    \Norm{\ket{\gamma(\tfrac{(j-1)t}{M})}}\\ 
  & \leq 2M\delta 
    + 2M\left(\frac{e\norm{H}t}{M N_0}\right)^{N_0}(1+\delta)
\end{align*}
where in the second line we have used \propo{hybrid} and in the last step we have used \propo{trunc_prop} and the fact that $\Vert |\gamma(t)\rangle\Vert \leq 1+\delta$.
Now, for some $\eta>1$, choose
\[
  M= \left\lceil \frac{\eta e\norm{H}t}{N_0} \right\rceil
\]
for $0<t\leq T$ to get
\begin{align*}
  \norm{(e^{-iHt}-e^{-i\tilde{H}t}) \ket{\Phi}}
  &\leq 2M\left(\delta + \eta^{-N_0}(1+\delta)\right) \\
  &\leq 2\left(\frac{\eta e\norm{H}t}{N_0} + 1 \right) 
        \left(\delta + \eta^{-N_0}(1+\delta)\right).
\end{align*}
The choice $\eta=2$ gives the stated conclusion.
\end{proof}

Note that it would be slightly better to take a smaller value of $\eta$.  However, this does not significantly improve the final result; the above bound is simpler and sufficient for our purposes.


\subsection{Nullspace Projection Lemma}

\begin{lemma}[Nullspace Projection Lemma]
Let $H_A$ and $H_B$ be positive semi-definite matrices.  Suppose that the nullspace, $S$, of $H_A$ is nonempty, and that 
\begin{equation}
  \gamma\big(H_B|_S\big) \geq c > 0 \qquad \text{and} \qquad \gamma(H_A) \geq d > 0.
\end{equation}
Then,
\begin{equation}
  \gamma(H_A + H_B) \geq \frac{c d}{d + \norm{H_B}} .
\end{equation}
\end{lemma}
\begin{proof}
Let $|\psi\rangle$ be a normalized state satisfying 
\begin{equation}
\langle\psi|H_{A}+H_{B}|\psi\rangle=\gamma(H_{A}+H_{B}).
\end{equation}
Let $\Pi_{S}$ be the projector onto the nullspace of $H_{A}$. First suppose that $\Pi_{S}|\psi\rangle=0$, in which case 
\begin{equation}
\langle\psi|H_{A}+H_{B}|\psi\rangle\geq\langle\psi|H_{A}|\psi\rangle\geq\gamma(H_{A})
\end{equation}
and the result follows. On the other hand, if $\Pi_{S}|\psi\rangle\neq0$ then we can write 
\begin{equation}
|\psi\rangle=\alpha|a\rangle+\beta|a^{\perp}\rangle
\end{equation}
with $|\alpha|^{2}+|\beta|^{2}=1$, $\alpha\neq0$, and two normalized states $|a\rangle$ and $|a^{\perp}\rangle$ such that $|a\rangle\in S$ and $|a^{\perp}\rangle\in S^{\perp}$. (If $\beta=0$ then we may choose $|a^{\perp}\rangle$ to be an arbitrary state in $S^{\perp}$ but in the following we fix one specific choice for concreteness.) Note that any state $|\phi\rangle$ in the nullspace of $H_{A}+H_{B}$ satisfies $H_{A}|\phi\rangle=0$ and hence $\langle\phi|a^{\perp}\rangle=0$. Since $\langle\phi|\psi\rangle=0$ and $\alpha\neq0$ we also see that $\langle\phi|a\rangle=0$. Hence any state
\begin{equation}
|f(q,r)\rangle=q|a\rangle+r|a^{\perp}\rangle
\end{equation}
is orthogonal to the nullspace of $H_{A}+H_{B}$, and
\begin{equation}
\gamma(H_{A}+H_{B})=\min_{|q|^{2}+|r|^{2}=1}\langle f(q,r)|H_{A}+H_{B}|f(q,r)\rangle.
\end{equation}

Within the subspace $Q$ spanned by $\ket{a}$ and $\ket{a^\perp}$, note that
\begin{equation}
  H_A|_Q = \begin{pmatrix} w & v^*\\
    v & z\end{pmatrix} \qquad H_B|_Q = \begin{pmatrix} 0 & 0\\
    0 & y \end{pmatrix}
\end{equation}
where $w=\langle a|H_{B}|a\rangle$, $v=\langle a^{\perp}|H_{B}|a\rangle$, $y=\langle a^{\perp}|H_{A}|a^{\perp}\rangle$, and $z=\langle a^{\perp}|H_{B}|a^{\perp}\rangle$, and that we are interested in the smaller eigenvalue of 
\begin{equation}
M = H_A|_Q + H_B|_Q =  \begin{pmatrix}
w & v^{*}\\
v & y+z
\end{pmatrix}.
\end{equation}
Letting $\epsilon_+$ and $\epsilon_-$ be the two eigenvalues of $M$ with $\epsilon_+\geq \epsilon_-$, note that 
\begin{equation}
  \epsilon_+ = \norm{M} \leq \norm{H_A|_Q} + \norm{H_B|_Q} \leq y+ \norm{H_B|_Q} \leq y + \norm{H_B},
\end{equation}
where we have used the Cauchy interlacing theorem to note that $\norm{H_B|_Q} \leq \norm{H_B}$.
Additionally, we have that
\begin{equation}
  \epsilon_+ \epsilon_- = \det (M) = w (y+z) - |v|^2 \geq wy
\end{equation}
where we used the fact that $H_B|_Q$ is positive-semidefinite.  Putting this together, we have that
\begin{equation}
 \gamma(H_A + H_B) = \min_{|q|^{2}+|r|^{2}=1}\langle f(q,r)|H_{A}+H_{B}|f(q,r)\rangle = \epsilon_- \geq \frac{w y}{y + \norm{H_B}}.
\end{equation}
As the right hand side increased monotonically with both $w$ and $y$, and as $w \geq \gamma(H_B|_S) \geq c$ and $y \geq \gamma(H_A) \geq d$, we have
\begin{equation}
  \gamma(H_A + H_B) \geq \frac{c d}{d + \norm{H_B}}
\end{equation}
as required.
\end{proof}

\end{document}