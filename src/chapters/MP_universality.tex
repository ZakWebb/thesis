%======================================================================
%   Zak Webb
%   Ph.D. Thesis
%   Department of Physics and Astronomy
%   University of Waterloo
% 
%   Universality of multi-particle scattering
%======================================================================


\documentclass[../thesis-main/thesis-main]{subfiles}
\begin{document}

\chapter{Universality of multi-particle scattering}


I should really give a broad overview of the technique.  Maybe not in any great detail, but I should really explain why things are going to go the way they are.



\section{Multi-particle quantum walk}

Now that we have analyzed the single particle quantum walk, we will want to understand how the multi-particle system works together.  Unfortunately, this system is difficult in general to analyze, as the various interactions become intractable.  In fact, we will eventually show that this is as hard as understanding the amplitudahedron, also known as universal quantum computation.


\subsection{Two-particle scattering on an infinite path}

While understanding the interactions of multi-particle interactions on an arbitrary graph is beyond our current understanding, we can simplify the model, and see what we can understand.  Along those lines, we can restrict ourselves to the case where only two particles interact.  Similarly, we can restrict ourselves to understanding their interaction on the most simple infinite graph; namely the infinite path.

As such, let us assume that there is some interaction with finite range between the particles, that depends only on the distance between the particles.  


Here we derive scattering states of the two-particle quantum walk on an infinite path. We write the Hamiltonian in the basis $\ket{x,y}$, where $x$ denotes the location of the first particle and $y$ denotes the location of the second particle, with the understanding that bosonic states are symmetrized and fermionic states are antisymmetrized. The Hamiltonian \eq{dist_ham} can be written as
\begin{equation}
  H^{(2)} = H^{(1)}_x \otimes \II_y + \II_x \otimes H^{(1)}_y + \sum_{x,y\in\ZZ} 
     \mathcal{V}(|x-y|) \, \ket{x,y}\bra{x,y}\label{eq:twoham}
\end{equation}
where $\mathcal{V}$ corresponds to the interaction term $\mathcal{U}$ and (with a slight abuse of notation) the subscript indicates which variable is acted on. Here
\[
  H^{(1)} = \sum_{x\in\ZZ} \ket{x+1}\bra{x} + \ket{x}\bra{x+1}
\]
is the adjacency matrix of an infinite path. Our assumption that $\mathcal{U}$ has finite range $C$ means that $\mathcal{V}(r)= 0$ for $r>C$.  

The scattering states we are interested in provide information about the dynamics of two particles initially prepared in spatially separated wave packets moving toward each other along the path with momenta $k_1\in(-\pi,0) $ and $k_2\in (0,\pi)$.

We derive scattering eigenstates of this Hamiltonian by transforming to the new variables $s = x+y$ and $r = x-y$ and exploiting translation symmetry.  Here the allowed values $(s,r)$ range over the pairs of integers where either both are even or both are odd.  Writing states in this basis as $|s;r\rangle$, the Hamiltonian takes the form
\begin{equation}
  H^{(1)}_s\otimes H^{(1)}_r+ \II_s\otimes \sum_{r\in \ZZ} \mathcal{V}(|r|) \, \ket{r}\bra{r}.
\label{eq:twopart_ham}
\end{equation}
For each $p_1\in (-\pi,\pi)$ and $p_2\in(0,\pi)$ there is a scattering eigenstate $|\mathrm{sc}(p_1;p_2)\rangle$ of the form
\[
\langle s;r |\mathrm{sc}(p_1;p_2)\rangle=e^{-ip_1 s/2} \langle r|\psi(p_1;p_2)\rangle,
\]
where the state $|\psi(p_1;p_2)\rangle$ can be viewed as an effective single-particle scattering state of the Hamiltonian
\begin{equation}
 2\cos\left(\frac{p_1}{2}\right) H^{(1)}_r + \sum_{r\in \ZZ} \mathcal{V}(|r|) \, \ket{r}\bra{r}\label{eq:vr_eqn}
\end{equation}
with eigenvalue $4 \cos( p_1/2) \cos(p_2)$.  For a given $\mathcal{V}$, the state $|\psi(p_1;p_2)\rangle$ can be obtained explicitly by solving a set of linear equations (see for example \cite{Childs_Gosset}). It has the form
\begin{equation}
\langle r|\psi (p_1;p_2)\rangle= \begin{cases}  e^{-i p_2 r} + R(p_1,p_2) e^{i p_2 r} &  \text{if } r \leq -C\\
  	f(p_1,p_2,r) &  \text{if } |r| < C\\
  	T(p_1,p_2) e^{- i p_2 r}  & \text{if } r \geq C\end{cases}
\label{eq:psip1p2}
\end{equation}
for $p_2\in (0,\pi)$. Here the reflection and transmission coefficients $R$ and $T$ and the amplitudes of the scattering state for $|r|<C$ (described by the function $f$) depend on both momenta as well as the interaction $\mathcal{V}$.  With $R$, $T$, and $f$ chosen appropriately, the state $|\mathrm{sc}(p_1;p_2)\rangle$ is an eigenstate of $H^{(2)}$ with eigenvalue $4\cos(p_1/2)\cos(p_2)$.

Since $\mathcal{V}(|r|)$ is an even function of $r$, we can also define scattering states for $p_2\in (-\pi,0)$ by
\[
\langle s;r|\mathrm{sc}(p_1;p_2)\rangle=\langle s;-r|\mathrm{sc}(p_1;-p_2)\rangle.
\]
These other states are obtained by swapping $x$ and $y$, corresponding to interchanging the two particles.

The states $\{|\mathrm{sc}(p_1;p_2)\rangle \colon p_1\in (-\pi,\pi),\,p_2\in(-\pi,0)\cup(0,\pi)\}$ are (delta-function) orthonormal:
\begin{align*}
\langle  \mathrm{sc}(p_1';p_2')|\mathrm{sc}(p_1;p_2)\rangle &= \langle \mathrm{sc}(p_1'; p_2')|\left(\sum_{\text{$r,s$ even}}|r\rangle\langle r| \otimes |s\rangle \langle s| \right)|\mathrm{sc}(p_1;p_2)\rangle\\
&\quad + \langle \mathrm{sc}(p_1'; p_2')|\left(\sum_{\text{$r,s$ odd}}|r\rangle \langle r|\otimes  |s\rangle \langle s|\right)|\mathrm{sc}(p_1;p_2)\rangle\\
&= \sum_{\text{$s$ even}} e^{-i(p_1-p_1') {s}/{2}}\sum_{\text{$r$ even}}\langle \psi(p_1';p_2')|r\rangle\langle r|\psi(p_1;p_2)\rangle \\
& \quad + \sum_{\text{$s$ odd}} e^{-i(p_1-p_1') {s}/{2}}\sum_{\text{$r$ odd}}\langle  \psi(p_1';p_2')|r\rangle\langle r|\psi(p_1;p_2)\rangle\\
&= 2\pi \delta(p_1-p_1') \sum_{r=-\infty}^{\infty}\langle \psi(p_1;p_2')|r\rangle\langle r|\psi(p_1;p_2)\rangle\\
&= 4\pi^2 \delta(p_1-p_1')\delta(p_2-p_2')
\end{align*}
where in the last step we used the fact that $\langle\psi(p_1;p_2')|\psi(p_1;p_2)\rangle=2\pi\delta(p_2-p_2')$.
%\begin{figure}
%\centering
%\begin{tikzpicture}[label distance= -6pt,
%    verts/.style={circle,draw=black,fill=black,inner sep=.5pt,minimum size=0pt},
%    dots/.style={circle,fill=black,inner sep=.25pt,minimum width=0pt}]
%  \draw (0,0) -- (7,0);
%
%  
%  \draw (3.15,.15) -- (3,.15) -- (3,.6) -- (1.5,.6) 
%     -- (1.5,.15) -- (1.35,.15);
%  
%  \draw (3.85,.15) -- (4,.15) -- (4,.6) -- (5.5,.6) 
%     -- (5.5,.15) -- (5.65,.15);
%     
%  \node at (2.25,.8) {$\leftarrow k_1$};
%  \node at (4.75,.8) {$k_2\rightarrow$};
%
%  \node at (8.5,.5) {$\times \; (T\pm R)$};
%
%  \node at (10,.5)[label=right:$\frac{D}{2|{\sin k_1}|+ 2|{\sin k_2}|}$]{$t>$};
%  
%
%  \foreach \x in {.1,.25,...,6.9}
%  \node at (\x ,0) [verts] {};
%  
%  \begin{scope}[yshift=2.5cm]
%    \draw (0,0) -- (7,0);
%
%    \draw[xshift=-.75cm] (3.15,.15) -- (3,.15) -- (3,.6) -- (1.5,.6) 
%       -- (1.5,.15) -- (1.35,.15);
%  
%    \draw[xshift=.75cm] (3.85,.15) -- (4,.15) -- (4,.6) -- (5.5,.6) 
%       -- (5.5,.15) -- (5.65,.15);
% 
%    \draw [|-|] (.75,-.2) to node[below] {$D$} (6.25,-.2);
%    
%    \node at (1.5,.8) {$k_2\rightarrow$};
%    \node at (5.5,.8) {$\leftarrow k_1$};
%
%    \node at (10,.5)[label=right:$0$] {$t=$};
%
%    \foreach \x in {.1,.25,...,6.9}
%    \node at (\x ,0) [verts] {};
%
%  \end{scope}
%
%  \foreach \xsh in {-0.45cm, 7.15cm}{
%  \foreach \ysh in {0cm, 2.5cm}{
%    \begin{scope}[xshift=\xsh,yshift=\ysh]
%      \node at (0,0) [dots]{};
%      \node at (0.15,0) [dots] {};
%      \node at (0.3,0) [dots]{};
%    \end{scope}
%  }}
%
%\end{tikzpicture}
%\caption{Scattering of two particles on an infinite path.}
%\label{fig:wte}
%\end{figure}
%	
To construct bosonic or fermionic scattering states, we symmetrize or antisymmetrize as follows. For $p_1\in (-\pi,\pi)$ and $p_2\in (0,\pi)$, we define
\[
  \ket{\mathrm{sc}(p_1;p_2)}_\pm = \frac{1}{\sqrt2}(\ket{\mathrm{sc}(p_1;p_2)} \pm \ket{\mathrm{sc}(p_1;-p_2)}).
\]
Then
\begin{align}
    \braket{s;r}{\mathrm{sc}(p_1;p_2)}_\pm
      &= \frac{1}{\sqrt{2}}e^{-i p_1 s/2} \begin{cases}  e^{-i p_2 r} \pm e^{i\theta_{\pm}(p_1,p_2)} e^{i p_2 r} &  \text{if } r \leq -C\\
  	f(p_1,p_2,r) \pm f(p_1,p_2,-r) & \text{if }  |r| < C\\
  	e^{i\theta_{\pm}(p_1,p_2)}e^{- i p_2 r} \pm e^{i p_2 r}  & \text{if } r \geq C\end{cases}
\label{eq:symscatter}
\end{align}
where $\theta_{\pm}(p_1,p_2)$ is a real function defined through
\begin{equation}
e^{i\theta_{\pm}(p_1,p_2)}= T(p_1,p_2)\pm R(p_1,p_2). \label{eq:delta_pm}
\end{equation}
Note that $|T\pm R| = 1$; this follows from the potential $\mathcal{V}(|r|)$ being even in $r$ and from unitarity of the S-matrix.  These eigenstates allow us to understand what happens when two particles with momenta $k_1\in(-\pi,0)$ and $k_2\in(0,\pi)$ move toward each other. Here $p_1=-k_1-k_2$ and $p_2=(k_2-k_1)/2$.  Recall (from the main text of the paper) that we defined $e^{i\theta}$ to be the phase acquired by the two-particle wavefunction when $k_1=-{\pi}/{2}$ and $k_2={\pi}/{4}$ ($\theta$ depends implicitly on the interaction $\mathcal{V}$ and the particle type), so $\theta=\theta_\pm ({\pi}/{4},{3\pi}/{8})$ .

For $|r|\geq C$ the scattering state is a sum of two terms, one corresponding to the two particles moving toward each other and one corresponding to the two particles moving apart after scattering. The outgoing term has a  phase of $T\pm R$ relative to the incoming term (as depicted in \fig{wte}). This phase arises from the interaction between the two particles.

For example, consider the Bose-Hubbard model, where $\mathcal{V}(|r|) = U\delta_{r,0}$. Here $C=0$ and $T=1+R$.  In this case the scattering state $|\mathrm{sc}(p_1;p_2)\rangle_+$ is
\[
\langle x,y|\mathrm{sc}(p_1;p_2)\rangle_+=\frac{1}{\sqrt{2}}e^{-ip_1 \left(\frac{x+y}{2}\right)}\left(e^{ip_2 |x-y|}+e^{i\theta_+(p_1,p_2)}e^{-ip_2 |x-y|}\right).
\]
The first term describes the two particles moving toward each other and the second term describes them moving away from each other. To solve for the applied phase $e^{i\theta_+(p_1,p_2)}$ we look at the eigenvalue equation for $|\psi(p_1;p_2)\rangle$ at $r=0$. This gives
\[
  R(p_1,p_2) =- \frac{U}{U - 4i\cos({p_1}/{2})\sin(p_2)}.
\]
So for the Bose-Hubbard model,
\[
  e^{i \theta_{+} (p_1,p_2)} = T(p_1,p_2) + R(p_1,p_2) = - \frac{ U + 4 i \cos({p_1}/{2}) \sin(p_2)}{U - 4 i \cos({p_1}/{2}) \sin(p_2)} =  \frac{2 \left(\sin(k_2) - \sin(k_1)\right) - i U}{2 \left(\sin(k_2) - \sin(k_1)\right) + i U}.
\]
For example, if $U = 2+\sqrt{2}$ then two particles with momenta $k_1 =-{ \pi}/{2}$ and $k_2={\pi}/{4}$ acquire a phase of $e^{-i\pi/2}= -i$ after scattering.

For a multi-particle quantum walk with nearest-neighbor interactions, $\mathcal{V}(|r|)=U\delta_{|r|,1}$ and $C=1$.  In this case the eigenvalue equations for $|\psi(p_1;p_2)\rangle$ at $r=-1$, $r=1$, and $r=0$ are
\begin{align*}
 4 \cos\left(\frac{p_1}{2}\right)  \cos(p_2) ( e^{i p_2} + R(p_1,p_2) e^{-i p_2} ) &= U ( e^{i p_2} + R(p_1,p_2) e^{-i p_2}) \\
& \quad + 2\cos\left(\frac{p_1}{2}\right) \left( e^{2i p_2} + R(p_1,p_2) e^{-2i p_2}+f(p_1,p_2,0)\right) \\
 4 \cos\left(\frac{p_1}{2}\right)  \cos(p_2) T(p_1,p_2) e^{-ip_2} & =UT(p_1,p_2)e^{-ip_2}\\
& \quad +2\cos \left(\frac{p_1}{2}\right)\left(f(p_1,p_2,0)+T(p_1,p_2)e^{-2ip_2}\right)\\
2 \cos(p_2) f(p_1,p_2,0) &=T(p_1,p_2)e^{-ip_2}+e^{ip_2}+R(p_1,p_2)e^{-ip_2},
\end{align*}
respectively.

Solving these equations for $R$, $T$, and $f(p_1,p_2,0)$, we can construct the corresponding scattering states for bosons, fermions, or distinguishable particles (for more on the last case, see \sec{distinguishable}). Unlike the case of the Bose-Hubbard model, we may not have $1+R=T$. For example, when $U=-2-\sqrt{2}$, $p_1={\pi}/{4}$, and $p_2={3\pi}/{8}$, we get $R=0$ and $T=i$ (see \sec{distinguishable}).


%%%%%%%%%%%%%%%%%
%  Finite Truncation

\subsection{Finite truncation}


\begin{theorem}
\label{thm:twopart}Let $H^{(2)}$ be a two-particle Hamiltonian of the form \eq{twoham} with interaction range at most $C$, i.e., $\mathcal{V}(|r|)=0$ for all $|r|>C$. Let $\theta_{\pm}(p_1,p_2)$ be given by equation \eq{delta_pm}. Define $\theta=\theta_{\pm}({\pi}/{4},{3\pi}/{8})$. Let $L\in\NN^+$, let $M\in\{C+1,C+2,\ldots\}$, and define
\begin{align*}
|\chi_{z,k}\rangle & =  \frac{1}{\sqrt{L}}\sum_{x=z-L}^{z-1}e^{ikx}|x\rangle\\
|\psi(0)\rangle & =  \frac{1}{\sqrt{2}}\left(|\chi_{-M,-\frac{\pi}{2}}\rangle|\chi_{M+L+1,\frac{\pi}{4}}\rangle 
	\pm |\chi_{M+L+1,\frac{\pi}{4}}\rangle|\chi_{-M,-\frac{\pi}{2}}\rangle\right).
\end{align*}
Let $c_{0}$ be a constant independent of $L$. Then, for all $0\leq t\leq c_{0}L$, we have
\[
\left\Vert e^{-iH^{(2)}t}|\psi(0)\rangle-|\alpha(t)\rangle\right\Vert =\O(L^{-{1}/{8}}),
\]
where 
\begin{equation}
|\alpha(t)\rangle=\sum_{x,y}a_{xy}(t)|x,y\rangle,
\label{eq:alpha}
\end{equation}
$a_{xy}(t)=\pm a_{yx}(t)$, and, for $x\leq y$, 
\begin{align}
a_{xy}(t) & =  \frac{1}{\sqrt{2}L}e^{-\sqrt{2}it}\left[e^{-i \pi x/2} e^{i \pi y/4} F(x,y,t) 
    \pm e^{i\theta} e^{i \pi x/4} e^{-i \pi y/2} F(y,x,t) \right]
\label{eq:a_xy}
\end{align}
where
\begin{align*}
F(u,v,t) & =  \begin{cases}
	1 & \text{if }u-2 \lfloor t \rfloor\in\{-M-L,\ldots,-M-1\}\text{ and }v+2\left\lfloor \frac{t}{\sqrt{2}}
	\right\rfloor \in\{M+1,\ldots,M+L\}\\
	0 & \text{otherwise.}\end{cases}
\end{align*}
\end{theorem}

In this section we prove \thm{twopart}. The main proof appears in \sec{twopartproof}, relying on several technical lemmas proved in \sec{techlem}. The proof follows the method used in the single-particle case, which is based on the calculation from the appendix of reference \cite{FGG08}.

Recall from \eq{symscatter} that for each $p_{1}\in(-\pi,\pi)$ and $p_{2}\in(0,\pi)$ there is an eigenstate $|\text{sc}(p_{1};p_{2})\rangle_{\pm}$ of $H^{(2)}$ of the form 
\begin{align}
  \langle x,y|\text{sc}(p_{1};p_{2})\rangle_\pm &=
    \frac{e^{-ip_{1}\left(\frac{x+y}{2}\right)}}{\sqrt{2}} 
    \begin{cases} e^{-i p_2 (x-y)} \pm  e^{i \theta_{\pm}(p_1,p_2)} e^{i p_2 (x - y)} & \text{if } x - y \leq -C\\
     e^{- i p_2 (x-y)}e^{i\theta_{\pm}(p_1,p_2)} \pm e^{i p_2 (x-y)} & \text{if }x-y \geq C\\
    f(p_1,p_2,x-y) \pm f(p_1,p_2,y-x)& \text{if }|x-y|< C \end{cases}\label{eq:sc}
\end{align}
where
\begin{align*}
e^{i \theta_\pm (p_{1},p_{2})} & =  T(p_1,p_2) \pm R(p_1,p_2),
\end{align*}
$C$ is the range of the interaction, $T$ and $R$ are the transmission
and reflection coefficients of the interaction at the chosen momentum, $f$
describes the amplitudes of the scattering state within the interaction range, and the $\pm$ depends
on the type of particle ($+$ for bosons, $-$ for fermions).  The state
$|\text{sc}(p_{1};p_{2})\rangle_\pm$ satisfies 
\[
H^{(2)}|\text{sc}(p_{1};p_{2})\rangle_{\pm} = 4\cos \frac{p_{1}}{2}\cos p_{2} |\text{sc}(p_{1};p_{2})\rangle_\pm
\]
and is delta-function normalized as
\begin{equation}
_{\pm}\langle\text{sc}(p_{1}';p_{2}')|\text{sc}(p_{1};p_{2})\rangle_{\pm}=
  4 \pi^{2} \delta(p_{1} - p_{1}') \delta(p_{2} - p_{2}').
\label{eq:del_func_norm}
\end{equation}

\begin{proof}
Expand $|\psi(0)\rangle$ in the basis of eigenstates of the Hamiltonian
to get
\begin{align*}
|\psi(t)\rangle & = e^{-iH^{(2)}t}|\psi(0)\rangle = |\psi_{1}(t)\rangle+|\psi_{2}(t)\rangle
\end{align*}
where 
\[
  \ket{\psi_{1}(t)} = \iint_{D_{\epsilon}} \frac{d\phi_{1}d\phi_{2}} 
    {4\pi^{2}} e^{-it 4\cos(\frac{p_{1}}{2}+\frac{\phi_{1}}{2})
      \cos(p_{2} + \phi_{2})}|\text{sc}(p_{1}+\phi_{1};p_{2}+ 
    \phi_{2})\rangle_{\pm} \left({_{\pm}\langle\text{sc}(p_{1}+\phi_{1};p_{2}+\phi_{2})|\psi(0)\rangle}\right)
\]
with $D_{\epsilon}=\left[-\epsilon,\epsilon\right]\times\left[-\epsilon,\epsilon\right]$, $p_{1}= {\pi}/{2}-{\pi}/{4}=
{\pi}/{4}$, $p_{2}=({\pi}/{2} +
{\pi}/{4})/2={3\pi}/{8}$,
and with $|\psi_{2}(t)\rangle$ orthogonal to $|\psi_{1}(t)\rangle$.
We take $\epsilon=a/\sqrt{L}$ for some constant $a$. Using equation \eq{sc}
we get 
\[
  \ket{\psi_{1}(t)}=|\psi_{A}(t)\rangle\pm|\psi_{B}(t)\rangle
\]
where
\begin{align}
  |\psi_{A}(t)\rangle & = \iint_{D_{\epsilon}} \frac{d\phi_{1}d\phi_{2}}
    {4\pi^{2}} e^{-it 4\cos(\frac{\pi}{8}+\frac{\phi_{1}}{2})
    \cos(\frac{3\pi}{8}+\phi_{2})}
     A(\phi_{1},\phi_{2})|\text{sc}(\tfrac{\pi}{4}+\phi_{1};
     \tfrac{3\pi}{8}+\phi_{2})\rangle_{\pm} \label{eq:psiA} \\
 \ket{\psi_{B}(t)} & = \iint_{D_{\epsilon}} \frac{d\phi_{1}d\phi_{2}}
    {4\pi^{2}} e^{-it 4\cos(\frac{\pi}{8}+\frac{\phi_{1}}{2})
    \cos(\frac{3\pi}{8}+\phi_{2})}  e^{-i\theta_{\pm}(\tfrac{\pi}{4}+\phi_{1},\tfrac{3\pi}{8}+\phi_{2} )}B(\phi_{1},\phi_{2},\tfrac{3\pi}{8})
 	|\text{sc}(\tfrac{\pi}{4}\! +\!\phi_{1};\tfrac{3\pi}{8}\! +\!\phi_{2})\rangle_\pm \nonumber 
\end{align}
with
\begin{align}
A(\phi_{1},\phi_{2}) & = \frac{1}{L}\sum_{x=-(M+L)}^{-(M+1)}\sum_{y=M+1}^{M+L} 
    e^{i\phi_{1}\frac{x+y}{2}}e^{i\phi_{2}\left(x-y\right)}\label{eq:A}\\
B(\phi_{1},\phi_{2},k) & = \frac{1}{L}\sum_{x=-(M+L)}^{-(M+1)}
    \sum_{y=M+1}^{M+L}e^{i\phi_{1}\frac{x+y}{2}} 	
    e^{i\left(\phi_{2}+2 k\right)\left(y-x\right)}.\nonumber 
\end{align}
 Using the delta-function normalization of the scattering states (equation
\eq{del_func_norm}) we get 
\begin{align*}
\langle\psi_{B}(t)|\psi_{B}(t)\rangle & = \iint_{D_{\epsilon}}\frac{d\phi_{1}d\phi_{2}}{4\pi^{2}}\left|B(\phi_{1},\phi_{2},\tfrac{3\pi}{8})\right|^{2}\\
 & \leq \frac{16\pi^{2}}{L^{2}\epsilon^{2}}
\end{align*}
by \lem{Let--and} (as long as $\epsilon<{3\pi}/{8}$, which holds for $L$ sufficiently large).
Similarly, 
\begin{align*}
1 &\geq\langle\psi_{A}(t)|\psi_{A}(t)\rangle \\
&= \iint_{D_{\epsilon}}\frac{d\phi_{1}d\phi_{2}}{4\pi^{2}}\left|A(\phi_{1},\phi_{2})\right|^{2}\\
 & \geq 1-\frac{4\pi}{L\epsilon}
\end{align*}
(from the first two facts in \lem{Let--and}) and therefore
\begin{align*}
\langle\psi_{1}(t)|\psi_{1}(t)\rangle & = \langle\psi_{A}(t)|\psi_{A}(t)\rangle+\langle\psi_{B}(t)|\psi_{B}(t)\rangle+\langle\psi_{A}(t)|\psi_{B}(t)\rangle+\langle\psi_{B}(t)|\psi_{A}(t)\rangle\\
 & \geq 1-\frac{4\pi}{L\epsilon}-2\left|\langle\psi_{A}(t)|\psi_{B}(t)\rangle\right|\\
 & \geq 1-\frac{4\pi}{L\epsilon}-2\left|\langle\psi_{A}(t)|\psi_{A}(t)\rangle\right|^{\frac{1}{2}}\left|\langle\psi_{B}(t)|\psi_{B}(t)\rangle\right|^{\frac{1}{2}}\\
 & \geq 
 1-\frac{12\pi}{L\epsilon}.
\end{align*}

Hence 
\[
  \langle\psi_{2}(t)|\psi_{2}(t)\rangle\leq\frac{12\pi}{L\epsilon}
\]
since 
\[
\langle\psi(t)|\psi(t)\rangle=\langle\psi_{1}(t)|\psi_{1}(t)\rangle+\langle\psi_{2}(t)|\psi_{2}(t)\rangle=1.
\]
Thus
\begin{align*}
\left\Vert \,|\psi(t)\rangle-|\psi_{A}(t)\rangle\right\Vert  & = \left\Vert \,|\psi_{B}(t)\rangle+|\psi_{2}(t)\rangle\right\Vert \\
 & \leq \left\Vert \,|\psi_{B}(t)\rangle\right\Vert +\left\Vert \,|\psi_{2}(t)\rangle\right\Vert \\
 & \leq \frac{4\pi}{L\epsilon}+\sqrt{\frac{12\pi}{L\epsilon}}.
\end{align*}
Now 
\begin{align*}
\left\Vert \,|\psi(t)\rangle-|\alpha(t)\rangle\right\Vert  & \leq \left\Vert \,|\psi(t)\rangle-|\psi_{A}(t)\rangle\right\Vert +\left\Vert \,|\psi_{A}(t)\rangle-|\alpha(t)\rangle\right\Vert \\
 & \leq \frac{4\pi}{L\epsilon}+\sqrt{\frac{12\pi}{L\epsilon}}+\left\Vert \,|\psi_{A}(t)\rangle-|\alpha(t)\rangle\right\Vert \\
 & = \O(L^{-{1}/{4}})+\left\Vert \,|\psi_{A}(t)\rangle-|\alpha(t)\rangle\right\Vert 
\end{align*}
using our choice $\epsilon=a/\sqrt{L}$. To complete the
proof, we now show that the second term in this expression is bounded by $\O(L^{-{1}/{8}})$.
\begin{lemma}
With $|\psi_{A}(t)\rangle$ and $|\alpha(t)\rangle$ defined through
equations \eq{psiA} and \eq{alpha}, with $t\leq c_{0}L$ (for some constant $c_{0}$), 
\[
\left\Vert \,|\psi_{A}(t)\rangle-|\alpha(t)\rangle\right\Vert = \O(L^{-{1}/{8}}).
\]
\end{lemma}
\begin{proof}
To simplify matters, note that both $|\psi_{A}(t)\rangle$ and $|\alpha(t)\rangle$ are either symmetric or anti-symmetric (i.e., $\langle x,y|\alpha(t)\rangle=\pm\langle y,x|\alpha(t)\rangle$ and $\langle x,y|\psi_{A}(t)\rangle=\pm\langle y,x|\psi_{A}(t)\rangle$).  Taking $C$ to be the maximum range of the interaction in our Hamiltonian, we have
\[
\left\Vert \,|\psi_{A}(t)\rangle-|\alpha(t)\rangle\right\Vert \leq2\left\Vert P_{1}|\psi_{A}(t)\rangle-P_{1}|\alpha(t)\rangle\right\Vert  + \left\Vert P_2 \ket{\psi_A(t)}\right\Vert + \left \Vert P_2 \ket{\alpha(t)}\right\Vert,
\]
where
\[
     P_{1}=\sum_{y-x \geq C}|x,y\rangle\langle x,y| 
     \qquad P_2 = \sum_{|x-y| < C} \ket{x,y}\bra{x,y}.
\]

Now, for $y-x\geq C$, 
\begin{align*}
\langle x,y|\psi_{A}(t)\rangle & = \iint_{D_{\epsilon}}
  \frac{d\phi_{1}d\phi_{2}}{4\pi^{2}}
  e^{-it 4\cos(\frac{\pi}{8}+\frac{\phi_{1}}{2})
        \cos(\frac{3\pi}{8}+\phi_{2})} A(\phi_{1},\phi_{2})\frac{e^{-i\left(\frac{\pi}{4}+\phi_{1}\right)
       \left(\frac{x+y}{2}\right)}}{\sqrt{2}} \\
 & \qquad \left(e^{i\left(\frac{3\pi}
       {8}+\phi_{2}\right)\left(y-x\right)}\pm e^{-i\left(\frac{3\pi}{8}
       +\phi_{2}\right)
     \left(y-x\right)+
     i\theta_{\pm}(\frac{\pi}{4}+\phi_{1},\frac{3\pi}{8}+\phi_{2})}
      \right)\\
 & = \iint_{D_{\epsilon}}\frac{d\phi_{1}d\phi_{2}}{4\pi^{2}}
      \bigg[\frac{1}{\sqrt{2}} e^{-it 4\cos(\frac{\pi}{8} + \frac{\phi_{1}}{2}) 
      \cos(\frac{3\pi}{8}+\phi_{2})} A(\phi_{1},\phi_{2}) \\
&    \qquad \left( e^{-i\pi x/2} e^{i\pi y/4} e^{-i\phi_{1}\left(\frac{x+y}{2}\right)}
 	 e^{i\phi_{2}\left(y-x\right)}\right.\\
& \qquad\quad\left.\pm  e^{i \pi x/4} e^{-i\pi y/2} e^{-i\phi_{1}
      \left(\frac{x+y}{2}\right)}
 	 e^{-i\phi_{2}\left(y-x\right)} e^{i \theta_{\pm}\left(\frac{\pi}{4} 
             + \phi_1 , \frac{3\pi}{8} + 
 	 \phi_2\right)} \right)\bigg].
\end{align*}

From \lem{a_xy}, for $x\leq y$, the state $\ket{\alpha(t)}$ takes the form
\begin{align*}
\langle x,y|\alpha(t)\rangle & = \frac{1}{\sqrt{2}}e^{-it\sqrt{2}}\left[ e^{-i\pi x /2}e^{i \pi y/4}  
	\left(\iint_{D_{\pi}}\frac{d\phi_{1}d\phi_{2}}{4\pi^{2}}\right.\right.\\
 &	\qquad  \left.\left. A(\phi_{1},\phi_{2}) 		
	e^{- i\phi_{1}\left(-\left\lfloor t\right\rfloor +\left\lfloor \frac{t}{\sqrt{2}}\right\rfloor +\frac{x+y}
	{2}\right)}e^{-2i\phi_{2}\left(-\left\lfloor t\right\rfloor -\left\lfloor \frac{t}{\sqrt{2}}\right\rfloor +\frac{x-y}
	{2}\right)}\right)\right.\\
 & \quad \pm e^{i\theta} e^{i\pi x/4}e^{-i \pi y/2} 
 	\left(\iint_{D_{\pi}}\frac{d\phi_{1}d\phi_{2}}{4\pi^{2}}\right.\\
 & \qquad \left.\left. A(\phi_{1},\phi_{2})
 	e^{-i\phi_{1}\left(-\left\lfloor t\right\rfloor +\left\lfloor \frac{t}{\sqrt{2}}\right\rfloor +\frac{x+y}{2}\right)}
 	e^{-2i\phi_{2}\left(-\left\lfloor t\right\rfloor -\left\lfloor \frac{t}{\sqrt{2}}\right\rfloor +\frac{y-x}
 	{2}\right)}\right)\right],
 \end{align*}
 where $D_{\pi}=[-\pi,\pi]\times[-\pi,\pi]$. Using these expressions for $\ket{\psi_A(t)}$ and $\ket{\alpha(t)}$,
we now write 
\[
P_{1}|\psi_{A}(t)\rangle-P_{1}|\alpha(t)\rangle=\pm |e_{1}(t)\rangle+|e_{2}(t)\rangle \pm|f_{1}(t)\rangle+|f_{2}(t)\rangle\pm|g_{1}(t)\rangle+|g_{2}(t)\rangle\pm|h(t)\rangle
\]
where each term in the above equation is supported only on states
$|x,y\rangle$ such that $y-x \geq C$, and (for $y - x \geq C$)
\begin{align*}
\langle x,y|e_{1}(t)\rangle & = 
	\frac{e^{i\theta}}{\sqrt{2}} e^{-it\sqrt{2}}
	e^{i \pi x/4}e^{-i\pi y/2}\iint_{D_{\pi}}\frac{d\phi_{1}d\phi_{2}}{4\pi^{2}}
	A(\phi_{1},\phi_{2}) \bigg[e^{-i\phi_{1}\left(-t+\frac{t}{\sqrt{2}}+\frac{x+y}{2}\right)}\\
&  \qquad\qquad e^{-2i\phi_{2}\left(-t-\frac{t}{\sqrt{2}}+\frac{y-x}{2}\right)}
	-e^{-i\phi_{1}\left(-\left\lfloor t\right\rfloor +\left\lfloor \frac{t}{\sqrt{2}}\right\rfloor +\frac{x+y}{2}
 		\right)}e^{-2i\phi_{2}\left(-\left\lfloor t\right\rfloor -\left\lfloor \frac{t}{\sqrt{2}}\right\rfloor 
 		+\frac{y-x}{2}\right)}\bigg]\\
\langle x,y|e_{2}(t)\rangle & = 
	\frac{1}{\sqrt{2}}e^{-it\sqrt{2}}e^{-i \pi x/2}e^{i\pi y/4}
	\iint_{D_{\pi}}\frac{d\phi_{1}d\phi_{2}}{4\pi^{2}}A(\phi_{1},\phi_{2})\bigg[e^{-i\phi_{1}
	\left(-t+\frac{t}{\sqrt{2}}+\frac{x+y}{2}\right)}\\
& \qquad\qquad e^{-2i\phi_{2}\left(-t-\frac{t}{\sqrt{2}}+\frac{x-y}{2}\right)}
		-e^{-i\phi_{1}\left(-\left\lfloor t\right\rfloor +\left\lfloor \frac{t}{\sqrt{2}}\right\rfloor +
		\frac{x+y}{2}\right)}e^{-2i\phi_{2}\left(-\left\lfloor t\right\rfloor -\left\lfloor 
		\frac{t}{\sqrt{2}}\right\rfloor +\frac{x-y}{2}\right)}\bigg]\\
\langle x,y|f_{1}(t)\rangle & = -
	\frac{e^{i\theta}}{\sqrt{2}} e^{-it\sqrt{2}}e^{i \pi x/4} 
	e^{-i\pi y/2}\iint_{D_{\pi}\setminus D_{\epsilon}}\frac{d\phi_{1}d\phi_{2}}{4\pi^{2}}
	A(\phi_{1},\phi_{2})\\
&  \qquad\qquad\qquad\qquad \qquad\qquad \qquad 
	e^{-i\phi_{1}\left(-t+\frac{t}{\sqrt{2}}+\frac{x+y}{2}\right)}e^{-2i\phi_{2}
	\left(-t-\frac{t}{\sqrt{2}}+\frac{y-x}{2}\right)}\\
\langle x,y|f_{2}(t)\rangle & =  -\frac{1}{\sqrt{2}}e^{-it\sqrt{2}}e^{-i\pi x/2}e^{i\pi y/4 }
	\iint_{D_{\pi}\setminus D_{\epsilon}}\frac{d\phi_{1}d\phi_{2}}{4\pi^{2}}
	A(\phi_{1},\phi_{2})\\
&  \qquad\qquad\qquad\qquad \qquad\qquad \qquad e^{-i\phi_{1}\left(-t+\frac{t}{\sqrt{2}}+\frac{x+y}{2}\right)}
	e^{-2i\phi_{2}\left(-t-\frac{t}{\sqrt{2}}+\frac{x-y}{2}\right)}\\
\langle x,y|g_{1}(t)\rangle & =   \frac{e^{i\theta}}{\sqrt{2}}e^{i \pi x/4}e^{- i \pi y/2}
	\iint_{D_{\epsilon}}\frac{d\phi_{1}d\phi_{2}}{4\pi^{2}}A(\phi_{1},\phi_{2})
	e^{-i\phi_{1}\left(\frac{x+y}{2}\right)}e^{-2i\phi_{2}\left(\frac{y-x}{2}\right)}\\
& \qquad \qquad 
	\left[e^{-it 4\cos(\frac{\pi}{8}+\frac{\phi_{1}}{2})\cos(\frac{3\pi}{8}
	+\phi_{2})} -e^{-it\left(\sqrt{2}+\sqrt{2}\left(\frac{\phi_{1}}{2}-\phi_{2}\right)
 	-2\left(\frac{\phi_{1}}{2}+\phi_{2}\right)\right)}\right]\\
\langle x,y|g_{2}(t)\rangle & =  \frac{1}{\sqrt{2}}e^{-i\pi x/2}e^{i\pi y/4}
	\iint_{D_{\epsilon}}\frac{d\phi_{1}d\phi_{2}}{4\pi^{2}}A(\phi_{1},\phi_{2})
	e^{-i\phi_{1}\left(\frac{x+y}{2}\right)}e^{-2i\phi_{2}\left(\frac{x-y}{2}\right)}\\
 & \qquad\qquad\left[e^{-it 4\cos(\frac{\pi}{8}+\frac{\phi_{1}}{2})\cos(\frac{3\pi}{8}+\phi_{2})} - e^{-it\left(\sqrt{2}+\sqrt{2}\left(\frac{\phi_{1}}{2}-\phi_{2}\right)
 	-2\left(\frac{\phi_{1}}{2}+\phi_{2}\right)\right)}\right]\\
\langle x,y|h(t)\rangle & =  \frac{1}{\sqrt{2}}e^{i\pi x/4}e^{-i \pi y/2}
	\iint_{D_{\epsilon}}\frac{d\phi_{1}d\phi_{2}}{4\pi^{2}}
	A(\phi_{1},\phi_{2})e^{-i\phi_{1}\left(\frac{x+y}{2}\right)}e^{-2i\phi_{2}\left(\frac{y-x}{2}\right)}\\
& \qquad \qquad e^{-it 4\cos(\frac{\pi}{8}+\frac{\phi_{1}}{2}) \cos(\frac{3\pi}{8}+\phi_{2})} \left(e^{i\theta_{\pm}(\frac{\pi}{4}+\phi_{1},\frac{3\pi}{8}+\phi_{2})}-e^{i\theta}\right).
\end{align*}
We now proceed to bound the norm of each of these states. We repeatedly
use the fact that, for $(\phi_{1},\phi_{2})\in D_{\pi}$, 
\begin{align*}
\sum_{x,y=-\infty}^{\infty}e^{ix\left(\frac{1}{2}\left(\phi_{1}-\tilde{\phi}_{1}\right)-\left(\phi_{2}-\tilde{\phi}_{2}\right)\right)}e^{iy\left(\frac{1}{2}\left(\phi_{1}-\tilde{\phi}_{1}\right)+\left(\phi_{2}-\tilde{\phi}_{2}\right)\right)}
 & =  4\pi^{2}\delta(\phi_{1}-\tilde{\phi}_{1}) \delta(\phi_{2}-\tilde{\phi}_{2}).
 \end{align*}
 Using this formula we get
 \begin{align*}
\langle e_{1}(t)|e_{1}(t)\rangle & =  \sum_{y-x\geq C}\langle e_{1}(t)|x,y\rangle\langle x,y|e_{1}(t)\rangle\\
   & \leq  \sum_{x=-\infty}^{\infty}\sum_{y=-\infty}^{\infty}\bigg|\frac{1}{\sqrt{2}}
	 	\iint_{D_{\pi}}\frac{d\phi_{1}d\phi_{2}}{4\pi^{2}}A(\phi_{1},\phi_{2})
	 	\bigg[e^{-i\phi_{1}\left(-t+\frac{t}{\sqrt{2}}+\frac{x+y}{2}\right)}\\
	&\qquad e^{-2i\phi_{2}\left(-t-\frac{t}
	 	{\sqrt{2}}+\frac{y-x}{2}\right)}-e^{-i\phi_{1}\left(-\left\lfloor t\right\rfloor +\left\lfloor \frac{t}{\sqrt{2}}\right\rfloor 
 		+\frac{x+y}{2}\right)}e^{-2i\phi_{2}\left(-\left\lfloor t\right\rfloor -\left\lfloor \frac{t}{\sqrt{2}}
 		\right\rfloor +\frac{y-x}{2}\right)}\bigg]\bigg|^{2}\\
 & =  \frac{1}{2}\iint_{D_{\pi}}\frac{d\phi_{1}d\phi_{2}}{4\pi^{2}}\left|A(\phi_{1},\phi_{2})
 		\right|^{2}\bigg|e^{-i\phi_{1}\left(-t+\frac{t}{\sqrt{2}}\right)}e^{-2i\phi_{2}\left(-t-\frac{t}
 		{\sqrt{2}}\right)}\\
 & \qquad -e^{-i\phi_{1}\left(-\left\lfloor t\right\rfloor +\left\lfloor \frac{t}
 		{\sqrt{2}}\right\rfloor \right)}e^{-2i\phi_{2}\left(-\left\lfloor t\right\rfloor -\left\lfloor 
 		\frac{t}{\sqrt{2}}\right\rfloor \right)}\bigg|^{2}.
\end{align*}
 Now use the fact that $\left|e^{-ic}-1\right|^{2}\leq c^{2}$ for
$c\in\mathbb{R}$ to get 
\begin{align*}
\langle e_{1}(t)|e_{1}(t)\rangle & \leq  \frac{1}{2}\iint_{D_{\pi}}\left(\frac{d\phi_{1}d\phi_{2}}
	{4\pi^{2}}\right)\left|A(\phi_{1},\phi_{2})\right|^{2}\Bigg(-\phi_{1}\left(-t+\frac{t}{\sqrt{2}}+
	\left\lfloor t\right\rfloor -\left\lfloor \frac{t}{\sqrt{2}}\right\rfloor \right)\\
 &  \quad \quad -2\phi_{2}\left(-t-\frac{t}{\sqrt{2}}+\left\lfloor t\right\rfloor +\left\lfloor \frac{t}{\sqrt{2}}\right\rfloor \right)\Bigg)^{2}\\
 & \leq  4\iint_{D_{\pi}}\frac{d\phi_{1}d\phi_{2}}{4\pi^{2}}\left|A(\phi_{1},\phi_{2})\right|^{2}\left(\phi_{1}^{2}+4\phi_{2}^{2}\right)
\end{align*}
using the Cauchy-Schwarz inequality and the fact that $\left|t-{t}/{\sqrt{2}}-\left\lfloor t\right\rfloor -\left\lfloor {t}/{\sqrt{2}}\right\rfloor \right|\leq2$.
So 
\begin{align*}
\langle e_{1}(t)|e_{1}(t)\rangle & \leq 4\left(\iint_{D_{\pi}\setminus D_{\epsilon}}\frac{d\phi_{1}d\phi_{2}}{4\pi^{2}}+\iint_{D_{\epsilon}}\frac{d\phi_{1}d\phi_{2}}{4\pi^{2}}\right)\left|A(\phi_{1},\phi_{2})\right|^{2}\left(\phi_{1}^{2}+4\phi_{2}^{2}\right)\\
 & \leq  4\left(5\pi^{2}\right)\left(\frac{4\pi}{L\epsilon}\right)+20\epsilon^{2}\\
 & =  \frac{80\pi^{3}}{L\epsilon}+20\epsilon^{2}
\end{align*}
where we have used \lem{Let--and} and the fact that $\phi_{1}^{2}+4\phi_{2}^{2}\leq 5\epsilon^{2}$ on $D_\epsilon$. Similarly, 
\[
\langle e_{2}(t)|e_{2}(t)\rangle\leq\frac{80\pi^{3}}{L\epsilon}+20\epsilon^{2}.
\]
 Now
 \begin{align*}
\langle f_{1}(t)|f_{1}(t)\rangle & \leq  \frac{1}{2}\iint_{D_{\pi}\setminus D_{\epsilon}}\frac{d\phi_{1}d\phi_{2}}{4\pi^{2}}\left|A(\phi_{1},\phi_{2})\right|^{2}\\
 & \leq  \frac{2\pi}{L\epsilon}
 \end{align*}
 by \lem{Let--and}, and similarly
 \[
\langle f_{2}(t)|f_{2}(t)\rangle\leq\frac{2\pi}{L\epsilon}.
\]
Moving on to the next term, 
\begin{align}
\langle g_{1}(t)|g_{1}(t)\rangle & \leq  \frac{1}{2}\iint_{D_{\epsilon}}\frac{d\phi_{1}d\phi_{2}}{4\pi^{2}}
	\left|A(\phi_{1},\phi_{2})\right|^{2}\Bigg|e^{-it 4\cos(\frac{\pi}{8}+\frac{\phi_{1}}{2})
	\cos(\frac{3\pi}{8}+\phi_{2})}\nonumber \\
 &  \qquad\qquad\qquad -e^{-it\left(\sqrt{2}+\sqrt{2}\left(\frac{\phi_{1}}{2}-\phi_{2}\right)-2\left(\frac{\phi_{1}}{2}+\phi_{2}\right)\right)}
 	\Bigg|^{2}\nonumber \\
 & \leq  \frac{1}{2}\iint_{D_{\epsilon}}\frac{d\phi_{1}d\phi_{2}}{4\pi^{2}}
 	\Bigg[\left|A(\phi_{1},\phi_{2})\right|^{2}t^{2} \left(4\cos\left(\frac{\pi}{8}+\frac{\phi_{1}}{2}\right)\cos\left(\frac{3\pi}{8}+\phi_{2}\right)\right.\nonumber\\
&\qquad\qquad\qquad\left.
 	-\sqrt{2}-\sqrt{2}\left(\frac{\phi_{1}}{2}-\phi_{2}\right)+2\left(\frac{\phi_{1}}{2}+\phi_{2}\right)\right)^{2}\Bigg]
 	\label{eq:g_bound}
\end{align}
using $\left|e^{-ic}-1\right|^{2}\leq c^{2}$ for $c\in\mathbb{R}$.
Now 
\begin{align*}
4\cos\left(\frac{\pi}{8}+\frac{\phi_{1}}{2}\right)\cos\left(\frac{3\pi}{8}+\phi_{2}\right) & =  
	2\cos\left(\frac{\pi}{2}+\frac{\phi_{1}}{2}+\phi_{2}\right)+2\cos\left(-\frac{\pi}{4}+\frac{\phi_{1}}{2}-\phi_{2}\right)\\
 & =  - 2 \sin\left(\frac{\phi_1}{2} + \phi_2\right) + \sqrt{2} \cos\left(\frac{\phi_1}{2}-\phi_2\right) + 
 	\sqrt{2} \sin\left(\frac{\phi_1}{2} - \phi_2\right)
 \end{align*}
so 
\begin{align*}
 &   \left|4\cos\left(\frac{\pi}{8}+\frac{\phi_{1}}{2}\right)\cos\left(\frac{3\pi}{8}+\phi_{2}\right)-\sqrt{2}-\sqrt{2}
 	\left(\frac{\phi_{1}}{2}-\phi_{2}\right)+2\left(\frac{\phi_{1}}{2}+\phi_{2}\right)\right|\\
 & \quad \leq  \left|\sqrt{2}\left(\cos\left(\frac{\phi_{1}}{2}-\phi_{2}\right)-1\right)\right| 
 	+\left|\sqrt{2}\left(\sin\left(\frac{\phi_{1}}{2}-\phi_{2}\right)-\left(\frac{\phi_{1}}{2}-\phi_{2}\right)\right)\right|\\
 & \qquad +\left|2\left(\sin\left(\frac{\phi_{1}}{2}+\phi_{2}\right)-\left(\frac{\phi_{1}}{2}+\phi_{2}\right)\right)\right|\\
 & \quad \leq  \sqrt{2}\left(\frac{\phi_1}{2}-\phi_{2}\right)^{2}+\sqrt{2}\left(\frac{\phi_{1}}{2}
 	-\phi_{2}\right)^{2}+2\left(\frac{\phi_{1}}{2}+\phi_{2}\right)^{2} \\
 & \quad \leq  4\left(\left(\frac{\phi_{1}}{2}+\phi_{2}\right)^{2}+\left(\frac{\phi_{1}}{2}-\phi_{2}\right)^{2}\right),
\end{align*}
using $|{\cos x -1}|\leq x^2$ and $|{\sin x-x}|\leq x^2$ for $x\in \mathbb{R}$. Plugging this into equation \eq{g_bound} we get 
\begin{align*}
\langle g_{1}(t)|g_{1}(t)\rangle & \leq \frac{1}{2}\iint_{D_{\epsilon}}
	\frac{d\phi_{1}d\phi_{2}}{4\pi^{2}} 16 \left|A(\phi_{1},\phi_{2})\right|^{2}t^{2}
	\left(\left(\frac{\phi_{1}}{2} + \phi_2\right)^2+\left(\frac{\phi_1}{2} - \phi_{2}\right)^2\right)^{2} \\
 & \leq  16 t^2 \iint_{D_{\epsilon}}\frac{d\phi_{1}d\phi_{2}}{4\pi^{2}}
 	\left|A(\phi_{1},\phi_{2})\right|^{2}\left(\left(\frac{\phi_1}{2} + \phi_2\right)^4 
 	+ \left(\frac{\phi_1}{2} - \phi_2\right)^4\right)\\
 & \leq  \frac{16 t^{2}}{L^2} \iint_{D_{\epsilon}}\frac{d\phi_{1}d\phi_{2}}{4\pi^{2}}
	  \frac{\sin^2(\frac{L}{2} [\frac{\phi_1}{2} + \phi_2])}
	 {\sin^2(\frac{1}{2} [\frac{\phi_1}{2} + \phi_2])}
	 \frac{\sin^2(\frac{L}{2} [-\frac{\phi_1}{2} + \phi_2])}
	 {\sin^2(\frac{1}{2} [-\frac{\phi_1}{2} + \phi_2])}\\
 & \qquad \left(\left(\frac{\phi_1}{2}+ \phi_2\right)^4 
 	+ \left(\frac{\phi_1}{2} - \phi_2\right)^4\right)
\end{align*}
where we used the Cauchy-Schwarz inequality in the second line and equation \eq{A_summed} in the last line.  Changing coordinates to
\[
  \alpha_1 = \phi_1 + \frac{ \phi_2}{2} \qquad \alpha_2 = \frac{\phi_1}{2} -\phi_2 
 \]
and realizing that $|\alpha_1|,|\alpha_2| < 3\epsilon/2$ for $(\phi_1,\phi_2)\in D_{\epsilon}$, we see that
\begin{align*}
  \bra{g_1(t)} g_1(t)\rangle &\leq \frac{16t^2}{L^2} \int_{-{3\epsilon/2}}^{3\epsilon/2} \frac{d\alpha_1}{2\pi}
      \int_{-{3\epsilon/2}}^{3\epsilon/2} \frac{d\alpha_1}{2\pi} 
      \frac{\sin^2(\frac{1}{2} L\alpha_1)}
	 {\sin^2(\frac{1}{2} \alpha_1)}
	 \frac{\sin^2(\frac{1}{2} L\alpha_2)}
	 {\sin^2(\frac{1}{2} \alpha_2)} \left(\alpha_1^4+\alpha_2^4\right)\\
   &= \frac{32t^2}{L^2} \int_{-{3\epsilon/2}}^{3\epsilon/2} \frac{d\alpha_1}{2\pi}
      \int_{-{3\epsilon/2}}^{3\epsilon/2} \frac{d\alpha_1}{2\pi}
      \frac{\sin^2(\frac{1}{2} L\alpha_1)}
	 {\sin^2(\frac{1}{2} \alpha_1)}
	 \frac{\sin^2(\frac{1}{2} L\alpha_2)}
	 {\sin^2(\frac{1}{2} \alpha_2)} \alpha_1^4\\
   &\leq \frac{32t^2}{L} \int_{-3\epsilon/2}^{3\epsilon/2} \frac{d\alpha_1}{2\pi}  
     \frac{\sin^2(\frac{1}{2} L\alpha_1)}
	 {\sin^2(\frac{1}{2} \alpha_1)} \alpha_1^4\\
   &\leq \frac{32 t^2}{L} \int_{-3\epsilon/2}^{3\epsilon/2} \frac{d\alpha_1}{2\pi} \frac{\pi^2}{\alpha_1^2} \alpha_1^4\\
   &= \frac{36 \pi t^2\epsilon^3}{L},
\end{align*}
with a similar bound on $\langle g_{2}(t)|g_{2}(t)\rangle$.
 
Finally, 
\begin{align*}
\langle h(t)|h(t)\rangle & \leq  \frac{1}{2}\iint_{D_{\epsilon}}\frac{d\phi_{1}d\phi_{2}}{4\pi^{2}}
	\left|A(\phi_{1},\phi_{2})\right|^{2}\left|e^{i\theta_\pm (\tfrac{\pi}{4}+\phi_{1},
	\tfrac{3\pi}{8}+\phi_{2})}-e^{i\theta}\right|^{2}.
 \end{align*}
Recall that $e^{i\theta_\pm (p_1,p_2)}=T(p_1,p_2) \pm R(p_1,p_2)$ is obtained by solving for the effective single-particle S-matrix for the Hamiltonian \eq{vr_eqn}. For $p_1$ near ${\pi}/{4}$ we divide this Hamiltonian by $2\cos({p_1}/{2})$ to put it in the form considered in \cite{Childs_Gosset}, where the potential term is now $\mathcal{V}(|r|)/(2\cos({p_1}/{2}))$. The entries $T(p_1,p_2)$ and $R(p_1,p_2)$ of this S-matrix are bounded rational functions of $z=e^{ip_2}$ and $(2\cos({p_1}/{2}))^{-1}$ \cite{Childs_Gosset}, so they are differentiable as a function of $p_1$ and $p_2$ on some neighborhood $U$ of $({\pi}/{4},{3\pi}/{8})$ (and have bounded partial derivatives on this neighborhood).

For $\epsilon$ small enough that $D_\epsilon\subset U$ we get, using the mean value theorem and the fact that $\theta=\theta_\pm ({\pi}/{4}, {3\pi}/{8})$,
\begin{align*}
\left|e^{i\theta_\pm (\tfrac{\pi}{4}+\phi_{1}, \tfrac{3\pi}{8}+\phi_{2})}-e^{i\theta}\right| & \leq \sqrt{\phi_1^2+\phi_2^2} \max_{U} \big|\vec{\nabla} e^{i\theta_\pm}\big| \quad \text{for }(\phi_1,\phi_2)\in D_{\epsilon}\\
& \leq  \epsilon \Gamma
\end{align*}
for some constant $\Gamma$ (independent of $L$).
Therefore
\begin{align*}
\langle h(t)|h(t)\rangle & \leq  \frac{1}{2}\iint_{D_{\epsilon}}\frac{d\phi_{1}d\phi_{2}}{4\pi^{2}}
	\left|A(\phi_{1},\phi_{2})\right|^{2} \epsilon^2 \Gamma^2\\
& \leq \frac{1}{2}\Gamma^2 \epsilon^2.
\end{align*}
 
Putting these bounds together, we get 
\begin{align*}
\Norm{P_1|\psi_A(t)\rangle-P_1|\alpha(t)\rangle}  
 & \leq \norm{|e_{1}(t)\rangle} + \norm{|e_{2}(t)\rangle} + 
        \norm{|f_{1}(t)\rangle} + \norm{|f_{2}(t)\rangle} \\
 &  \qquad + \norm{|g_{1}(t)\rangle} + \norm{|g_{2}(t)\rangle} 
           + \norm{|h(t)\rangle} \\
 & \leq 2\left(\frac{80\pi^{3}}{L\epsilon}+20\epsilon^{2}\right)^{\frac{1}{2}}
 	+2\left(\frac{2\pi}{L\epsilon}\right)^{\frac{1}{2}}+2\left(\frac{36\pi t^{2}\epsilon^{3}}{L}\right)^{\frac{1}{2}}+\frac{1}{\sqrt{2}}\Gamma \epsilon.
\end{align*}
 Letting $\epsilon={a}/{\sqrt{L}}$ and $t\leq c_{0}L$ we get 
\begin{equation}
\left\Vert \,P_1|\psi_A(t)\rangle-P_1|\alpha(t)\rangle\right\Vert = \O(L^{-{1}/{4}}).\label{eq:psiA_alpha}
\end{equation}

Since $P_2\ket{\alpha(t)}$ has support on at most $4CL$ basis states $|x,y\rangle$, and since $|\langle x,y|P_2|\alpha(t)\rangle|^2=\O( L^{-2})$, we get
\begin{equation}
  \left\Vert P_2\ket{\alpha(t)}\right\Vert = \O(L^{-{1}/{2}}).\label{eq:P2alpha_bound}
\end{equation}

We now use the bounds \eq{psiA_alpha} and \eq{P2alpha_bound} and \lem{alpha} to show that
\begin{equation}
\left\Vert |\psi_{A}(t)\rangle-|\alpha(t)\rangle\right\Vert =\O(L^{-{1}/{8}}).\label{eq:psiA_alpha_bound}\end{equation}
 First consider the case where the interaction range is $C=0$ (as in
the Bose-Hubbard model). In this case equation \eq{psiA_alpha_bound}
follows directly from equation \eq{psiA_alpha} and the facts that
$\langle x,y|\alpha(t)\rangle=\pm\langle y,x|\alpha(t)\rangle$ and
$\langle x,y|\psi_{A}(t)\rangle=\pm\langle y,x|\psi_{A}(t)\rangle$. 

Now suppose $C\neq0$. In this case
\begin{align*}
\left\Vert \left(1-P_{2}\right)|\psi_{A}(t)\rangle\right\Vert ^{2} & = 2\left\Vert P_{1}|\psi_{A}(t)\rangle\right\Vert ^{2}\\
 & = 2\left(\left\Vert P_{1}|\alpha(t)\rangle\right\Vert +\O(L^{-{1}/{4}})\right)^2\\
 & = 2\left(\frac{1}{2}\left\Vert (1-P_{2})|\alpha(t)\rangle\right\Vert ^{2}+\O(L^{-{1}/{4}})\right)\\
 & = 1+\O(L^{-1})-\langle\alpha(t)|P_{2}|\alpha(t)\rangle+\O(L^{-{1}/{4}})\\
 & = 1+\O(L^{-{1}/{4}})
\end{align*}
where in the next-to-last line we have used \lem{alpha}. So
\begin{align*}
\left\Vert |\psi_{A}(t)\rangle-|\alpha(t)\rangle\right\Vert  & \leq 2\left\Vert P_{1}|\psi_{A}(t)\rangle-P_{1}|\alpha(t)\rangle\right\Vert +\left\Vert P_{2}|\alpha(t)\rangle\right\Vert +\left\Vert P_{2}|\psi_{A}(t)\rangle\right\Vert \\
 & = \O(L^{-{1}/{4}})+\O(L^{-{1}/{2}})+\left(1-\left\Vert \left(1-P_{2}\right)|\psi_{A}(t)\rangle\right\Vert \right)^{\frac{1}{2}}\\
 & = \O(L^{-{1}/{4}})+\O(L^{-{1}/{2}})+\O(L^{-{1}/{8}})\\
 & = \O(L^{-{1}/{8}})
 \end{align*}
which completes the proof. 
\end{proof}
\end{proof}

%%%%%%%%%%%%%%%%%
% Technical Lemmas

\subsubsection{Technical lemmas}
\label{sec:techlem}

In this section we prove three lemmas that are used in the proof of \thm{twopart}.

\begin{lemma} \label{lem:alpha}
Let $|\alpha(t)\rangle$ be defined as in \thm{twopart}.
Then
\[
\langle\alpha(t)|\alpha(t)\rangle=1+\O(L^{-1}).
\]
\end{lemma}

\begin{proof}
Define
\[
\Pi=\sum_{x\leq y}|x,y\rangle\langle x,y|.
\]
Note that, since $\langle x,y|\alpha(t)\rangle=\pm\langle y,x|\alpha(t)\rangle$,
\begin{align*}
\langle\alpha(t)|\alpha(t)\rangle & = 2\langle\alpha(t)|\Pi|\alpha(t)\rangle-\sum_{x=-\infty}^{\infty}\langle\alpha(t)|x,x\rangle\langle x,x|\alpha(t)\rangle\\
 & = 2\langle\alpha(t)|\Pi|\alpha(t)\rangle+\O(L^{-1})
\end{align*}
where the last line follows since $|\langle x,x|\alpha(t)\rangle|^{2}$
is nonzero for at most $L$ values of $x$ and $|\langle x,x|\alpha(t)\rangle|^{2}=\O(L^{-2})$.
We now show that
\[
\langle\alpha(t)|\Pi|\alpha(t)\rangle=\frac{1}{2}+O(L^{-1}).
\]
Note that
\begin{align*}
\langle\alpha(t)|\Pi|\alpha(t)\rangle &= \frac{1}{2L^{2}}\sum_{x\leq y}\Bigg(F(x,y,t) +F(y,x,t)  \\
&\qquad \pm e^{i\theta}e^{\frac{3i\pi}{4}x}e^{-\frac{3i\pi}{4}y}F(x,y,t) F(y,x,t) \\
&\qquad \pm e^{-i\theta}e^{-\frac{3i\pi}{4}x}e^{\frac{3i\pi}{4}y}F(x,y,t) F(y,x,t) \Bigg).
\end{align*}
Now $F(x,y,t)= 1$ if and only if $x\in\{-M-L+2\lfloor t \rfloor,\ldots,-M-1+2 \lfloor t \rfloor \}$
and $y\in\{M+1-2 \lfloor {t}/{\sqrt{2}} \rfloor ,\ldots,M+L-2 \lfloor {t}/{\sqrt{2}} \rfloor \}$.
Similarly $F(y,x,t) =1$ if and only if $x\in\{M+1-2 \lfloor {t}/{\sqrt{2}} \rfloor ,\ldots,M+L-2 \lfloor {t}/{\sqrt{2}} \rfloor \}$
and $y\in\{-M-L+2 \lfloor t \rfloor ,\ldots,-M-1+2 \lfloor t \rfloor \}$.
So
\[
\sum_{x\leq y}F(y,x,t) =\sum_{y\leq x}F(x,y,t) 
\]
and
\begin{align*}
\frac{1}{2L^{2}}\sum_{x\leq y}\left[F(x,y,t) +F(y,x,t) \right] &= \frac{1}{2L^{2}}\left(\sum_{x=-\infty}^{\infty}\sum_{y=-\infty}^{\infty}F(x,y,t) -\sum_{x=-\infty}^{\infty}F(x,x,t) \right)\\
 &= \frac{1}{2}+\O(L^{-1}).
\end{align*}

We now establish the bound
\[
\left|\frac{1}{2L^{2}}\sum_{x\leq y}e^{\frac{3i\pi}{4}x}e^{-\frac{3i\pi}{4}y}F(x,y,t) F(y,x,t) \right|=\O(L^{-1})
\]
to complete the proof.  To get this bound, note that both $F(x,y,t) =1$ and $F(y,x,t) =1$
if and only if
\begin{align*}
x,y & \in \{-M-L+2\left\lfloor t\right\rfloor ,\ldots,-M-1+2\left\lfloor t\right\rfloor \}\\
\text{ and }x,y & \in \left\{M+1-2\left\lfloor \frac{t}{\sqrt{2}}\right\rfloor ,\ldots,M+L-2\left\lfloor \frac{t}{\sqrt{2}}\right\rfloor \right\}.
\end{align*}
Letting
\[
B=\{-M-L+2\left\lfloor t\right\rfloor ,\ldots,-M-1+2\left\lfloor t\right\rfloor \} \cap \left\{M+1-2\left\lfloor \frac{t}{\sqrt{2}}\right\rfloor ,\ldots,M+L-2\left\lfloor \frac{t}{\sqrt{2}}\right\rfloor \right\},
\]
we have
 \[
B=\{j,j+1,\ldots,j+l\}
\]
for some $j,l\in\mathbb{Z}$ with $l<L$. So
\begin{align*}
\frac{1}{2L^{2}}\left|\sum_{x\leq y}e^{\frac{3i\pi}{4}x}e^{-\frac{3i\pi}{4}y}F(x,y,t) F(y,x,t) \right| & = \frac{1}{2L^{2}}\left|\sum_{x,y\in B,\, x\leq y}e^{\frac{3i\pi}{4}x}e^{-\frac{3i\pi}{4}y}\right|\\
 &  = \frac{1}{2L^{2}}\left|\sum_{y=j}^{j+l}\sum_{x=j}^{y}e^{\frac{3i\pi}{4}x}e^{-\frac{3i\pi}{4}y}\right|\\
 &= \frac{1}{2L^{2}}\left|\sum_{y=j}^{j+l}e^{-\frac{3i\pi}{4}y}e^{3i\frac{\pi}{4}j}\frac{e^{3i\frac{\pi}{4}\left(y+1-j\right)}-1}{e^{3i\frac{\pi}{4}}-1}\right|\\
 & \leq \frac{(l+1)}{2L^{2}}\frac{2}{\left|e^{3i\frac{\pi}{4}}-1\right|}\\
 &= \O(L^{-1})
 \end{align*}
since $l<L$.
\end{proof}
\begin{lemma}
\label{lem:Let--and}Let $k\in(-\pi,0)\cup(0,\pi)$ and $0<\epsilon<\min\left\{ \pi-|k|,|k|\right\}$.
Let
\begin{align*}
D_{\epsilon} & =  \left[-\epsilon,\epsilon\right]\times\left[-\epsilon,\epsilon\right]\\
D_{\pi} & =  \left[-\pi,\pi\right]\times\left[-\pi,\pi\right].
\end{align*}
Then 
\begin{align*}
\iint_{D_{\pi}}\frac{d\phi_{1}d\phi_{2}}{4\pi^{2}}|A(\phi_{1},\phi_{2})|^{2} & =  1\\
\iint_{D_{\pi}\setminus D_{\epsilon}}\frac{d\phi_{1}d\phi_{2}}{4\pi^{2}}|A(\phi_{1},\phi_{2})|^{2} & \leq  \frac{4\pi}{L\epsilon}\\
\iint_{D_{\epsilon}}\frac{d\phi_{1}d\phi_{2}}{4\pi^{2}}|B(\phi_{1},\phi_{2},k)|^{2} & \leq  \frac{4\pi^{2}}{L^{2}\epsilon^{2}}.\end{align*}
where $A(\phi_1,\phi_2)$ and $B(\phi_1,\phi_2,k)$ are given by equation \eq{A}.
 \end{lemma}
\begin{proof}
Using equation \eq{A} we get
\begin{align*}
|A(\phi_{1},\phi_{2})|^{2} & =  \frac{1}{L^{2}} \sum_{x,\tilde{x}=-(M+L)}
  ^{-(M+1)} \sum_{y,\tilde{y}=M+1}^{M+L}
	e^{i\frac{\phi_{1}}{2}\left(x+y-(\tilde{x}+\tilde{y})\right)}
        e^{i\phi_{2}\left(x-y-(\tilde{x}-\tilde{y})\right)}.
\end{align*}
Now 
\[
\int_{-\pi}^{\pi}\frac{d\phi_{2}}{2\pi}e^{i\phi_{2}
  \left(x-y-\tilde{x}+\tilde{y}\right)}=\delta_{x-y,\tilde{x}-\tilde{y}},
\]
so (suppressing the limits of summation for readability) 
\begin{align*}
\iint_{D_{\pi}}\frac{d\phi_{1}d\phi_{2}}{4\pi^{2}}
   |A(\phi_{1},\phi_{2})|^{2} 
   & =  \frac{1}{L^{2}} \int_{-\pi}^{\pi}\frac{d\phi_{1}}{2\pi}
     \sum_{x,\tilde{x}}\sum_{y,\tilde{y}}	e^{i\phi_{1}\left(y-\tilde{y}\right)}
     \delta_{x-y,\tilde{x}-\tilde{y}} \\
 & =  \frac{1}{L^{2}}\sum_{x,\tilde{x}}\sum_{y,\tilde{y}}\delta_{y,\tilde{y}}
 	\delta_{x-y,\tilde{x}-\tilde{y}}\\
 & =  1
\end{align*}
which proves the first part.

By performing the sums in equation \eq{A} we get 
\begin{equation}
|A(\phi_{1},\phi_{2})|^{2}
=\frac{1}{L^{2}}\frac{\sin^{2}(\frac{1}{2}L[\frac{\phi_1}{2}+\phi_{2}])} {\sin^{2}(\frac{1}{2}[\frac{\phi_1}{2}+\phi_{2}])}
\frac{\sin^{2}(\frac{1}{2}L[\frac{\phi_1}{2} -\phi_{2}])}
{\sin^{2}(\frac{1}{2}[\frac{\phi_1}{2}-\phi_{2}])}.
\label{eq:A_summed}
\end{equation}
Letting $\alpha_{1}={\phi_1}/{2}+\phi_{2}$ and
$\alpha_{2}={\phi_1}/{2}-\phi_{2}$, we see that
$|\alpha_{1}|\leq 3\pi/2$, $|\alpha_{2}|\leq 3\pi/2$,
and $\alpha_{1}^{2}+\alpha_{2}^{2}\geq  5\epsilon^{2}/2$ whenever
$(\phi_{1},\phi_{2})\in D_{\pi}\setminus D_{\epsilon}$. Defining
$D_{3\pi/2}=[-3\pi/2,3\pi/2]^{2}$
we get $(\alpha_{1},\alpha_{2})\in D_{3\pi/2}\setminus D_{\epsilon}$
whenever $(\phi_{1},\phi_{2})\in D_{\pi}\setminus D_{\epsilon}$.
Hence 
\begin{align*}
\iint_{D_{\pi}\setminus D_{\epsilon}}\frac{d\phi_{1}d\phi_{2}}{4\pi^{2}}|A(\phi_{1},\phi_{2})|^{2} 
	& \leq \frac{1}{L^{2}} \iint_{D_{3\pi/2}\setminus D_{\epsilon}} \frac{d\alpha_{1}d\alpha_{2}}{4\pi^{2}}
	\frac{\sin^{2}(\frac{1}{2}L\alpha_{1})}{\sin^{2}
	(\frac{1}{2}\alpha_{1})}\frac{\sin^{2}(\frac{1}{2}L\alpha_{2})}
	{\sin^{2}(\frac{1}{2}\alpha_{2})}\\
 & \leq  \frac{4}{L}\left(\frac{1}{L}\int_{-\frac{3\pi}{2}}^{\frac{3\pi}{2}}\frac{d\alpha_{1}}{2\pi}\frac{\sin^{2}(
 	\frac{1}{2}L\alpha_{1})}{\sin^{2}(\frac{1}{2}\alpha_{1})}\right)
 	\left(\int_{\epsilon}^{3\pi/2}\frac{d\alpha_{2}}{2\pi}\frac{\sin^{2}(\frac{1}{2}L\alpha_{2})}
 	{\sin^{2}(\frac{1}{2}\alpha_{2})}\right)\\
 & \leq  \frac{4}{L}\left(\int_{-2\pi}^{2\pi}\frac{d\alpha_{1}}{2\pi}\frac{1}{L}\frac{\sin^{2}(\frac{1}{2}L\alpha_{1})}{\sin^{2}(\frac{1}{2}\alpha_{1})}\right)\left(\int_{\epsilon}^{\frac{3\pi}{2}}\frac{d\alpha_{2}}{2\pi}\frac{1}{\sin^{2}(\frac{1}{2}\alpha_{2})}\right)\\
 & = \frac{8}{L} \left(\int_{\epsilon}^{\pi}\frac{d\alpha_{2}}{2\pi}\frac{1}{\sin^{2}(\frac{1}{2}\alpha_{2})}+\int_{\pi}^{\frac{3\pi}{2}}\frac{d\alpha_{2}}{2\pi}\frac{1}{\sin^{2}(\frac{1}{2}\alpha_{2})}\right)\\
 & \leq  \frac{8}{L}\left(\int_{\epsilon}^{\pi}\frac{d\alpha_{2}}{2\pi}\frac{\pi^{2}}{\alpha_{2}^{2}}+2\int_{\pi}^{\frac{3\pi}{2}}\frac{d\alpha_{2}}{2\pi}\right)\\
 & =  \frac{4\pi}{L\epsilon}
 \end{align*}
which proves the second inequality (in the next-to-last line we
have used the fact that $\sin({x}/{2})>{x}/{\pi}$ for $x \in (0,\pi)$
and $\sin^{2}({x}/{2})>{1}/{2}$ for $x \in (\pi,{3\pi}/{2})$).

Now
\begin{align*}
|B(\phi_{1},\phi_{2},k)|^{2} & =  |A(\phi_{1},-\phi_{2}-2k)|^{2}\\
 & \leq  \frac{1}{L^{2}}\frac{1}{\sin^{2}\left(\frac{1}{2}
     \left[\frac{\phi_{1}}{2}+\phi_{2}+2k\right]\right)}
 \frac{1}{\sin^{2}\left(\frac{1}{2}
     \left[-\frac{\phi_{1}}{2}+{\phi_{2}}+2k\right]\right)}.
\end{align*}
If $(\phi_{1},\phi_{2})\in D_{\epsilon}$ then $|k|-{3\epsilon}/{4} 
\leq\left|\pm{\phi_{1}}/{4}+{\phi_{2}}/{2}+k\right|
\leq|k|+{3\epsilon}/{4}$.
Noting that $\epsilon$ is chosen such that $0 < \epsilon < 
\min \{\pi-|k|,|k|\}$, we get
\[
\frac{\epsilon}{4}\leq\left|\pm\frac{\phi_{1}}{4}+\frac{\phi_{2}}{2}+k\right|\leq\pi-\frac{\epsilon}{4}
\]
so 
\begin{align*}
|B(\phi_{1},\phi_{2},k)|^{2} & \leq  \frac{1}{L^{2}}\frac{1}{\sin^{4}(\frac{\epsilon}{4})}\\
 	& \leq  \frac{16\pi^{4}}{L^{2}\epsilon^{4}}
\end{align*}
and 
\begin{align*}
\iint_{D_{\epsilon}}\frac{d\phi_{1}d\phi_{2}}{4\pi^{2}}|B(\phi_{1},\phi_{2},k)|^{2} & \leq  \frac{1}{4\pi^{2}}\left(2\epsilon\right)^{2}\left(\frac{16\pi^{4}}{L^{2}\epsilon^{4}}\right)\\
 & =  \frac{16\pi^{2}}{L^{2}\epsilon^{2}}. \qedhere
 \end{align*}
\end{proof}

\begin{lemma}
\label{lem:a_xy}
Let $a_{xy}(t)$ be as in Theorem \ref{thm:twopart}. For $x\leq y$,
\begin{align*}
a_{xy}(t) & =  \frac{1}{\sqrt{2}} e^{- i t\sqrt{2}}\left[e^{-i \pi x/2} e^{i \pi y/4} \left(\iint_{D_{\pi}} 
	\frac{d\phi_1 d\phi_2}{4\pi^2} A(\phi_1,\phi_2)\right.\right.\\
	& \qquad\quad \left.e^{-i \phi_1 \left( -\lfloor t\rfloor + \lfloor
	\frac{t}{\sqrt{2}}\rfloor + \frac{x+y}{2} \right)} e^{-2 i \phi_2 \left(-\lfloor t 
	\rfloor -\lfloor \frac{t}{\sqrt{2}}\rfloor + \frac{x-y}{2}\right)}\right)\\
& \quad \pm e^{i\theta} e^{i\pi x/4}e^{-i \pi y/2} \left(\iint_{D_{\pi}} \frac{d\phi_1 d\phi_2}{4\pi^2}
	A(\phi_1,\phi_2) \right.\\
&	\qquad\quad\left.\left.e^{- i \phi_1 \left(- \lfloor t\rfloor + \lfloor \frac{t}{\sqrt{2}}\rfloor + 
	\frac{x+y}{2}\right) } e^{-2 i \phi_2 \left(-\lfloor t \rfloor -\lfloor \frac{t}{\sqrt{2}}\rfloor
	+ \frac{y-x}{2}\right)}\right)
	\right].
\end{align*}
\end{lemma}

\begin{proof}
The lemma follows from \eq{a_xy} and the fact that, for any two numbers $\gamma_{1},\gamma_{2}$ such that $\gamma_{1}+\gamma_{2},\gamma_{1}-\gamma_{2}\in \mathbb{Z}$,
\[
\iint_{D_\pi}\frac{d\phi_{1} d\phi_{2}}{4\pi^2} A(\phi_{1},\phi_{2})e^{i\gamma_{1}\phi_{1}+2i \gamma_{2}\phi_{2}}=\begin{cases}
\frac{1}{L} &\text{ if }(-\gamma_{1}-\gamma_{2},-\gamma_{1}+\gamma_{2})\in S \\
0 &\text{ otherwise}
\end{cases}
\]
where $S = \{-M-L,\ldots, -M-1\} \times \{M+1,\ldots, M+L\}$.  To establish this formula, observe that
\begin{align*}
  \iint_{D_{\pi}}\frac{d\phi_1 d\phi_{2}}{4\pi^2}A(\phi_{1},\phi_{2})e^{ i\gamma_{1}\phi_{1}+2i\gamma_{2}\phi_{2}} 
  	& =  \frac{1}{L}\sum_{x=-M-L}^{-M-1}\sum_{y=M+1}^{M+L}\iint_{D_{\pi}}\frac{d\phi_1 d\phi_{2}}{4\pi^2}
  		e^{i\phi_{1}\left(\gamma_{1}+\frac{x+y}{2}\right)}e^{i\phi_{2}\left(x-y+2\gamma_{2}\right)}\\
 & =  \frac{1}{L}\sum_{x=-M-L}^{-M-1}\,\sum_{y=M+1}^{M+L}\int_{-\pi}^{\pi}\frac{d\phi_{1}}{2\pi}
 	e^{i\phi_{1}\left(\gamma_{1}+\frac{x-y}{2}\right)}\delta_{y,-x-2\gamma_{2}}.
\end{align*}
 Here we have performed the integral over $\phi_{2}$ using the fact
that $2\gamma_{2}$ is an integer.  We then have
\begin{align*}
\iint_{D_\pi}\frac{d\phi_{1} d\phi_{2}}{4\pi^2} A(\phi_{1},
	\phi_{2})e^{i\gamma_{1}\phi_{1}+2i\gamma_{2}\phi_{2}} 
 & =  \frac{1}{L}\sum_{x=-M-L}^{-M-1}\,
	\sum_{y=M+1}^{M+L}\int_{-\pi}^{\pi}\frac{d\phi_{1}}{2\pi}e^{i\phi_{1}
	\left(\gamma_{1}+x+\gamma_{2}\right)}\delta_{y,-x-2\gamma_{1}}\\
 & =  \frac{1}{L}\sum_{x=-M-L}^{-M-1}\,\sum_{y=M+1}^{M+L}\delta_{x,-\gamma_{1}-\gamma_{2}}\delta_{y,\gamma_{2}-\gamma_{1}}
 \end{align*}
as claimed.
\end{proof}

%%%%%%%%%%%%%%%%%%%%%%%%%%%%%%%%%%%%%%%%%%%%%%%%%%%%%%%%%%%%%%%%%%%%%%%%%%
%  Applying an encoded C\theta gate
%

\section{Applying an encoded $C\theta$-gate}



To implement the controlled phase gate between the mediator qubit and a computational qubit we use some facts about two-particle scattering on a long path. Recall  that two indistinguishable particles of momentum $k_1$ and $k_2$ initially traveling toward each other will, after scattering, continue to travel as if no interaction occurred, except that the phase of the wave function is modified by the interaction. In general this phase depends on $k_1$ and $k_2$ (as well as the interaction $\mathcal{U}$ and the particle statistics).  For us, $k_1=-{\pi}/{2}$ and $k_2={\pi}/{4}$ (moving in opposite directions).  We write $e^{i\theta}$ for the phase acquired at these momenta.


\subsection{Momentum switch}

In our scheme we design a subgraph that routes a computational particle and a mediator particle toward each other along a long path only when the two associated qubits are in state $\ket{11}$. This allows us to implement the two-qubit gate
\[
\CD=
\begin{pmatrix}
1 & 0 & 0 & 0\\
0 & 1 & 0 & 0\\
0 & 0 & 1 &0\\
0 & 0 & 0 & e^{i\theta}\\
\end{pmatrix}.
\]
For some models $\CD =\CP$. We show in \sec{twopart_scat} that this holds in the Bose-Hubbard model (where the interaction term is  $\mathcal{U}_{ij}(\hat{n}_i,\hat{n}_j)=  (U/2) \delta_{i,j} \hat{n}_i(\hat{n}_i-1)$) when the interaction strength is chosen to be $U=2+\sqrt{2}$, since in this case $e^{i\theta}=-i$. For nearest-neighbor interactions with fermions, with $\mathcal{U}_{ij}(\hat{n}_i,\hat{n}_j) = U \delta_{(i,j) \in E(G)} \hat{n}_i \hat{n}_j$, the choice $U=-2-\sqrt{2}$ gives $e^{i\theta}=i$, so $\CP=(\CD)^3$. While tuning the interaction strength makes the $\CP$ gate easier to implement, almost any interaction between indistinguishable particles allows for universal computation. We can approximate the required CP gate by repeating the $\CD$  gate $a$ times, where $e^{i a \theta} \approx -i$ (which is possible for most values of $\theta$, assuming $\theta$ is known \cite{note2}). 

Our strategy requires routing the particles onto a long path.  This is done via a subgraph we call the \emph{momentum switch}, as depicted in \fig{onepsplit}(a). The S-matrices for this graph at momenta $-{\pi}/{4}$ and $-{\pi}/{2}$ are
\begin{equation}
  S_{\text{switch}}\left(-\pi/4\right) = \begin{pmatrix} 0 & 0 & e^{-i\pi/4}\\
    0 & -1 & 0\\
    e^{-i\pi/4} & 0 & 0\end{pmatrix}\qquad
  S_{\text{switch}}\left(-\pi/2\right) = \begin{pmatrix}1 & 0 &0\\
    0 & 0 & -1\\
    0 & -1 & 0\end{pmatrix}.
\label{eq:switch_S}
\end{equation}
 The momentum switch has perfect transmission between vertices 1 and 3 at momentum $-{\pi}/{4}$ and perfect transmission between vertices 2 and 3 at momentum $-{\pi}/{2}$. In other words, in the schematic shown in \fig{onepsplit}(a), the path a particle follows through the switch depends on its momentum. A particle with momentum $-{\pi}/{2}$ follows the double line, while a particle with momentum $-{\pi}/{4}$ follows the single line.

The graph used to implement the $\CD$ gate has the form shown in \fig{onepsplit}(b).  We specify the number of vertices on each of the paths in \sec{description}. To see why this graph implements a $\CD$ gate, consider the movement of two particles as they pass through the graph. If either particle begins in the state $\ket{0_{\text{in}}}$, then it travels along a path to the output without interacting with the second particle. When the computational particle (qubit $c$ in the figure) begins in the state $\ket{1_{\text{in}}}^c$, it is routed downward as it passes through the top momentum switch (following the single line). It travels down the vertical path and then is routed to the right (along the single line) as it passes through the bottom switch.  Similarly, when the mediator particle begins in the state $\ket{1_{\text{in}}}^\med$, it is routed upward (along the double line) through the vertical path at the bottom switch and then to the right (along the double line) at the top switch. If both particles begin in the state $\ket{1_{\text{in}}}$, then they interact on the vertical path. In this case, as the two particles move past each other, the wave function acquires a phase $e^{i\theta}$ arising from this interaction. 

%\begin{figure}
%\centering
%\capstart
%\subfigure{(a)}
%\begin{tikzpicture}
%  [ scale = 0.6,
%    thin,
%    inner/.style={circle,draw=black!100,fill=black!100,inner sep=1.25pt},
%    attach/.style={circle,draw=black!100,fill=black!0,thin,inner sep=1.25pt},
%    sin/.style={line width=.7pt},
%    doub/.style={line width=2.1pt},
%    trip/.style={draw=white,line width=.7pt}]
%    \node at (0,0){};
%\begin{scope}[yshift=1.8cm]
%  \node (2)  at (0,0) [attach,label=left:3] {};
%  \node (1)  at (0,1) [attach,label=left:1] {};
%  \node (3)  at (3,1) [attach,label=right:2] {};
%  \node (4)  at (1,0) [inner]  {};
%  \node (5)  at (2,0) [inner]  {};
%  \node (6)  at (1,1) [inner]  {};
%  \node (7)  at (2,1) [inner]  {};
%  \node (8)  at (2,2) [inner]  {};
%  \node (9)  at (2.866,2.5)  [inner] {};
%  \node (10) at (1.134,2.5)  [inner] {};
%  \node (11) at (2,-1)       [inner] {};
%  \node (12) at (2.866,-1.5) [inner] {};
%  \node (13) at (1.134,-1.5) [inner] {};
%
%  \draw (2) to (4);
%  \draw (4) to (5);
%  \draw (3) to (7);
%  \draw (1) to (6);
%  \draw (6) to (4);
%  \draw (6) to (7);
%  \draw (7) to (5);
%  \draw (7) to (8);
%  \draw (8) to (9);
%  \draw (8) to (10);
%  \draw (11) to (5);
%  \draw (11) to (12);
%  \draw (11) to (13);
%
%  \node (split) at (-3.6,.8) [draw=black,circle,inner sep=3mm,
%         label=left:1,label=right:2,label=below:3] {};
% 
%  \node at (-1.6,.5) {=};
%
%  \draw[sin]  (split.west) to[out=0,in=90]   (split.south);
%  \draw[doub] (split.east) to[out=180,in=90] (split.south);
%  \draw[trip] (split.east) to[out=180,in=90] (split.south);
%  
%  \node at (split.east) [attach] {};
%  \node at (split.west) [attach] {};
%  \node at (split.south) [attach]{};
%\end{scope}
%\end{tikzpicture}
%\hspace{0.5cm}
%\subfigure{(b)}
%\begin{tikzpicture}
%  [ scale = 1,
%    yscale = .8,
%    attach/.style={circle,draw=black!100,fill=white,thick,
%    minimum size = 6mm},
%    cross/.style={line width=4pt, draw=white},
%    drawn/.style={draw=black},
%    vert/.style = {circle,fill=black,inner sep=.6pt, minimum size=0},
%    nofill/.style = {circle,draw=black,fill=white,inner sep = 1.25pt,minimum size=0},
%    decoration={markings,
%		mark=between positions 0 and 10 step .1cm
% 		with { \node at (0,0) [vert]{}; }}]
%
%  \node (bottom) at ( 1, 0) [attach] {};
%  \node (top)    at ( 1, 2.5) [attach] {};
%  
%  \foreach \x in {0,.1,...,3.5}{
%  \foreach \y in {.75, 3.25}{
%    \node at (\x,\y) [vert] {};
%  }}
%
%  \draw[postaction={decorate}] (0,0) node[left] {$1_{\med,\text{in}}$} 
%    -- (bottom.west);
%  \draw (0,.75) node [left] {$0_{\med,\text{in}}$} 
%    -- (3.5,.75) node [right] {$0_{\med,\text{out}}$};
%  \draw (0,3.25) node [left] {$0_{c,\text{in}}$}
%    -- (3.5,3.25) node [right] {$0_{c,\text{out}}$};
%  \draw[postaction={decorate}] (0,2.5) node [left] {$1_{c,\text{in}}$} 
%    -- (top.west) ;
%  \draw (top.south) -- (bottom.north) [cross];
%
%  \draw (top.south) -- (bottom.north) [drawn,postaction={decorate}];
%  \draw (top.east) -- ( 2, 2.5)  .. controls (2.5,2.5) and (2.5, 0) 
%                   .. (3,0) -- (3.5, 0)  [cross];
%  \draw (top.east) -- ( 2, 2.5)  .. controls (2.5,2.5) and (2.5, 0) 
%                    .. (3,0) -- (3.5, 0) 
%                    node [right] {$1_{\med,\text{out}}$} [drawn,postaction={decorate}];
%  \draw (bottom.east) -- ( 2, 0) .. controls (2.5,0) and (2.5,2.5)  
%                      .. (3,2.5) -- (3.5, 2.5)  [cross];
%  \draw[drawn,postaction={decorate}] (bottom.east) -- ( 2, 0) .. controls (2.5,0) and (2.5,2.5)  
%                      .. (3,2.5) -- (3.5, 2.5) 
%                      node [right] {$1_{c,\text{out}}$} ;
%
%  \draw (top.west) to[out=0,in=90] (top.south) [line width = .7pt];
%  \draw (bottom.east) to[out=-180,in=-90] (bottom.north) [line width=.7pt];
%  \draw (top.east) to[out=-180,in=90] (top.south) [line width=2.1pt];
%  \draw (top.east) to[out=-180,in=90] (top.south) [line width=.7pt,draw=white];
%  \draw (bottom.west) to[out=0,in=-90] (bottom.north) [line width=2.1pt];
%  \draw (bottom.west) to[out=0,in=-90] (bottom.north) [line width=.7pt,draw=white];  
%  
%  
%  \foreach \x in {0, 3.5}{
%  \foreach \y in {0,.75, 2.5, 3.25}{
%    \node at (\x,\y) [nofill] {};
%  }}
% 
% \node at (top.west) [nofill]{};
% \node at (top.east) [nofill]{};
% \node at (top.south) [nofill]{};
% \node at (bottom.west) [nofill]{};
% \node at (bottom.east) [nofill]{};
% \node at (bottom.north) [nofill]{};
%  
%\end{tikzpicture}
%
%\caption{(a) Momentum switch. (b) $\CD$ gate.}
%\label{fig:onepsplit}
%\end{figure}

Note that timing is important: the wave packets of the two particles must be on the vertical path at the same time. We achieve this by choosing the number of vertices on each of the segments in the graph appropriately, taking into account the different propagation speeds of the two wave packets (see \sec{description} for details). 

The $\CD$ gate is implemented using the graph shown in \fig{Graph-used-to-1}. In this section we specify the logical input states, the logical output states, the distances $X$, $Z$, and $W$ appearing in the figure, and the total evolution time. With these choices, we show that a $\CD$ gate is applied to the logical states at the end of the time evolution under the quantum walk Hamiltonian (up to error terms that are $\O(L^{-{1}/{4}})$). The results of this section pertain to the two-particle Hamiltonian $H^{(2)}_{G'}$ for the graph $G'$ shown in \fig{Graph-used-to-1}.

The logical input states are
\begin{equation*}
|0_{\text{in}}\rangle^c=\frac{1}{\sqrt{L}}\sum_{x=M(-\frac{\pi}{4})+1}^{M(-\frac{\pi}{4})+L}e^{-i\frac{\pi}{4}x}|x,1\rangle \qquad |1_\text{in}\rangle^c=\frac{1}{\sqrt{L}}\sum_{x=M(-\frac{\pi}{4})+1}^{M(-\frac{\pi}{4})+L}e^{-i\frac{\pi}{4}x}|x,2\rangle
\end{equation*}
for the computational qubit and
\begin{equation*}
|0_{\text{in}}\rangle^\med=\frac{1}{\sqrt{L}}\sum_{y=M(-\frac{\pi}{2})+1}^{M(-\frac{\pi}{2})+L}e^{-i\frac{\pi}{2}y}|y,4\rangle \qquad |1_\text{in}\rangle^\med=\frac{1}{\sqrt{L}}\sum_{y=M(-\frac{\pi}{2})+1}^{M(-\frac{\pi}{2})+L}e^{-i\frac{\pi}{2}y}|y,3\rangle
\end{equation*}
 for the mediator qubit. We define symmetrized (or antisymmetrized) logical input states for $a,b\in\{0,1\}$ as
\begin{align*}
|a b_{\text{in}}\rangle^{c,\med} &=\text{Sym}(|a_{\text{in}}\rangle^c |b_{\text{in}}\rangle^\med )\\
& =\frac{1}{\sqrt{2}} \left(|a_{\text{in}}\rangle^c |b_{\text{in}}\rangle^\med \pm |b_{\text{in}}\rangle^\med|a_{\text{in}}\rangle^c\right).
\end{align*}

%\begin{figure}
%\centering
%\capstart
%\begin{tikzpicture} [scale=1.5,
%	vert/.style={circle, draw=black, fill=black,inner sep=1pt, minimum width=0pt},
%	dots/.style={circle, fill=black,inner sep=1pt, minimum width=0pt},
%	switch/.style={circle,draw=black,inner sep=6pt,minimum width=0pt},
%	attach/.style={circle,fill=white, draw=black,inner sep=1pt, minimum width=0pt}]
%
%  \foreach \y in {0,0.5,3,3.5}{
%  \begin{scope}[yshift=\y cm]
%  \foreach \x in {0,1,2,6,7,8}{
%	\begin{scope}[xshift = \x cm]
%	  \node at (0,0) [vert]{};
%	\end{scope}}
%	\draw (0,0) -- (2,0);
%	\draw (6,0) -- (8,0);
%	\draw[densely dotted] (2,0)--(2.2,0);
%	\draw[densely dotted] (5.8,0)--(6,0);
%	\node at (0,0) [attach] {};
%	\node at (8,0) [attach] {};
% \end{scope}}
% 
%  
%  \node (switch2) at (4,0.5)[switch]{};
%  \node (switch1) at (4,3)[switch]{};
%  
%  \node at (3,0.5) [vert]{};
%  \node at (3,3) [vert]{};
%  \node at (5,0.5) [vert]{};
%  \node at (5,3) [vert]{};
%  
%  \draw (switch2.west) -- (3,0.5);
%  \draw[densely dotted] (3,0.5) -- (2.8,0.5);
%  \draw (switch2.east) -- (5,0.5);
%  \draw[densely dotted] (5,0.5) -- (5.2,0.5);
%  
%  \draw (switch1.west) -- (3,3);
%  \draw[densely dotted] (3,3) -- (2.8,3);
%  \draw (switch1.east) -- (5,3);
%  \draw[densely dotted] (5,3) -- (5.2,3);
%  
%  \node at (3.5,0) [vert]{};
%  \node at (4.5,0) [vert]{};
%  \node at (3.5,3.5) [vert]{};
%  \node at (4.5,3.5) [vert]{};
%  
%  \draw (3.5,0) -- (4.5,0);
%  \draw (3.5,3.5) -- (4.5,3.5);
%  \draw[densely dotted] (3.3,0) -- (4.7,0);
%  \draw[densely dotted] (3.3,3.5) -- (4.7,3.5);
%  
%
%  \node at (4,1.25) [vert] {};
%  \node at (4,2.25) [vert] {};
%  
%  \draw (switch2.north) -- (4,1.25);
%  \draw[densely dotted] (4,1.25) -- (4,1.45);
%  
%  \draw (switch1.south) -- (4,2.25);
%  \draw[densely dotted] (4,2.25)-- (4,2.05);
%  
%  \draw[line width=.7pt] (switch1.west) to[out=0,in=90] (switch1.south);
%  \draw[line width=2.1pt] (switch1.east) to[out=180,in=90] (switch1.south);
%  \draw[line width=.7pt,white] (switch1.east) to[out=180,in=90] (switch1.south);
%
%  \draw[line width=.7pt] (switch2.east) to[out=180,in=-90] (switch2.north);
%  \draw[line width=2.1pt] (switch2.west) to[out=0,in=-90] (switch2.north);
%  \draw[line width=.7pt,white] (switch2.west) to[out=0,in=-90] (switch2.north);
%  
%  \node at (switch2.north) [attach] {};
%  \node at (switch1.south) [attach] {};
%  \node at (switch2.east) [attach] {};
%  \node at (switch1.east) [attach] {};
%  \node at (switch2.west) [attach] {};
%  \node at (switch1.west) [attach] {};
%  
%  \node at (0,0) [below] {(1,4)};
%  \node at (1,0) [below] {(2,4)};
%  \node at (2,0) [below] {(3,4)};
%  \node at (3.5,0) [below] {};
%  \node at (4.5,0) [below] {};
%  \node at (6,0) [below] {($2X\!+\! Z\! + \! 4$,4)};
%  \node at (8,0) [below] {($2X\! +\! Z\! +\! 6$,4)};
%  
%  \node at (0,3.5) [above] {(1,1)};
%  \node at (1,3.5) [above] {(2,1)};
%  \node at (2,3.5) [above] {(3,1)};
%  \node at (3.5,3.5) [above] {};
%  \node at (4.5,3.5) [above] {};
%  \node at (6,3.5) [above] {($2W\! + \! Z\! + \! 2$,1)};
%  \node at (8,3.5) [above] {($2W\! + \! Z\! + \! 4$,1)};
%  
%  \node at (0,3) [below] {(1,2)};
%  \node at (1,3) [below] {(2,2)};
%  \node at (2,3) [below] {(3,2)};
%  \node at (3,3) [below] {($W\!-\! 1$,2)};
%  \node[anchor=south east] at (3.87,3) {($W$,2)};
%  
%  \node at (0,.5) [above] {(1,3)};
%  \node at (1,.5) [above] {(2,3)};
%  \node at (2,.5) [above] {(3,3)};
%  \node at (3,.5) [above] {($X\! -\! 1$,3)};
%  \node at (3.87,.5) [below left] {($X$,3)};
%  
%  \node at (8,.5) [above] {($2W\! + \! Z\! + \! 4$,2)};
%  \node at (6,.5) [above] {($2W\! + \! Z\! + \! 2$,2)};
%  \node at (4.13,.5) [below right] {($W\! + \! Z\! + \! 5$,2)};  
%  
%  \node at (8,3) [below] {($2X\! + \! Z\! + \! 6$,3)};
%  \node at (6,3) [below] {($2X\! + \! Z\! + \! 4$,3)};
%  \node at (4.13,3) [above right] {($X\! + \! Z\! + \! 7$,3)};
%  
%  \node at (4,2.87) [below right] {(1,5)};
%  \node at (4,2.25) [below right] {(2,5)};
%  
%  \node at (4,.63) [above right] {($Z$,5)};
%  \node at (4,1.25) [above right] {($Z\! -\! 1$,5)};
%  
%  \node at (-.4,0) [left] {$0_{\med,\text{in}}$};
%  \node at (-.4,.5) [left] {$1_{\med,\text{in}}$};
%  \node at (-.4,3) [left] {$1_{c,\text{in}}$};
%  \node at (-.4,3.5) [left] {$0_{c,\text{in}}$};
%  
%  \node at (8.7,0) [right] {$0_{\med,\text{out}}$};
%  \node at (8.7,.5) [right] {$1_{c,\text{out}}$};
%  \node at (8.7,3) [right] {$1_{\med,\text{out}}$};
%  \node at (8.7,3.5) [right] {$0_{c,\text{out}}$};
%\end{tikzpicture}
%\caption{Graph $G'$ used to implement the $\CD$ gate. The integers $Z$, $X$, and $W$ are specified in equations \eq{Z_eq}, \eq{X_eq}, and \eq{W_eq}, respectively.}
%\label{fig:Graph-used-to-1}
%\end{figure}

We choose the distances $Z$, $X$, and $W$ from \fig{Graph-used-to-1}
to be \begin{align}
Z & = 4L \label{eq:Z_eq} \\
X & = d_{2}+L+M\left(-\frac{\pi}{2}\right) \label{eq:X_eq}\\
W & = d_{1}+L+M\left(-\frac{\pi}{4}\right) \label{eq:W_eq}
\end{align}
where
\begin{align*}
d_{1} & = M\left(-\frac{\pi}{4}\right) \\
d_{2} & = \left\lceil \frac{5L+2d_{1}}{\sqrt{2}}-\frac{5}{2}L\right\rceil. \end{align*}
With these choices, a wave packet moving with speed $\sqrt{2}$ travels
a distance $Z+2d_{1}+L=5L+2d_{1}$ in approximately the same time that
a wave packet moving with speed $2$ takes to travel a distance $Z+2d_{2}+L=5L+2d_{2}$,
since
\[
t_{\mathrm{II}}=\frac{5L+2d_{1}}{\sqrt{2}}\approx\frac{5L+2d_{2}}{2}.
\]

We claim that the logical input states evolve into logical output states (defined below) with a phase of $e^{i\theta}$ applied in the case where both particles are in the logical state $1$.  Specifically,
\begin{align}
\left\Vert e^{-iH_{G'}^{(2)}t_{\mathrm{II}}}|00_{\text{in}}\rangle^{c,\med}-|00_{\text{out}}\rangle^{c,\med}\right\Vert  & = \O(L^{-{1}/{4}})\label{eq:bound00}\\
\left\Vert e^{-iH_{G'}^{(2)}t_{\mathrm{II}}}|01_{\text{in}}\rangle^{c,\med}-|01_{\text{out}}\rangle^{c,\med}\right\Vert  & = \O(L^{-{1}/{4}})\label{eq:bound01}\\
\left\Vert e^{-iH_{G'}^{(2)}t_{\mathrm{II}}}|10_{\text{in}}\rangle^{c,\med}-|10_{\text{out}}\rangle^{c,\med}\right\Vert  & = \O(L^{-{1}/{4}})\label{eq:bound10}\\
\left\Vert e^{-iH_{G'}^{(2)}t_{\mathrm{II}}}|11_{\text{in}}\rangle^{c,\med} - e^{i\theta}|11_{\text{out}}\rangle^{c,\med}\right\Vert  & = \O(L^{-{1}/{4}})\label{eq:bound11}
\end{align}
 where, letting $Q_{1}=2W+Z+4-M\left(-{\pi}/{4}\right)-L$ and $Q_{2}=2X+Z+6-M\left(-{\pi}/{2}\right)-L$,
\begin{align*}
|0_\text{out}\rangle^c &=\frac{e^{-it_{\mathrm{II}}\sqrt{2}}}{\sqrt{L}}\sum_{x=Q_{1}+1}^{Q_{1}+L}e^{-i\frac{\pi}{4}x}|x,1\rangle &
|1_\text{out}\rangle^c &=\frac{e^{-it_{\mathrm{II}}\sqrt{2}}}{\sqrt{L}}\sum_{x=Q_{1}+1}^{Q_{1}+L}e^{-i\frac{\pi}{4}x}|x,2\rangle \\
|0_\text{out}\rangle^\med &=\frac{1}{\sqrt{L}}\sum_{y=Q_{2}+1}^{Q_{2}+L}e^{-i\frac{\pi}{2}y}|y,4\rangle &|1_\text{out}\rangle^\med &=\frac{1}{\sqrt{L}}\sum_{y=Q_{2}+1}^{Q_{2}+L}e^{-i\frac{\pi}{2}y}|y,3\rangle
\end{align*}
and $|a b_\text{out}\rangle^{c,\med}=\text{Sym}\left(|a_{\text{out}}\rangle^c|b_{\text{out}}\rangle^\med\right)$. 

Note that the input states are wave packets located a distance $M(k)$ from the ends of the input paths on the left-hand side of the graph in \fig{Graph-used-to-1}. Similarly, the output logical states are wave packets located a distance $M(k)$ from the ends of the output paths on the right-hand side.

The first three bounds \eq{bound00}, \eq{bound01}, and \eq{bound10} are relatively easy to show, since in each case the two particles are supported on disconnected subgraphs and therefore do not interact. In each of these three cases we can simply analyze the propagation of the one-particle starting states through the graph. The symmetrized (or antisymmetrized) starting state then evolves into the symmetrized (or antisymmetrized) tensor product of the two output states.

For example, with input state $|00_{\text{in}}\rangle^{c,\med}$, the evolution of the particle with momentum $-{\pi}/{4}$ occurs only on the top path and the evolution of the particle with momentum $-{\pi}/{2}$ occurs only on the bottom path. Starting from the initial state $|0_\text{in}\rangle^c$ and evolving for time $t_{\mathrm{II}}$ with the single-particle Hamiltonian for the top path, we obtain the final state
\[
|0_\text{out}\rangle^c+\O(L^{-{1}/{4}})
\]
using the method of \sec{truncating}. Similarly, starting from the initial state $|0_\text{in}\rangle^\med$ and evolving for time $t_{\mathrm{II}}$ with the single-particle Hamiltonian
for the bottom path of the graph we obtain the final state
\[
|0_\text{out}\rangle^{\med}+\O(L^{-{1}/{4}}).
\]
Putting these bounds together we get the bound \eq{bound00}.

In the case where the input state is $|10_{\text{in}}\rangle^{c,\med}$ (or $|01_{\text{in}}\rangle^{c,\med}$) the single-particle evolution for the particle with momentum $-{\pi}/{4}$ (or $-{\pi}/{2}$) is slightly more complicated, as in this case the particle moves through the momentum switches and the vertical path. The S-matrix of the momentum switch at the relevant momenta is given by equation \eq{switch_S}. At momentum $-{\pi}/{4}$, the momentum switch has the same S-matrix as a path with $4$ vertices (including the input and output vertices). At momentum $-{\pi}/{2}$, it has the same S-matrix as a path with $5$ vertices (including input and output vertices). Note that our labeling of vertices on the output paths (in \fig{Graph-used-to-1}) takes this into account. The first vertices on the output paths connected to the momentum switches are labeled $(X+Z+7,3)$ and $(W+Z+5,2)$, respectively, reflecting the fact that a particle with momentum $-{\pi}/{4}$ has traveled $W$ vertices on the input path, $Z$ vertices through the middle segment, and has effectively traveled an additional $4$ vertices inside the two switches. Similarly, a particle with momentum $-{\pi}/{2}$ effectively sees an additional $6$ vertices from the two momentum switches.

To get the bound \eq{bound10} we have to analyze the single-particle evolution
for the computational particle initialized in the state $|1_\text{in}\rangle^c$. 
We claim that, after time $t_{\mathrm{II}}$, the time-evolved state is
\[
|1_\text{out}\rangle^c+\O(L^{-{1}/{4}}).
\]
It is easy to see why this should be the case in light of our discussion above: when scattering at momentum $-{\pi}/{4}$, the graph in \fig{Graph-used-to-1} is equivalent to one where each momentum switch is replaced by a path with $2$ internal vertices connecting the relevant input/output vertices.

To make this precise, we use the method described in \sec{more_complicated_graphs} for analyzing scattering through sequences of overlapping graphs using the truncation lemma. Here we should choose subgraphs $G_{1}$ and $G_{2}$ of the graph $G'$ in \fig{Graph-used-to-1} that overlap on the vertical path but where each subgraph contains only one of the momentum switches. A convenient choice is to take $G_{1}$ to be the subgraph containing the top switch and the paths connected to it (the vertices $(1,2),\ldots,(W,2)$, $(1,5),\ldots,(Z,5)$ and $(X+Z+7,3),\ldots,(2X+Z+6,3)$). Similarly, choose $G_{2}$ to be the bottom switch along with the three paths connected to it. The graphs $G_{1}$ and $G_{2}$ both contain the vertices $(1,5),\ldots,(Z,5)$ along the vertical path. Break up the total evolution time into two intervals $[0,t_{\alpha}]$ and $[t_{\alpha},t_{\mathrm{II}}]$. Choose $t_{\alpha}$ so that the wave packet, evolved for this time with $H_{G_1}^{(1)}$, travels through the top switch and ends up a distance $\Theta(L)$ from each switch, partway along the vertical path (up to terms bounded as $\O(L^{-{1}/{4}})$, as in \sec{truncating}). With this choice, the single-particle evolution with the Hamiltonian for the full graph is approximated by the evolution with $H_{G_1}^{(1)}$ on this time interval (see \sec{more_complicated_graphs}). At time $t_\alpha$, the particle is outgoing with respect to scattering from the graph $G_1$, but incoming with respect to $G_2$. On the interval $[t_{\alpha},t_{\mathrm{II}}]$ the time evolution is approximated by evolving the state with $H_{G_2}^{(1)}$. During this time interval the particle travels through the bottom switch onto the final path, and at $t_{\mathrm{II}}$ is a distance $M(-{\pi}/{4})$ from the endpoint of the output path. Both switches have the same S-matrix (at momentum $-{\pi}/{4}$) as a path of length $4$, so this analysis gives the output state $|10_\text{out}\rangle^{c,\med}$ up to terms bounded as $\O(L^{-{1}/{4}})$, establishing \eq{bound10}. For the bound \eq{bound01}, we apply a similar analysis to the trajectory of the mediator particle.

The case where the input state is $|11_{\text{in}}\rangle^{c,\med}$ is more involved but proceeds similarly. In this case, to analyze the time evolution we divide the time interval $[0,t_{\mathrm{II}}]$ into three segments $[0,t_{A}]$, $[t_{A},t_{B}]$, and $[t_{B},t_{\mathrm{II}}]$. For each of these three time intervals we choose a subgraph $G_{A}$, $G_{B}$, $G_C$ of the graph $G'$ in \fig{Graph-used-to-1} and we approximate the time evolution by evolving with the Hamiltonian on the associated subgraph. We then use the truncation lemma to show that, on each time interval, the evolution generated by the Hamiltonian for the appropriate subgraph approximates the evolution generated by the full Hamiltonian, with error $\O(L^{-{1}/{4}})$. Up to these error terms, at times $t=0$, $t=t_A$, $t=t_B$, and $t=t_{\mathrm{II}}$ the time-evolved state 
\[
e^{-iH_{G'}^{(2)}t}|11_{\text{in}}\rangle^{c,\med}
\]
has both particles in square wave packet states, each with support only on $L$ vertices of the graph, as depicted in \fig{11_scattering_cartoon}.

We take $G_A$ to be the subgraph obtained from $G'$ by removing the vertices labeled $(\lceil 1.85L\rceil,5)\allowbreak, \ldots,\allowbreak (\lceil 1.90 L\rceil,5)$ in the vertical path. By removing this interval of consecutive vertices, we disconnect the graph into two components where the initial state $|11_{\text{in}}\rangle^{c,\med}$ has one particle in each component. This could be achieved by removing a single vertex, but instead we remove an interval of approximately $0.05L$ vertices to separate the components of $G_A$ by more than the interaction range $C$ (for sufficiently large $L$), simplifying our use of the truncation lemma.

 We choose $t_{A}={3L}/{2}$. Consider the time evolution of the initial state $|11_\text{in}\rangle^{c,\med}$ with the two-particle Hamiltonian $H_{G_A}^{(2)}$ for time $t_A$. The states $|1_ {\text{in}}\rangle^c$ and $|1_\text{in}\rangle^{\med}$ are supported on disconnected components of the graph $G_A$, so we can analyze the time evolution of the state $|11_\text{in}\rangle^{c,\med}$ under $H_{G_A}^{(2)}$ by analyzing two single-particle problems, using the results of \sec{truncating} for each particle. During the interval $[0,t_A]$,  each particle passes through one switch, ending up a distance $\Theta(L)$ from the switch that it passed through and $\Theta(L)$ from the vertices that have been removed, as shown in \fig{11_scattering_cartoon}(b) (with error at most $\O(L^{-{1}/{4}})$). Up to these error terms, the support of each particle remains at least $N_0=\Theta(L)$ vertices from the endpoints of the graph, so we can apply the truncation lemma using $H=H_{G'}^{(2)}$, $W=\tilde{H}=H_{G_A}^{(2)}$, $T=t_{\mathrm{A}}$, and $\delta=\O(L^{-{1}/{4}})$. Here $P$ is the projector onto states where both particles are located at vertices of $G_A$. We have $P H_{G'}^{(2)}P=H_{G_A}^{(2)}$ since the number of vertices in the removed segment is greater than the interaction range $C$. Applying the truncation lemma gives
\[
\left\Vert e^{-iH_{G_A}^{(2)}t_A}|11_\text{in}\rangle^{c,\med}-e^{-iH_{G'}^{(2)}t_A}|11_\text{in}\rangle^{c,\med}\right\Vert=\O(L^{-{1}/{4}}).
\]

We approximate the evolution on the interval $[t_A,t_B]$ using the two-particle  Hamiltonian $H_{G_B}^{(2)}$, where $G_B$  is the vertical path $(1,5),\ldots,(Z,5)$. Using the result of \sec{truncating}, we know that (up to terms bounded as $\O(L^{-{1}/{4}})$) the wave packets move with their respective speeds and acquire a phase of $e^{i\theta}$ as they pass each other. We choose $t_B={5L}/{2}$ so that during the evolution the wave packets have no support on vertices within a distance $\Theta(L)$ from the endpoints of the vertical segment where the graph has been truncated (again up to terms bounded as $\O(L^{-{1}/{4}})$). Using $H_{G_B}^{(2)}$ (rather than $H_{G'}^{(2)}$) to evolve the state on this interval, we incur errors bounded as $\O(L^{-{1}/{4}})$ (using the truncation lemma with $N_0=\Theta(L)$, $W=\tilde{H}=H_{G_B}^{(2)}$, $H=H_{G'}^{(2)}$, and $\delta=\O(L^{-{1}/{4}})$).

We choose $G_C=G_A$; in the final interval $[t_{B},t_{\mathrm{II}}]$ we evolve using the Hamiltonian  $H_{G_A}^{(2)}$ again, and we use the truncation lemma as we did for the first interval. The initial state is approximated by two wave packets supported on disconnected sections of $G_A$ and the evolution of this initial state reduces to two single-particle scattering problems. During the interval $[t_B,t_{\mathrm{II}}]$, each particle passes through a second switch, and at time $t_{\mathrm{II}}$ is a distance $M(k)$ from the end of the appropriate output path. 

Our analysis shows that for the input state $|11_\text{in}\rangle^{c,\med}$ the only effect of the interaction is to alter the global phase of the final state by a factor of $e^{i\theta}$ relative to the case where no interaction is present, up to error terms bounded as $\O(L^{-{1}/{4}})$. This establishes equation \eq{bound11}. In \fig{11_scattering_cartoon} we illustrate the movement of the two wave packets through the graph when the initial state is $|11_\text{in}\rangle^{c,\med}$.

%\begin{figure}
%\centering
%\capstart
%\begin{tikzpicture}[   scale=0.7,   
%	dots/.style={circle,draw=black,fill=black,inner sep=0pt,minimum size=.5 mm}, 
%	splitter/.style={circle,draw=black,fill=white,inner sep=0pt,minimum size=4mm},   
%	attach/.style={circle,draw=black,fill=white,inner sep=0pt,minimum size=1mm}]
%	
%\begin{scope}[yshift= 6 cm]
%  \node at (-.75,4) {(a)};
%
%  \draw (0,0) to (10,0);   
%  \draw (0,3) to (10,3);
%  \foreach \x in {0,.2,.4,...,10}{
%  \node at (\x, 0) [dots] {};
%  \node at (\x, 3) [dots] {};
%  }
%  \foreach \y in {0,.2,.4,...,3}
%  \node at (5,\y) [dots] {};
%  
%  \node (splitter42) at (5,0) [splitter] {};   
%  \node (splitter41) at (5,3) [splitter] {};   
%  \draw (splitter41.south) to (splitter42.north);      
%  \draw (splitter41.west) to[out=0,in=90] (splitter41.south);   
%  \draw[line width=1.2pt] (splitter41.south) to[out=90,in=180]           
%  (splitter41.east);   
%  \draw[line width=.4pt,white] (splitter41.south) to[out=90,in=180] (splitter41.east);
%  \draw[line width=1.2pt] (splitter42.west) to[out=0,in=-90]           (splitter42.north);   
%  \draw[line width=.4pt,white] (splitter42.west) to[out=0,in=-90]           (splitter42.north);   
%  \draw (splitter42.north) to[out=-90,in=180] (splitter42.east);
%  \node at (splitter41.east) [attach]{};
%  \node at (splitter41.west) [attach]{};
%  \node at (splitter41.south) [attach]{};
%  \node at (splitter42.east) [attach]{};
%  \node at (splitter42.west) [attach]{};
%  \node at (splitter42.north) [attach]{};
%  
%  \node at (0,0) [attach]{};
%  \node at (0,3) [attach]{};
%  \node at (10,0) [attach]{};
%  \node at (10,3) [attach]{};  
%
%  \draw (1.05,3.15) to (1.25,3.15) to (1.25,3.6) to node[above]          
%    {${\pi}/{4}\rightarrow$} (3.25,3.6) to (3.25,3.15) to (3.45,3.15);
%  \draw (1.05,0.15) to (1.25,0.15) to (1.25,0.6) to node[above]          
%    {${\pi}/{2}\rightarrow$} (3.25,0.6) to (3.25,0.15) to (3.45,0.15);
%  \draw[|-|] (0,2.5) to node[below] {$M(-\frac{\pi}{4})$} (1.25,2.5);   
%  \draw[|-|] (1.25,2.5) to node[below]{$L$}   (3.25,2.5);   
%  \draw[|-|] (3.25,2.5) to node[below]{$d_1$} (4.75,2.5);
%  \draw[|-|] (0,-.5) to node[below] {$M(-\frac{\pi}{2})$} (1.25,-.5);   
%  \draw[|-|] (1.25,-.5) to node[below]{$L$}   (3.25,-.5);   
%  \draw[|-|] (3.25,-.5) to node[below]{$d_2$} (4.75,-.5);
%\end{scope}
%  
%%----------------------------------------------------------------------------------------%
%
%\begin{scope}[xshift= 11.5cm, yshift=6cm]
%  \node at (-.75,4) {(b)};
%
%  \draw (0,0) to (10,0);   
%  \draw (0,3) to (10,3);
%  \foreach \x in {0,.2,.4,...,10}{
%  \node at (\x, 0) [dots] {};
%  \node at (\x, 3) [dots] {};
%  }
%  \foreach \y in {0,.2,.4,...,3}
%  \node at (5,\y) [dots] {};
%  
%  \node (splitter42) at (5,0) [splitter] {};   
%  \node (splitter41) at (5,3) [splitter] {};   
%  \draw (splitter41.south) to (splitter42.north);      
%  \draw (splitter41.west) to[out=0,in=90] (splitter41.south);   
%  \draw[line width=1.2pt] (splitter41.south) to[out=90,in=180]           
%  (splitter41.east);   
%  \draw[line width=.4pt,white] (splitter41.south) to[out=90,in=180] (splitter41.east);
%  \draw[line width=1.2pt] (splitter42.west) to[out=0,in=-90]           (splitter42.north);   
%  \draw[line width=.4pt,white] (splitter42.west) to[out=0,in=-90]           (splitter42.north);   
%  \draw (splitter42.north) to[out=-90,in=180] (splitter42.east);
%  \node at (splitter41.east) [attach]{};
%  \node at (splitter41.west) [attach]{};
%  \node at (splitter41.south) [attach]{};
%  \node at (splitter42.east) [attach]{};
%  \node at (splitter42.west) [attach]{};
%  \node at (splitter42.north) [attach]{};
%  
%  \node at (0,0) [attach]{};
%  \node at (0,3) [attach]{};
%  \node at (10,0) [attach]{};
%  \node at (10,3) [attach]{};
%  \begin{scope}[yshift = -10 cm]
%    \draw (5.1,11.7) to (5.1,11.8) to (5.5,11.8) to node[right]          
%      {$\frac{\pi}{4}\downarrow$} (5.5,12.4) to (5.1,12.4) to (5.1,12.5);
%    \draw (4.9,11.3) to (4.9,11.2) to (4.5,11.2) to node[left]            
%      {$\frac{\pi}{2}\uparrow$} (4.5,10.6) to (4.9,10.6) to (4.9,10.5);
%    \draw[|-|] (7.95,12.85) to node[right] {$\approx 0.56L$} (7.95,12.4);   
%    \draw[|-|] (7.35,12.4)  to node[right] {$L$}   (7.35,11.8);   
%    \draw[|-|] (7.95,11.8)  to node[right] {$\approx 1.27L$} (7.95,11.2);   
%    \draw[|-|] (7.35,11.2)  to node[right] {$L$}   (7.35,10.6);   
%    \draw[|-|] (7.95,10.6)  to node[right] {$\approx 0.17L$} (7.95,10.15);
%   \end{scope}  
%\end{scope} 
%  
%%----------------------------------------------------------------------------------------%
%
%  \node at (-.75,4) {(c)};
%
%  \draw (0,0) to (10,0);   
%  \draw (0,3) to (10,3);
%  \foreach \x in {0,.2,.4,...,10}{
%  \node at (\x, 0) [dots] {};
%  \node at (\x, 3) [dots] {};
%  }
%  \foreach \y in {0,.2,.4,...,3}
%  \node at (5,\y) [dots] {};
%  
%  \node (splitter42) at (5,0) [splitter] {};   
%  \node (splitter41) at (5,3) [splitter] {};   
%  \draw (splitter41.south) to (splitter42.north);      
%  \draw (splitter41.west) to[out=0,in=90] (splitter41.south);   
%  \draw[line width=1.2pt] (splitter41.south) to[out=90,in=180]           
%  (splitter41.east);   
%  \draw[line width=.4pt,white] (splitter41.south) to[out=90,in=180] (splitter41.east);
%  \draw[line width=1.2pt] (splitter42.west) to[out=0,in=-90]           (splitter42.north);   
%  \draw[line width=.4pt,white] (splitter42.west) to[out=0,in=-90]           (splitter42.north);   
%  \draw (splitter42.north) to[out=-90,in=180] (splitter42.east);
%  \node at (splitter41.east) [attach]{};
%  \node at (splitter41.west) [attach]{};
%  \node at (splitter41.south) [attach]{};
%  \node at (splitter42.east) [attach]{};
%  \node at (splitter42.west) [attach]{};
%  \node at (splitter42.north) [attach]{};
% 
%  \node at (0,0) [attach]{};
%  \node at (0,3) [attach]{};
%  \node at (10,0) [attach]{};
%  \node at (10,3) [attach]{};
%  \begin{scope}[yshift = - 5cm]
%  \draw (5.1,6.3) to (5.1,6.20) to (5.5,6.20) to node[right]            {$\frac{\pi}{4}\downarrow$} (5.5,5.6) to (5.1,5.6) to (5.1,5.5);
%  \draw (4.9,6.7) to (4.9,6.8) to (4.5,6.8) to node[left]            {$\frac{\pi}{2}\uparrow$} (4.5,7.4) to (4.9,7.4) to (4.9,7.5);
%  \draw[|-|] (7.95,7.85) to node[right] {$\approx 0.83L$} (7.95,7.4);   
%  \draw[|-|] (7.35,7.4)  to node[right] {$L$}   (7.35,6.8);   
%  \draw[|-|] (7.95,6.8)  to node[right] {$\approx 0.14L$} (7.95,6.2);   
%  \draw[|-|] (7.35,6.2)  to node[right] {$L$}   (7.35,5.6);   
%  \draw[|-|] (7.95,5.6)  to node[right] {$\approx 1.03L$} (7.95,5.15);
%  \end{scope}
%  
%%----------------------------------------------------------------------------------------%
%\begin{scope}[xshift=11.5cm]
%  \node at (-.75,4) {(d)};
%
%  \draw (0,0) to (10,0);   
%  \draw (0,3) to (10,3);
%  \foreach \x in {0,.2,.4,...,10}{
%  \node at (\x, 0) [dots] {};
%  \node at (\x, 3) [dots] {};
%  }
%  \foreach \y in {0,.2,.4,...,3}
%  \node at (5,\y) [dots] {};
%  
%  \node (splitter42) at (5,0) [splitter] {};   
%  \node (splitter41) at (5,3) [splitter] {};   
%  \draw (splitter41.south) to (splitter42.north);      
%  \draw (splitter41.west) to[out=0,in=90] (splitter41.south);   
%  \draw[line width=1.2pt] (splitter41.south) to[out=90,in=180]           
%  (splitter41.east);   
%  \draw[line width=.4pt,white] (splitter41.south) to[out=90,in=180] (splitter41.east);
%  \draw[line width=1.2pt] (splitter42.west) to[out=0,in=-90]           (splitter42.north);   
%  \draw[line width=.4pt,white] (splitter42.west) to[out=0,in=-90]           (splitter42.north);   
%  \draw (splitter42.north) to[out=-90,in=180] (splitter42.east);
%  \node at (splitter41.east) [attach]{};
%  \node at (splitter41.west) [attach]{};
%  \node at (splitter41.south) [attach]{};
%  \node at (splitter42.east) [attach]{};
%  \node at (splitter42.west) [attach]{};
%  \node at (splitter42.north) [attach]{};
% 
%  \node at (0,0) [attach]{};
%  \node at (0,3) [attach]{};
%  \node at (10,0) [attach]{};
%  \node at (10,3) [attach]{};
%  \draw (6.55,3.15) to (6.75,3.15) to (6.75,3.6) to node[above]            {${\pi}/{2}\rightarrow$} (8.75,3.6) to (8.75,3.15) to (8.95,3.15);
%  \draw (6.55,.15) to (6.75,.15) to (6.75,.6) to node[above]            {${\pi}/{4}\rightarrow$} (8.75,.6)to (8.75,.15) to (8.95,.15);
%  \begin{scope}[yshift=-.25cm]
%  \draw[|-|] (5.25,2.75) to node[below] {$d_2$} (6.75,2.75);   
%  \draw[|-|] (6.75,2.75) to node[below]{$L$}   (8.75,2.75);   
%  \draw[|-|] (8.75,2.75) to node[below]{$M(-\frac{\pi}{2})$} (10,2.75);
%  \draw[|-|] (5.25,-.25) to node[below] {$d_1$} (6.75,-.25);   
%  \draw[|-|] (6.75,-.25) to node[below]{$L$}   (8.75,-.25);   
%  \draw[|-|] (8.75,-.25) to node[below]{$M(-\frac{\pi}{4})$} (10,-.25);
%  \end{scope}
%\end{scope}
%\end{tikzpicture}
%
%\caption{This picture illustrates the scattering process for two wave packets that are incident on the input paths as shown in figure (a) at time $t=0$. Figure (b) shows the location of the two wave packets after a time $t_{A}={3L}/{2}$ and figure (c) shows the wave packets after a time $t_{B}=t_{A}+L$. After the particles pass one another they acquire an overall phase of $e^{i\theta}$. Figure (d) shows the final configuration of the wave packets after a total evolution time $t_{\mathrm{II}}={(Z+2d_{1}+L)}/{\sqrt{2}}$.}
%\label{fig:11_scattering_cartoon}
%\end{figure}






\section{Universal Computation}
\subsection{Two-qubit blocks}
\subsection{Combining blocks}


\section{Improvements and Modifications}

What about long-range interactions, but where the interactions die off?
Additionally, what about error correction?

\end{document}