%======================================================================
%   Zak Webb
%   Ph. D. Thesis
%   Department of Physics and Astronomy
%   University of Waterloo
% 
%   Universality of multi-particle scattering
%======================================================================


\documentclass[../thesis-main/thesis-main]{subfiles}
\begin{document}

\chapter{Universality of multi-particle scattering}


Hard, but worthwhile



\section{Multi-particle quantum walk}

Note that this is exactly what I wanted to talk about.

Very difficult in general.

\subsection{Two-particle scattering on an infinite path}

The one thing we can actually compute
It might be interesting to talk about what happens with spins.




\section{Applying an encoded $C\theta$-gate}

\subsection{Finite truncation}


\begin{theorem}
\label{thm:twopart}Let $H^{(2)}$ be a two-particle Hamiltonian of the form \eq{twoham} with interaction range at most $C$, i.e., $\mathcal{V}(|r|)=0$ for all $|r|>C$. Let $\theta_{\pm}(p_1,p_2)$ be given by equation \eq{delta_pm}. Define $\theta=\theta_{\pm}({\pi}/{4},{3\pi}/{8})$. Let $L\in\natural$, let $M\in\{C+1,C+2,\ldots\}$, and define
\begin{align*}
|\chi_{z,k}\rangle & =  \frac{1}{\sqrt{L}}\sum_{x=z-L}^{z-1}e^{ikx}|x\rangle\\
|\psi(0)\rangle & =  \frac{1}{\sqrt{2}}\left(|\chi_{-M,-\frac{\pi}{2}}\rangle|\chi_{M+L+1,\frac{\pi}{4}}\rangle 
	\pm |\chi_{M+L+1,\frac{\pi}{4}}\rangle|\chi_{-M,-\frac{\pi}{2}}\rangle\right).
\end{align*}
Let $c_{0}$ be a constant independent of $L$. Then, for all $0\leq t\leq c_{0}L$, we have
\[
\left\Vert e^{-iH^{(2)}t}|\psi(0)\rangle-|\alpha(t)\rangle\right\Vert =\O(L^{-{1}/{8}}),
\]
where 
\begin{equation}
|\alpha(t)\rangle=\sum_{x,y}a_{xy}(t)|x,y\rangle,
\label{eq:alpha}
\end{equation}
$a_{xy}(t)=\pm a_{yx}(t)$, and, for $x\leq y$, 
\begin{align}
a_{xy}(t) & =  \frac{1}{\sqrt{2}L}e^{-\sqrt{2}it}\left[e^{-i \pi x/2} e^{i \pi y/4} F(x,y,t) 
    \pm e^{i\theta} e^{i \pi x/4} e^{-i \pi y/2} F(y,x,t) \right]
\label{eq:a_xy}
\end{align}
where
\begin{align*}
F(u,v,t) & =  \begin{cases}
	1 & \text{if }u-2 \lfloor t \rfloor\in\{-M-L,\ldots,-M-1\}\text{ and }v+2\left\lfloor \frac{t}{\sqrt{2}}
	\right\rfloor \in\{M+1,\ldots,M+L\}\\
	0 & \text{otherwise.}\end{cases}
\end{align*}
\end{theorem}

In this section we prove \thm{twopart}. The main proof appears in \sec{twopartproof}, relying on several technical lemmas proved in \sec{techlem}. The proof follows the method used in the single-particle case, which is based on the calculation from the appendix of reference \cite{FGG08}.

Recall from \eq{symscatter} that for each $p_{1}\in(-\pi,\pi)$ and $p_{2}\in(0,\pi)$ there is an eigenstate $|\text{sc}(p_{1};p_{2})\rangle_{\pm}$ of $H^{(2)}$ of the form 
\begin{align}
  \langle x,y|\text{sc}(p_{1};p_{2})\rangle_\pm &=
    \frac{e^{-ip_{1}\left(\frac{x+y}{2}\right)}}{\sqrt{2}} 
    \begin{cases} e^{-i p_2 (x-y)} \pm  e^{i \theta_{\pm}(p_1,p_2)} e^{i p_2 (x - y)} & \text{if } x - y \leq -C\\
     e^{- i p_2 (x-y)}e^{i\theta_{\pm}(p_1,p_2)} \pm e^{i p_2 (x-y)} & \text{if }x-y \geq C\\
    f(p_1,p_2,x-y) \pm f(p_1,p_2,y-x)& \text{if }|x-y|< C \end{cases}\label{eq:sc}
\end{align}
where
\begin{align*}
e^{i \theta_\pm (p_{1},p_{2})} & =  T(p_1,p_2) \pm R(p_1,p_2),
\end{align*}
$C$ is the range of the interaction, $T$ and $R$ are the transmission
and reflection coefficients of the interaction at the chosen momentum, $f$
describes the amplitudes of the scattering state within the interaction range, and the $\pm$ depends
on the type of particle ($+$ for bosons, $-$ for fermions).  The state
$|\text{sc}(p_{1};p_{2})\rangle_\pm$ satisfies 
\[
H^{(2)}|\text{sc}(p_{1};p_{2})\rangle_{\pm} = 4\cos \frac{p_{1}}{2}\cos p_{2} |\text{sc}(p_{1};p_{2})\rangle_\pm
\]
and is delta-function normalized as
\begin{equation}
_{\pm}\langle\text{sc}(p_{1}';p_{2}')|\text{sc}(p_{1};p_{2})\rangle_{\pm}=
  4 \pi^{2} \delta(p_{1} - p_{1}') \delta(p_{2} - p_{2}').
\label{eq:del_func_norm}
\end{equation}

\begin{proof}
Expand $|\psi(0)\rangle$ in the basis of eigenstates of the Hamiltonian
to get
\begin{align*}
|\psi(t)\rangle & = e^{-iH^{(2)}t}|\psi(0)\rangle = |\psi_{1}(t)\rangle+|\psi_{2}(t)\rangle
\end{align*}
where 
\[
  \ket{\psi_{1}(t)} = \iint_{D_{\epsilon}} \frac{d\phi_{1}d\phi_{2}} 
    {4\pi^{2}} e^{-it 4\cos(\frac{p_{1}}{2}+\frac{\phi_{1}}{2})
      \cos(p_{2} + \phi_{2})}|\text{sc}(p_{1}+\phi_{1};p_{2}+ 
    \phi_{2})\rangle_{\pm} \left({_{\pm}\langle\text{sc}(p_{1}+\phi_{1};p_{2}+\phi_{2})|\psi(0)\rangle}\right)
\]
with $D_{\epsilon}=\left[-\epsilon,\epsilon\right]\times\left[-\epsilon,\epsilon\right]$, $p_{1}= {\pi}/{2}-{\pi}/{4}=
{\pi}/{4}$, $p_{2}=({\pi}/{2} +
{\pi}/{4})/2={3\pi}/{8}$,
and with $|\psi_{2}(t)\rangle$ orthogonal to $|\psi_{1}(t)\rangle$.
We take $\epsilon=a/\sqrt{L}$ for some constant $a$. Using equation \eq{sc}
we get 
\[
  \ket{\psi_{1}(t)}=|\psi_{A}(t)\rangle\pm|\psi_{B}(t)\rangle
\]
where
\begin{align}
  |\psi_{A}(t)\rangle & = \iint_{D_{\epsilon}} \frac{d\phi_{1}d\phi_{2}}
    {4\pi^{2}} e^{-it 4\cos(\frac{\pi}{8}+\frac{\phi_{1}}{2})
    \cos(\frac{3\pi}{8}+\phi_{2})}
     A(\phi_{1},\phi_{2})|\text{sc}(\tfrac{\pi}{4}+\phi_{1};
     \tfrac{3\pi}{8}+\phi_{2})\rangle_{\pm} \label{eq:psiA} \\
 \ket{\psi_{B}(t)} & = \iint_{D_{\epsilon}} \frac{d\phi_{1}d\phi_{2}}
    {4\pi^{2}} e^{-it 4\cos(\frac{\pi}{8}+\frac{\phi_{1}}{2})
    \cos(\frac{3\pi}{8}+\phi_{2})}  e^{-i\theta_{\pm}(\tfrac{\pi}{4}+\phi_{1},\tfrac{3\pi}{8}+\phi_{2} )}B(\phi_{1},\phi_{2},\tfrac{3\pi}{8})
 	|\text{sc}(\tfrac{\pi}{4}\! +\!\phi_{1};\tfrac{3\pi}{8}\! +\!\phi_{2})\rangle_\pm \nonumber 
\end{align}
with
\begin{align}
A(\phi_{1},\phi_{2}) & = \frac{1}{L}\sum_{x=-(M+L)}^{-(M+1)}\sum_{y=M+1}^{M+L} 
    e^{i\phi_{1}\frac{x+y}{2}}e^{i\phi_{2}\left(x-y\right)}\label{eq:A}\\
B(\phi_{1},\phi_{2},k) & = \frac{1}{L}\sum_{x=-(M+L)}^{-(M+1)}
    \sum_{y=M+1}^{M+L}e^{i\phi_{1}\frac{x+y}{2}} 	
    e^{i\left(\phi_{2}+2 k\right)\left(y-x\right)}.\nonumber 
\end{align}
 Using the delta-function normalization of the scattering states (equation
\eq{del_func_norm}) we get 
\begin{align*}
\langle\psi_{B}(t)|\psi_{B}(t)\rangle & = \iint_{D_{\epsilon}}\frac{d\phi_{1}d\phi_{2}}{4\pi^{2}}\left|B(\phi_{1},\phi_{2},\tfrac{3\pi}{8})\right|^{2}\\
 & \leq \frac{16\pi^{2}}{L^{2}\epsilon^{2}}
\end{align*}
by \lem{Let--and} (as long as $\epsilon<{3\pi}/{8}$, which holds for $L$ sufficiently large).
Similarly, 
\begin{align*}
1 &\geq\langle\psi_{A}(t)|\psi_{A}(t)\rangle \\
&= \iint_{D_{\epsilon}}\frac{d\phi_{1}d\phi_{2}}{4\pi^{2}}\left|A(\phi_{1},\phi_{2})\right|^{2}\\
 & \geq 1-\frac{4\pi}{L\epsilon}
\end{align*}
(from the first two facts in \lem{Let--and}) and therefore
\begin{align*}
\langle\psi_{1}(t)|\psi_{1}(t)\rangle & = \langle\psi_{A}(t)|\psi_{A}(t)\rangle+\langle\psi_{B}(t)|\psi_{B}(t)\rangle+\langle\psi_{A}(t)|\psi_{B}(t)\rangle+\langle\psi_{B}(t)|\psi_{A}(t)\rangle\\
 & \geq 1-\frac{4\pi}{L\epsilon}-2\left|\langle\psi_{A}(t)|\psi_{B}(t)\rangle\right|\\
 & \geq 1-\frac{4\pi}{L\epsilon}-2\left|\langle\psi_{A}(t)|\psi_{A}(t)\rangle\right|^{\frac{1}{2}}\left|\langle\psi_{B}(t)|\psi_{B}(t)\rangle\right|^{\frac{1}{2}}\\
 & \geq 
 1-\frac{12\pi}{L\epsilon}.
\end{align*}

Hence 
\[
  \langle\psi_{2}(t)|\psi_{2}(t)\rangle\leq\frac{12\pi}{L\epsilon}
\]
since 
\[
\langle\psi(t)|\psi(t)\rangle=\langle\psi_{1}(t)|\psi_{1}(t)\rangle+\langle\psi_{2}(t)|\psi_{2}(t)\rangle=1.
\]
Thus
\begin{align*}
\left\Vert \,|\psi(t)\rangle-|\psi_{A}(t)\rangle\right\Vert  & = \left\Vert \,|\psi_{B}(t)\rangle+|\psi_{2}(t)\rangle\right\Vert \\
 & \leq \left\Vert \,|\psi_{B}(t)\rangle\right\Vert +\left\Vert \,|\psi_{2}(t)\rangle\right\Vert \\
 & \leq \frac{4\pi}{L\epsilon}+\sqrt{\frac{12\pi}{L\epsilon}}.
\end{align*}
Now 
\begin{align*}
\left\Vert \,|\psi(t)\rangle-|\alpha(t)\rangle\right\Vert  & \leq \left\Vert \,|\psi(t)\rangle-|\psi_{A}(t)\rangle\right\Vert +\left\Vert \,|\psi_{A}(t)\rangle-|\alpha(t)\rangle\right\Vert \\
 & \leq \frac{4\pi}{L\epsilon}+\sqrt{\frac{12\pi}{L\epsilon}}+\left\Vert \,|\psi_{A}(t)\rangle-|\alpha(t)\rangle\right\Vert \\
 & = \O(L^{-{1}/{4}})+\left\Vert \,|\psi_{A}(t)\rangle-|\alpha(t)\rangle\right\Vert 
\end{align*}
using our choice $\epsilon=a/\sqrt{L}$. To complete the
proof, we now show that the second term in this expression is bounded by $\O(L^{-{1}/{8}})$.
\begin{lemma}
With $|\psi_{A}(t)\rangle$ and $|\alpha(t)\rangle$ defined through
equations \eq{psiA} and \eq{alpha}, with $t\leq c_{0}L$ (for some constant $c_{0}$), 
\[
\left\Vert \,|\psi_{A}(t)\rangle-|\alpha(t)\rangle\right\Vert = \O(L^{-{1}/{8}}).
\]
\end{lemma}
\begin{proof}
To simplify matters, note that both $|\psi_{A}(t)\rangle$ and $|\alpha(t)\rangle$ are either symmetric or anti-symmetric (i.e., $\langle x,y|\alpha(t)\rangle=\pm\langle y,x|\alpha(t)\rangle$ and $\langle x,y|\psi_{A}(t)\rangle=\pm\langle y,x|\psi_{A}(t)\rangle$).  Taking $C$ to be the maximum range of the interaction in our Hamiltonian, we have
\[
\left\Vert \,|\psi_{A}(t)\rangle-|\alpha(t)\rangle\right\Vert \leq2\left\Vert P_{1}|\psi_{A}(t)\rangle-P_{1}|\alpha(t)\rangle\right\Vert  + \left\Vert P_2 \ket{\psi_A(t)}\right\Vert + \left \Vert P_2 \ket{\alpha(t)}\right\Vert,
\]
where
\[
     P_{1}=\sum_{y-x \geq C}|x,y\rangle\langle x,y| 
     \qquad P_2 = \sum_{|x-y| < C} \ket{x,y}\bra{x,y}.
\]

Now, for $y-x\geq C$, 
\begin{align*}
\langle x,y|\psi_{A}(t)\rangle & = \iint_{D_{\epsilon}}
  \frac{d\phi_{1}d\phi_{2}}{4\pi^{2}}
  e^{-it 4\cos(\frac{\pi}{8}+\frac{\phi_{1}}{2})
        \cos(\frac{3\pi}{8}+\phi_{2})} A(\phi_{1},\phi_{2})\frac{e^{-i\left(\frac{\pi}{4}+\phi_{1}\right)
       \left(\frac{x+y}{2}\right)}}{\sqrt{2}} \\
 & \qquad \left(e^{i\left(\frac{3\pi}
       {8}+\phi_{2}\right)\left(y-x\right)}\pm e^{-i\left(\frac{3\pi}{8}
       +\phi_{2}\right)
     \left(y-x\right)+
     i\theta_{\pm}(\frac{\pi}{4}+\phi_{1},\frac{3\pi}{8}+\phi_{2})}
      \right)\\
 & = \iint_{D_{\epsilon}}\frac{d\phi_{1}d\phi_{2}}{4\pi^{2}}
      \bigg[\frac{1}{\sqrt{2}} e^{-it 4\cos(\frac{\pi}{8} + \frac{\phi_{1}}{2}) 
      \cos(\frac{3\pi}{8}+\phi_{2})} A(\phi_{1},\phi_{2}) \\
&    \qquad \left( e^{-i\pi x/2} e^{i\pi y/4} e^{-i\phi_{1}\left(\frac{x+y}{2}\right)}
 	 e^{i\phi_{2}\left(y-x\right)}\right.\\
& \qquad\quad\left.\pm  e^{i \pi x/4} e^{-i\pi y/2} e^{-i\phi_{1}
      \left(\frac{x+y}{2}\right)}
 	 e^{-i\phi_{2}\left(y-x\right)} e^{i \theta_{\pm}\left(\frac{\pi}{4} 
             + \phi_1 , \frac{3\pi}{8} + 
 	 \phi_2\right)} \right)\bigg].
\end{align*}

From \lem{a_xy}, for $x\leq y$, the state $\ket{\alpha(t)}$ takes the form
\begin{align*}
\langle x,y|\alpha(t)\rangle & = \frac{1}{\sqrt{2}}e^{-it\sqrt{2}}\left[ e^{-i\pi x /2}e^{i \pi y/4}  
	\left(\iint_{D_{\pi}}\frac{d\phi_{1}d\phi_{2}}{4\pi^{2}}\right.\right.\\
 &	\qquad  \left.\left. A(\phi_{1},\phi_{2}) 		
	e^{- i\phi_{1}\left(-\left\lfloor t\right\rfloor +\left\lfloor \frac{t}{\sqrt{2}}\right\rfloor +\frac{x+y}
	{2}\right)}e^{-2i\phi_{2}\left(-\left\lfloor t\right\rfloor -\left\lfloor \frac{t}{\sqrt{2}}\right\rfloor +\frac{x-y}
	{2}\right)}\right)\right.\\
 & \quad \pm e^{i\theta} e^{i\pi x/4}e^{-i \pi y/2} 
 	\left(\iint_{D_{\pi}}\frac{d\phi_{1}d\phi_{2}}{4\pi^{2}}\right.\\
 & \qquad \left.\left. A(\phi_{1},\phi_{2})
 	e^{-i\phi_{1}\left(-\left\lfloor t\right\rfloor +\left\lfloor \frac{t}{\sqrt{2}}\right\rfloor +\frac{x+y}{2}\right)}
 	e^{-2i\phi_{2}\left(-\left\lfloor t\right\rfloor -\left\lfloor \frac{t}{\sqrt{2}}\right\rfloor +\frac{y-x}
 	{2}\right)}\right)\right],
 \end{align*}
 where $D_{\pi}=[-\pi,\pi]\times[-\pi,\pi]$. Using these expressions for $\ket{\psi_A(t)}$ and $\ket{\alpha(t)}$,
we now write 
\[
P_{1}|\psi_{A}(t)\rangle-P_{1}|\alpha(t)\rangle=\pm |e_{1}(t)\rangle+|e_{2}(t)\rangle \pm|f_{1}(t)\rangle+|f_{2}(t)\rangle\pm|g_{1}(t)\rangle+|g_{2}(t)\rangle\pm|h(t)\rangle
\]
where each term in the above equation is supported only on states
$|x,y\rangle$ such that $y-x \geq C$, and (for $y - x \geq C$)
\begin{align*}
\langle x,y|e_{1}(t)\rangle & = 
	\frac{e^{i\theta}}{\sqrt{2}} e^{-it\sqrt{2}}
	e^{i \pi x/4}e^{-i\pi y/2}\iint_{D_{\pi}}\frac{d\phi_{1}d\phi_{2}}{4\pi^{2}}
	A(\phi_{1},\phi_{2}) \bigg[e^{-i\phi_{1}\left(-t+\frac{t}{\sqrt{2}}+\frac{x+y}{2}\right)}\\
&  \qquad\qquad e^{-2i\phi_{2}\left(-t-\frac{t}{\sqrt{2}}+\frac{y-x}{2}\right)}
	-e^{-i\phi_{1}\left(-\left\lfloor t\right\rfloor +\left\lfloor \frac{t}{\sqrt{2}}\right\rfloor +\frac{x+y}{2}
 		\right)}e^{-2i\phi_{2}\left(-\left\lfloor t\right\rfloor -\left\lfloor \frac{t}{\sqrt{2}}\right\rfloor 
 		+\frac{y-x}{2}\right)}\bigg]\\
\langle x,y|e_{2}(t)\rangle & = 
	\frac{1}{\sqrt{2}}e^{-it\sqrt{2}}e^{-i \pi x/2}e^{i\pi y/4}
	\iint_{D_{\pi}}\frac{d\phi_{1}d\phi_{2}}{4\pi^{2}}A(\phi_{1},\phi_{2})\bigg[e^{-i\phi_{1}
	\left(-t+\frac{t}{\sqrt{2}}+\frac{x+y}{2}\right)}\\
& \qquad\qquad e^{-2i\phi_{2}\left(-t-\frac{t}{\sqrt{2}}+\frac{x-y}{2}\right)}
		-e^{-i\phi_{1}\left(-\left\lfloor t\right\rfloor +\left\lfloor \frac{t}{\sqrt{2}}\right\rfloor +
		\frac{x+y}{2}\right)}e^{-2i\phi_{2}\left(-\left\lfloor t\right\rfloor -\left\lfloor 
		\frac{t}{\sqrt{2}}\right\rfloor +\frac{x-y}{2}\right)}\bigg]\\
\langle x,y|f_{1}(t)\rangle & = -
	\frac{e^{i\theta}}{\sqrt{2}} e^{-it\sqrt{2}}e^{i \pi x/4} 
	e^{-i\pi y/2}\iint_{D_{\pi}\setminus D_{\epsilon}}\frac{d\phi_{1}d\phi_{2}}{4\pi^{2}}
	A(\phi_{1},\phi_{2})\\
&  \qquad\qquad\qquad\qquad \qquad\qquad \qquad 
	e^{-i\phi_{1}\left(-t+\frac{t}{\sqrt{2}}+\frac{x+y}{2}\right)}e^{-2i\phi_{2}
	\left(-t-\frac{t}{\sqrt{2}}+\frac{y-x}{2}\right)}\\
\langle x,y|f_{2}(t)\rangle & =  -\frac{1}{\sqrt{2}}e^{-it\sqrt{2}}e^{-i\pi x/2}e^{i\pi y/4 }
	\iint_{D_{\pi}\setminus D_{\epsilon}}\frac{d\phi_{1}d\phi_{2}}{4\pi^{2}}
	A(\phi_{1},\phi_{2})\\
&  \qquad\qquad\qquad\qquad \qquad\qquad \qquad e^{-i\phi_{1}\left(-t+\frac{t}{\sqrt{2}}+\frac{x+y}{2}\right)}
	e^{-2i\phi_{2}\left(-t-\frac{t}{\sqrt{2}}+\frac{x-y}{2}\right)}\\
\langle x,y|g_{1}(t)\rangle & =   \frac{e^{i\theta}}{\sqrt{2}}e^{i \pi x/4}e^{- i \pi y/2}
	\iint_{D_{\epsilon}}\frac{d\phi_{1}d\phi_{2}}{4\pi^{2}}A(\phi_{1},\phi_{2})
	e^{-i\phi_{1}\left(\frac{x+y}{2}\right)}e^{-2i\phi_{2}\left(\frac{y-x}{2}\right)}\\
& \qquad \qquad 
	\left[e^{-it 4\cos(\frac{\pi}{8}+\frac{\phi_{1}}{2})\cos(\frac{3\pi}{8}
	+\phi_{2})} -e^{-it\left(\sqrt{2}+\sqrt{2}\left(\frac{\phi_{1}}{2}-\phi_{2}\right)
 	-2\left(\frac{\phi_{1}}{2}+\phi_{2}\right)\right)}\right]\\
\langle x,y|g_{2}(t)\rangle & =  \frac{1}{\sqrt{2}}e^{-i\pi x/2}e^{i\pi y/4}
	\iint_{D_{\epsilon}}\frac{d\phi_{1}d\phi_{2}}{4\pi^{2}}A(\phi_{1},\phi_{2})
	e^{-i\phi_{1}\left(\frac{x+y}{2}\right)}e^{-2i\phi_{2}\left(\frac{x-y}{2}\right)}\\
 & \qquad\qquad\left[e^{-it 4\cos(\frac{\pi}{8}+\frac{\phi_{1}}{2})\cos(\frac{3\pi}{8}+\phi_{2})} - e^{-it\left(\sqrt{2}+\sqrt{2}\left(\frac{\phi_{1}}{2}-\phi_{2}\right)
 	-2\left(\frac{\phi_{1}}{2}+\phi_{2}\right)\right)}\right]\\
\langle x,y|h(t)\rangle & =  \frac{1}{\sqrt{2}}e^{i\pi x/4}e^{-i \pi y/2}
	\iint_{D_{\epsilon}}\frac{d\phi_{1}d\phi_{2}}{4\pi^{2}}
	A(\phi_{1},\phi_{2})e^{-i\phi_{1}\left(\frac{x+y}{2}\right)}e^{-2i\phi_{2}\left(\frac{y-x}{2}\right)}\\
& \qquad \qquad e^{-it 4\cos(\frac{\pi}{8}+\frac{\phi_{1}}{2}) \cos(\frac{3\pi}{8}+\phi_{2})} \left(e^{i\theta_{\pm}(\frac{\pi}{4}+\phi_{1},\frac{3\pi}{8}+\phi_{2})}-e^{i\theta}\right).
\end{align*}
We now proceed to bound the norm of each of these states. We repeatedly
use the fact that, for $(\phi_{1},\phi_{2})\in D_{\pi}$, 
\begin{align*}
\sum_{x,y=-\infty}^{\infty}e^{ix\left(\frac{1}{2}\left(\phi_{1}-\tilde{\phi}_{1}\right)-\left(\phi_{2}-\tilde{\phi}_{2}\right)\right)}e^{iy\left(\frac{1}{2}\left(\phi_{1}-\tilde{\phi}_{1}\right)+\left(\phi_{2}-\tilde{\phi}_{2}\right)\right)}
 & =  4\pi^{2}\delta(\phi_{1}-\tilde{\phi}_{1}) \delta(\phi_{2}-\tilde{\phi}_{2}).
 \end{align*}
 Using this formula we get
 \begin{align*}
\langle e_{1}(t)|e_{1}(t)\rangle & =  \sum_{y-x\geq C}\langle e_{1}(t)|x,y\rangle\langle x,y|e_{1}(t)\rangle\\
   & \leq  \sum_{x=-\infty}^{\infty}\sum_{y=-\infty}^{\infty}\bigg|\frac{1}{\sqrt{2}}
	 	\iint_{D_{\pi}}\frac{d\phi_{1}d\phi_{2}}{4\pi^{2}}A(\phi_{1},\phi_{2})
	 	\bigg[e^{-i\phi_{1}\left(-t+\frac{t}{\sqrt{2}}+\frac{x+y}{2}\right)}\\
	&\qquad e^{-2i\phi_{2}\left(-t-\frac{t}
	 	{\sqrt{2}}+\frac{y-x}{2}\right)}-e^{-i\phi_{1}\left(-\left\lfloor t\right\rfloor +\left\lfloor \frac{t}{\sqrt{2}}\right\rfloor 
 		+\frac{x+y}{2}\right)}e^{-2i\phi_{2}\left(-\left\lfloor t\right\rfloor -\left\lfloor \frac{t}{\sqrt{2}}
 		\right\rfloor +\frac{y-x}{2}\right)}\bigg]\bigg|^{2}\\
 & =  \frac{1}{2}\iint_{D_{\pi}}\frac{d\phi_{1}d\phi_{2}}{4\pi^{2}}\left|A(\phi_{1},\phi_{2})
 		\right|^{2}\bigg|e^{-i\phi_{1}\left(-t+\frac{t}{\sqrt{2}}\right)}e^{-2i\phi_{2}\left(-t-\frac{t}
 		{\sqrt{2}}\right)}\\
 & \qquad -e^{-i\phi_{1}\left(-\left\lfloor t\right\rfloor +\left\lfloor \frac{t}
 		{\sqrt{2}}\right\rfloor \right)}e^{-2i\phi_{2}\left(-\left\lfloor t\right\rfloor -\left\lfloor 
 		\frac{t}{\sqrt{2}}\right\rfloor \right)}\bigg|^{2}.
\end{align*}
 Now use the fact that $\left|e^{-ic}-1\right|^{2}\leq c^{2}$ for
$c\in\mathbb{R}$ to get 
\begin{align*}
\langle e_{1}(t)|e_{1}(t)\rangle & \leq  \frac{1}{2}\iint_{D_{\pi}}\left(\frac{d\phi_{1}d\phi_{2}}
	{4\pi^{2}}\right)\left|A(\phi_{1},\phi_{2})\right|^{2}\Bigg(-\phi_{1}\left(-t+\frac{t}{\sqrt{2}}+
	\left\lfloor t\right\rfloor -\left\lfloor \frac{t}{\sqrt{2}}\right\rfloor \right)\\
 &  \quad \quad -2\phi_{2}\left(-t-\frac{t}{\sqrt{2}}+\left\lfloor t\right\rfloor +\left\lfloor \frac{t}{\sqrt{2}}\right\rfloor \right)\Bigg)^{2}\\
 & \leq  4\iint_{D_{\pi}}\frac{d\phi_{1}d\phi_{2}}{4\pi^{2}}\left|A(\phi_{1},\phi_{2})\right|^{2}\left(\phi_{1}^{2}+4\phi_{2}^{2}\right)
\end{align*}
using the Cauchy-Schwarz inequality and the fact that $\left|t-{t}/{\sqrt{2}}-\left\lfloor t\right\rfloor -\left\lfloor {t}/{\sqrt{2}}\right\rfloor \right|\leq2$.
So 
\begin{align*}
\langle e_{1}(t)|e_{1}(t)\rangle & \leq 4\left(\iint_{D_{\pi}\setminus D_{\epsilon}}\frac{d\phi_{1}d\phi_{2}}{4\pi^{2}}+\iint_{D_{\epsilon}}\frac{d\phi_{1}d\phi_{2}}{4\pi^{2}}\right)\left|A(\phi_{1},\phi_{2})\right|^{2}\left(\phi_{1}^{2}+4\phi_{2}^{2}\right)\\
 & \leq  4\left(5\pi^{2}\right)\left(\frac{4\pi}{L\epsilon}\right)+20\epsilon^{2}\\
 & =  \frac{80\pi^{3}}{L\epsilon}+20\epsilon^{2}
\end{align*}
where we have used \lem{Let--and} and the fact that $\phi_{1}^{2}+4\phi_{2}^{2}\leq 5\epsilon^{2}$ on $D_\epsilon$. Similarly, 
\[
\langle e_{2}(t)|e_{2}(t)\rangle\leq\frac{80\pi^{3}}{L\epsilon}+20\epsilon^{2}.
\]
 Now
 \begin{align*}
\langle f_{1}(t)|f_{1}(t)\rangle & \leq  \frac{1}{2}\iint_{D_{\pi}\setminus D_{\epsilon}}\frac{d\phi_{1}d\phi_{2}}{4\pi^{2}}\left|A(\phi_{1},\phi_{2})\right|^{2}\\
 & \leq  \frac{2\pi}{L\epsilon}
 \end{align*}
 by \lem{Let--and}, and similarly
 \[
\langle f_{2}(t)|f_{2}(t)\rangle\leq\frac{2\pi}{L\epsilon}.
\]
Moving on to the next term, 
\begin{align}
\langle g_{1}(t)|g_{1}(t)\rangle & \leq  \frac{1}{2}\iint_{D_{\epsilon}}\frac{d\phi_{1}d\phi_{2}}{4\pi^{2}}
	\left|A(\phi_{1},\phi_{2})\right|^{2}\Bigg|e^{-it 4\cos(\frac{\pi}{8}+\frac{\phi_{1}}{2})
	\cos(\frac{3\pi}{8}+\phi_{2})}\nonumber \\
 &  \qquad\qquad\qquad -e^{-it\left(\sqrt{2}+\sqrt{2}\left(\frac{\phi_{1}}{2}-\phi_{2}\right)-2\left(\frac{\phi_{1}}{2}+\phi_{2}\right)\right)}
 	\Bigg|^{2}\nonumber \\
 & \leq  \frac{1}{2}\iint_{D_{\epsilon}}\frac{d\phi_{1}d\phi_{2}}{4\pi^{2}}
 	\Bigg[\left|A(\phi_{1},\phi_{2})\right|^{2}t^{2} \left(4\cos\left(\frac{\pi}{8}+\frac{\phi_{1}}{2}\right)\cos\left(\frac{3\pi}{8}+\phi_{2}\right)\right.\nonumber\\
&\qquad\qquad\qquad\left.
 	-\sqrt{2}-\sqrt{2}\left(\frac{\phi_{1}}{2}-\phi_{2}\right)+2\left(\frac{\phi_{1}}{2}+\phi_{2}\right)\right)^{2}\Bigg]
 	\label{eq:g_bound}
\end{align}
using $\left|e^{-ic}-1\right|^{2}\leq c^{2}$ for $c\in\mathbb{R}$.
Now 
\begin{align*}
4\cos\left(\frac{\pi}{8}+\frac{\phi_{1}}{2}\right)\cos\left(\frac{3\pi}{8}+\phi_{2}\right) & =  
	2\cos\left(\frac{\pi}{2}+\frac{\phi_{1}}{2}+\phi_{2}\right)+2\cos\left(-\frac{\pi}{4}+\frac{\phi_{1}}{2}-\phi_{2}\right)\\
 & =  - 2 \sin\left(\frac{\phi_1}{2} + \phi_2\right) + \sqrt{2} \cos\left(\frac{\phi_1}{2}-\phi_2\right) + 
 	\sqrt{2} \sin\left(\frac{\phi_1}{2} - \phi_2\right)
 \end{align*}
so 
\begin{align*}
 &   \left|4\cos\left(\frac{\pi}{8}+\frac{\phi_{1}}{2}\right)\cos\left(\frac{3\pi}{8}+\phi_{2}\right)-\sqrt{2}-\sqrt{2}
 	\left(\frac{\phi_{1}}{2}-\phi_{2}\right)+2\left(\frac{\phi_{1}}{2}+\phi_{2}\right)\right|\\
 & \quad \leq  \left|\sqrt{2}\left(\cos\left(\frac{\phi_{1}}{2}-\phi_{2}\right)-1\right)\right| 
 	+\left|\sqrt{2}\left(\sin\left(\frac{\phi_{1}}{2}-\phi_{2}\right)-\left(\frac{\phi_{1}}{2}-\phi_{2}\right)\right)\right|\\
 & \qquad +\left|2\left(\sin\left(\frac{\phi_{1}}{2}+\phi_{2}\right)-\left(\frac{\phi_{1}}{2}+\phi_{2}\right)\right)\right|\\
 & \quad \leq  \sqrt{2}\left(\frac{\phi_1}{2}-\phi_{2}\right)^{2}+\sqrt{2}\left(\frac{\phi_{1}}{2}
 	-\phi_{2}\right)^{2}+2\left(\frac{\phi_{1}}{2}+\phi_{2}\right)^{2} \\
 & \quad \leq  4\left(\left(\frac{\phi_{1}}{2}+\phi_{2}\right)^{2}+\left(\frac{\phi_{1}}{2}-\phi_{2}\right)^{2}\right),
\end{align*}
using $|{\cos x -1}|\leq x^2$ and $|{\sin x-x}|\leq x^2$ for $x\in \mathbb{R}$. Plugging this into equation \eq{g_bound} we get 
\begin{align*}
\langle g_{1}(t)|g_{1}(t)\rangle & \leq \frac{1}{2}\iint_{D_{\epsilon}}
	\frac{d\phi_{1}d\phi_{2}}{4\pi^{2}} 16 \left|A(\phi_{1},\phi_{2})\right|^{2}t^{2}
	\left(\left(\frac{\phi_{1}}{2} + \phi_2\right)^2+\left(\frac{\phi_1}{2} - \phi_{2}\right)^2\right)^{2} \\
 & \leq  16 t^2 \iint_{D_{\epsilon}}\frac{d\phi_{1}d\phi_{2}}{4\pi^{2}}
 	\left|A(\phi_{1},\phi_{2})\right|^{2}\left(\left(\frac{\phi_1}{2} + \phi_2\right)^4 
 	+ \left(\frac{\phi_1}{2} - \phi_2\right)^4\right)\\
 & \leq  \frac{16 t^{2}}{L^2} \iint_{D_{\epsilon}}\frac{d\phi_{1}d\phi_{2}}{4\pi^{2}}
	  \frac{\sin^2(\frac{L}{2} [\frac{\phi_1}{2} + \phi_2])}
	 {\sin^2(\frac{1}{2} [\frac{\phi_1}{2} + \phi_2])}
	 \frac{\sin^2(\frac{L}{2} [-\frac{\phi_1}{2} + \phi_2])}
	 {\sin^2(\frac{1}{2} [-\frac{\phi_1}{2} + \phi_2])}\\
 & \qquad \left(\left(\frac{\phi_1}{2}+ \phi_2\right)^4 
 	+ \left(\frac{\phi_1}{2} - \phi_2\right)^4\right)
\end{align*}
where we used the Cauchy-Schwarz inequality in the second line and equation \eq{A_summed} in the last line.  Changing coordinates to
\[
  \alpha_1 = \phi_1 + \frac{ \phi_2}{2} \qquad \alpha_2 = \frac{\phi_1}{2} -\phi_2 
 \]
and realizing that $|\alpha_1|,|\alpha_2| < 3\epsilon/2$ for $(\phi_1,\phi_2)\in D_{\epsilon}$, we see that
\begin{align*}
  \bra{g_1(t)} g_1(t)\rangle &\leq \frac{16t^2}{L^2} \int_{-{3\epsilon/2}}^{3\epsilon/2} \frac{d\alpha_1}{2\pi}
      \int_{-{3\epsilon/2}}^{3\epsilon/2} \frac{d\alpha_1}{2\pi} 
      \frac{\sin^2(\frac{1}{2} L\alpha_1)}
	 {\sin^2(\frac{1}{2} \alpha_1)}
	 \frac{\sin^2(\frac{1}{2} L\alpha_2)}
	 {\sin^2(\frac{1}{2} \alpha_2)} \left(\alpha_1^4+\alpha_2^4\right)\\
   &= \frac{32t^2}{L^2} \int_{-{3\epsilon/2}}^{3\epsilon/2} \frac{d\alpha_1}{2\pi}
      \int_{-{3\epsilon/2}}^{3\epsilon/2} \frac{d\alpha_1}{2\pi}
      \frac{\sin^2(\frac{1}{2} L\alpha_1)}
	 {\sin^2(\frac{1}{2} \alpha_1)}
	 \frac{\sin^2(\frac{1}{2} L\alpha_2)}
	 {\sin^2(\frac{1}{2} \alpha_2)} \alpha_1^4\\
   &\leq \frac{32t^2}{L} \int_{-3\epsilon/2}^{3\epsilon/2} \frac{d\alpha_1}{2\pi}  
     \frac{\sin^2(\frac{1}{2} L\alpha_1)}
	 {\sin^2(\frac{1}{2} \alpha_1)} \alpha_1^4\\
   &\leq \frac{32 t^2}{L} \int_{-3\epsilon/2}^{3\epsilon/2} \frac{d\alpha_1}{2\pi} \frac{\pi^2}{\alpha_1^2} \alpha_1^4\\
   &= \frac{36 \pi t^2\epsilon^3}{L},
\end{align*}
with a similar bound on $\langle g_{2}(t)|g_{2}(t)\rangle$.
 
Finally, 
\begin{align*}
\langle h(t)|h(t)\rangle & \leq  \frac{1}{2}\iint_{D_{\epsilon}}\frac{d\phi_{1}d\phi_{2}}{4\pi^{2}}
	\left|A(\phi_{1},\phi_{2})\right|^{2}\left|e^{i\theta_\pm (\tfrac{\pi}{4}+\phi_{1},
	\tfrac{3\pi}{8}+\phi_{2})}-e^{i\theta}\right|^{2}.
 \end{align*}
Recall that $e^{i\theta_\pm (p_1,p_2)}=T(p_1,p_2) \pm R(p_1,p_2)$ is obtained by solving for the effective single-particle S-matrix for the Hamiltonian \eq{vr_eqn}. For $p_1$ near ${\pi}/{4}$ we divide this Hamiltonian by $2\cos({p_1}/{2})$ to put it in the form considered in \cite{Childs_Gosset}, where the potential term is now $\mathcal{V}(|r|)/(2\cos({p_1}/{2}))$. The entries $T(p_1,p_2)$ and $R(p_1,p_2)$ of this S-matrix are bounded rational functions of $z=e^{ip_2}$ and $(2\cos({p_1}/{2}))^{-1}$ \cite{Childs_Gosset}, so they are differentiable as a function of $p_1$ and $p_2$ on some neighborhood $U$ of $({\pi}/{4},{3\pi}/{8})$ (and have bounded partial derivatives on this neighborhood).

For $\epsilon$ small enough that $D_\epsilon\subset U$ we get, using the mean value theorem and the fact that $\theta=\theta_\pm ({\pi}/{4}, {3\pi}/{8})$,
\begin{align*}
\left|e^{i\theta_\pm (\tfrac{\pi}{4}+\phi_{1}, \tfrac{3\pi}{8}+\phi_{2})}-e^{i\theta}\right| & \leq \sqrt{\phi_1^2+\phi_2^2} \max_{U} \big|\vec{\nabla} e^{i\theta_\pm}\big| \quad \text{for }(\phi_1,\phi_2)\in D_{\epsilon}\\
& \leq  \epsilon \Gamma
\end{align*}
for some constant $\Gamma$ (independent of $L$).
Therefore
\begin{align*}
\langle h(t)|h(t)\rangle & \leq  \frac{1}{2}\iint_{D_{\epsilon}}\frac{d\phi_{1}d\phi_{2}}{4\pi^{2}}
	\left|A(\phi_{1},\phi_{2})\right|^{2} \epsilon^2 \Gamma^2\\
& \leq \frac{1}{2}\Gamma^2 \epsilon^2.
\end{align*}
 
Putting these bounds together, we get 
\begin{align*}
\Norm{P_1|\psi_A(t)\rangle-P_1|\alpha(t)\rangle}  
 & \leq \norm{|e_{1}(t)\rangle} + \norm{|e_{2}(t)\rangle} + 
        \norm{|f_{1}(t)\rangle} + \norm{|f_{2}(t)\rangle} \\
 &  \qquad + \norm{|g_{1}(t)\rangle} + \norm{|g_{2}(t)\rangle} 
           + \norm{|h(t)\rangle} \\
 & \leq 2\left(\frac{80\pi^{3}}{L\epsilon}+20\epsilon^{2}\right)^{\frac{1}{2}}
 	+2\left(\frac{2\pi}{L\epsilon}\right)^{\frac{1}{2}}+2\left(\frac{36\pi t^{2}\epsilon^{3}}{L}\right)^{\frac{1}{2}}+\frac{1}{\sqrt{2}}\Gamma \epsilon.
\end{align*}
 Letting $\epsilon={a}/{\sqrt{L}}$ and $t\leq c_{0}L$ we get 
\begin{equation}
\left\Vert \,P_1|\psi_A(t)\rangle-P_1|\alpha(t)\rangle\right\Vert = \O(L^{-{1}/{4}}).\label{eq:psiA_alpha}
\end{equation}

Since $P_2\ket{\alpha(t)}$ has support on at most $4CL$ basis states $|x,y\rangle$, and since $|\langle x,y|P_2|\alpha(t)\rangle|^2=\O( L^{-2})$, we get
\begin{equation}
  \left\Vert P_2\ket{\alpha(t)}\right\Vert = \O(L^{-{1}/{2}}).\label{eq:P2alpha_bound}
\end{equation}

We now use the bounds \eq{psiA_alpha} and \eq{P2alpha_bound} and \lem{alpha} to show that
\begin{equation}
\left\Vert |\psi_{A}(t)\rangle-|\alpha(t)\rangle\right\Vert =\O(L^{-{1}/{8}}).\label{eq:psiA_alpha_bound}\end{equation}
 First consider the case where the interaction range is $C=0$ (as in
the Bose-Hubbard model). In this case equation \eq{psiA_alpha_bound}
follows directly from equation \eq{psiA_alpha} and the facts that
$\langle x,y|\alpha(t)\rangle=\pm\langle y,x|\alpha(t)\rangle$ and
$\langle x,y|\psi_{A}(t)\rangle=\pm\langle y,x|\psi_{A}(t)\rangle$. 

Now suppose $C\neq0$. In this case
\begin{align*}
\left\Vert \left(1-P_{2}\right)|\psi_{A}(t)\rangle\right\Vert ^{2} & = 2\left\Vert P_{1}|\psi_{A}(t)\rangle\right\Vert ^{2}\\
 & = 2\left(\left\Vert P_{1}|\alpha(t)\rangle\right\Vert +\O(L^{-{1}/{4}})\right)^2\\
 & = 2\left(\frac{1}{2}\left\Vert (1-P_{2})|\alpha(t)\rangle\right\Vert ^{2}+\O(L^{-{1}/{4}})\right)\\
 & = 1+\O(L^{-1})-\langle\alpha(t)|P_{2}|\alpha(t)\rangle+\O(L^{-{1}/{4}})\\
 & = 1+\O(L^{-{1}/{4}})
\end{align*}
where in the next-to-last line we have used \lem{alpha}. So
\begin{align*}
\left\Vert |\psi_{A}(t)\rangle-|\alpha(t)\rangle\right\Vert  & \leq 2\left\Vert P_{1}|\psi_{A}(t)\rangle-P_{1}|\alpha(t)\rangle\right\Vert +\left\Vert P_{2}|\alpha(t)\rangle\right\Vert +\left\Vert P_{2}|\psi_{A}(t)\rangle\right\Vert \\
 & = \O(L^{-{1}/{4}})+\O(L^{-{1}/{2}})+\left(1-\left\Vert \left(1-P_{2}\right)|\psi_{A}(t)\rangle\right\Vert \right)^{\frac{1}{2}}\\
 & = \O(L^{-{1}/{4}})+\O(L^{-{1}/{2}})+\O(L^{-{1}/{8}})\\
 & = \O(L^{-{1}/{8}})
 \end{align*}
which completes the proof. 
\end{proof}
\end{proof}

\subsubsection{Technical lemmas}
\label{sec:techlem}

In this section we prove three lemmas that are used in the proof of \thm{twopart}.

\begin{lemma} \label{lem:alpha}
Let $|\alpha(t)\rangle$ be defined as in \thm{twopart}.
Then
\[
\langle\alpha(t)|\alpha(t)\rangle=1+\O(L^{-1}).
\]
\end{lemma}

\begin{proof}
Define
\[
\Pi=\sum_{x\leq y}|x,y\rangle\langle x,y|.
\]
Note that, since $\langle x,y|\alpha(t)\rangle=\pm\langle y,x|\alpha(t)\rangle$,
\begin{align*}
\langle\alpha(t)|\alpha(t)\rangle & = 2\langle\alpha(t)|\Pi|\alpha(t)\rangle-\sum_{x=-\infty}^{\infty}\langle\alpha(t)|x,x\rangle\langle x,x|\alpha(t)\rangle\\
 & = 2\langle\alpha(t)|\Pi|\alpha(t)\rangle+\O(L^{-1})
\end{align*}
where the last line follows since $|\langle x,x|\alpha(t)\rangle|^{2}$
is nonzero for at most $L$ values of $x$ and $|\langle x,x|\alpha(t)\rangle|^{2}=\O(L^{-2})$.
We now show that
\[
\langle\alpha(t)|\Pi|\alpha(t)\rangle=\frac{1}{2}+O(L^{-1}).
\]
Note that
\begin{align*}
\langle\alpha(t)|\Pi|\alpha(t)\rangle &= \frac{1}{2L^{2}}\sum_{x\leq y}\Bigg(F(x,y,t) +F(y,x,t)  \\
&\qquad \pm e^{i\theta}e^{\frac{3i\pi}{4}x}e^{-\frac{3i\pi}{4}y}F(x,y,t) F(y,x,t) \\
&\qquad \pm e^{-i\theta}e^{-\frac{3i\pi}{4}x}e^{\frac{3i\pi}{4}y}F(x,y,t) F(y,x,t) \Bigg).
\end{align*}
Now $F(x,y,t)= 1$ if and only if $x\in\{-M-L+2\lfloor t \rfloor,\ldots,-M-1+2 \lfloor t \rfloor \}$
and $y\in\{M+1-2 \lfloor {t}/{\sqrt{2}} \rfloor ,\ldots,M+L-2 \lfloor {t}/{\sqrt{2}} \rfloor \}$.
Similarly $F(y,x,t) =1$ if and only if $x\in\{M+1-2 \lfloor {t}/{\sqrt{2}} \rfloor ,\ldots,M+L-2 \lfloor {t}/{\sqrt{2}} \rfloor \}$
and $y\in\{-M-L+2 \lfloor t \rfloor ,\ldots,-M-1+2 \lfloor t \rfloor \}$.
So
\[
\sum_{x\leq y}F(y,x,t) =\sum_{y\leq x}F(x,y,t) 
\]
and
\begin{align*}
\frac{1}{2L^{2}}\sum_{x\leq y}\left[F(x,y,t) +F(y,x,t) \right] &= \frac{1}{2L^{2}}\left(\sum_{x=-\infty}^{\infty}\sum_{y=-\infty}^{\infty}F(x,y,t) -\sum_{x=-\infty}^{\infty}F(x,x,t) \right)\\
 &= \frac{1}{2}+\O(L^{-1}).
\end{align*}

We now establish the bound
\[
\left|\frac{1}{2L^{2}}\sum_{x\leq y}e^{\frac{3i\pi}{4}x}e^{-\frac{3i\pi}{4}y}F(x,y,t) F(y,x,t) \right|=\O(L^{-1})
\]
to complete the proof.  To get this bound, note that both $F(x,y,t) =1$ and $F(y,x,t) =1$
if and only if
\begin{align*}
x,y & \in \{-M-L+2\left\lfloor t\right\rfloor ,\ldots,-M-1+2\left\lfloor t\right\rfloor \}\\
\text{ and }x,y & \in \left\{M+1-2\left\lfloor \frac{t}{\sqrt{2}}\right\rfloor ,\ldots,M+L-2\left\lfloor \frac{t}{\sqrt{2}}\right\rfloor \right\}.
\end{align*}
Letting
\[
B=\{-M-L+2\left\lfloor t\right\rfloor ,\ldots,-M-1+2\left\lfloor t\right\rfloor \} \cap \left\{M+1-2\left\lfloor \frac{t}{\sqrt{2}}\right\rfloor ,\ldots,M+L-2\left\lfloor \frac{t}{\sqrt{2}}\right\rfloor \right\},
\]
we have
 \[
B=\{j,j+1,\ldots,j+l\}
\]
for some $j,l\in\mathbb{Z}$ with $l<L$. So
\begin{align*}
\frac{1}{2L^{2}}\left|\sum_{x\leq y}e^{\frac{3i\pi}{4}x}e^{-\frac{3i\pi}{4}y}F(x,y,t) F(y,x,t) \right| & = \frac{1}{2L^{2}}\left|\sum_{x,y\in B,\, x\leq y}e^{\frac{3i\pi}{4}x}e^{-\frac{3i\pi}{4}y}\right|\\
 &  = \frac{1}{2L^{2}}\left|\sum_{y=j}^{j+l}\sum_{x=j}^{y}e^{\frac{3i\pi}{4}x}e^{-\frac{3i\pi}{4}y}\right|\\
 &= \frac{1}{2L^{2}}\left|\sum_{y=j}^{j+l}e^{-\frac{3i\pi}{4}y}e^{3i\frac{\pi}{4}j}\frac{e^{3i\frac{\pi}{4}\left(y+1-j\right)}-1}{e^{3i\frac{\pi}{4}}-1}\right|\\
 & \leq \frac{(l+1)}{2L^{2}}\frac{2}{\left|e^{3i\frac{\pi}{4}}-1\right|}\\
 &= \O(L^{-1})
 \end{align*}
since $l<L$.
\end{proof}
\begin{lemma}
\label{lem:Let--and}Let $k\in(-\pi,0)\cup(0,\pi)$ and $0<\epsilon<\min\left\{ \pi-|k|,|k|\right\}$.
Let
\begin{align*}
D_{\epsilon} & =  \left[-\epsilon,\epsilon\right]\times\left[-\epsilon,\epsilon\right]\\
D_{\pi} & =  \left[-\pi,\pi\right]\times\left[-\pi,\pi\right].
\end{align*}
Then 
\begin{align*}
\iint_{D_{\pi}}\frac{d\phi_{1}d\phi_{2}}{4\pi^{2}}|A(\phi_{1},\phi_{2})|^{2} & =  1\\
\iint_{D_{\pi}\setminus D_{\epsilon}}\frac{d\phi_{1}d\phi_{2}}{4\pi^{2}}|A(\phi_{1},\phi_{2})|^{2} & \leq  \frac{4\pi}{L\epsilon}\\
\iint_{D_{\epsilon}}\frac{d\phi_{1}d\phi_{2}}{4\pi^{2}}|B(\phi_{1},\phi_{2},k)|^{2} & \leq  \frac{4\pi^{2}}{L^{2}\epsilon^{2}}.\end{align*}
where $A(\phi_1,\phi_2)$ and $B(\phi_1,\phi_2,k)$ are given by equation \eq{A}.
 \end{lemma}
\begin{proof}
Using equation \eq{A} we get
\begin{align*}
|A(\phi_{1},\phi_{2})|^{2} & =  \frac{1}{L^{2}} \sum_{x,\tilde{x}=-(M+L)}
  ^{-(M+1)} \sum_{y,\tilde{y}=M+1}^{M+L}
	e^{i\frac{\phi_{1}}{2}\left(x+y-(\tilde{x}+\tilde{y})\right)}
        e^{i\phi_{2}\left(x-y-(\tilde{x}-\tilde{y})\right)}.
\end{align*}
Now 
\[
\int_{-\pi}^{\pi}\frac{d\phi_{2}}{2\pi}e^{i\phi_{2}
  \left(x-y-\tilde{x}+\tilde{y}\right)}=\delta_{x-y,\tilde{x}-\tilde{y}},
\]
so (suppressing the limits of summation for readability) 
\begin{align*}
\iint_{D_{\pi}}\frac{d\phi_{1}d\phi_{2}}{4\pi^{2}}
   |A(\phi_{1},\phi_{2})|^{2} 
   & =  \frac{1}{L^{2}} \int_{-\pi}^{\pi}\frac{d\phi_{1}}{2\pi}
     \sum_{x,\tilde{x}}\sum_{y,\tilde{y}}	e^{i\phi_{1}\left(y-\tilde{y}\right)}
     \delta_{x-y,\tilde{x}-\tilde{y}} \\
 & =  \frac{1}{L^{2}}\sum_{x,\tilde{x}}\sum_{y,\tilde{y}}\delta_{y,\tilde{y}}
 	\delta_{x-y,\tilde{x}-\tilde{y}}\\
 & =  1
\end{align*}
which proves the first part.

By performing the sums in equation \eq{A} we get 
\begin{equation}
|A(\phi_{1},\phi_{2})|^{2}
=\frac{1}{L^{2}}\frac{\sin^{2}(\frac{1}{2}L[\frac{\phi_1}{2}+\phi_{2}])} {\sin^{2}(\frac{1}{2}[\frac{\phi_1}{2}+\phi_{2}])}
\frac{\sin^{2}(\frac{1}{2}L[\frac{\phi_1}{2} -\phi_{2}])}
{\sin^{2}(\frac{1}{2}[\frac{\phi_1}{2}-\phi_{2}])}.
\label{eq:A_summed}
\end{equation}
Letting $\alpha_{1}={\phi_1}/{2}+\phi_{2}$ and
$\alpha_{2}={\phi_1}/{2}-\phi_{2}$, we see that
$|\alpha_{1}|\leq 3\pi/2$, $|\alpha_{2}|\leq 3\pi/2$,
and $\alpha_{1}^{2}+\alpha_{2}^{2}\geq  5\epsilon^{2}/2$ whenever
$(\phi_{1},\phi_{2})\in D_{\pi}\setminus D_{\epsilon}$. Defining
$D_{3\pi/2}=[-3\pi/2,3\pi/2]^{2}$
we get $(\alpha_{1},\alpha_{2})\in D_{3\pi/2}\setminus D_{\epsilon}$
whenever $(\phi_{1},\phi_{2})\in D_{\pi}\setminus D_{\epsilon}$.
Hence 
\begin{align*}
\iint_{D_{\pi}\setminus D_{\epsilon}}\frac{d\phi_{1}d\phi_{2}}{4\pi^{2}}|A(\phi_{1},\phi_{2})|^{2} 
	& \leq \frac{1}{L^{2}} \iint_{D_{3\pi/2}\setminus D_{\epsilon}} \frac{d\alpha_{1}d\alpha_{2}}{4\pi^{2}}
	\frac{\sin^{2}(\frac{1}{2}L\alpha_{1})}{\sin^{2}
	(\frac{1}{2}\alpha_{1})}\frac{\sin^{2}(\frac{1}{2}L\alpha_{2})}
	{\sin^{2}(\frac{1}{2}\alpha_{2})}\\
 & \leq  \frac{4}{L}\left(\frac{1}{L}\int_{-\frac{3\pi}{2}}^{\frac{3\pi}{2}}\frac{d\alpha_{1}}{2\pi}\frac{\sin^{2}(
 	\frac{1}{2}L\alpha_{1})}{\sin^{2}(\frac{1}{2}\alpha_{1})}\right)
 	\left(\int_{\epsilon}^{3\pi/2}\frac{d\alpha_{2}}{2\pi}\frac{\sin^{2}(\frac{1}{2}L\alpha_{2})}
 	{\sin^{2}(\frac{1}{2}\alpha_{2})}\right)\\
 & \leq  \frac{4}{L}\left(\int_{-2\pi}^{2\pi}\frac{d\alpha_{1}}{2\pi}\frac{1}{L}\frac{\sin^{2}(\frac{1}{2}L\alpha_{1})}{\sin^{2}(\frac{1}{2}\alpha_{1})}\right)\left(\int_{\epsilon}^{\frac{3\pi}{2}}\frac{d\alpha_{2}}{2\pi}\frac{1}{\sin^{2}(\frac{1}{2}\alpha_{2})}\right)\\
 & = \frac{8}{L} \left(\int_{\epsilon}^{\pi}\frac{d\alpha_{2}}{2\pi}\frac{1}{\sin^{2}(\frac{1}{2}\alpha_{2})}+\int_{\pi}^{\frac{3\pi}{2}}\frac{d\alpha_{2}}{2\pi}\frac{1}{\sin^{2}(\frac{1}{2}\alpha_{2})}\right)\\
 & \leq  \frac{8}{L}\left(\int_{\epsilon}^{\pi}\frac{d\alpha_{2}}{2\pi}\frac{\pi^{2}}{\alpha_{2}^{2}}+2\int_{\pi}^{\frac{3\pi}{2}}\frac{d\alpha_{2}}{2\pi}\right)\\
 & =  \frac{4\pi}{L\epsilon}
 \end{align*}
which proves the second inequality (in the next-to-last line we
have used the fact that $\sin({x}/{2})>{x}/{\pi}$ for $x \in (0,\pi)$
and $\sin^{2}({x}/{2})>{1}/{2}$ for $x \in (\pi,{3\pi}/{2})$).

Now
\begin{align*}
|B(\phi_{1},\phi_{2},k)|^{2} & =  |A(\phi_{1},-\phi_{2}-2k)|^{2}\\
 & \leq  \frac{1}{L^{2}}\frac{1}{\sin^{2}\left(\frac{1}{2}
     \left[\frac{\phi_{1}}{2}+\phi_{2}+2k\right]\right)}
 \frac{1}{\sin^{2}\left(\frac{1}{2}
     \left[-\frac{\phi_{1}}{2}+{\phi_{2}}+2k\right]\right)}.
\end{align*}
If $(\phi_{1},\phi_{2})\in D_{\epsilon}$ then $|k|-{3\epsilon}/{4} 
\leq\left|\pm{\phi_{1}}/{4}+{\phi_{2}}/{2}+k\right|
\leq|k|+{3\epsilon}/{4}$.
Noting that $\epsilon$ is chosen such that $0 < \epsilon < 
\min \{\pi-|k|,|k|\}$, we get
\[
\frac{\epsilon}{4}\leq\left|\pm\frac{\phi_{1}}{4}+\frac{\phi_{2}}{2}+k\right|\leq\pi-\frac{\epsilon}{4}
\]
so 
\begin{align*}
|B(\phi_{1},\phi_{2},k)|^{2} & \leq  \frac{1}{L^{2}}\frac{1}{\sin^{4}(\frac{\epsilon}{4})}\\
 	& \leq  \frac{16\pi^{4}}{L^{2}\epsilon^{4}}
\end{align*}
and 
\begin{align*}
\iint_{D_{\epsilon}}\frac{d\phi_{1}d\phi_{2}}{4\pi^{2}}|B(\phi_{1},\phi_{2},k)|^{2} & \leq  \frac{1}{4\pi^{2}}\left(2\epsilon\right)^{2}\left(\frac{16\pi^{4}}{L^{2}\epsilon^{4}}\right)\\
 & =  \frac{16\pi^{2}}{L^{2}\epsilon^{2}}. \qedhere
 \end{align*}
\end{proof}

\begin{lemma}
\label{lem:a_xy}
Let $a_{xy}(t)$ be as in Theorem \ref{thm:twopart}. For $x\leq y$,
\begin{align*}
a_{xy}(t) & =  \frac{1}{\sqrt{2}} e^{- i t\sqrt{2}}\left[e^{-i \pi x/2} e^{i \pi y/4} \left(\iint_{D_{\pi}} 
	\frac{d\phi_1 d\phi_2}{4\pi^2} A(\phi_1,\phi_2)\right.\right.\\
	& \qquad\quad \left.e^{-i \phi_1 \left( -\lfloor t\rfloor + \lfloor
	\frac{t}{\sqrt{2}}\rfloor + \frac{x+y}{2} \right)} e^{-2 i \phi_2 \left(-\lfloor t 
	\rfloor -\lfloor \frac{t}{\sqrt{2}}\rfloor + \frac{x-y}{2}\right)}\right)\\
& \quad \pm e^{i\theta} e^{i\pi x/4}e^{-i \pi y/2} \left(\iint_{D_{\pi}} \frac{d\phi_1 d\phi_2}{4\pi^2}
	A(\phi_1,\phi_2) \right.\\
&	\qquad\quad\left.\left.e^{- i \phi_1 \left(- \lfloor t\rfloor + \lfloor \frac{t}{\sqrt{2}}\rfloor + 
	\frac{x+y}{2}\right) } e^{-2 i \phi_2 \left(-\lfloor t \rfloor -\lfloor \frac{t}{\sqrt{2}}\rfloor
	+ \frac{y-x}{2}\right)}\right)
	\right].
\end{align*}
\end{lemma}

\begin{proof}
The lemma follows from \eq{a_xy} and the fact that, for any two numbers $\gamma_{1},\gamma_{2}$ such that $\gamma_{1}+\gamma_{2},\gamma_{1}-\gamma_{2}\in \mathbb{Z}$,
\[
\iint_{D_\pi}\frac{d\phi_{1} d\phi_{2}}{4\pi^2} A(\phi_{1},\phi_{2})e^{i\gamma_{1}\phi_{1}+2i \gamma_{2}\phi_{2}}=\begin{cases}
\frac{1}{L} &\text{ if }(-\gamma_{1}-\gamma_{2},-\gamma_{1}+\gamma_{2})\in S \\
0 &\text{ otherwise}
\end{cases}
\]
where $S = \{-M-L,\ldots, -M-1\} \times \{M+1,\ldots, M+L\}$.  To establish this formula, observe that
\begin{align*}
  \iint_{D_{\pi}}\frac{d\phi_1 d\phi_{2}}{4\pi^2}A(\phi_{1},\phi_{2})e^{ i\gamma_{1}\phi_{1}+2i\gamma_{2}\phi_{2}} 
  	& =  \frac{1}{L}\sum_{x=-M-L}^{-M-1}\sum_{y=M+1}^{M+L}\iint_{D_{\pi}}\frac{d\phi_1 d\phi_{2}}{4\pi^2}
  		e^{i\phi_{1}\left(\gamma_{1}+\frac{x+y}{2}\right)}e^{i\phi_{2}\left(x-y+2\gamma_{2}\right)}\\
 & =  \frac{1}{L}\sum_{x=-M-L}^{-M-1}\,\sum_{y=M+1}^{M+L}\int_{-\pi}^{\pi}\frac{d\phi_{1}}{2\pi}
 	e^{i\phi_{1}\left(\gamma_{1}+\frac{x-y}{2}\right)}\delta_{y,-x-2\gamma_{2}}.
\end{align*}
 Here we have performed the integral over $\phi_{2}$ using the fact
that $2\gamma_{2}$ is an integer.  We then have
\begin{align*}
\iint_{D_\pi}\frac{d\phi_{1} d\phi_{2}}{4\pi^2} A(\phi_{1},
	\phi_{2})e^{i\gamma_{1}\phi_{1}+2i\gamma_{2}\phi_{2}} 
 & =  \frac{1}{L}\sum_{x=-M-L}^{-M-1}\,
	\sum_{y=M+1}^{M+L}\int_{-\pi}^{\pi}\frac{d\phi_{1}}{2\pi}e^{i\phi_{1}
	\left(\gamma_{1}+x+\gamma_{2}\right)}\delta_{y,-x-2\gamma_{1}}\\
 & =  \frac{1}{L}\sum_{x=-M-L}^{-M-1}\,\sum_{y=M+1}^{M+L}\delta_{x,-\gamma_{1}-\gamma_{2}}\delta_{y,\gamma_{2}-\gamma_{1}}
 \end{align*}
as claimed.
\end{proof}

\subsection{Construction of $C\theta$-gate}

\section{Impossibility of some momentum switches}

\section{Universal Computation}
\subsection{Two-qubit blocks}
\subsection{Combining blocks}


\section{Improvements and Modifications}

What about long-range interactions, but where the interactions die off?
Additionally, what about error correction?

\end{document}