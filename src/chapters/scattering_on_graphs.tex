%======================================================================
%   Zak Webb
%   Ph. D. Thesis
%   Department of Physics and Astronomy
%   University of Waterloo
% 
%   Scattering on Graphs
%======================================================================


\documentclass[../thesis-main/thesis-main]{subfiles}
\begin{document}

\chapter{Scattering on graphs}

Scattering has a long history of study in the physics literature.  Ranging from the classical study of colliding objects to the analysis of high energy collisions of protons, studying the interactions of particles can be very interesting.


\section{Introduction and motivation}

% I'm being rather cavalier with these definitions, I should go back over them and make sure that I'm making correct statements.

Let us first take motivation from one of the most simple quantum systems: a free particle in one dimension.  Without any potential or interactions, we have that the time independent Schr\"{o}dinger equation reads
\[
  \frac{\partial^2}{\partial x^2} \psi(x) = -\frac{2m}{\hbar^2}E \psi(x) = -k^2 \psi(x),
\]
which requires the (unnormalizable) solutions,
\[
  \psi(x) = \exp(- i k x) 
\]
for real $k$.  These \textit{momentum states} correspond to particles travelling with momentum $k$ along the real line, and form a basis for the possible states of the system.


If we now also include some finite-range potential, or a potential $V$ that is non-zero only for $|x| < d$ for some range $d$, then outside this range the eigenstates remain unchanged.  The only difference is that we will deal with a superposition of states for each energy instead of the pure momentum states.  In particular, the scattering eigenbasis for this system will become
\[
  \psi(x) = \begin{cases}
    \exp(-i k x) + R(k) \exp(i k x) & x \leq -d\\
    T(k) \exp(- i k x) & x \geq d\\
    \phi(x,k) & |x| \leq d
    \end{cases}
\]
for some functions $R(k)$, $T(k)$, and $\phi(x,k)$.  


In addition to these scattering states, it is possible for bound states to exist.  These states are only nonzero for $|x| <d$, as the potential allows for the particles to simply sit at a particular location. One of the canonical examples is a finite well in one dimension, in which depending on the depth of the well, any number of bound states can exist.


\subsection{Infinite path}

With this motivation in mind, let us now look at the discretized system corresponding to a graph.  In particular, instead of a continuum of positions states in one dimension, we restrict the position states to integer values, with transport only between neighboring integers.  Explicitly, the Hilbert space of such a system corresponds to $n\in \NN$, with the discretized second derivative taking the form
\[
  \sum_{x=-\infty}^\infty (\ket{x + 1} - 2\ket{x} + \ket{x-1}) \bra{x} = 2\sum_{x=-\infty}^\infty \ket{x+1} \bra{x} - 2 \II.
\]
If we then rescale the energy levels, we have that the second ????

Altogether, we end up with the equation
\begin{equation}
  \Bigg(\sum_{x=-\infty}^\infty \ketbra{x+1}{x} + \ketbra{x}{x+1} \Bigg) \ket{\psi} = E_\psi \ket{\psi}.
\end{equation}
We can then break this vector equation into an equation for each basis vector $\ket{x}$, to get
\begin{equation}
  \braket{x+1}{\psi} + \braket{x-1}{\psi} = E_\psi \braket{x}{\psi}.
\end{equation}
for all $x\in \ZZ$.  If we then make the ansatz that $\braket{x}{\psi} = e^{i k x}$ for some $k$, we find that
\begin{equation}
  \braket{x+1}{\psi} + \braket{x-1}{\psi} = e^{i k } e^{i k x} + e^{-ik} e^{ikx} = E_\psi e^{ik x} = E_\psi \braket{x}{\psi} \qquad \Rightarrow \qquad E_\psi = e^{ik} + e^{-i k} = 2 \cos(k).
\end{equation}
If we then use the fact that $E_\psi$ must be real, and that the amplitudes should not diverge to infinity as $x\rightarrow \pm \infty$, we find that the only possible values of $k$ are between $[-\pi,\pi)$.  

Now show how these form a basis for the states.

Hence, in analogy with the continuous case, our basis of states corresponds to momentum states, but where the possible momenta only range over $[-\pi,\pi)$.  (As an aside, this maximum momenta is commonly know as a Leib-Robinson bound, and corresponds to a maximum speed of information propagation.)

We can then talk about the ``speed'' of these states, which is given by
\begin{equation}
  s = \Big| \frac{d E_k}{d k} \Big| = 2 \sin (|k|).
\end{equation}
Note that in the case of small $k$, we recover the linear relationship between speed and momentum.  In this way, as the distance between the vertices grows smaller, we recover the continuum case.



\section{Scattering off of a graph}

Now that we have an example based on a free particle, we should examine how to generalize potentials.  One method to do this is to add a potential function, with explicit potential energies at various vertices of the infinite path, but if we wish to only examine scattering on unweighted graphs, we need to be a little more clever.  The interesting way to do this is to take a finite graph $\hat{G}$, and attach two semi-infinite paths to this graph.  In this way, we get something that is similar to a finite-range potential.

With this construction, the eigenvalue equation must still be satisfied along the semi-infinite paths, and thus the form of the eigenstates along the paths must still be of the form $e^{i k x}$ for some $k$ and $x$.  However, we can no longer assume that $k$ is real, as the fact that the attached semi-infinite paths are only infinite in one direction allow for an exponentially decaying amplitudes along the paths.


\subsection{Infinite path and a Graph}

In the most simple example, let us attached a graph $\widetilde{G}$ to an infinite path.  In particular, we assume that $\widetilde{G}$ is attached to a single vertex of the infinite path, and that the graph is attached by adding an edge from each vertex in $S\subset V(\widetilde{G})$ to one specific vertex of the infinite path, which we label $0$.  With this, the adjacency matrix of the graph $G$ is then
\begin{equation}
  A(G) = A(\widetilde{G}) + \sum_{v\in S \subset V(\widetilde{G})} \ketbra{v}{0} + \ketbra{0}{v} + \sum_{x=-\infty}^\infty \ketbra{x}{x+1} + \ketbra{x+1}{x}.
\end{equation}

If we then want to inspect the eigenvectors of this Hamiltonian, we find that the eigenvalue equation on the infinite path is identical to that of an infinite path without the graph attached.  Hence, we can see that any eigenstate of the Hamiltonian must take the form $A_ke^{i k x} + B_{-k}e^{-i k x}$ for some $k$ along the infinite paths.  

With this assumption, we can see that there are three distinct cases for the form of the eigenstates.  In particular, the eigenstate could have no amplitude along the infinite paths, being confined to the finite graph $\widehat{G}$.  It could also be a normalizable state not confined to the finite graph $\widehat{G}$, in that the amplitude along the infinite paths decays exponentially.  Finally, the eigenstate could be an unnormalizable state, in which case we will call the state a scattering state.

Let us assume that the state is a scattering state.  Note that the eigenvalue of the state must be between $[-2,2]$, and that the form of the eigenstate along the paths must be scalar multiples of $e^{ikx}$ and $e^{-ikx}$.  Explicitly, the state must be of the form
\begin{align}
  \braket{x}{\psi} = \begin{cases} A_k e^{i k x} + B_k e^{i k x} & x \leq 0\\
   C_k e^{i k x} + D_k e^{i k x} & x\geq 0\end{cases}
\end{align}
where we note that the amplitude can change at $x=0$ since we have attached the graph $\widetilde{G}$.  However, we do have that $A_k + B_k=C_k +D_k$, since the amplitude at $0$ is single valued.  Additionally, we have that the eigenvalue of this state is given by $2\cos(k)$.  Note that we have not yet determined the form of the eigenstate inside the graph $\widetilde{G}$, but if we define $\ket{\phi}$ to be the restriction of $\ket{\psi}$ to the finite graph $\widetilde{G}$, then $\ket{\phi}$ must satisfy the equation
\begin{equation}
  A(G) \ket{\phi} + (A_k + B_k)\sum_{v\in S} \ket{v}\braket{v}{\phi} = 2\cos(k) \ket{\phi},
\end{equation}
where the additional term arises from the fact that the vertices in $S$ are connected to the vertex $0$.  Finally, we have that
\begin{equation}
  2 \cos(k)\braket{0}{\psi} = Ae^{-ik} + B_k e^{ik} + C_k e^{i k} + D_k e^{-ik} + \sum_{v\in S} \braket{v}{\phi},
\end{equation}
since the eigenvalue equation must be satisfied at $0$.


While all of this seems rather complicated, we can focus on the case where $A_k = 1$ and $D_k=0$ and the case where $A_k = 0$ and $D_k=1$ individually, so that along one of the semi-infinite paths (corresponding to $x>0$ or $x<0$), the amplitude is given by $e^{ikx}$ or $e^{-ikx}$.  These two states correspond to the eigenstates of the infinite path with the amplitudes given by $e^{-ikx}$ and $e^{ikx}$, with changes representing how adding the graph $\widetilde{G}$ affect the eigenstates.  

With this assumption, let us first look at the case where $A_k = 1$ and $D_k = 0$.  We then have that the eigenstates take the form
\begin{align}
  \braket{x}{\psi} = \begin{cases} e^{-ikx} + B_k e^{i kx} & x\leq 0\\
  C_k e^{-ikx} & x \geq 0\end{cases}
\end{align}
so that $1+ B_k  =C_k$.  Note that this is reminiscent of a scattering state, with reflection amplitude $B_k$ and transmission amplitude $C_k$, so that we take this intuition.  



\subsection{General graphs}


More concretely, let $\widehat{G}$ be any finite graph, with $n+m$ vertices and an adjacency matrix
\begin{equation}
  A(\widehat{G}) = \begin{pmatrix}A & B^\dag\\ B & D\end{pmatrix},
\end{equation}
where $A$ is an $N\times N$ matrix, $B$ is an $m\times N$ matrix, and $D$ is an $m\times m$ matrix.  When examining graph scattering, we will be interested in the graph $G$ given by the graph-join of $\widehat{G}$ and $N$ semi-infinite paths, with an additional edge between each of the first $N$ vertices of $\widehat{G}$ and the first vertex of one semi-infinite path.  If each semi-infinite path is labeled as $(x,i)$, where $x\geq 2$ is an integer and $i\in[N]$, then the adjacency matrix for $G$ will be
\begin{equation}
  A(G) = A(\widehat{G}) + \sum_{j=1}^N \sum_{x=1}^\infty \big(\ketbra{x,j}{x+1,j} + \ketbra{x+1,j}{x,j}\big).
\end{equation}

At this point, we want to examine the possible eigenstates of the matrix $A(G)$.  It turns out that there are 3 different kinds of eigenstates, corresponding to the different forms of the state on the infinite path.  We can easily see that the eigenstates along the path must have amplitude of the form $e^{\kappa x}$ for some $\kappa$, but the form of the $\kappa$ determines these form.

\subsubsection{Confined bound states}

The easiest states to analyze are the confined bound states, which are eigenstates in which the only nonzero amplitudes are on vertices inside the finite graph $\widehat{G}$.  In particular, we have that these states are eigenstates of the matrix $D$, with the further restriction that they are in the null space of the matrix $B^\dag$.

Note that there are no restrictions on the eigenvalues of these states, other than those that are inherited from any restrictions placed on it by $D$.

\subsubsection{Unconfined bound states}

The next interesting states are those that are not confined to the finite graph $\widehat{G}$, and thus they must take the form $e^{i k x}$ along the semi-infinite paths.  However, as we assume that the state is normalizable, we have to assume that $k\notin \RR$, and further that $\Im(k) < 0$.   

With this assumption, we have that the amplitudes along the paths decay exponentially, so that the state is bound to the graph $\widehat{G}$.  

Note that the energy of the eigenstate is given by 
\begin{equation}
  E = e^{i k } + e^{-i k}
\end{equation}
and as $E$ must be real, we have that $k = i \kappa + n\pi$ for $\kappa <0$.  We can assume that $n$ is either 0 or 1, as well.

\todo{Finish this section, and figure out what values of $\kappa$ are possible}

\todo{Are there only finitely many such $\kappa$, or is there a range of values?}

\subsubsection{Half-bound states}

The half-bound states are the limit of the states as $\kappa \rightarrow 0$.  In particular, they are those states where the amplitude along the infinite paths take the form $(\pm 1)^x$.  I don't really know much about them.

\todo{Finish this section, make it important}


\subsubsection{Scattering states}

We finally reach the point of scattering states, or those states we can use for computational tasks.  We first assume that we are orthogonal to all bound states, and in particular that we are othogonal to all confined bound states.  This allows us to uniquely construct the scattering states (without this assumption, if there existed a confined bound state at the appropriate energy, then we could simply add any multiple of the confined bound state to get a different scattering state).

Taking some intuition from the classical case, we will construct a state corresponding to sending a particle in along one of the semi-infinite paths.  Namely, we will assume that one of the paths has a portion of its amplitude of the form $e^{i k x} + S_{i,i}(k)e^{-i k x}$ for $k\in (-\pi, 0)$, and that the rest of the paths have amplitudes given by $S_{i,q}(k)e^{ikx}$.  More concretely, we assume that the form of the states is given on the infinite paths by
\begin{equation}
  \braket{x,q}{\scat_{j}(k)} = \delta_{j,q} e^{i k x} + S_{qj} e^{i k x}.
\end{equation}
We then need to see whether such an eigenstate exists.

In particular, if we assume that such an eigenstate exists, and that 
\todo{Finish this section}



\subsection{Scattering matrix properties}

While the use of the $\gamma$ matrix gives an explicit construction of the form of the eigenstates on the internal vertices, it is also useful to note that the scattering matrix at a particular momentum $k$ can be expressed as
\begin{equation}
  S(k) = - Q(z)^{-1} Q(z^{-1}),
\end{equation}
where $z=e^{i k}$, and the matrices $Q(z)$ are given by
\begin{equation}
  Q(z) = \II - z \Big( A + B^\dag \frac{1}{ \frac{1}{z} + z - D} B\Big).
\end{equation}
Note that $Q(z)$ and $Q(z^{-1})$ commute for all $z\in \CC$, as they can both be written as $\II + z H(z + z^{-1})$.

Using this expression for the scattering matrix, it is easy to see that $S(k)$ is a unitary matrix, as 
\begin{equation}
  S(k)^{\dag} = - Q(z^{-1})^\dag (Q(z)^{-1})^\dag
\end{equation}
and that
\begin{equation}
  Q(z)^\dag = \II^\dag - z^\dag\Big( A^\dag + B^\dag \Big(\frac{1}{\frac{1}{z} + z - D}\Big)^\dag (B^\dag)^\dag\Big) 
     = \II - z^\dag \Big(A + B^\dag \frac{1}{\frac{1}{z^\dag} + z^\dag - D} B \Big) = Q(z^\dag)
\end{equation}
and thus
\begin{equation}
  S(k)^\dag = -Q(z^{-1})^\dag (Q(z)^{-1})^\dag = - Q(z) Q(z^{-1})^{-1} = Q(z^{-1})^{-1} Q(z) = S(k)^{-1}
\end{equation}
where we used the fact that $z=e^{ik}$ so that $z^\dag = z^{-1}$, and the fact that $Q(z)$ and $Q(z^{-1})$ commute.

Additionally, we can make use of the fact that $S$ is derived from an unweighted graph to show that the scattering matrices are symmetric.  In particular, note that $Q(z)$ is symmetric for all $z$, since $D$ is symmetric, symmetric matrices are closed under inversion, $A$ is symmetric and $B$ is a 0-1 matrix.  As such, we have that
\begin{align}
  S(k)^T &= -\big( Q(z)^{-1} Q(z^{-1})\big)^T = -Q(z^{-1})^T (Q(z)^{-1})^T \\
    &= -Q(z^{-1}) Q(z)^{-1} = -Q(z)^{-1} Q(z^{-1}) = S(k)
\end{align}
where we used the fact that $Q(z)$ and $Q(z^{-1})$ commute.

Putting this together, we have that $S(k)$ is a symmetric, unitary matrix for all $k$.


\subsection{Orthonormality of the scattering states}

We now establish the delta-function normalization of the scattering states. Let 
\begin{align*}
\Pi_{1} & = \sum_{x=1}^{\infty}\sum_{q=1}^{N}|x,q\rangle\langle x,q|\\
\Pi_{2} & = \mathbb{I}-\sum_{x=2}^{\infty}\sum_{q=1}^{N}|x,q\rangle\langle x,q|\\
\Pi_{3} & = \sum_{q=1}^{N}|1,q\rangle\langle1,q|.\end{align*}
We show that, for $k\in(-\pi,0)$, $p\in(-\pi,0)$, and $i,j\in\{1,\ldots,N\}$,
\begin{equation}
\langle\mathrm{sc}_{i}(p)|\mathrm{sc}_{j}(k)\rangle=\langle\mathrm{sc}_{i}(p)|\Pi_{1}+\Pi_{2}-\Pi_{3}|\mathrm{sc}_{j}(k)\rangle=2\pi\delta_{ij}\delta(k-p).\label{eq:delta}\end{equation}
First write 
\begin{align*}
\langle\mathrm{sc}_{i}(p)|\Pi_{1}|\mathrm{sc}_{j}(k)\rangle & = \sum_{x=1}^{\infty}\sum_{q=1}^{N}(\delta_{iq}e^{ipx}+S_{qi}^{\ast}(p)e^{-ipx})(\delta_{jq}e^{-ikx}+S_{qj}(k)e^{ikx})\\
 & = \frac{1}{2}\left(\delta_{ij}+\sum_{q=1}^{N}S_{qi}^{\ast}(p)S_{qj}(k)\right)\left(\sum_{x=1}^{\infty}e^{i(p-k)x}+\sum_{x=1}^{\infty}e^{-i(p-k)x}\right)\\
 & \quad +\frac{1}{2}\left(\delta_{ij}-\sum_{q=1}^{N}S_{qi}^{\ast}(p)S_{qj}(k)\right)\left(\sum_{x=1}^{\infty}e^{i(p-k)x}-\sum_{x=1}^{\infty}e^{-i(p-k)x}\right)\\
 & \quad +\frac{1}{2}(S_{ji}^{\ast}(p)+S_{ij}(k))\left(\sum_{x=1}^{\infty}e^{-i(p+k)x}+\sum_{x=1}^{\infty}e^{i(p+k)x}\right)\\
 & \quad +\frac{1}{2}(S_{ji}^{\ast}(p)-S_{ij}(k))\left(\sum_{x=1}^{\infty}e^{-i(p+k)x}-\sum_{x=1}^{\infty}e^{i(p+k)x}\right).\end{align*}
We use the following identities for $p,k\in(-\pi,0)$: 
\begin{align*}
\sum_{x=1}^{\infty}e^{i(p-k)x}+\sum_{x=1}^{\infty}e^{-i(p-k)x} & = 2\pi\delta(p-k)-1 \\
\sum_{x=1}^{\infty}e^{i(p+k)x}+\sum_{x=1}^{\infty}e^{-i(p+k)x} &=-1 \\
\sum_{x=1}^{\infty}e^{i(p-k)x}-\sum_{x=1}^{\infty}e^{-i(p-k)x} & = i\cot\left(\frac{p-k}{2}\right) \\
\sum_{x=1}^{\infty}e^{i(p+k)x}-\sum_{x=1}^{\infty}e^{-i(p+k)x} &= i\cot\left(\frac{p+k}{2}\right).
\end{align*}
These identities hold when both sides are integrated against a smooth function of $p$ and $k$. Substituting, we get
\begin{align}
\langle\mathrm{sc}_{i}(p)|\Pi_{1}|\mathrm{sc}_{j}(k)\rangle & =  2\pi\delta_{ij}\delta(p-k)+\delta_{ij}\left(\frac{i}{2}\cot\left(\frac{p-k}{2}\right)-\frac{1}{2}\right)\nonumber\\
&\quad+\sum_{q=1}^{N}S_{qi}^{\ast}(p)S_{qj}(k)\left(-\frac{i}{2}\cot\left(\frac{p-k}{2}\right)-\frac{1}{2}\right)\nonumber \\
 &\quad+S_{ji}^{\ast}(p)\left(-\frac{1}{2}-\frac{i}{2}\cot\left(\frac{p+k}{2}\right)\right)\nonumber\\
&\quad+S_{ij}(k)\left(-\frac{1}{2}+\frac{i}{2}\cot\left(\frac{p+k}{2}\right)\right)\label{eq:pi1}
\end{align}
where we used unitarity of the $S$-matrix to simplify the first term.
Now turning to $\Pi_{2}$ we have \[
\langle\mathrm{sc}_{i}(p)|H\Pi_{2}|\mathrm{sc}_{j}(k)\rangle=2\cos(p)\langle\mathrm{sc}_{i}(p)|\Pi_{2}|\mathrm{sc}_{j}(k)\rangle\]
 and 
\begin{align*}
\langle\mathrm{sc}_{i}(p)|H\Pi_{2}|\mathrm{sc}_{j}(k)\rangle & = \langle\mathrm{sc}_{i}(p)|\bigg(2\cos(k)\Pi_{2}|\mathrm{sc}_{j}(k)\rangle+\sum_{q=1}^{N}(e^{-ik}\delta_{qj}+S_{qj}(k)e^{ik})|2,q\rangle\\
&\quad -\sum_{q=1}^{N}(e^{-2ik}\delta_{qj}+S_{qj}(k)e^{2ik})|1,q\rangle\bigg).
\end{align*}
 Using these two equations we get 
\begin{align*}
(2\cos(p)-2\cos(k))\langle\mathrm{sc}_{i}(p)|\Pi_{2}|\mathrm{sc}_{j}(k)\rangle & =  \delta_{ij} (e^{2ip-ik}-e^{-2ik+ip})+S_{ji}^{\ast}(p)(e^{-2ip-ik}-e^{-2ik-ip})\\
 & \quad +S_{ij}(k)(e^{2ip+ik}-e^{2ik+ip})\\
& \quad +\sum_{q=1}^{N}S_{qi}^{\ast}(p)S_{qj}(k)(e^{-2ip+ik}-e^{2ik-ip}).
\end{align*}
 Noting that
\[
\langle\mathrm{sc}_{i}(p)|\Pi_{3}|\mathrm{sc}_{j}(k)\rangle=\sum_{q=1}^{N}(\delta_{iq}e^{ip}+S_{qi}^{\ast}(p)e^{-ip})(\delta_{jq}e^{-ik}+S_{qj}(k)e^{ik}),
\]
we have 
\begin{align}
\langle\mathrm{sc}_{i}(p)|\Pi_{2}-\Pi_{3}|\mathrm{sc}_{j}(k)\rangle & = \delta_{ij}\left(\frac{e^{2ip-ik}-e^{-2ik+ip}}{2\cos(p)-2\cos(k)}-e^{ip-ik}\right)\nonumber\\
&\quad+S_{ji}^{\ast}(p)\left(\frac{e^{-2ip-ik}-e^{-2ik-ip}}{2\cos(p)-2\cos(k)}-e^{-ip-ik}\right)\nonumber \\
 & \quad +S_{ij}(k)\left(\frac{e^{2ip+ik}-e^{2ik+ip}}{2\cos(p)-2\cos(k)}-e^{ip+ik}\right)\nonumber\\
&\quad+\sum_{q=1}^{N}S_{qi}^{\ast}(p)S_{qj}(k)\left(\frac{e^{-2ip+ik}-e^{2ik-ip}}{2\cos(p)-2\cos(k)}-e^{-ip+ik}\right)\nonumber \\
  & = \delta_{ij}\left(\frac{1}{2}-\frac{i}{2}\cot\left(\frac{p-k}{2}\right)\right)+S_{ji}^{\ast}(p)\left(\frac{1}{2}+\frac{i}{2}\cot\left(\frac{p+k}{2}\right)\right)\nonumber\\
&\quad+S_{ij}(k)\left(\frac{1}{2}-\frac{i}{2}\cot\left(\frac{p+k}{2}\right)\right)\nonumber\\
&\quad+\sum_{q=1}^{N}S_{qi}^{\ast}(p)S_{qj}(k)\left(\frac{1}{2}+\frac{i}{2}\cot\left(\frac{p-k}{2}\right)\right).
\label{eq:pi2_pi3}
\end{align}
 Adding equation \eq{pi1} to equation \eq{pi2_pi3} gives
equation \eq{delta}.

\todo{Do they span the space of states, if you also include the bound states?}

\section{Applications of graph scattering}

\subsection{NAND Trees}

\todo{I need to give an explanation of this}

\subsection{Momentum dependent actions}

While the NAND trees gives a good example of how the process works, we can generalize the idea to work at momenta other than \todo{what momenta}.  In particular, we can attempt to find graph gadgets such that the scattering behaviour at some particular momenta is fixed.

%%%%%%%%%%%%%%%%%%%%%%%%%%%%%%%%%%%%%%%%%%%%%%%%%%%%%%%%%%%%%%%%%

\subsubsection{R/T gadets}

The easiest thing we could hope for are exactly similar to the NAND trees experiment, in that if there are only two attached semi-infite paths, then at some fixed momenta it either completely transmits, or it completely reflects.  

%%%%%%%%%%%%%%%%%%%%%%%%%%%%%%%%%%%%%%%%%%%%%%%%%%%%%%%%%%%%%%%%%

\subsubsection{Momentum Switches}

We can generalize this idea of complete reflection or transmission to something called a momentum switch, in that with three inputs/outputs, for some chosen semi-infinite path, all incoming wavepackets at some momenta completely transmit to a second path, while all incoming wavepackets at some other particular momenta transmit to the third.

%%%%%%%%%%%%%%%%%%%%%%%%%%%%%%%%%%%%%%%%%%%%%%%%%%%%%%%%%%%%%%%%%

\subsection{Encoded unitary}


4 input/output, must go from in to out.

This is as very particular behavior.

%%%%%%%%%%%%%%%%%%%%%%%%%%%%%%%%%%%%%%%%%%%%%%%%%%%%%%%%%%%%%%%%%
%  Constructing graphs
%%%%%%%%%%%%%%%%%%%%%%%%%%%%%%%%%%%%%%%%%%%%%%%%%%%%%%%%%%%%%%%%%

\section{Construction of graphs with particular scattering behavior}

Note that while these scattering behaviors at particular momenta are easy to calculate, no efficient way currently exists to find a graph with a given scattering matrix, or even to tell whether or not such a graph exists.  However, there are some special types of graphs that allow us to do this.

%%%%%%%%%%%%%%%%%%%%%%%%%%%%%%%%%%%%%%%%%%%%%%%%%%%%%%%%%%%%%%%%%

\subsection{R/T gadgets}

\todo{Go over this section, and revise}

Perhaps the most simple behavior will be two-terminal gadgets that either perfectly reflect at some particular momenta, or perfectly transmit.  While this is still a rather complicated problem when the terminals can be any vertices of the graph, things become tractable when we want to only attach a graph to a single vertex of an infinite path.  In this case, everything works out as expected.

We refer to the graph shown in \fig{reversal_orig} as $\hat{G}$, and we write $G$ for the full graph obtained by attaching two semi-infinite paths to terminals $(1,1)$ and $(1,2)$.  As shown in the Figure, the graph $\hat{G}$ for a type 1 gadget is determined by a finite graph $G_0$ and a subset $P = \{p_1,\ldots,p_n\} \subseteq V(G_0)$ of its vertices, called the \emph{periphery}.  Each vertex in the periphery is connected to a vertex denoted $a$, and $a$ is also connected to two terminals $(1,1)$ and $(1,2)$. A type 1 R/T gadget with $n=1$ has only one edge between $G_0$ and $a$; in this special case we also call it a \emph{type 2 R/T gadget} (see \fig{reversalRT}).

Looking at the eigenvalue equation for the scattering state $\ket{\sc_{1} (k)}$ at vertices $(1,1)$ and $(1,2)$, we see that the amplitude at vertex $a$ satisfies
\[
  \langle{a}\ket{\sc_{1} (k)} = 1 + R(k) = T(k).
\] 
Thus perfect reflection at momentum $k$ occurs if and only if $R(k)=-1$ and $\langle{a}\ket{\sc_{1} (k)}=0$, while perfect transmission occurs if and only if $T(k)=1$ and $\langle{a}\ket{\sc_{1} (k)}=1$. Using this fact, we now derive conditions on the graph $G_0$ that determine when perfect transmission and reflection occur.

For type 1 gadgets, we give a necessary and sufficient condition for perfect reflection: $G_0$ should have an eigenvector for which the sum of amplitudes on the periphery is nonzero.

\begin{lemma}\label{lem:reflect_reqs}
Let $\hat{G}$ be a type 1 R/T gadget. A momentum $k\in (-\pi,0)$ is in the reflection set $\mathcal{R}$ if and only if $G_0$ has an eigenvector $\ket{\chi_k}$ with eigenvalue $2\cos(k)$ satisfying
\begin{equation}
  \sum_{i=1}^{n} \braket{p_i}{\chi_k} \neq 0. \label{eq:sum_condition}
\end{equation}
\end{lemma}

\begin{proof}
First suppose that $\hat{G}$ has perfect reflection at momentum $k$, i.e., $R(k)=-1$ and $\langle{a}\ket{\sc_{1} (k)}=0$. Since $\langle{(1,1)}\ket{\sc_1(k)} = e^{-ik} - e^{ik}\neq 0$ and $\langle{(1,2)}\ket{\sc_1(k)}=0$, to satisfy the eigenvalue equation at vertex $a$, we have
\[
  \sum_{j=1}^{n} \langle{p_j}\ket{\sc_1(k)} = e^{ik} - e^{-ik} \neq 0.
\]
Further, since $G_0$ only connects to vertex $a$ and the amplitude at this vertex is zero, the restriction of $\ket{\sc_1(k)}$ to $G_0$ must be an eigenvector of $G_0$ with eigenvalue $2\cos(k)$. Hence the condition is necessary for perfect reflection. 
 
Next suppose that $G_0$ has an eigenvector $\ket{\chi_k}$ with eigenvalue $2\cos(k)$ satisfying \eq{sum_condition}, with the sum equal to some nonzero constant $c$. Define a scattering state $\ket{\psi_k}$ on the Hilbert space of the full graph $G$ with amplitudes
\[
  \braket{v}{\psi_k} = \frac{e^{ik} - e^{-ik}}{c} \braket{v}{\chi_k}
\]
for all $v \in V(G_0)$, $\braket{a}{\psi_k}=0$, and 
\[
 \braket{(x,j)}{\psi_k}=\begin{cases} e^{-ikx}-e^{ikx} & j=1\\
0 & j=2
\end{cases}
\]
for all $x \in \posint$.

We claim that $\ket{\psi_k}$ is an eigenvector of $G$ with eigenvalue $2 \cos(k)$.  The state clearly satisfies the eigenvalue equation on the semi-infinite paths since it is a linear combination of states with momentum $\pm k$.  At vertices of $G_0$, the state is proportional to an eigenvector of $G_0$, and since the state as no amplitude at $a$, the eigenvalue equation is also satisfied at these vertices.  It remains to see that the eigenvalue equation is satisfied at $a$, but this follows immediately by a simple calculation.

Since $\ket{\psi_k}$ has the form of a scattering state with perfect reflection, we see that $R(k)=-1$ and $T(k)=0$ as claimed.
\end{proof}

The following Lemma gives a sufficient condition for perfect transmission (which is also necessary for type 2 gadgets).  Let $g_0$ denote the induced subgraph on $V(G_0)\setminus P$ where $P = \{p_i\colon i\in [n]\}$ is the periphery.

\begin{lemma}\label{lem:transmit_reqs}
Let $\hat{G}$ be a type 1 R/T gadget and let $k\in (-\pi,0)$. Suppose $\ket{\xi_k}$ is an eigenvector of $g_0$ with eigenvalue $2\cos{k}$ and with the additional property that, for all $i \in [n]$,
\begin{equation}
\label{eq:trans_cond}
  \sum_{\substack{v \in V(g_0): \\ (v,p_i)\in E(G_0)}} \braket{v}{\xi_k} = c \neq 0 
\end{equation}
for some constant $c$ that does not depend on $i$. Then $k$ is in the transmission set $\mathcal{T}$. If $\hat{G}$ is a type 2 R/T gadget, then this condition is also necessary.
\end{lemma}

\begin{proof}
If $g_0$ has a suitable eigenvector $\ket{\xi_k}$ satisfying \eq{trans_cond}, define a scattering state $\ket{\psi_k}$ on the full graph $G$, with amplitudes $\langle a\ket{\psi_k}=1$, 
\begin{equation}
  \braket{v}{\psi_k} 
  = \begin{cases} -\frac{1}{c} \braket{v}{\xi_k} & v\in V(g_0)\\
  	0 & v \in P
\end{cases}
\label{eq:psik_c}
\end{equation}
in the graph $G_0$, and 
\[
 \langle{(x,j)} \ket{\psi_k}=\begin{cases} e^{-ikx} & j=1\\
 e^{ikx} & j=2
\end{cases}
\]
for $x \in \posint$.  As in the proof of \lem{reflect_reqs}, the state $\ket{\psi_k}$ is clearly satisfies the eigenvalue equation (with eigenvalue $2\cos(k)$) at vertices on the semi-infinite paths and vertices of $g_0$.  The factor of $-\frac{1}{c}$ in \eq{psik_c} is chosen so that the eigenvalue condition is satisfied at vertices in $P$.  It is easy to see that the eigenvalue condition is also satisfied at $a$.

Since $\ket{\psi_k}$ is a scattering eigenvector of $G$ with eigenvalue $2\cos(k)$ and perfect transmission, we have $T(k)=1$.

Now suppose $\hat{G}$ is a type 2 R/T gadget (as shown in \fig{reversalRT}), with $P = \{p\}$.  Perfect transmission along with the eigenvalue equation at vertex $a$ implies
\[
\braket{p}{\sc_1(k)} = 0,
\]
so the restriction of $\ket{\sc_1(k)}$ to $g_0$ must be an eigenvector (since $p$ is the only vertex connected to $g_0$).  The eigenvalue equation at $p$ gives
\[
  \braket{a}{\sc_1(k)} 
  + \sum_{w \colon (w,p)\in E(G_0)} \braket{w}{\sc_1(k)} = 0 
  \quad\implies\quad
  \sum_{w \colon (w,p)\in E(G_0)} \braket{w}{\sc_1(k)} = -1.
\]
Hence the restriction of $\ket{\sc_1(k)}$ to $V(g_0)$ is an eigenvector of the induced subgraph, with the additional property that the sum of the amplitudes at vertices connected to $p$ is nonzero.
\end{proof}

%%%%%%%%%%%%%%%%%%%%%%%%%%%%%%%%%%%%%%%%%%%%%%%%%%%%%%%%%%%%%%%%%

\subsection{Momentum switches}

\todo{Go over this section, and revise/make it fit}

To construct momentum switches between pairs of momentum, it will be worthwhile to first construct two R/T gadgets between the two momenta, with the two gadgets having swapped reflection and transmission sets.  We will then construct something like a railroad switch, by placing the two gadgets immediately after a 3-claw; this construction will then be that the wavepacket will only see one of the two outgoing paths, and function exactly how we want it to.


We now construct a momentum switch between the reflection and transmission sets $\mathcal{R}$ and $\mathcal{T}$ of a type 2 R/T gadget.  We attach the gadget and its reversal (defined in \sec{reversal}) to the leaves of a claw, as shown in \fig{gen_mom_con}.  Specifically, given a type 2 R/T gadget $\hat{G}$, the corresponding momentum switch $\hat{G}^{\prec}$ consists of a copy of $G_0$, a copy of $G_{0}^{\leftrightarrow}$, and a claw.  The three leaves of the claw are the terminals.  Vertex $p$ of $G_0$ is connected to leaf $2$ of the claw, and vertices $w_1^{(1)},\ldots,w_r^{(1)}$ of $G_{0}^{\leftrightarrow}$ are each connected to leaf $3$ of the claw.

The high-level idea of the switch construction is as follows.  For momenta in the transmission set, the gadget perfectly transmits while its reversal perfectly reflects, so the claw is effectively a path connecting terminals 1 and 2.  For momenta in the reflection set, the roles of transmission and reflection are reversed, so the claw is effectively a path connecting terminals 1 and 3.


\begin{lemma}\label{lem:mom_switch_construction}
Let $\hat{G}$ be a type 2 R/T gadget with reflection set $\mathcal{R}$ and transmission set $\mathcal{T}$.  The gadget $\hat{G}^{\prec}$ described above is a momentum switch between $\mathcal{R}$ and $\mathcal{T}$.
\end{lemma}


\begin{proof}

We construct a scattering eigenstate for each momentum $k\in \mathcal{T}$ with perfect transmission from path 1 to path 2, and similarly construct a scattering eigenstate for each momentum $k^{\prime}\in \mathcal{R}$ with perfect transmission from 1 to 3.  These eigenstates show that $S_{2,1}(k) = 1$ and $S_{3,1}(k^\prime) = 1$. Since the S-matrix is symmetric and unitary, this gives the complete form of the S-matrix for all momenta in $\mathcal{R}\cup\mathcal{T}$.  In particular, this shows that $\hat{G}^{\prec}$ is a momentum switch between $\mathcal{R}$ and $\mathcal{T}$.

We first construct the scattering states for momenta $k\in \mathcal{T}$.  \lem{transmit_reqs} shows that the graph $g_0$ has a $2\cos(k)$-eigenvector $\ket{\xi_k}$ satisfying equation \eq{trans_cond} with some nonzero constant $c$. We define a state $\ket{\mu_k}$ on $G^{\prec}$ and we show that it is a scattering eigenstate with perfect transmission between paths $1$ and $2$.   The amplitudes of $\ket{\mu_k}$ on the semi-infinite paths and the claw are
\[
  \langle (x,1)|\mu_k\rangle=e^{-ikx} \qquad 
  \langle 0|\mu_k\rangle=1 \qquad 
  \langle (x,2)|\mu_k\rangle=e^{ikx} \qquad
  \langle (x,3)|\mu_k\rangle=0.
\]
The rest of the graph consists of the three copies of the subgraph $g_0$ and the vertices $p$ and $u_{\leftrightarrow}$. The corresponding amplitudes are
\[
  \braket{v}{\mu_k} =
  \begin{cases}
	  -\frac{1}{c}\braket{v}{\xi_k} & v\in V(g^{(1)}_0) \\
    \frac{1}{c}\braket{v}{\xi_k} & v \in V(g^{(2)}_0) \\
	  -\frac{e^{ik}}{c} \braket{v}{\xi_k} & v \in V(g^{(3)}_0) \\
  	0 & v=p \text{ or } v=u_{\leftrightarrow}.
  \end{cases}
\]

We claim that $\ket{\mu_k}$ is an eigenstate of the Hamiltonian with eigenvalue $2\cos(k)$.  As in previous proofs, the state clearly satisfies the eigenvalue condition on the semi-infinite paths and at the vertices of $G_0$ and $G_0^\leftrightarrow$, and the factors of $\frac{1}{c}$ in the above equation are chosen so that it also satisfies the eigenvalue condition at vertices $p$ and $u_\leftrightarrow$. Since $\ket{\mu_k}$ is a scattering state with perfect transmission from path 1 to path 2, we see that $S_{2,1}(k) = 1$.

Finally, we construct an eigenstate $\ket{\nu_{k^\prime}}$ with perfect transmission from path 1 to path 3 for each momentum $k^\prime \in \mathcal{R}$.  This state has the form
\[
  \langle (x,1)|\nu_{k^\prime}\rangle=e^{-i k^\prime x} \qquad 
  \langle 0|\nu_{k^\prime}\rangle=1 \qquad 
  \langle (x,2)|\nu_{k^\prime}\rangle=0 \qquad
  \langle (x,3)|\nu_{k^\prime}\rangle=e^{i k^\prime x}
\]
on the semi-infinite paths and the claw.  \lem{reflect_reqs} shows that $G_0$ has a $2\cos(k^\prime)$-eigenstate $\ket{\chi_{k^\prime}}$ with $\braket{p}{\chi_{k'}} \ne 0$, which determines the form of $\ket{\nu_{k^\prime}}$ on the remaining vertices:
\[
  \braket{v}{\nu_{k^\prime}} =
  \begin{cases}
	  -\frac{1}{\braket{p}{\chi_{k'}}} \braket{v}{\chi_{k^\prime}} & v \in V(G_0) \\
	  -\frac{e^{ik'}}{\braket{p}{\chi_{k'}}} \braket{v}{\chi_{k^\prime}} & v \in V(g_0^{(2)}) \\
	  -e^{ik'} & v=u^\leftrightarrow \\
    0 & \text{otherwise}.
  \end{cases} 
\]
As before, it is easy to check that this a momentum-$k^\prime$ scattering state with perfect transmission from path 1 to path 3, so $S_{3,1}(k^\prime)=1$.

Thus the gadget from \fig{gen_mom_con} is a momentum switch between $\mathcal{R}$ and $\mathcal{T}$.
\end{proof}



\subsection{Encoded unitaries}

While there is no efficient method to find graphs that apply some fixed encoded unitary, it is possible to search over all small graphs that have some particular implementation.

\missingcite{Find this small graphs thing}

In particular, we have that these graphs have nice scattering behaviors, and will be useful in the long run.

Additionally, it is possible to combine some graphs in a manner that can be used 


\section{Various facts about scattering}

These are facts that will be of use to us.

\subsection{Degree-3 graphs are sufficent}

Replace each vertex by a path of fixed length.  I should turn this into a lemma.

\begin{lemma} Let $\widehat{G}$ be a finite graph, and let $M$ be a finite set of rational multiples of $\pi$.  There exists a degree 3 graph $\widehat{H}$ such that the scattering matrix arising from $\widehat{H}$ is equal to the scattering matrix arising from $\widehat{G}$ for each $ki\in M$.
\end{lemma}
\begin{proof}
  The graph $\widehat{H}$ is constructed from $\widehat{G}$ by replacing each vertex with large degree by a path of some length related to the momenta in $M$.  If the path is taken to be the correct size, by inspecting the resulting eigenvectors we can show that the scattering amplitudes of the new graph are equal to the scattering graphs of the old graph.
  
  In particular, let $\widehat{G}$ be as described, and let $w\in V(G)$ be any vertex of degree $d>3$.  Additionally, let $L$ be the least integer such that $Lk/\pi \in \ZZ$ for each $k\in M$ (i.e., the least common multiple of the denominators of each $k\in M$).  Let $\widetilde{H}$ be the graph obtained by 
  
  \todo{Finish this proof}
\end{proof}

%%%%%%%%%%%%%%%%%%%%%%%%%%%%%%%%%%%%%%%%%%%%%%%%%%%%%%%%%%%%%%%%%

\subsection{Not all momenta can be split}

In addition, it might be useful to see when particular scattering behavior is possible or not.  As such ,we will show that no momentum switch can exist between the pairs of momenta $-\frac{\pi}{4}$ and $-\frac{3\pi}{4}$.  The proof will actually show that no R/T gadget exists between these two momenta, but as any momentum switch can be turned into an R/T gadget, this will be sufficient.

\todo{Go over this section, and revise}

\subsubsection{Basis vectors with entries in $\QQ(\sqrt{2})$}
\label{sec:vecs_over_field}

Recall the general setup shown in \fig{basic_scattering}: $N$ semi-infinite paths are attached to a finite graph $\hat G$. Consider an eigenvector $\ket{\tau_k}$ of the adjacency matrix of $G$ with eigenvalue $2\cos(k)$ for $k\in (-\pi,0)$. In general this eigenspace is spanned by incoming scattering states with momentum $k$ and confined bound states \cite{CG12} (which have zero amplitude on the semi-infinite paths). We can thus write the amplitudes of $\ket{\tau_k}$ on the semi-infinite paths as
\[
  \braket{(x,j)}{\tau_k} 
  = \kappa_j \cos(k (x-1)) + \sigma_j \sin(k (x-1))
\]
for $x \in \posint$, $j \in [N]$, and $\kappa_j,\sigma_j \in \CC$, and the amplitudes on the internal vertices as
\[
  \braket{w}{\tau_k} = \iota_w
\]
for $\iota_w \in \CC$, where $w$ indexes the internal vertices. We write the adjacency matrix of $\hat{G}$ as a block matrix as in \eq{blockadj}.  Since the state $\ket{\tau_k}$ satisfies the eigenvalue equation on the semi-infinite paths, it remains to satisfy the conditions specified by the block matrix equation
\begin{equation*}
  \begin{pmatrix} A & B^\dag\\ B & D\end{pmatrix}
	\begin{pmatrix} \kappa \\ \iota \end{pmatrix}
	+ \cos(k) \begin{pmatrix} \kappa \\ 0 \end{pmatrix}
	+ \sin(k) \begin{pmatrix} \sigma \\ 0 \end{pmatrix} 
	= 2\cos(k) \begin{pmatrix} \kappa \\ \iota \end{pmatrix}.
\end{equation*}

Hence, the nullspace of the matrix
\[
  M= \begin{pmatrix} A -\cos(k) \II & \sin(k) \II & B^\dag\\
    0 & 0 & 0\\
    B & 0 & D-2\cos(k)\II\end{pmatrix}
\]
is in one-to-one correspondence with the $2\cos(k)$-eigenspace of the infinite matrix (here the first block corresponds to $\kappa$, the second to $\sigma$, and the third to $\iota$). Further, $M$ only has entries in $\QQ(\cos(k),\sin(k))$, so its nullspace has a basis with amplitudes in $\QQ(\cos(k),\sin(k))$, as can be seen using Gaussian elimination.

We are interested in the specific cases $2\cos(k)=\pm\sqrt{2}$ corresponding to $k = -\frac{\pi}{4}$ or $k = -\frac{3\pi}{4}$.  In these cases $\QQ(\cos(k),\sin(k)) = \QQ(\sqrt{2})$, and we may choose a basis for the nullspace of $M$ with amplitudes from $\QQ(\sqrt{2})$. Furthermore, $\cos(kx), \sin(kx) \in \QQ(\sqrt{2})$ for all $x\in \posint$, so with such a choice of basis, each amplitude of $\ket{\tau_k}$ is also an element of $\QQ(\sqrt{2})$.

As noted above, the spectrum of $G$ may include confined bound states \cite{CG12} with eigenvalue $\pm\sqrt2$.  However, any such states are eigenstates of the adjacency matrix of $\hat{G}$ subject to the additional (rational) constraints that the amplitudes on the terminals are zero.  As such, the confined bound states have a basis over $\QQ(\sqrt{2})$. We can use this basis to restrict attention to those states orthogonal to confined bound states using only constraints over $\QQ(\sqrt{2})$, so there exists a basis over $\QQ(\sqrt{2})$ for the $N$-dimensional subspace of scattering states with energy $\pm\sqrt{2}$ that are orthogonal to the confined bound states. Finally, since $\QQ(\sqrt{2})$ can be seen as a two-dimensional vector space over $\QQ$, note that for any member of this basis $\ket{\tau_k}$ there exist rational vectors $\ket{u_k},\ket{w_k}$ such that $\ket{\tau_k}=\ket{u_k} + \sqrt{2}\ket{w_k}$. Since $H^2\ket{\tau_k}=2\ket{\tau_k}$, we have $H\ket{u_k}=\pm 2\ket{w_k}$ and $H\ket{w_k} = \pm \ket{u_k}$, so
\begin{equation}
  \ket{\tau_k}=(H \pm \sqrt{2} \II)\ket{w_k}.\label{eq:form_of_tau}
\end{equation}

%================================================================
\subsubsection{No R/T gadget and hence no momentum switch}

Recall from \sec{mswitch} that a momentum switch between two momenta $k$ and $p$ can always be converted into an R/T gadget between $k$ and $p$. Here we show that if an R/T gadget perfectly reflects at momentum $-\frac{\pi}{4}$, then it must also perfectly reflect at momentum $-\frac{3\pi}{4}$. This implies that no R/T gadget exists between these two momenta, and thus no momentum switch exists.

We use the following basic fact about two-terminal gadgets several times: 


\begin{fact}\label{fct:zero_ampl}
If a two-terminal gadget has a momentum-$k$ scattering state $\ket{\phi}$ with zero amplitude along path $2$, then the gadget perfectly reflects at momentum $k$.
\end{fact}


\begin{proof}
Without loss of generality, we may assume that $\ket{\phi}$ is orthogonal to all confined bound states.
If $\ket{\phi}$ has zero amplitude along path $2$, then there exist some $\mu,\nu \in \CC$ such that
\[
  \braket{(x,2)}{\phi}
  = \mu \braket{(x,2)}{\sc_2 (k)} + \nu \braket{(x,2)}{\sc_1(k)}
  = \mu e^{-ikx} + \mu R e^{ikx} + \nu T e^{ikx}
  = 0
\] 
for all $x\in \ZZ^{+}$.  Since this holds for all $x$, we have $\mu = \mu R + \nu T = 0$.  Since $\mu$ and $\nu$ cannot both be zero, we have $T=0$.
\end{proof}

For an R/T gadget, the scattering states (at some fixed momentum) that are orthogonal to the confined bound states span a two-dimensional space. As shown in \sec{vecs_over_field}, we can expand each scattering eigenstate at momentum $k=-\frac{\pi}{4}$ in a basis with entries in $\QQ(\sqrt{2})$, where each basis vector takes the form \eq{form_of_tau}. This gives
\begin{equation*}
  \ket{\sc_{1}(-\tfrac{\pi}{4})} 
  = (H + \sqrt{2}\II) (\alpha \ket{a} + \beta \ket{b}) \label{eq:rat_expansion}
\end{equation*}
where $\alpha,\beta \in \CC$, $\alpha\neq 0$, and $\ket{a}$ and $\ket{b}$ are rational 2-eigenvectors of $H^2$.

If $T(-\frac{\pi}{4}) = 0$, then for all $x \geq 0$, 
\[
  \langle x,2\ket{\sc_1(-\tfrac{\pi}{4})} 
  = 0 
  = \langle{x,2} |(H + \sqrt{2}\II) (\alpha \ket{a} + \beta \ket{b}).
\]
Dividing through by $\alpha$ and rearranging, we get that for all $x\geq 0$,
\begin{equation*}
\frac{\beta}{\alpha} (\langle{x,2}| H\ket{b}+\sqrt{2} \langle{x,2}\ket{b})
  =-\langle{x,2}| H\ket{a}  -\sqrt{2} \langle{x,2}\ket{a}.
\label{eq:cases_eqn}
\end{equation*}
If the left-hand side is not zero, then $\beta/\alpha \in \QQ(\sqrt{2})$ since $H$, $\ket{a}$, and $\ket{b}$ are rational.  If the left-hand side is zero, then $(H+ \sqrt{2}\II)\ket{a}$ is an eigenstate at energy $2\cos(k)$ with no amplitude along path 2, so $\beta = 0$ (using \fct{zero_ampl}), and again $\beta/\alpha \in \QQ(\sqrt{2})$.

Now write $\beta/\alpha=r+s\sqrt{2}$ with $r,s\in\QQ$, and consider the rational 2-eigenvector of $H^2$
\[
  \ket{c} := \ket{a} + (r+sH) \ket{b}.
\]
Note that
\[
  \alpha (H + \sqrt{2} \II) \ket{c} 
= \alpha(H+ \sqrt{2} \II) \ket{a} + \alpha (rH+r\sqrt{2}+ sH^2+sH\sqrt{2}) \ket{b}.
\]
Since $\ket{b}$ is a 2-eigenvector of $H^2$ and $\beta/\alpha=r+s\sqrt{2}$, this simplifies to
\begin{equation}
  \alpha (H + \sqrt{2}\II) \ket{c} 
  = \alpha (H + \sqrt{2}\II) \ket{a} + \beta(H + \sqrt{2}\II) \ket{b} 
  = \ket{\sc_1(-\tfrac{\pi}{4})}, \label{eq:sc1_c}
\end{equation}
so $\ket{\sc_1(-\tfrac{\pi}{4})}$ can be written as $\alpha(H+\sqrt{2}\II)$ times a rational 2-eigenvector of $H^2$.

Since $\langle{x,2}\ket{\sc_1(-\tfrac{\pi}{4})} = 0$ for all $x\geq 1$ (and $\alpha\neq 0$), we have
\[
  \langle{x,2}| (H+\sqrt{2}\II) \ket{c} 
  = \langle{x,2}| H \ket{c} + \sqrt{2} \langle{x,2}\ket{c} 
  = 0.
\]
As $H$ is a rational matrix and $\ket{c}$ is a rational vector, the rational and irrational components must both be zero, implying $\langle{x,2}\ket{c}  =\langle{x,2}|H\ket{c} = 0$ for all $x\geq 1$. Furthermore, since $ \ket{\sc_1(-\tfrac{\pi}{4})}$ is a scattering state with zero amplitude on path $2$, it must have some nonzero amplitude on path 1 and thus there is some $x_0\in \mathbb{Z}^+$ for which $\langle{x_0,1}\ket{c}\neq 0$ or $\langle{x_0,1}|H\ket{c} \neq 0$.

Now consider the state obtained by replacing $\sqrt{2}$ with $-\sqrt{2}$:
\[
  \ket{\overline{\sc}_1(-\tfrac{\pi}{4})} := \alpha (H-\sqrt{2}\II) \ket{c}.
\]
This is a $-\sqrt{2}$-eigenvector of $H$, which can be confirmed using the fact that $\ket{c}$ is a $2$-eigenvector of $H^2$. As $\langle{x,2} | H \ket{c} = \langle x,2 \ket{c} = 0$ for all $x\geq 1$, $\langle x,2 \ket{\overline{\sc}_1(-\tfrac{\pi}{4})} = 0$ for all $x\geq 1$. Furthermore the amplitude at vertex $(x_0,1)$ is nonzero, i.e.,  $\langle x_0,1 \ket{\overline{\sc}_1(-\tfrac{\pi}{4})} \neq 0$, and hence $\ket{\overline{\sc}_1(-\tfrac{\pi}{4})}$ has a component orthogonal to the space of confined bound states (which have zero amplitude on both semi-infinite paths).  Hence, there exists a scattering state with eigenvalue $-\sqrt{2}$ with no amplitude on path 2. By \fct{zero_ampl}, the gadget perfectly reflects at momentum $-\frac{3\pi}{4}$.  It follows that no perfect R/T gadget (and hence no perfect momentum switch) exists between these momenta.

This proof technique can also establish non-existence of momentum switches between other pairs of momenta $k$ and $p$.  For example, a slight modification of the above proof shows that no momentum switch exists between $k = -\frac{\pi}{6}$ and $p = -\frac{5\pi}{6}$. 



\end{document}