%======================================================================
%   Zak Webb
%   Ph. D. Thesis
%   Department of Physics and Astronomy
%   University of Waterloo
% 
%   Ground energy of multi-particle quantum walk
%======================================================================


\documentclass[../thesis-main/thesis-main]{subfiles}
\begin{document}

\chapter{Ground energy of multi-particle quantum walk}

\section{Introduction}
\subsection{Containment in QMA}
\subsection{Reduction to frustration-free case}


\section{Constructing the underlying graph for QMA-hardness}

\subsection{Gate graphs}

In this Section we define a class of graphs (\emph{gate graphs}) and a diagrammatic notation for them (\emph{gate diagrams}). We also discuss the Bose-Hubbard model on these graphs.

Every gate graph is constructed using a specific $128$-vertex graph $g_{0}$ as a building block. This graph is shown in \fig{g_0} and discussed in \sec{Encoding-a-Computation}. In \sec{Gate-graphs-and} we define gate graphs and gate diagrams. A gate graph is obtained by adding edges and self-loops (in a prescribed way) to a collection of disjoint copies of $g_{0}$.

In \sec{FF_State} we discuss the ground states of the Bose-Hubbard model on gate graphs. For any gate graph $G$, the smallest eigenvalue $\mu(G)$ of the adjacency matrix $A(G)$ satisfies $\mu(G)\geq-1-3\sqrt{2}$. It is convenient to define the constant
\begin{equation}
e_{1}=-1-3\sqrt{2}.\label{eq:e1_defn}
\end{equation}
When $\mu(G)=e_{1}$ we say $G$ is an $e_{1}$-gate graph. We focus on the frustration-free states of $e_1$-gate graphs (recall from \defn{FF_states} that $|\phi\rangle\in \mathcal{Z}_N(G)$ is frustration free iff $H(G,N)|\phi\rangle=0$). We show that all such states live in a convenient subspace (called $\mathcal{I}(G,N)$) of the $N$-particle Hilbert space. This subspace has the property that no two (or more) particles ever occupy vertices of the same copy of $g_{0}$. The restriction to this subspace makes it easier to analyze the ground space.

In \sec{Occupancy-constraints} we consider a class of subspaces that, like $\mathcal{I}(G,N)$, are defined by a set of constraints on the locations of $N$ particles in an $e_{1}$-gate graph $G$. We state an ``Occupancy Constraints Lemma'' (proven in \app{Occupancy-Constraints-Lemma}) that relates a subspace of this form to the ground space of the Bose-Hubbard model on a graph derived from $G$.


\subsubsection{The graph $g_0$}

The graph $g_{0}$ shown in \fig{g_0} is closely related to a single-qubit circuit $\mathcal{C}_{0}$ with eight gates $U_{j}$ for $j\in[8]$, where 
\begin{align*}
U_{1} & =U_{2}=U_{7}=U_{8}=H\qquad U_{3}=U_{5}=HT\qquad U_{4}=U_{6}=\left(HT\right)^{\dagger}
\end{align*}
with 
\[
H=\frac{1}{\sqrt{2}}\begin{pmatrix}
1 & 1\\
1 & -1
\end{pmatrix}\qquad T=\begin{pmatrix}
1 & 0\\
0 & e^{i\frac{\pi}{4}}
\end{pmatrix}.
\]
In this section we map this circuit to the graph $g_{0}$. The mapping we use can be generalized to map an arbitrary quantum circuit with any number of qubits to a graph, but for simplicity we focus here on $g_{0}$. In \app{complexity_smallest_graph_eig} we discuss the more general mapping and use it to prove that computing (in a certain precise sense specified in the Appendix) the smallest eigenvalue of a sparse, efficiently row-computable symmetric $0$-$1$ matrix is QMA-complete.

%\begin{figure}
%\centering
%\begin{tikzpicture}[scale = 0.75]
%
%  % These are the connections from t to t+1
%
%  \foreach \sym in {1,-1}{  % note that the graph is symmetric about x=0
%  \begin{scope}[xscale = \sym]
%    % note that each connection is a shift of the registers by some constant amount 
%    % write all four explicitly (a = logic 0, b= logic 1), and needed rotation
%  \foreach \theta/\shiftaa /\shiftab /\shiftba /\shiftbb in {
%      0/4/4/4/0,
%      -45/4/4/4/0,
%      -90/4/4/5/1,
%      -135/4/3/4/7}{
%  \begin{scope}[rotate = \theta]
%    \foreach \zstart / \zend /\shift in {6/6/\shiftaa,6/1.5/\shiftab,1.5/6/\shiftba,1.5/1.5/\shiftbb}{
%    \foreach \w in {0,...,7}{
%      \draw[draw=black!70] let \n1={int(mod(\w + \shift,8)) /2 + \zend} in (90:{\w/2 + \zstart} ) -- (45: \n1 cm);
%    }}
%  \end{scope}}
%  \end{scope}}
%
%  % Now just write all of the penalty terms (S^3+S^4+S^5)
%  % can shift for logic 0 and 1, and rotate for different t
%  % Also draw the nodes
%  \foreach \theta in {0,45,...,315}{
%  \foreach \yshift in {1.5,6}{
%  \begin{scope}[rotate = \theta,yshift=\yshift cm]
%  \begin{scope}[draw=black!40]
%    \draw (0,3.5) to[out=210,in = 150] (0,1);
%    \draw (0,3.5) to[out=225,in = 135] (0,1.5);
%    \draw (0,3.5) to[out=240,in=120] (0,2);
%  
%    \draw (0,3) to[out=330,in = 30] (0,.5);
%    \draw (0,3) to[out=315,in=45] (0,1);
%    \draw (0,3) to[out=300,in=60] (0,1.5);
%  
%    \draw (0,0) to[out=30,in=330] (0,2.5);
%    \draw (0,0) to[out=45,in=315] (0,2);
%    \draw (0,0) to[out=60,in=300] (0,1.5);
%  
%    \draw (0,.5) to[out=135,in=225] (0,2.5);
%    \draw (0,.5) to[out=120,in=240] (0,2);
%  
%    \draw (0,2.5) to[out=240,in=120] (0,1);
%  \end{scope}
%   \foreach \y in {0,.5,...,3.5}{
%    \draw[fill = black,draw=black] (0,\y cm) circle (.33mm);
%  }
%  
%  \end{scope}}}
%  
%  % Label the times and unitaries applied (I couldn't get a counter to work here for some reason)
%  \foreach \t/\gate in {1/H,2/H,3/HT,4/{(HT)^\dag},5/HT,6/{(HT)^\dag},7/H,8/H}{
%    \node at ({135 - \t * 45} :10.25) {$t = \t$};
%    \draw[->,draw=black] ({120-\t*45} :10.25) arc[radius=10.25, start angle = {120-\t*45} ,end angle= {105-\t*45}];
%    \node[fill=white] at ({112.5 - \t * 45} :10.25) {$\gate$ };
%  }
%\end{tikzpicture}
%
%\caption{The graph $g_{0}$.\label{fig:g_0}}
%\end{figure}

Starting with the circuit $\mathcal{C}_{0}$, we apply the Feynman-Kitaev circuit-to-Hamiltonian mapping \cite{Fey85,KSV02} (up to a constant term and overall multiplicative factor) to get the Hamiltonian
\begin{equation}
-\sqrt{2}\sum_{t=1}^{8}\left(U_{t}^{\dagger}\otimes|t\rangle\langle t+1|+U_{t}\otimes|t+1\rangle\langle t|\right).\label{eq:single_qubit_ham}
\end{equation}
This Hamiltonian acts on the Hilbert space $\CC^{2}\otimes\CC^{8}$, where the second register (the ``clock register'') has periodic boundary conditions (i.e., we let $|8+1\rangle=|1\rangle$). The ground space of \eq{single_qubit_ham} is spanned by so-called history states
\begin{align*}
  |\phi_{z}\rangle & 
  =\frac{1}{\sqrt{8}} \big(|z\rangle(|1\rangle+|3\rangle+|5\rangle+|7\rangle)
    +H|z\rangle(|2\rangle+|8\rangle)
    +HT|z\rangle(|4\rangle+|6\rangle)\big),
  \quad z\in\{0,1\},
\end{align*}
that encode the history of the computation where the circuit $\mathcal{C}_{0}$ is applied to $|z\rangle$. One can easily check that $|\phi_{z}\rangle$ is an eigenstate of the Hamiltonian with eigenvalue $-2\sqrt{2}$. 

Now we modify \eq{single_qubit_ham} to give a symmetric $0$-$1$ matrix. The trick we use is a variant of one used in references \cite{JW06,JGL10} for similar purposes. 

The nonzero standard basis matrix elements of \eq{single_qubit_ham} are integer powers of $\omega=e^{i\frac{\pi}{4}}$. Note that $\omega$ is an eigenvalue of the $8\times8$ shift operator 
\[
S=\sum_{j=0}^{7}|j+1\bmod 8\rangle\langle j|
\]
with eigenvector 
\[
|\omega\rangle=\frac{1}{\sqrt{8}}\sum_{j=0}^{7}\omega^{-j}|j\rangle.
\]
For each operator $-\sqrt{2}H,-\sqrt{2}HT,$ or $-\sqrt{2}(HT)^{\dagger}$ appearing in equation \eq{single_qubit_ham}, define another operator acting on $\CC^{2}\otimes\CC^{8}$ by replacing nonzero matrix elements with powers of the operator $S$, namely $\omega^k \mapsto S^k$. Write $B(U)$ for the operator obtained by making this replacement in $U$, e.g.,
\begin{align*}
-\sqrt{2}HT & = \begin{pmatrix}
\omega^{4} & \omega^{5}\\
\omega^{4} & \omega
\end{pmatrix} 
\mapsto B(HT)
=\begin{pmatrix}
S^{4} & S^{5}\\
S^{4} & S
\end{pmatrix}.
\end{align*}
We adjoin an 8-level ancilla and we make this replacement in equation \eq{single_qubit_ham}. This gives
\begin{align}
H_{\text{prop}} & =\sum_{t=1}^{8}\left(B(U_{t})^{\dagger}_{13}\otimes|t\rangle\langle t+1|_2+B(U_{t})_{13}\otimes|t+1\rangle\langle t|_2\right),\label{eq:Hprop}
\end{align}
a symmetric $0$-1 matrix acting on $\CC^{2}\otimes\CC^{8}\otimes\CC^{8}$, where the second register is the clock register and the third register is the ancilla register on which the $S$ operators act (the subscripts indicate which registers are acted upon). It is an insignificant coincidence that the clock and ancilla registers have the same dimension. 

Note that $H_{\text{prop}}$ commutes with $S$ (acting on the $8$-level ancilla) and therefore is block diagonal with eight sectors. In the sector where $S$ has eigenvalue $\omega$, it is identical to the Hamiltonian we started with, equation \eq{single_qubit_ham}. There is also a sector (where $S$ has eigenvalue $\omega^*$) where the Hamiltonian is the entrywise complex conjugate of the one we started with. We add a term to $H_{\text{prop}}$ that assigns an energy penalty to states in any of the other six sectors, ensuring that none of these states lie in the ground space of the resulting operator.

Now we can define the graph $g_{0}$. Each vertex in $g_0$ corresponds to a standard basis vector in the Hilbert space $\CC^{2}\otimes\CC^{8}\otimes\CC^{8}$. We label the vertices $(z,t,j)$ with $z\in\{0,1\}$ describing the state of the computational qubit, $t\in[8]$ giving the state of the clock, and $j\in\{0,\ldots,7\}$ describing the state of the ancilla. The adjacency matrix is
\[
A(g_{0})=H_{\text{prop}}+H_{\text{penalty}}
\]
where the penalty term
\[
H_{\text{penalty}}=\II\otimes\II\otimes\left(S^{3}+S^{4}+S^{5}\right)
\]
acts nontrivially on the third register. The graph $g_{0}$ is shown in \fig{g_0}.

Now consider the ground space of $A(g_{0})$. Note that $H_{\text{prop}}$ and $H_{\text{penalty}}$ commute, so they can be simultaneously diagonalized. Furthermore, $H_{\text{penalty}}$ has smallest eigenvalue $-1-\sqrt{2}$ (with eigenspace spanned by $|\omega\rangle$ and $|\omega^*\rangle$) and first excited energy $-1$. The norm of $H_{\text{prop}}$ satisfies $\left\Vert H_{\text{prop}}\right\Vert \leq4$, which follows from the fact that $H_{\text{prop}}$ has four ones in each row and column (with the remaining entries all zero).

The smallest eigenvalue of $A(g_{0})$ lives in the sector where $H_{\text{penalty}}$
has eigenvalue $-1-\sqrt{2}$ and is equal to 
\begin{equation}
  -2\sqrt{2}+(-1-\sqrt{2})=-1-3\sqrt{2}=-5.24\ldots.\label{eq:e_1_definition}
\end{equation}
This is the constant $e_{1}$ from equation \eq{e1_defn}. To see this, note that in any other sector $H_{\text{penalty}}$ has eigenvalue at least $-1$ and every eigenvalue of $A(g_{0})$ is at least $-5$ (using the fact that $H_{\text{prop}}\geq-4$). An orthonormal basis for the ground space of $A(g_{0})$ is furnished by the states
\begin{align}
|\psi_{z,0}\rangle & =\frac{1}{\sqrt{8}}\big(|z\rangle(|1\rangle+|3\rangle+|5\rangle+|7\rangle)+H|z\rangle(|2\rangle+|8\rangle)+HT|z\rangle(|4\rangle+|6\rangle)\big)|\omega\rangle\label{eq:psi0m}\\
|\psi_{z,1}\rangle & =|\psi_{z,0}\rangle^{*}\label{eq:psi1m}
\end{align}
where $z\in \{0,1\}$.

Note that the amplitudes of $|\psi_{z,0}\rangle$ in the above basis contain the result of computing either the identity, Hadamard, or $HT$ gate acting on the ``input'' state $|z\rangle$.


\subsubsection{Gate graphs}

We use three different schematic representations of the graph $g_{0}$ (defined in \sec{Encoding-a-Computation}), as depicted in \fig{diagram_elements}. We call these Figures \emph{diagram elements}; they are also the simplest examples of \emph{gate diagrams}, which we define shortly.

%\begin{figure}
%\centering
%%%%%%%%%%%%%%%%%%%%%% 
%\subfloat[][]{ 
%%%%%%%%%% Hadamard %%%%%%%%%
%\label{fig:diagram_elementH}
%\begin{tikzpicture}
%  \draw[rounded corners=2mm,thick] (0,0) rectangle (3.24 cm, 2cm);
%  % Drawing the nodes 
% \foreach \y in {.33,.66,1.33,1.66}
%{ 
% \foreach \x/\color in {0/black,3.24/gray}
%{      
%\draw[fill=\color,draw=\color] (\x cm, \y cm) circle (.66mm);   
%}
%}   
%  % Labels   
%\foreach \z in {0,1}
%{    
%\node[right] at (0,1.66-\z) {\footnotesize $(\z,1)$}; 
%\node[right] at (0,1.33-\z) {\footnotesize $(\z,3)$};    
%% Output nodes change for each type 
%   \node[left] at (3.24,1.66-\z) {\footnotesize $(\z,2)$}; 
%   \node[left] at (3.24,1.33-\z) {\footnotesize $(\z,8)$};  
%}   
%  % Type of Graph
%  \node at (1.62,1) {\huge $H$};   
%\end{tikzpicture} 
%} 
%\qquad
%\subfloat[][]{
%%%%%%%%%% Hadamard * T %%%%%%%%%
%\label{fig:diagram_elementHT}
%\begin{tikzpicture}
%  \draw[rounded corners=2mm,thick] (0,0) rectangle (3.24 cm, 2cm);
%  % Drawing the nodes   
%\foreach \y in {.33,.66,1.33,1.66}
%{  
%\foreach \x/\color in {0/black,3.24/gray}
%{     
%\draw[fill=\color,draw=\color] (\x cm, \y cm) circle (.66mm);   
%}
%}     
%% Labels  
%\foreach \z in {0,1}
%{    
%\node[right] at (0,1.66-\z) {\footnotesize $(\z,1)$}; 
%   \node[right] at (0,1.33-\z) {\footnotesize $(\z,3)$};     
%% Output nodes change for each type  
%  \node[left] at (3.24,1.66-\z) {\footnotesize $(\z,4)$}; 
%   \node[left] at (3.24,1.33-\z) {\footnotesize $(\z,6)$}; 
% }    
% % Type of Graph  
%\node at (1.62,1) {\huge $HT$}; 
% \end{tikzpicture}
%} 
%\qquad 
%\subfloat[][]
%{
%%%%%%%%%% Identity %%%%%%%%%
%\label{fig:diagram_element1}
%\begin{tikzpicture}
%  \draw[rounded corners=2mm,thick] (0,0) rectangle (3.24 cm, 2cm);
%  % Drawing the nodes  
%\foreach \y in {.33,.66,1.33,1.66}
%{   
%\foreach \x/\color in {0/black,3.24/gray}
%{     
%\draw[fill=\color,draw=\color] (\x cm, \y cm) circle (.66mm);   
%}
%}    
% % Labels  
%\foreach \z in {0,1}
%{     
%\node[right] at (0,1.66-\z) {\footnotesize $(\z,1)$};   
% \node[right] at (0,1.33-\z) {\footnotesize $(\z,3)$};    
%% Output nodes change for each type   
% \node[left] at (3.24,1.66-\z) {\footnotesize $(\z,5)$};  
%  \node[left] at (3.24,1.33-\z) {\footnotesize $(\z,7)$};   }    
% % Type of Graph  
%\node at (1.62,1) {\huge $1$};  
%\end{tikzpicture}  
%}
%
%\caption{Diagram elements from which a gate diagram is constructed. Each diagram element is a schematic representation of the graph $g_{0}$ shown
%in \fig{g_0}. \label{fig:diagram_elements}}
%\end{figure}

The black and grey circles in a diagram element are called ``nodes.'' Each node has a label $(z,t)$. The only difference between the three diagram elements is the labeling of their nodes. In particular, the nodes in the diagram element $U\in\{\II,H,HT\}$ correspond to values of $t\in[8]$ where the first register in equation \eq{psi0m} is either $|z\rangle$ or $U|z\rangle$. For example, the nodes for the $H$ diagram element have labels with $t\in\{1,3\}$ (where $|\psi_{z,0}\rangle$ contains the ``input'' $|z\rangle$) or $t=\{2,8\}$ (where $|\psi_{z,0}\rangle$ contains the ``output'' $H|z\rangle$). We draw the input nodes in black and the output nodes in grey.

The rules for constructing gate diagrams are simple. A gate diagram consists of some number $R \in \{1,2,\ldots\}$ of diagram elements, with self-loops attached to a subset $\mathcal{S}$ of the nodes and edges connecting a set $\mathcal{E}$ of pairs of nodes. A node may have a single edge or a single self-loop attached to it, but never more than one edge or self-loop and never both an edge and a self-loop. Each node in a gate diagram has a label $(q,z,t)$ where $q \in [R]$ indicates the diagram element it belongs to. An example is shown in \fig{simple_gate_diagram}.
% \dg{Note that the diagram element names $q\in [R]$ are indicated in the Figure.} 
Sometimes it is convenient to draw the input nodes on the right-hand side of a diagram element; e.g., in \fig{W_gadget} the node closest to the top left corner is labeled $(q,z,t)=(3,0,2)$.

%\begin{figure}
%\centering \begin{tikzpicture}
%  % The connections 
% \foreach \y in {1.2,.55}
%{   
% \draw[thick] (2.43,\y) -- (3.43,\y);  
%}      
%\draw[thick,looseness = 200] (0,1.2) to [out = 180, in = 100] (-.01,1.2);
%  % The two rectangles 
% \foreach \offset/\unitary/\label in {0/HT/1,3.43/1/2}
%{ 
% \begin{scope}[xshift=\offset cm] 
% \draw[rounded corners=2mm,thick] (0,0) rectangle (2.43cm,1.5 cm);
% \foreach \x /\color in {0/black,2.43/gray}
%{     
%\foreach \y in {1.2,.95,.55,.3}
%{      
%\draw[fill=\color,draw=\color] (\x cm, \y cm) circle (.66mm);  
%  }
%}    
%\node at (1.22cm, .75cm) {\huge $\unitary$};   
% \node at (1.22cm, 1.85cm){\Large \label};  
%\end{scope}
%}    
%\end{tikzpicture}
%\caption{A gate diagram with two diagram elements labeled $q=1$ (left) and $q=2$ (right).
%\label{fig:simple_gate_diagram}}
%\end{figure}

To every gate diagram we associate a \emph{gate graph} $G$ with
vertex set 
\[
\left\{ (q,z,t,j)\colon q\in[R],\, z\in\{0,1\},\, t\in[8],\, j\in\{0,\ldots,7\}\right\} 
\]
and adjacency matrix 
\begin{align}
A(G) & =\II_{q}\otimes A(g_{0})+h_{\mathcal{S}}+h_{\mathcal{E}}\label{eq:adj_gate_graph}\\
h_{\mathcal{S}} & =\sum_{\mathcal{S}}|q,z,t\rangle\langle q,z,t|\otimes\II_{j}\label{eq:h_loops}\\
h_{\mathcal{E}} & =\sum_{\mathcal{E}}\left(|q,z,t\rangle+|q^{\prime},z^{\prime},t^{\prime}\rangle\right)\left(\langle q,z,t|+\langle q^{\prime},z^{\prime},t^{\prime}|\right)\otimes\II_{j}.\label{eq:h_edges}
\end{align}
The sums in equations \eq{h_loops} and \eq{h_edges} run over the set of nodes with self-loops $(q,z,t)\in\mathcal{S}$ and the set of pairs of nodes connected by edges $\{(q,z,t),(q^{\prime},z^{\prime},t^{\prime})\}\in\mathcal{E}$, respectively. We write $\II_{q}$ and $\II_{j}$ for the identity operator on the registers with variables $q$ and $j$, respectively. We see from the above expression that each self-loop in the gate diagram corresponds to $8$ self-loops in the graph $G$, and an edge in the gate diagram corresponds to $8$ edges and $16$ self-loops in $G$.

Since a node in a gate graph never has more than one edge or self-loop attached to it, equations \eq{h_loops} and \eq{h_edges} are sums of orthogonal Hermitian operators.  Therefore 
\begin{align}
\left\Vert h_{\mathcal{S}}\right\Vert  & =\max_{\mathcal{S}}\left\Vert |q,z,t\rangle\langle q,z,t|\otimes\II_{j}\right\Vert =1\quad\text{if }\mathcal{S}\neq\emptyset\label{eq:h_S_bound}\\
\left\Vert h_{\mathcal{E}}\right\Vert  & =\max_{\mathcal{E}}\left\Vert \left(|q,z,t\rangle+|q^{\prime},z^{\prime},t^{\prime}\rangle\right)\left(\langle q,z,t|+\langle q^{\prime},z^{\prime},t^{\prime}|\right)\otimes\II_{j}\right\Vert =2\quad\text{if }\mathcal{E}\neq\emptyset\label{eq:h_E_bound}
\end{align}
for any gate graph. (Of course, this also shows that $\|{h_{\mathcal{S}^\prime}}\|=1$ and $\|{h_{\mathcal{E}^\prime}}\|=2$ for any nonempty subsets $\mathcal{S}^\prime\subseteq \mathcal{S}$ and $\mathcal{E}^\prime\subseteq \mathcal{E}$.)



\todo{Change this to the updated types with every vertex having a self-loop}

\subsubsection{Frustration-free states for a given interaction range}

Consider the adjacency matrix $A(G)$ of a gate graph $G$, and note (from equation \eq{adj_gate_graph} that its smallest eigenvalue $\mu(G)$ satisfies
\[
\mu(G)\geq e_{1}
\]
since $h_{\mathcal{S}}$ and $h_{\mathcal{E}}$ are positive semidefinite and $A(g_{0})$ has smallest eigenvalue $e_{1}$. In the special case where $\mu(G)=e_{1}$, we say $G$ is an $e_{1}$-gate graph.

\begin{definition}
An $e_{1}$-gate graph is a gate graph $G$ such that the smallest eigenvalue of its adjacency matrix is $e_{1}=-1-3\sqrt{2}$.
\end{definition}

When $G$ is an $e_{1}$-gate graph, a single-particle ground state $|\Gamma\rangle$ of $A(G)$ satisfies 
\begin{align}
\left(\II\otimes A(g_{0})\right)|\Gamma\rangle & =e_{1}|\Gamma\rangle\label{eq:Gamma_disc}\\
h_{\mathcal{S}}|\Gamma\rangle & =0\label{eq:h_s_0}\\
h_{\mathcal{E}}|\Gamma\rangle & =0.\label{eq:h_e_0}
\end{align}
Indeed, to show that a given gate graph $G$ is an $e_{1}$-gate graph, it suffices to find a state $|\Gamma\rangle$ satisfying these conditions. Note that equation \eq{Gamma_disc} implies that $|\Gamma\rangle$ can be written as a superposition of the states
\[
  |\psi_{z,a}^{q}\rangle=|q\rangle|\psi_{z,a}\rangle,\quad
  z,a\in\{0,1\},\, q\in[R]
\]
where $|\psi_{z,a}\rangle$ is given by equations \eq{psi0m} and \eq{psi1m}. The coefficients in the superposition are then constrained by equations \eq{h_s_0} and \eq{h_e_0}.
 
\begin{example}\label{ex:As-an-example}
As an example, we show the gate graph in \fig{simple_gate_diagram} is an $e_{1}$-gate graph. As noted above, equation \eq{Gamma_disc} lets us restrict our attention to the space spanned by the eight states $|\psi_{z,a}^{q}\rangle$ with $z,a\in \{0,1\}$ and $q\in \{1,2\}$. In this basis, the operators $h_{\mathcal{S}}$ and $h_{\mathcal{E}}$ only have nonzero matrix elements between states with the same value of $a\in\{0,1\}$. We therefore solve for the $e_{1}$ energy ground states with $a=0$ and those with $a=1$ separately. Consider a ground state of the form
\[
\left(\tau_{1}|\psi_{0,a}^{1}\rangle+\nu_{1}|\psi_{1,a}^{1}\rangle\right)+\left(\tau_{2}|\psi_{0,a}^{2}\rangle+\nu_{2}|\psi_{1,a}^{2}\rangle\right)
\]
and note that in this case \eq{h_s_0} implies $\tau_{1}=0$. Equation \eq{h_e_0} gives
\[
\begin{pmatrix}
\tau_{2}\\
\nu_{2}
\end{pmatrix}=\begin{cases}
HT\begin{pmatrix}
-\tau_{1}\\
-\nu_{1}
\end{pmatrix} & a=0\\
(HT)^{*}\begin{pmatrix}
-\tau_{1}\\
-\nu_{1}
\end{pmatrix} & a=1.
\end{cases}
\]
We find two orthogonal $e_{1}$-energy states, which are (up to normalization)
\begin{align}
|\psi_{1,0}^{1}\rangle-\frac{e^{i\frac{\pi}{4}}}{\sqrt{2}}\left(|\psi_{0,0}^{2}\rangle-|\psi_{1,0}^{2}\rangle\right)\label{eq:example_ffstate_1}\\
|\psi_{1,1}^{1}\rangle-\frac{e^{-i\frac{\pi}{4}}}{\sqrt{2}}\left(|\psi_{0,1}^{2}\rangle-|\psi_{1,1}^{2}\rangle\right) & .\label{eq:example_ffstate_2}
\end{align}
We interpret each of these states as encoding a qubit that is transformed at each set of input/output nodes in the gate diagram in \fig{simple_gate_diagram}. The encoded qubit begins on the input nodes of the first diagram element in the state 
\[
\begin{pmatrix}
\tau_{1}\\
\nu_{1}
\end{pmatrix}=\begin{pmatrix}
0\\
1
\end{pmatrix}
\]
because the self-loop penalizes the basis vectors $|\psi_{0,a}^{1}\rangle$. On the output nodes of diagram element $1$, the encoded qubit is in the state where either $HT$ (if $a=0$) or its complex conjugate (if $a=1$) has been applied. The edges in the gate diagram ensure that the encoded qubit on the input nodes of diagram element 2 is minus the state on the output nodes of diagram element $1$.
\end{example}

In this example, each single-particle ground state encodes a single-qubit computation. Later we show how $N$-particle frustration-free states on $e_{1}$-gate graphs can encode computations on $N$ qubits. Recall from \defn{FF_states} that a state $|\Gamma\rangle\in\mathcal{Z}_{N}(G)$ is said to be frustration free iff $H(G,N)|\Gamma\rangle=0.$ Note that $H(G,N)\geq0$, so an $N$-particle frustration-free state is necessarily a ground state. Putting this together with \lem{increase_part_number}, we see that the existence of an $N$-particle frustration-free state implies
\[
\lambda_{N}^{1}(G)=\lambda_{N-1}^{1}(G)=\ldots=\lambda_{1}^{1}(G)=0,
\]
i.e., there are $N^{\prime}$-particle frustration-free states for all $N^{\prime}\leq N$. 

We prove that the graph $g_{0}$ has no two-particle frustration-free states. By \lem{increase_part_number}, it follows that $g_0$ has no $N$-particle frustration-free states for $N\geq 2$.

\begin{lemma}
\label{lem:2particle}$\lambda_{2}^{1}(g_{0})>0$.
\end{lemma}

\begin{proof}
Suppose (for a contradiction) that $|Q\rangle\in\mathcal{Z}_{2}(g_{0})$ is a nonzero vector in the nullspace of $H(g_{0},2)$, so 
\[
H_{g_{0}}^{2}|Q\rangle=\bigg(A(g_{0})\otimes\II+\II\otimes A(g_{0})+2\sum_{v\in g_{0}}|v\rangle\langle v|\otimes|v\rangle\langle v|\bigg)|Q\rangle=2e_{1}|Q\rangle.
\]
This implies 
\[
A(g_{0})\otimes\II|Q\rangle=\II\otimes A(g_{0})|Q\rangle=e_{1}|Q\rangle
\]
since $A(g_0)$ has smallest eigenvalue $e_1$ and the interaction term is positive semidefinite. We can therefore write 
\[
|Q\rangle=\sum_{z,a,x,y\in\{0,1\}}Q_{za,xy}|\psi_{z,a}\rangle|\psi_{x,y}\rangle
\]
with $Q_{za,xy}=Q_{xy,za}$ (since $|Q\rangle\in\mathcal{Z}_{2}(g_{0})$) and 
\begin{equation}
\left(|v\rangle\langle v|\otimes|v\rangle\langle v|\right)|Q\rangle=0\label{eq:eqn_twoparticleannihilate}
\end{equation}
for all vertices $v=(z,t,j)\in g_{0}.$ Using this equation with
$|v\rangle=|0,1,j\rangle$ gives 
\begin{align*}
&Q_{00,00}\langle0,1,j|\psi_{0,0}\rangle^{2}  +2Q_{01,00}\langle0,1,j|\psi_{0,1}\rangle\langle0,1,j|\psi_{0,0}\rangle+Q_{01,01}\langle0,1,j|\psi_{0,1}\rangle^{2}\\
 &\quad =\frac{1}{64}\left(Q_{00,00}i^{-j}+2Q_{01,00}+Q_{01,01}i^{j}\right)\\
 &\quad =0
\end{align*}
for each $j\in\{0,\ldots,7\}$. The only solution to this set of equations is $Q_{00,00}=Q_{01,00}=Q_{01,01}=0$. The same analysis, now using $|v\rangle=|1,1,j\rangle$, gives $Q_{10,10}=Q_{11,10}=Q_{11,11}=0$. Finally, using equation \eq{eqn_twoparticleannihilate} with $|v\rangle=|0,2,j\rangle$ gives
\begin{align*}
 & \frac{1}{64}\langle0|H|1\rangle\langle0|H|0\rangle\left(2Q_{10,00}i^{-j}+2Q_{10,01}+2Q_{11,00}+2Q_{11,01}i^{j}\right)\\
 &\quad =\frac{1}{64}\left(Q_{10,00}i^{-j}+Q_{10,01}+Q_{11,00}+Q_{11,01}i^{j}\right)\\
 &\quad =0
\end{align*}
for all $j\in\{0,\ldots,7\}$, which implies that $Q_{10,00}=Q_{11,01}=0$ and $Q_{11,00}=-Q_{10,01}$. Thus, up to normalization, 
\[
|Q\rangle=|\psi_{1,0}\rangle|\psi_{0,1}\rangle+|\psi_{0,1}\rangle|\psi_{1,0}\rangle-|\psi_{11}\rangle|\psi_{00}\rangle-|\psi_{00}\rangle|\psi_{11}\rangle.
\]
Now applying equation \eq{eqn_twoparticleannihilate} with $|v\rangle=|0,4,j\rangle$, we see that the quantity 
\begin{align*}
\frac{1}{64}\left(2\langle0|HT|1\rangle\langle0|(HT)^{*}|0\rangle-2\langle0|(HT)^{*}|1\rangle\langle0|HT|0\rangle\right) & =\frac{1}{64}\left(e^{i\frac{\pi}{4}}-e^{-i\frac{\pi}{4}}\right)
\end{align*}
must be zero, which is a contradiction. Hence we conclude that the nullspace of $H(g_{0},2)$ is empty.
\end{proof}
We now characterize the space of $N$-particle frustration-free states on an $e_{1}$-gate graph $G$. Define the subspace $\mathcal{I}(G,N)\subset\mathcal{Z}_{N}(G)$ where each particle is in a ground state of $A(g_{0})$ and no two particles are located within the same diagram element: 
\begin{equation}
  \mathcal{I}(G,N)=\spn\{
  \Sym(|\psi_{z_{1},a_{1}}^{q_{1}}\rangle
  % |\psi_{z_{2},a_{2}}^{q_{2}}\rangle
  \ldots|\psi_{z_{N},a_{N}}^{q_{N}}\rangle)\colon 
  z_{i},a_{i}\in\{0,1\},\; q_{i}\in[R],\; 
  q_{i}\neq q_{j}\;\text{whenever}\; i\neq j\}.\label{eq:Ign}
\end{equation}

\begin{lemma}\label{lem:FF_characterization}
Let $G$ be an $e_{1}$-gate graph. A state $|\Gamma\rangle\in\mathcal{Z}_{N}(G)$ is frustration free if and only if 
\begin{align}
\left(A(G)-e_{1}\right)^{(w)}|\Gamma\rangle & =0\;\text{ for all }w\in[N]\label{eq:ff_condition1}\\
|\Gamma\rangle & \in\mathcal{I}(G,N).\label{eq:ff_condition2}
\end{align}
\end{lemma}

\begin{proof}
First suppose that equations \eq{ff_condition1} and \eq{ff_condition2} hold. From \eq{ff_condition2} we see that $|\Gamma\rangle$ has no support on states where two or more particles are located at the same vertex. Hence 
\begin{equation}
\sum_{k\in V}\hat{n}_{k}\left(\hat{n}_{k}-1\right)|\Gamma\rangle=0.\label{eq:ff_condition3}
\end{equation}
Putting together equations \eq{ff_condition1} and \eq{ff_condition3}, we get 
\[
H(G,N)|\Gamma\rangle=\left(H_{G}^{N}-Ne_{1}\right)|\Gamma\rangle=0,
\]
so $|\Gamma\rangle$ is frustration free.

To complete the proof, we show that if $|\Gamma\rangle$ is frustration free, then conditions \eq{ff_condition1} and \eq{ff_condition2} hold. By definition, a frustration-free state $|\Gamma\rangle$ satisfies 
\begin{equation}
H(G,N)|\Gamma\rangle=\left(\sum_{w=1}^{N}\left(A(G)-e_{1}\right)^{(w)}+\sum_{k\in V}\hat{n}_{k}\left(\hat{n}_{k}-1\right)\right)|\Gamma\rangle=0.\label{eq:defn_frustr}
\end{equation}
Since both terms in the large parentheses are positive semidefinite, they must both annihilate $|\Gamma\rangle$ (similarly, each term in the first summation must be zero). Hence equation \eq{ff_condition1} holds. Let $G_{\mathrm{rem}}$ be the graph obtained from $G$ by removing all of the edges and self-loops in the gate diagram of $G$. In other words,
\[
A(G_{\mathrm{rem}})=\sum_{q=1}^{R}|q\rangle\langle q|\otimes A(g_{0})=\II\otimes A(g_{0}).
\]
Noting that 
\[
H(G,N)\geq H(G_{\mathrm{rem}},N)\geq0,
\]
we see that equation \eq{defn_frustr} also implies 
\begin{equation}
H(G_{\mathrm{rem}},N)|\Gamma\rangle=0.\label{eq:Grem_gamma}
\end{equation}
Since each of the $R$ components of $G_{\mathrm{rem}}$ is an identical copy of $g_{0}$, the eigenvalues and eigenvectors of $H(G_{\mathrm{rem}},N)$ are characterized by \lem{BH_disconnected_graphs} (along with knowledge of the eigenvalues and eigenvectors of $g_{0}$). By \lem{2particle} and \lem{increase_part_number}, no component has a two- (or more) particle frustration-free state. Combining these two facts, we see that in an $N$-particle frustration-free state, every component of $G_{\mathrm{rem}}$ must contain either $0$ or $1$ particles, and the nullspace of $H(G_{\mathrm{rem}},N)$ is the space $\mathcal{I}(G,N).$ From equation \eq{Grem_gamma} we get $|\Gamma\rangle\in\mathcal{I}(G,N)$.
\end{proof}

Note that if $\mathcal{I}(G,N)$ is empty then \lem{FF_characterization} says that $G$ has no $N$-particle frustration-free states. For example, this holds for any $e_{1}$-gate graph $G$ whose gate diagram has $R<N$ diagram elements.

A useful consequence of \lem{FF_characterization} is the fact that every $k$-particle reduced density matrix of an $N$-particle frustration-free state $|\Gamma\rangle$ on an $e_{1}$-gate graph $G$ (with $k\leq N$) has all of its support on $k$-particle frustration-free states. To see this, note that for any partition of the $N$ registers into subsets $A$ (of size $k$) and $B$ (of size $N-k$), we have
\[
\mathcal{I}(G,N)\subseteq\mathcal{I}(G,k)_{A}\otimes\mathcal{Z}_{N-k}(G)_{B}.
\]
Thus, if condition \eq{ff_condition2} holds, then all $k$-particle reduced density matrices of $|\Gamma\rangle$ are contained in $\mathcal{I}(G,k)$. Furthermore, \eq{ff_condition1} is a statement about the single-particle reduced density matrices, so it also holds for each $k$-particle reduced density matrix. From this we see that each reduced density matrix of $|\Gamma\rangle$ is frustration free.


\subsection{Gadgets}

In \ex{As-an-example} we saw how a single-particle ground state can encode a single-qubit computation. In this Section we see how a two-particle frustration-free state on a suitably designed $e_{1}$-gate graph can encode a two-qubit computation. We design specific $e_{1}$-gate graphs (called \emph{gadgets}) that we use in \sec{From-circuits-to} to prove that Bose-Hubbard Hamiltonian is QMA-hard. For each gate graph we discuss, we show that the smallest eigenvalue of its adjacency matrix is $e_{1}$ and we solve for all of the frustration-free states.

We first design a gate graph where, in any two-particle frustration-free state, the locations of the particles are synchronized. This ``move-together'' gadget is presented in \sec{move-together}. In \sec{Gadgets-for-two-qubit}, we design gadgets for two-qubit gates using four move-together gadgets, one for each two-qubit computational basis state. Finally, in \sec{Other-gate-graph} we describe a small modification of a two-qubit gate gadget called the ``boundary gadget.''

The circuit-to-gate graph mapping described in \sec{From-circuits-to} uses a two-qubit gate gadget for each gate in the circuit, together with boundary gadgets in parts of the graph corresponding to the beginning and end of the computation.
%
%\subsubsection{The move-together gadget}
%
%\begin{figure}
%\centering \begin{tikzpicture}[scale=.7]
%  % Each connection
%\foreach \startpoint/\endpoint in {{(-1.618,1.16)}/{(0,.33)},{(-1.618,4.16)}/{(0,1.33)},{(-1.618,2.16)}/{(0,4.33)}, {(-1.618,5.16)}/{(0,5.33)},{(3.24,.66)}/{(4.85,1.16)},{(3.24,1.66)}/{(4.85,4.16)},{(3.24,4.66)}/{(4.85,5.16)},{(3.24,5.66)}/{(4.85,2.16)}}
%{   
%\node (a) at \startpoint {}; 
% \node (b) at \endpoint {};
%  \draw[looseness=.66,line width=4pt,color=white] (a) to [out=0,in=180] (b);
%  \draw[looseness=.66] (a) to [out=0,in=180] (b); 
%}
%  % Each Rectangle 
%\foreach \xshift / \yshift /\xscale /\lab in  {0/0/1/2,  0/4/1/1,  4.85/.5/1/6,  4.85/3.5/1/5,  -1.618/.5/-1/4,  -1.618/3.5/-1/3}{   \begin{scope}[shift={(\xshift,\yshift)},xscale=\xscale]   
% \draw[rounded corners=2mm,thick] (0,0) rectangle (3.24cm, 2cm);
% \node at (1.62,2.4) {\Large \lab};   
% \foreach \y in {.66,.33,1.33,1.66}
%{
%    \foreach \x /\color in {0/black,3.24 /gray}
%{       
%		\draw[fill=\color,draw=\color] (\x cm, \y cm) circle (.66mm);   
%}
%}
%  \node at (1.618,1) {\huge$H$}; \end{scope}
%}  
%% the output nodes 
%\foreach \point/\name in {{(0,.66)}/\gamma,{(0,1.66)}/\delta,{(0,4.66)}/\beta,{(0,5.66)}/\alpha}
%{ 
%\begin{scope}[shift=\point] 
% \node at (3mm,0mm){$\name$}; 
%\end{scope} 
%}
%\begin{scope}[shift={(10,3)},scale=1.5]
%  \node at (-.65 ,0) {\LARGE $=$};
%  \draw[rounded corners=2mm,thick] (0,-.46) rectangle (1.5cm,.46cm); 
%    \node at (.75,0) {\LARGE $W$};
%  \foreach \x/\y  in {0,1.5}
%{  
%\foreach \y in {-.2,.2}
%{    
%\draw[fill=black] (\x,\y) circle (.45mm);  
%}
%}      
%\node at (.2,.2) {$\alpha$};  
%\node at (.2,-.15) {$\gamma$}; 
% \node at (1.3,.2) {$\beta$}; 
% \node at (1.3,-.15) {$\delta$}; 
%  \end{scope} 
%\end{tikzpicture}
%
%\caption{The gate diagram for the move-together gadget. \label{fig:W_gadget}}
%\end{figure}

The gate diagram for the \emph{move-together gadget} is shown in \fig{W_gadget}. Using equation \eq{adj_gate_graph}, we write the adjacency matrix of the corresponding gate graph $G_W$ as 
\begin{equation}
A(G_W)=\sum_{q=1}^{6}|q\rangle\langle q|\otimes A(g_{0})+h_{\mathcal{E}}\label{eq:move_together_adj}
\end{equation}
where $h_{\mathcal{E}}$ is given by \eq{h_edges} and $\mathcal{E}$ is the set of edges in the gate diagram (in this case $h_{\mathcal{S}}=0$ as there are no self-loops).

We begin by solving for the single-particle ground states, i.e., the eigenvectors of \eq{move_together_adj} with eigenvalue $e_{1}=-1-3\sqrt{2}$. As in \ex{As-an-example}, we can solve for the states with $a=0$ and $a=1$ separately, since
\[
\langle\psi_{x,1}^{j}|h_{\mathcal{E}}|\psi_{z,0}^{i}\rangle=0
\]
for all $i,j\in\{1,\ldots,6\}$ and $x,z\in\{0,1\}$. We write a single-particle ground state as
\[
\sum_{i=1}^{6}\left(\tau_{i}|\psi_{0,a}^{i}\rangle+\nu_{i}|\psi_{1,a}^{i}\rangle\right)
\]
and solve for the coefficients $\tau_{i}$ and $\nu_{i}$ using equation \eq{h_e_0} (in this case equation \eq{h_s_0} is automatically satisfied since $h_{\mathcal{S}}=0$). Enforcing \eq{h_e_0} gives eight equations, one for each edge in the gate diagram:
\begin{align*}
  \tau_{3}&=-\tau_{1} & 
  \frac{1}{\sqrt{2}}(\tau_{1}+\nu_{1})&=-\tau_{6} \\
  \tau_{4}&=-\nu_{1} & 
  \frac{1}{\sqrt{2}}(\tau_{1}-\nu_{1})&=-\tau_{5}\\
  \nu_{3}&=-\tau_{2} & 
  \frac{1}{\sqrt{2}}(\tau_{2}+\nu_{2})&=-\nu_{5}\\
  \nu_{4}&=-\nu_{2} & 
  \frac{1}{\sqrt{2}}(\tau_{2}-\nu_{2})&=-\nu_{6}.
\end{align*}
There are four linearly independent solutions to this set of equations, given by 
\begin{align*}
  \text{\emph{Solution 1:}} && 
    \tau_{1} &= 1 & \tau_{3} &=-1 & 
    \tau_{5} &= -\frac{1}{\sqrt{2}} & \tau_{6} &= -\frac{1}{\sqrt{2}} &&
    \text{all other coefficients }0 \\
  \text{\emph{Solution 2:}} && 
    \nu_{1} &= 1 & \tau_{4} &= -1 &
    \tau_{5} &= \frac{1}{\sqrt{2}} & \tau_{6} &= -\frac{1}{\sqrt{2}} &&
    \text{all other coefficients }0 \\
  \text{\emph{Solution 3:}} && 
    \nu_{2} &= 1 & \nu_{4} &= -1 &
    \nu_{5} &= -\frac{1}{\sqrt{2}} & \nu_{6} &= \frac{1}{\sqrt{2}} &&
    \text{all other coefficients }0 \\
  \text{\emph{Solution 4:}} && 
    \tau_{2} &= 1 & \nu_{3} &= -1 &
    \nu_{5} &= -\frac{1}{\sqrt{2}} & \nu_{6} &= -\frac{1}{\sqrt{2}} &&
    \text{all other coefficients }0.
\end{align*}
For each of these solutions, and for each $a\in\{0,1\}$, we find a single-particle state with energy $e_1$. This result is summarized in the following Lemma.

\begin{lemma}
$G_{W}$ is an $e_{1}$-gate graph. A basis for the eigenspace
of $A(G_{W})$ with eigenvalue $e_1$ is 
\begin{align}
|\chi_{1,a}\rangle & =\frac{1}{\sqrt{3}}|\psi_{0,a}^{1}\rangle-\frac{1}{\sqrt{3}}|\psi_{0,a}^{3}\rangle-\frac{1}{\sqrt{6}}|\psi_{0,a}^{5}\rangle-\frac{1}{\sqrt{6}}|\psi_{0,a}^{6}\rangle\label{eq:chi_alpha}\\
|\chi_{2,a}\rangle & =\frac{1}{\sqrt{3}}|\psi_{1,a}^{1}\rangle-\frac{1}{\sqrt{3}}|\psi_{0,a}^{4}\rangle+\frac{1}{\sqrt{6}}|\psi_{0,a}^{5}\rangle-\frac{1}{\sqrt{6}}|\psi_{0,a}^{6}\rangle\label{eq:chi_beta}\\
|\chi_{3,a}\rangle & =\frac{1}{\sqrt{3}}|\psi_{1,a}^{2}\rangle-\frac{1}{\sqrt{3}}|\psi_{1,a}^{4}\rangle-\frac{1}{\sqrt{6}}|\psi_{1,a}^{5}\rangle+\frac{1}{\sqrt{6}}|\psi_{1,a}^{6}\rangle\label{eq:chi_gamma}\\
|\chi_{4,a}\rangle & =\frac{1}{\sqrt{3}}|\psi_{0,a}^{2}\rangle-\frac{1}{\sqrt{3}}|\psi_{1,a}^{3}\rangle-\frac{1}{\sqrt{6}}|\psi_{1,a}^{5}\rangle-\frac{1}{\sqrt{6}}|\psi_{1,a}^{6}\rangle\label{eq:chi_delta}
\end{align}
where $a\in\{0,1\}$. 
\end{lemma}

In \fig{W_gadget} we have used a shorthand $\alpha,\beta,\gamma,\delta$ to identify four nodes of the move-together gadget; these are the nodes with labels $(q,z,t)=(1,0,1),(1,1,1),(2,1,1),(2,0,1)$, respectively. We view $\alpha$ and $\gamma$ as ``input'' nodes and $\beta$ and $\delta$ as ``output'' nodes for this gate diagram. It is natural to associate each single-particle state $|\chi_{i,a}\rangle$ with one of these four nodes.  We also associate the set of 8 vertices represented by the node with the corresponding node, e.g.,
\[
  S_{\alpha}=\left\{ (1,0,1,j)\colon j\in\{0,\ldots,7\}\right\} .
\]
Looking at equation \eq{chi_alpha} (and perhaps referring back to equation \eq{psi0m}) we see that $|\chi_{1,a}\rangle$ has support on vertices in $S_{\alpha}$ but not on vertices in $S_{\beta}$, $S_{\gamma}$, or $S_{\delta}$. Looking at the picture on the right-hand side of the equality sign in \fig{W_gadget}, we think of $|\chi_{1,a}\rangle$ as localized at the node $\alpha$, with no support on the other three nodes. The states $|\chi_{2,a}\rangle,|\chi_{3,a}\rangle,|\chi_{4,a}\rangle$ are similarly localized at nodes $\beta,\gamma,\delta$. We view $|\chi_{1,a}\rangle$ and $|\chi_{3,a}\rangle$ as input states and $|\chi_{2,a}\rangle$ and $|\chi_{4,a}\rangle$ as output states.

Now we turn our attention to the two-particle frustration-free states of the move-together gadget, i.e., the states $|\Phi\rangle\in\mathcal{Z}_{2}(G_{W})$ in the nullspace of $H(G_W,2)$. Using \lem{FF_characterization} we can write
\begin{equation}
|\Phi\rangle=\sum_{a,b \in \{0,1\},\,I,J \in [4]}C_{(I,a),(J,b)}|\chi_{I,a}\rangle|\chi_{J,b}\rangle\label{eq:chi_superposition}
\end{equation}
where the coefficients are symmetric, i.e.,
\begin{equation}
C_{(I,a),(J,b)}=C_{(J,b),(I,a)},\label{eq:symmetric_coefs}
\end{equation}
and where 
\begin{equation}
\langle\psi_{z,a}^{q}|\langle\psi_{x,b}^{q}|\Phi\rangle=0\label{eq:frustration_free}
\end{equation}
for all $z,a,x,b\in\{0,1\}$ and $q\in[6].$

The move-together gadget is designed so that each solution $|\Phi\rangle$ to these equations  is a superposition of a term where both particles are in input states and a term where both particles are in output states. The particles move from input nodes to output nodes together. We now solve equations \eq{chi_superposition}--\eq{frustration_free} and prove the following.

\begin{lemma}
\label{lem:Wgadget_lemma}
A basis for the nullspace of $H(G_{W},2)$ is 
\begin{equation}
|\Phi_{a,b}\rangle=\Sym\left(\frac{1}{\sqrt{2}}|\chi_{1,a}\rangle|\chi_{3,b}\rangle+\frac{1}{\sqrt{2}}|\chi_{2,a}\rangle|\chi_{4,b}\rangle\right),\quad a,b\in\{0,1\}.\label{eq:phi_a1_a2}
\end{equation}
There are no $N$-particle frustration-free states on $G_{W}$ for $N\geq3$, i.e.,
\[
\lambda_{N}^{1}(G_{W})>0\quad\text{for }N\geq3.
\]
\end{lemma}

\begin{proof}
The states $|\Phi_{a,b}\rangle$ manifestly satisfy equations \eq{chi_superposition} and \eq{symmetric_coefs}, and one can directly verify that they also satisfy \eq{frustration_free} (the nontrivial cases to check are $q=5$ and $q=6$). 

To complete the proof that \eq{phi_a1_a2} is a basis for the nullspace of $H(G_W,2)$, we verify that any state satisfying these conditions must be a linear combination of these four states. Applying equation \eq{frustration_free} gives
\begin{align*}
\langle\psi_{0,a}^{1}|\langle\psi_{0,b}^{1}|\Phi\rangle & =\frac{1}{3} C_{(1,a),(1,b)}=0 &
\langle\psi_{1,a}^{1}|\langle\psi_{1,b}^{1}|\Phi\rangle & =\frac{1}{3} C_{(2,a),(2,b)}=0\\
\langle\psi_{1,a}^{2}|\langle\psi_{1,b}^{2}|\Phi\rangle & =\frac{1}{3} C_{(3,a),(3,b)}=0 &
\langle\psi_{0,a}^{2}|\langle\psi_{0,b}^{2}|\Phi\rangle & =\frac{1}{3} C_{(4,a),(4,b)}=0\\
\langle\psi_{0,a}^{1}|\langle\psi_{1,b}^{1}|\Phi\rangle & =\frac{1}{3} C_{(1,a),(2,b)}=0 &
\langle\psi_{0,a}^{2}|\langle\psi_{1,b}^{2}|\Phi\rangle & =\frac{1}{3} C_{(4,a),(3,b)}=0\\
\langle\psi_{0,a}^{3}|\langle\psi_{1,b}^{3}|\Phi\rangle & =\frac{1}{3} C_{(1,a),(4,b)}=0 &
\langle\psi_{0,a}^{4}|\langle\psi_{1,b}^{4}|\Phi\rangle & =\frac{1}{3} C_{(2,a),(3,b)}=0
\end{align*}
for all $a,b\in \{0,1\}$. Using the fact that all of these coefficients are zero, and using equation \eq{symmetric_coefs}, we get 
\[
|\Phi\rangle=\sum_{a,b\in\{0,1\}}\left(C_{(1,a),(3,b)}\left(|\chi_{1,a}\rangle|\chi_{3,b}\rangle+|\chi_{3,b}\rangle|\chi_{1,a}\rangle\right)+C_{(2,a),(4,b)}\left(|\chi_{2,a}\rangle|\chi_{4,b}\rangle+|\chi_{4,b}\rangle|\chi_{2,a}\rangle\right)\right).
\]
Finally, applying equation \eq{frustration_free} again gives
\[
\langle\psi_{0,a}^{6}|\langle\psi_{1,b}^{6}|\Phi\rangle=\frac{1}{6}C_{(2,a),(4,b)}-\frac{1}{6}C_{(1,a),(3,b)}=0.
\]
Hence
\[
|\Phi\rangle=\sum_{a,b\in\{0,1\}}C_{(1,a),(3,b)}\left(|\chi_{1,a}\rangle|\chi_{3,b}\rangle+|\chi_{3,b}\rangle|\chi_{1,a}\rangle+|\chi_{2,a}\rangle|\chi_{4,b}\rangle+|\chi_{4,b}\rangle|\chi_{2,a}\rangle\right),
\]
which is a superposition of the states $|\Phi_{a,b}\rangle.$ 

Finally, we prove that there are no frustration-free ground states of the Bose-Hubbard model on $G_{W}$ with more than two particles. By \lem{increase_part_number},
% To establish this 
it suffices to prove that there are no frustration-free three-particle states.
% , since by \lem{increase_part_number}, $\lambda_{3}^{1}(G_{W})>0$ implies $\lambda_{N}^{1}(G_{W})>0$ for $N\geq 3$.

Suppose (for a contradiction) that $|\Gamma\rangle\in\mathcal{Z}_{3}(G_{W})$ is a normalized three-particle frustration-free state. Write 
\[
|\Gamma\rangle=\sum D_{(i,a),(j,b),(k,c)}|\chi_{i,a}\rangle|\chi_{j,b}\rangle|\chi_{k,c}\rangle.
\]
Note that each reduced density matrix of $|\Gamma\rangle$ on two of the three subsystems must have all of its support on two-particle frustration-free states (see the remark following \lem{FF_characterization}), i.e., on the states $|\Phi_{a,b}\rangle$. Using this fact for the subsystem consisting of the first two particles, we see in particular that
\begin{equation}
(i,j)\notin\{(1,3),(3,1),(2,4),(4,2)\}\quad\Longrightarrow\quad D_{(i,a),(j,b),(k,c)}=0\label{eq:ij_constraint1}
\end{equation}
(since $|\Phi_{a_1,a_2}\rangle$ only has support on vectors $|\chi_{i,a}\rangle|\chi_{j,b}\rangle$ with $i,j\in \{(1,3),(3,1),(2,4),(4,2)\}$).

Using this fact for subsystems consisting of particles $2,3$ and $1,3$, respectively, gives 
\begin{align}
(j,k)\notin\{(1,3),(3,1),(2,4),(4,2)\}\quad\Longrightarrow\quad D_{(i,a),(j,b),(k,c)} & =0\label{eq:ij_constraint2}\\
(i,k)\notin\{(1,3),(3,1),(2,4),(4,2)\}\quad\Longrightarrow\quad D_{(i,a),(j,b),(k,c)} & =0.\label{eq:ij_constraint3}
\end{align}
Putting together equations \eq{ij_constraint1}, \eq{ij_constraint2}, and \eq{ij_constraint3}, we see that $|\Gamma\rangle=0$. This is a contradiction, so no three-particle frustration-free states exist.
\end{proof}

Next we show how this gadget can be used to build gadgets the implement two-qubit gates.

\subsubsection{Two-qubit gate gadget}


In this Section we define a gate graph for each of the two-qubit unitaries
\[
  \{\CNOT_{12}, \CNOT_{21}, \CNOT_{12}\left(H\otimes\II\right),
    \CNOT_{12}\left(HT\otimes\II\right)\}.
\]
Here $\CNOT_{12}$ is the standard controlled-not gate with the second qubit as a target, whereas $\CNOT_{21}$ has the first qubit as target.

%\begin{figure}
%\centering 
%\subfloat[][]{
%\begin{tikzpicture}[yscale=.929] \label{fig:GVucnot}
%
%\path[use as bounding box](-5.5,-3) rectangle (10,10.5);
%     % Each Hadamard Rectangle 
%\foreach \xshift / \yshift /\xscale / \lab / \unitary in{3.12/0.5/1/4/1,3.12/3.5/1/2/1,-3.62/0.5/1/3/1,-3.62/3.5/1/1/\tilde U, -3.62/-2.5/1/7/1,3.12/-2.5/1/8/1, 3.12/7/1/6/1, -3.62/7/1/5/1}
%{ 
%\begin{scope}[shift={(\xshift,\yshift)},xscale=\xscale]
%  \draw[rounded corners=0.75mm,thick] (0,0) rectangle (2 cm, 2cm);
%  \node at (1,1) {\huge$\unitary$};   
%\node at (1,2.4) {\large\lab};
%  \foreach \y in {.33,.66,1.33,1.66}
%	{   
%		\foreach \x /\color in {0/black,2/gray}
%			{    
%				\draw[fill=\color,draw=\color] (\x cm, \y cm) circle (.66mm);  
%			}
%	} 
%\end{scope}
%}
%  % Each W Gadget
%\foreach \yshift/\from in {0.25/11,1.75/10,3.25/01,4.75/00}
%{ 
%\begin{scope}[yshift=\yshift cm] 
% \draw[rounded corners=0.75mm,thick] (0,0) rectangle (1.5cm,.93cm); 
% \node at (.75,.465) {$W$};   
%\node at (.75,1.2) {$\from$};
%  \foreach \x/\color in {0/black,1.5/gray}
%	{   
%		\foreach \y in {0.33,.6}
%		{   
%			 \draw[fill=\color,draw=\color] (\x,\y) circle (.66mm); 
%		 }
%	}
%\end{scope}
%}
%  % Right Connections
%\foreach \l/\r in {   5.35/5.16,5.08/2.16,   .85/3.83,.57/1.83,   3.85/4.83,3.57/1.16,   2.35/4.16,2.08/.83} 
%{   
%\node (a) at (1.5,\l) {}; 
% \node (b) at (3.12,\r)   {}; 
% \draw[looseness=.66,line width=4pt,color=white] (a) to [out=0,in=180] (b);
%  \draw[looseness=.66] (a) to [out=0,in=180] (b); 
%}
%  % Left Connections
%\foreach \l/\r in {5.16/5.35,2.16/5.08,   3.83/.85,.83/.57,   4.83/3.85,1.83/2.08,   4.16/2.35,1.16/3.57}
%{   
%\node (a) at (-1.618,\l) {}; 
% \node (b) at (0,\r)   {};
%  \draw[looseness=.66,line width=4pt,color=white] (a) to [out=0,in=180] (b);
%  \draw[looseness=.66] (a) to [out=0,in=180] (b); 
%}
%
%%%%%%%%Edges which connect to the two lower diagram elements
%
%%The one on the left-hand side
%
%\node (c) at (-3.62,1.83){};
%\node (d) at (-3.62,-1.17){};
%
% \draw[looseness=.66,line width=4pt,color=white] (c) to [out=180,in=180] (d);
% \draw[looseness=.66] (c) to [out=180,in=180] (d); 
%
%\node (c) at (-3.62,0.83){};
%\node (d) at (-3.62,-2.17){};
%
% \draw[looseness=.66,line width=4pt,color=white] (c) to [out=180,in=180] (d);
% \draw[looseness=.66] (c) to [out=180,in=180] (d); 
%
%% The one on the right-hand side
%
%\node (c) at (5.12,1.83){};
%\node (d) at (5.12,-1.17){};
%
% \draw[looseness=.66,line width=4pt,color=white] (c) to [out=0,in=0] (d);
% \draw[looseness=.66] (c) to [out=0,in=0] (d); 
%
%\node (c) at (5.12,0.83){};
%\node (d) at (5.12,-2.17){};
%
% \draw[looseness=.66,line width=4pt,color=white] (c) to [out=0,in=0] (d);
% \draw[looseness=.66] (c) to [out=0,in=0] (d); 
%
%%%%%%%%Edges which connect to the two upper diagram elements
%
%%The ones on the left-hand side
%
%\node (c) at (-3.62,8.33){};
%\node (d) at (-3.62,4.83){};
%
% \draw[looseness=.66,line width=4pt,color=white] (c) to [out=180,in=180] (d);
% \draw[looseness=.66] (c) to [out=180,in=180] (d); 
%
%\node (c) at (-1.62,7.33){};
%\node (d) at (5.12,3.83){};
%
% \draw[looseness=1.5,line width=4pt,color=white] (c) to [out=0,in=10] (d);
% \draw[looseness=1.5] (c) to [out=0,in=10] (d); 
%
%%The ones on the right-hand side
%
%\node (c) at (5.12,8.33){};
%\node (d) at (5.12,4.83){};
%
% \draw[looseness=.66,line width=4pt,color=white] (c) to [out=0,in=0] (d);
% \draw[looseness=.66] (c) to [out=0,in=0] (d); 
%
%\node (c) at (3.12,7.33){};
%\node (d) at (-3.62,3.83){};
%
% \draw[looseness=1.5,line width=4pt,color=white] (c) to [out=180,in=170] (d);
% \draw[looseness=1.5] (c) to [out=180,in=170] (d); 
%
%
%
%
%  % Node Labels
%
% \node at (-3.9,5.2) {$\alpha$};
% \node at (-3.9,4.2) {$\beta$};
%\node at (-3.9,2.2) {$\gamma$};
%\node at (-3.9,1.2) {$\delta$};
%
% \node at (5.4,5.2) {$\epsilon$};
% \node at (5.4,4.2) {$\zeta$};
%\node at (5.4,2.2) {$\eta$};
%\node at (5.4,1.2) {$\theta$};
%
%
%\setcounter{mycount}{`a} \begin{scope}[xshift = 7.5cm,yshift = 3cm]
%  \node at (-1,0) {\huge $=$}; 
% \begin{scope}[xscale=-1,xshift=-2cm]   
%\draw[rounded corners = .75mm,thick] (0cm, -1.618 cm) -- (2 cm, -1.618 cm) -- (2cm, -.5 cm) -- (1.2cm, -.5 cm) -- (1.2 cm,.5cm) -- (2cm, .5cm) -- (2cm, 1.618cm) -- (0cm, 1.618cm) -- (0cm, .5cm) -- (.6cm, .5cm) -- (.6cm, -.5cm) -- (0cm, -.5cm) -- cycle; 
%    \draw (.15cm, 1.06cm) -- (1.85cm, 1.06cm); 
% \draw (.15cm, -1.06cm) -- (1.85cm, -1.06cm);
%  \draw (.9cm,1.06cm) -- (.9cm,-1.06cm);   
%\draw[fill=black] (.9,1.06cm) circle (.66mm); 
% \draw (.9,-1.06cm) circle (1.25mm); 
% \draw (.9,-1.19cm) -- (.9,-.93cm);   
%  \draw[rounded corners=.75mm,fill=white] (1.15cm, .7cm) rectangle (1.75cm, 1.42cm); 
% \node at (1.45cm, 1.06cm) {\small $\tilde U$}; 
% \end{scope}   
%
%
%\node at ( -0.2,1.31) {$\alpha$};
% \draw[fill=black] (0,1.31) circle (.66mm);   
%
%\node at ( -0.2,0.81) {$\beta$};
% \draw[fill=black] (0,0.81) circle (.66mm);   
%
%\node at ( -0.2,-0.81) {$\gamma$};
% \draw[fill=black] (0,-0.81) circle (.66mm);   
%
%\node at ( -0.2,-1.31) {$\delta$};
% \draw[fill=black] (0,-1.31) circle (.66mm);   
%
%\node at (2.2,1.31) {$\epsilon$};
% \draw[fill=black] (2,1.31) circle (.66mm);   
%
%\node at (2.2,0.81) {$\zeta$};
% \draw[fill=black] (2,0.81) circle (.66mm);   
%
%\node at ( 2.2,-0.81) {$\eta$};
% \draw[fill=black] (2,-0.81) circle (.66mm);   
%
%\node at ( 2.2,-1.31) {$\theta$};
% \draw[fill=black] (2,-1.31) circle (.66mm);   
%
%
%\end{scope} 
%\end{tikzpicture}}
%\vspace{0.63cm}
%\subfloat[][]{\begin{tikzpicture}[yscale=.92] \label{fig:GVcnot}
%
%  \setcounter{mycount}{`a} \begin{scope}[xshift = 7.2 cm,yshift = 3cm]   
%     
%\draw[rounded corners = 2mm,thick] (0cm, -1.618 cm) -- (2 cm, -1.618 cm) -- (2cm, -.5 cm) -- (1.2cm, -.5 cm) -- (1.2 cm,.5cm) -- (2cm, .5cm) -- (2cm, 1.618cm) -- (0cm, 1.618cm) -- (0cm, .5cm) -- (.6cm, .5cm) -- (.6cm, -.5cm) -- (0cm, -.5cm) -- cycle;      \draw (.15cm, 1.06cm) -- (1.85cm, 1.06cm);   \draw (.15cm, -1.06cm) -- (1.85cm, -1.06cm);   \draw (.9cm,1.06cm) -- (.9cm,-1.06cm);
% \draw[fill=black] (.9,-1.06cm) circle (.66mm);   \draw (.9,1.06cm) circle (1.25mm);   \draw (.9,1.19cm) -- (.9,.93cm); 
%
%
%\node at ( -0.2,-0.81) {$\alpha$};
% \draw[fill=black] (0,-0.81) circle (.66mm);   
%
%\node at ( -0.2,-1.31) {$\beta$};
% \draw[fill=black] (0,-1.31) circle (.66mm);   
%
%\node at ( -0.2,1.31) {$\gamma$};
% \draw[fill=black] (0,1.31) circle (.66mm);   
%
%\node at ( -0.2,0.81) {$\delta$};
% \draw[fill=black] (0,0.81) circle (.66mm);   
%
%\node at ( 2.2,-0.81) {$\epsilon$};
% \draw[fill=black] (2,-0.81) circle (.66mm);   
%
%\node at ( 2.2,-1.31) {$\zeta$};
% \draw[fill=black] (2,-1.31) circle (.66mm);   
%
%\node at (2.2,1.31) {$\eta$};
% \draw[fill=black] (2,1.31) circle (.66mm);   
%
%\node at (2.2,0.81) {$\theta$};
% \draw[fill=black] (2,0.81) circle (.66mm);   
%
%\end{scope}
%\end{tikzpicture}}
%
%\caption{\subfig{GVucnot} Gadget for the
% two-qubit unitary $U=(\tilde U\otimes\II)\CNOT_{12}$ with $\tilde U\in \{1,H,HT\}$.
%\subfig{GVcnot} For the $U=\CNOT_{21}$ gate (first qubit is the target), we use the same gate graph as in \subfig{GVucnot} with $\tilde U=1$; we represent it schematically as shown.}
%\end{figure}

We define the gate graphs by exhibiting their gate diagrams. For the three cases
\[
  U=\CNOT_{12}(\tilde U\otimes\II)
\]
with $\tilde U\in\{\II,H,HT\}$, we associate $U$ with the gate diagram shown in \fig{GVucnot}. In the Figure we also indicate a shorthand used to represent this gate diagram. As one might expect, for the case $U=\CNOT_{21}$, we use the same gate diagram as for $U=\CNOT_{12}$; however, we use the slightly different shorthand shown in \fig{GVcnot}.

Roughly speaking, the two-qubit gate gadgets work as follows. In \fig{GVucnot} there are four move-together gadgets, one for each two-qubit basis state $|00\rangle, |01\rangle, |10\rangle, |11\rangle$. These enforce the constraint that two particles must move through the graph together. The connections between the four diagram elements labeled $1,2,3,4$ and the move-together gadgets ensure that certain frustration-free two-particle states encode two-qubit computations, while the connections between diagram elements $1,2,3,4$ and $5,6,7,8$ ensure that there are no additional frustration-free two-particle states (i.e., states that do not encode computations).

To describe the frustration-free states of the gate graph depicted in \fig{GVucnot}, first recall the definition of the states $|\chi_{1,a}\rangle, |\chi_{2,a}\rangle, |\chi_{3,a}\rangle, |\chi_{4,a}\rangle$ from equations \eq{chi_alpha}--\eq{chi_delta}. For each of the move-together gadgets $xy\in\{00,01,10,11\}$ in \fig{GVucnot}, write 
\[
|\chi_{L,a}^{xy}\rangle
\]
for the state $|\chi_{L,a}\rangle$ with support (only) on the gadget labeled $xy$. Write 
\[
U(a)=\begin{cases}
U & \text{ if }a=0\\
U^{*} & \text{ if }a=1
\end{cases}
\]
and similarly for $\tilde U$ (we use this notation throughout the paper to indicate a unitary or its elementwise complex conjugate).

In \app{graph_gadgets}
we prove the following Lemma, which shows that $G_{U}$ is an $e_{1}$-gate
graph and solves for its frustration-free states.

\begin{restatable}{lemma}{Twoqub}\label{lem:2qub_gate}
Let $U=\CNOT_{12}(\tilde U\otimes\II)$ where $\tilde U\in\{\II,H,HT\}$. The corresponding gate graph $G_U$ is defined by its gate diagram shown in \fig{GVucnot}. The adjacency matrix $A(G_U)$ has ground energy $e_{1}$; a basis for the corresponding eigenspace is
\begin{align}
|\rho_{z,a}^{1,U}\rangle & =\frac{1}{\sqrt{8}}|\psi_{z,a}^{1}\rangle-\frac{1}{\sqrt{8}}|\psi_{z,a}^{5+z}\rangle-\sqrt{\frac{3}{8}}\sum_{x,y=0}^{1}\tilde U(a)_{yz}|\chi_{1,a}^{yx}\rangle
 & |\rho_{z,a}^{2,U}\rangle
 & =\frac{1}{\sqrt{8}}|\psi_{z,a}^{2}\rangle-\frac{1}{\sqrt{8}}|\psi_{z,a}^{6-z}\rangle-\sqrt{\frac{3}{8}}\sum_{x=0}^{1}|\chi_{2,a}^{zx}\rangle\label{eq:rho1_1}\\
|\rho_{z,a}^{3,U}\rangle & =\frac{1}{\sqrt{8}}|\psi_{z,a}^{3}\rangle-\frac{1}{\sqrt{8}}|\psi_{z,a}^{7}\rangle-\sqrt{\frac{3}{8}}\sum_{x=0}^{1}|\chi_{3,a}^{xz}\rangle 
& |\rho_{z,a}^{4,U}\rangle & =\frac{1}{\sqrt{8}}|\psi_{z,a}^{4}\rangle-\frac{1}{\sqrt{8}}|\psi_{z,a}^{8}\rangle-\sqrt{\frac{3}{8}}\sum_{x=0}^{1}|\chi_{4,a}^{x\left(z\oplus x\right)}\rangle\label{eq:rho2_1}
\end{align}
where $z,a\in\{0,1\}$. A basis for the nullspace of $H(G_U,2)$ is
\begin{equation}
\Sym(|T_{z_{1},a,z_{2},b}^U\rangle),\quad z_{1},z_{2},a,b\in\{0,1\}\label{eq:twopartstate_1}
\end{equation}
where 
\begin{equation}
|T_{z_{1},a,z_{2},b}^U\rangle=\frac{1}{\sqrt{2}}|\rho_{z_{1},a}^{1,U}\rangle|\rho_{z_{2},b}^{3,U}\rangle+\frac{1}{\sqrt{2}}\sum_{x_1,x_2=0}^{1}U(a)_{x_{1}x_{2},z_{1}z_{2}}|\rho_{x_{1},a}^{2,U}\rangle|\rho_{x_{2},b}^{4,U}\rangle\label{eq:twopartstate_2}
\end{equation}
for $z_{1},z_{2},a,b\in\{0,1\}$. There are no $N$-particle frustration-free
states on $G_U$ for $N\geq3$, i.e., 
\[
\lambda_{N}^{1}(G_U)>0\quad\text{for }N\geq3.
\]
\end{restatable}

We view the nodes labeled $\alpha,\beta,\gamma,\delta$ in \fig{GVucnot} as ``input'' nodes and those labeled $\epsilon, \zeta,\eta,\theta$ as ``output nodes''. Each of the states $|\rho_{x,y}^{i,U}\rangle$ is associated with one of the nodes, depending on the values of $i\in\{1,2,3,4\}$ and $x\in\{0,1\}$. For example, the states $|\rho_{0,0}^{1,U}\rangle$ and $|\rho_{0,1}^{1,U}\rangle$ are associated with input node $\alpha$ since they both have nonzero amplitude on vertices of the gate graph that are associated with $\alpha$ (and zero amplitude on vertices associated with other labeled nodes).


The two-particle state $\Sym(|T_{z_{1},a,z_{2},b}^{U}\rangle)$ is a superposition of a term
\[
\Sym\bigg(\frac{1}{\sqrt{2}}|\rho_{z_{1},a}^{1,U}\rangle|\rho_{z_{2},b}^{3,U}\rangle\bigg)
\]
with both particles located on vertices corresponding to input nodes and a term 
\[
\Sym\Bigg(\frac{1}{\sqrt{2}}\sum_{x_{1},x_{2}\in\{0,1\}}U(a)_{x_{1}x_{2},z_{1}z_{2}}|\rho_{x_{1},a}^{2,U}\rangle|\rho_{x_{2},b}^{4,U}\rangle\Bigg)
\]
with both particles on vertices corresponding to output nodes. The two-qubit gate $U(a)$
% \[
% U(a)=\begin{cases}
% U & \text{ if }a=0\\
% U^{*} & \text{ if }a=1
% \end{cases}
% \]
is applied as the particles move from input nodes to output nodes.

\subsubsection{Boundary gadget}

%\begin{figure}
%\centering
%\begin{tikzpicture}[yscale=0.92] 
%
%\path[use as bounding box](-5.5,-3) rectangle (10,10.5);
% % Each Hadamard Rectangle 
%\foreach \xshift / \yshift /\xscale / \lab / \unitary in{3.12/0.5/1/4/1,3.12/3.5/1/2/1,-3.62/0.5/1/3/1,-3.62/3.5/1/1/1, -3.62/-2.5/1/7/1,3.12/-2.5/1/8/1, 3.12/7/1/6/1, -3.62/7/1/5/1}
%{ 
%\begin{scope}[shift={(\xshift,\yshift)},xscale=\xscale]
%  \draw[rounded corners=0.75mm,thick] (0,0) rectangle (2 cm, 2cm);
%  \node at (1,1) {\huge$\unitary$};   
%\node at (1,2.4) {\large\lab};
%  \foreach \y in {.33,.66,1.33,1.66}
%	{   
%		\foreach \x /\color in {0/black,2/gray}
%			{    
%				\draw[fill=\color,draw=\color] (\x cm, \y cm) circle (.66mm);  
%			}
%	} 
%\end{scope}
%}
%  % Each W Gadget
%\foreach \yshift/\from in {0.25/11,1.75/10,3.25/01,4.75/00}
%{ 
%\begin{scope}[yshift=\yshift cm] 
% \draw[rounded corners=0.75mm,thick] (0,0) rectangle (1.5cm,.93cm); 
% \node at (.75,.465) {$W$};   
%\node at (.75,1.2) {$\from$};
%  \foreach \x/\color in {0/black,1.5/gray}
%	{   
%		\foreach \y in {0.33,.6}
%		{   
%			 \draw[fill=\color,draw=\color] (\x,\y) circle (.66mm); 
%		 }
%	}
%\end{scope}
%}
%  % Right Connections
%\foreach \l/\r in {   5.35/5.16,5.08/2.16,   .85/3.83,.57/1.83,   3.85/4.83,3.57/1.16,   2.35/4.16,2.08/.83} 
%{   
%\node (a) at (1.5,\l) {}; 
% \node (b) at (3.12,\r)   {}; 
% \draw[looseness=.66,line width=4pt,color=white] (a) to [out=0,in=180] (b);
%  \draw[looseness=.66] (a) to [out=0,in=180] (b); 
%}
%  % Left Connections
%\foreach \l/\r in {5.16/5.35,2.16/5.08,   3.83/.85,.83/.57,   4.83/3.85,1.83/2.08,   4.16/2.35,1.16/3.57}
%{   
%\node (a) at (-1.618,\l) {}; 
% \node (b) at (0,\r)   {};
%  \draw[looseness=.66,line width=4pt,color=white] (a) to [out=0,in=180] (b);
%  \draw[looseness=.66] (a) to [out=0,in=180] (b); 
%}
%
%%%%%%%%Edges which connect to the two lower diagram elements
%
%%The one on the left-hand side
%
%\node (c) at (-3.62,1.83){};
%\node (d) at (-3.62,-1.17){};
%
% \draw[looseness=.66,line width=4pt,color=white] (c) to [out=180,in=180] (d);
% \draw[looseness=.66] (c) to [out=180,in=180] (d); 
%
%\node (c) at (-3.62,0.83){};
%\node (d) at (-3.62,-2.17){};
%
% \draw[looseness=.66,line width=4pt,color=white] (c) to [out=180,in=180] (d);
% \draw[looseness=.66] (c) to [out=180,in=180] (d); 
%
%% The one on the right-hand side
%
%\node (c) at (5.12,1.83){};
%\node (d) at (5.12,-1.17){};
%
% \draw[looseness=.66,line width=4pt,color=white] (c) to [out=0,in=0] (d);
% \draw[looseness=.66] (c) to [out=0,in=0] (d); 
%
%\node (c) at (5.12,0.83){};
%\node (d) at (5.12,-2.17){};
%
% \draw[looseness=.66,line width=4pt,color=white] (c) to [out=0,in=0] (d);
% \draw[looseness=.66] (c) to [out=0,in=0] (d); 
%
%%%%%%%%Edges which connect to the two upper diagram elements
%
%%The ones on the left-hand side
%
%\node (c) at (-3.62,8.33){};
%\node (d) at (-3.62,4.83){};
%
% \draw[looseness=.66,line width=4pt,color=white] (c) to [out=180,in=180] (d);
% \draw[looseness=.66] (c) to [out=180,in=180] (d); 
%
%\node (c) at (-1.62,7.33){};
%\node (d) at (5.12,3.83){};
%
% \draw[looseness=1.5,line width=4pt,color=white] (c) to [out=0,in=10] (d);
% \draw[looseness=1.5] (c) to [out=0,in=10] (d); 
%
%%The ones on the right-hand side
%
%\node (c) at (5.12,8.33){};
%\node (d) at (5.12,4.83){};
%
% \draw[looseness=.66,line width=4pt,color=white] (c) to [out=0,in=0] (d);
% \draw[looseness=.66] (c) to [out=0,in=0] (d); 
%
%\node (c) at (3.12,7.33){};
%\node (d) at (-3.62,3.83){};
%
% \draw[looseness=1.5,line width=4pt,color=white] (c) to [out=180,in=170] (d);
% \draw[looseness=1.5] (c) to [out=180,in=170] (d); 
%
%
%
%
%  % Node Labels
%
%\draw[looseness=150] (-3.7,5.19) to [out=150,in=210] (-3.7,5.18) ;   
%\draw[looseness=150] (-3.7,4.19) to [out=150,in=210] (-3.7,4.18) ;   
%
%\draw[looseness=150] (-3.7,2.19) to [out=150,in=210] (-3.7,2.18) ;   
%\draw[looseness=150] (-3.7,1.19) to [out=150,in=210] (-3.7,1.18) ;   
%
%\draw[looseness=150] (5.2,5.19) to [out=30,in=-30] (5.2,5.18) ;   
%\draw[looseness=150] (5.2,4.19) to [out=30,in=-30] (5.2,4.18) ;   
%
%\node at (5.4,2.2) {$\alpha$};
%\node at (5.4,1.2) {$\beta$};
%\node at (5.4,-0.8) {$\gamma$};
%\node at (5.4,-1.8) {$\delta$};
%
%  \node at (6.5cm,3cm) {\huge $=$};
%\begin{scope}[xshift = 7.5cm,yshift = 3.81cm]       
%\draw[rounded corners = 2mm,thick] (0,0) -- (.6,0) -- (.6,-.5) -- (1.4,-.5) -- (1.4,-1.618) -- (0,-1.618) -- cycle;                    \draw[fill=black] (1,-.5) circle (.66mm);             
%\draw[fill=black] (1.4,-.81) circle (.66mm);       
%\draw[fill=black] (1.4,-1.41) circle (.66mm);       
%\draw[fill=black] (1,-1.618) circle (.66mm);              
%\node at (1,-.25) {$\alpha$};       
%\node at (1.6,-.76) {$\gamma$};       
%\node at (1.6,-1.46) {$\beta$};       
%\node at (1,-1.868) {$\delta$};              
%\node at (.7,-1.06) {\Large Bnd};          
%\end{scope}
%\end{tikzpicture}
%\caption{The gate diagram for the boundary gadget is obtained from \fig{GVucnot} by setting $\tilde U=1$ and adding 6 self-loops.}\label{fig:GVbdy}
%\end{figure}

The \emph{boundary gadget} is shown in \fig{GVbdy}. This gate diagram is obtained from \fig{GVucnot} (with $\tilde U=\II$) by adding self-loops. The adjacency matrix is
\[
  A(G_{\text{bnd}})=A(G_{\CNOT_{12}})+h_{\mathcal{S}}
\]
where 
\[
  h_{\mathcal{S}}
  =\sum_{z=0}^{1}(
    |1,z,1\rangle\langle1,z,1|\otimes\II_{j}
   +|2,z,5\rangle\langle2,z,5|\otimes\II_{j}
   +|3,z,1\rangle\langle3,z,1|\otimes\II_{j}
  ).
\]
The single-particle ground states (with energy $e_{1}$) are superpositions of the states $|\rho_{z,a}^{i,U}\rangle$ from \lem{2qub_gate} that are in the nullspace of $h_{\mathcal{S}}$. Note that 
\[
\langle\rho_{x,b}^{j,U}|h_{\mathcal{S}}|\rho_{z,a}^{i,U}\rangle=\delta_{a,b}\delta_{x,z}\left(\delta_{i,1}\delta_{j,1}+\delta_{i,2}\delta_{j,2}+\delta_{i,3}\delta_{j,3}\right)\frac{1}{8}\cdot\frac{1}{8}
\]
(one factor of $\frac{1}{8}$ comes from the normalization in equations \eq{rho1_1}--\eq{rho2_1} and the other factor comes from the normalization in equation \eq{psi0m}), so the only single-particle ground states are 
\[
|\rho_{z,a}^{\text{bnd}}\rangle = |\rho_{z,a}^{4,U}\rangle
\]
with $z,a\in\{0,1\}$. Thus there are no two- (or more) particle frustration-free states, because no superposition of the states \eq{twopartstate_1} lies in the subspace 
\[
\spn\{ \Sym(|\rho_{z,a}^{4,U}\rangle|\rho_{x,b}^{4,U}\rangle)\colon z,a,x,b\in\{0,1\}\} 
\]
of states with single-particle reduced density matrices in the ground space of $A(G_{\text{bnd}})$.  We summarize these results as follows.

\begin{lemma}\label{lem:boundary_lemma}
The smallest eigenvalue of $A(G_{\text{bnd}})$ is $e_{1}$, with corresponding eigenvectors 
\begin{equation}
|\rho_{z,a}^{\text{bnd}}\rangle=\frac{1}{\sqrt{8}}|\psi_{z,a}^{4}\rangle-\frac{1}{\sqrt{8}}|\psi_{z,a}^{8}\rangle-\sqrt{\frac{3}{8}}\sum_{x=0,1}|\chi_{4,a}^{x\left(z\oplus x\right)}\rangle.\label{eq:rho_bnd}
\end{equation}
There are no frustration-free states with two or more particles, i.e., $\lambda_{N}^{1}(G_{\text{bnd}})>0$ for $N\geq2$.
\end{lemma}


\subsection{Gate graph for a given circuit}
\subsubsection{Occupancy constraints graph}

\section{Proof of QMA-hardness for MPQW ground energy}

\subsection{Overview}

\subsection{Configurations}

\subsubsection{Legal configurations}

\subsection{The occupancy constraints lemma}

\subsection{Completeness and Soundness}

\section{Open questions}

\end{document}